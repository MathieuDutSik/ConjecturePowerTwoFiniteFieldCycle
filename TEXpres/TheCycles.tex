\documentclass[12pt]{article}
\usepackage{amsfonts, amsmath, latexsym, epic, eepic, pifont}
\usepackage{epsf}
\title{Sequence lists}
\usepackage{vmargin}
\setpapersize{custom}{21cm}{29.7cm}
\setmarginsrb{0.7cm}{2cm}{0.7cm}{3.5cm}{0pt}{0pt}{0pt}{0pt}
%marge gauche, marge haut, marge droite, marge bas.


\begin{document}
\newcommand{\RR}{\ensuremath{\mathbb{R}}}
\newcommand{\NN}{\ensuremath{\mathbb{N}}}
\newcommand{\QQ}{\ensuremath{\mathbb{Q}}}
\newcommand{\CC}{\ensuremath{\mathbb{C}}}
\newcommand{\ZZ}{\ensuremath{\mathbb{Z}}}
\newcommand{\TT}{\ensuremath{\mathbb{T}}}
\newcommand{\FF}{\ensuremath{\mathbb{F}}}
\newtheorem{prop}{Proposition}
\newtheorem{theorem}{Theorem}
\newtheorem{cor}{Corollary}
\newtheorem{lem}{Lemma}
\newtheorem{conjecture}{Conjecture}
\newtheorem{claim}{Claim}
\newtheorem{remark}{Remark}
\newtheorem{definition}{Definition}
\newtheorem{proposition}{Proposition}
\newcommand{\qed}{\hfill $\Box$ }
\newcommand{\proof}{\noindent{\bf Proof.}\ \ }


\maketitle


\begin{abstract}
We consider here some sequences arising from diagram geometry.
\end{abstract}


\section{Definition and basic properties}
Consider the Galois field $\FF_{2^n}$ with $2^n$ elements.
If $h$ is prime with $n$, then the function
\begin{equation*}
\begin{array}{rcl}
\phi_k:\FF_{2^n} &\rightarrow &\FF_{2^n}\\
x&\mapsto x^{2^h-1}
\end{array}
\end{equation*}
is bijective.

As a consequence the function
\begin{equation*}
\begin{array}{rcl}
\psi_{h,k}:\FF_{2^n} &\rightarrow &\FF_{2^n}\\
x&\mapsto (1+x^{\frac{2^h-1}{2^k-1}})^{\frac{2^k-1}{2^h-1}})-1
\end{array}
\end{equation*}
is bijective.

Of interest is to study the cycle structure of $\psi_{h,k}$ for $h$ and $k$ being primes with $n$, $1\leq h,k \leq n-1$.
This structure show up in the works of Antonio Pasini and Yoshiara Satoshi.





\section{Results for $n=5$}
\begin{itemize}
\item $(h,k)=(1,2)$ : $1^{2}$, $5^{2}$, $10^{2}$
\item $(h,k)=(1,3)$ : $1^{2}$, $15^{2}$
\item $(h,k)=(1,4)$ : $1^{2}$, $3^{10}$
\item $(h,k)=(2,3)$ : $1^{2}$, $3^{10}$
\item $(h,k)=(2,4)$ : $1^{2}$, $15^{2}$
\item $(h,k)=(3,4)$ : $1^{2}$, $5^{2}$, $10^{2}$
\end{itemize}

\section{Results for $n=6$}
\begin{itemize}
\item $(h,k)=(1,5)$ : $1^{4}$, $3^{20}$
\end{itemize}

\section{Results for $n=7$}
\begin{itemize}
\item $(h,k)=(1,2)$ : $1^{2}$, $7^{2}$, $14^{2}$, $21^{4}$
\item $(h,k)=(1,3)$ : $1^{2}$, $63^{2}$
\item $(h,k)=(1,4)$ : $1^{2}$, $63^{2}$
\item $(h,k)=(1,5)$ : $1^{2}$, $63^{2}$
\item $(h,k)=(1,6)$ : $1^{2}$, $3^{42}$
\item $(h,k)=(2,3)$ : $1^{2}$, $63^{2}$
\item $(h,k)=(2,4)$ : $1^{2}$, $14^{2}$, $49^{2}$
\item $(h,k)=(2,5)$ : $1^{2}$, $3^{42}$
\item $(h,k)=(2,6)$ : $1^{2}$, $63^{2}$
\item $(h,k)=(3,4)$ : $1^{2}$, $3^{42}$
\item $(h,k)=(3,5)$ : $1^{2}$, $14^{2}$, $49^{2}$
\item $(h,k)=(3,6)$ : $1^{2}$, $63^{2}$
\item $(h,k)=(4,5)$ : $1^{2}$, $63^{2}$
\item $(h,k)=(4,6)$ : $1^{2}$, $63^{2}$
\item $(h,k)=(5,6)$ : $1^{2}$, $7^{2}$, $14^{2}$, $21^{4}$
\end{itemize}

\section{Results for $n=8$}
\begin{itemize}
\item $(h,k)=(1,3)$ : $1^{4}$, $3^{4}$, $9^{16}$, $24^{4}$
\item $(h,k)=(1,5)$ : $1^{16}$, $4^{20}$, $7^{8}$, $8^{2}$, $11^{8}$
\item $(h,k)=(1,7)$ : $1^{4}$, $3^{84}$
\item $(h,k)=(3,5)$ : $1^{4}$, $3^{84}$
\item $(h,k)=(3,7)$ : $1^{16}$, $4^{20}$, $7^{8}$, $8^{2}$, $11^{8}$
\item $(h,k)=(5,7)$ : $1^{4}$, $3^{4}$, $9^{16}$, $24^{4}$
\end{itemize}

\section{Results for $n=9$}
\begin{itemize}
\item $(h,k)=(1,2)$ : $1^{2}$, $3^{2}$, $12^{6}$, $18^{2}$, $27^{2}$, $36^{2}$, $45^{6}$
\item $(h,k)=(1,4)$ : $1^{8}$, $3^{6}$, $9^{2}$, $10^{18}$, $36^{8}$
\item $(h,k)=(1,5)$ : $1^{2}$, $3^{2}$, $36^{2}$, $72^{6}$
\item $(h,k)=(1,7)$ : $1^{8}$, $3^{6}$, $6^{24}$, $9^{8}$, $12^{6}$, $15^{6}$, $18^{6}$
\item $(h,k)=(1,8)$ : $1^{2}$, $3^{170}$
\item $(h,k)=(2,4)$ : $1^{2}$, $3^{2}$, $60^{6}$, $72^{2}$
\item $(h,k)=(2,5)$ : $1^{8}$, $3^{6}$, $33^{6}$, $36^{8}$
\item $(h,k)=(2,7)$ : $1^{2}$, $3^{170}$
\item $(h,k)=(2,8)$ : $1^{8}$, $3^{6}$, $6^{24}$, $9^{8}$, $12^{6}$, $15^{6}$, $18^{6}$
\item $(h,k)=(4,5)$ : $1^{2}$, $3^{170}$
\item $(h,k)=(4,7)$ : $1^{8}$, $3^{6}$, $33^{6}$, $36^{8}$
\item $(h,k)=(4,8)$ : $1^{2}$, $3^{2}$, $36^{2}$, $72^{6}$
\item $(h,k)=(5,7)$ : $1^{2}$, $3^{2}$, $60^{6}$, $72^{2}$
\item $(h,k)=(5,8)$ : $1^{8}$, $3^{6}$, $9^{2}$, $10^{18}$, $36^{8}$
\item $(h,k)=(7,8)$ : $1^{2}$, $3^{2}$, $12^{6}$, $18^{2}$, $27^{2}$, $36^{2}$, $45^{6}$
\end{itemize}

\section{Results for $n=10$}
\begin{itemize}
\item $(h,k)=(1,3)$ : $1^{4}$, $7^{10}$, $10^{4}$, $12^{10}$, $14^{10}$, $15^{2}$, $18^{20}$, $65^{4}$
\item $(h,k)=(1,7)$ : $1^{4}$, $3^{10}$, $5^{2}$, $10^{2}$, $12^{20}$, $24^{20}$, $30^{2}$, $45^{4}$
\item $(h,k)=(1,9)$ : $1^{4}$, $3^{340}$
\item $(h,k)=(3,7)$ : $1^{4}$, $3^{340}$
\item $(h,k)=(3,9)$ : $1^{4}$, $3^{10}$, $5^{2}$, $10^{2}$, $12^{20}$, $24^{20}$, $30^{2}$, $45^{4}$
\item $(h,k)=(7,9)$ : $1^{4}$, $7^{10}$, $10^{4}$, $12^{10}$, $14^{10}$, $15^{2}$, $18^{20}$, $65^{4}$
\end{itemize}

\section{Results for $n=11$}
\begin{itemize}
\item $(h,k)=(1,2)$ : $1^{2}$, $11^{2}$, $22^{2}$, $33^{4}$, $44^{8}$, $55^{6}$, $66^{6}$, $77^{2}$, $88^{2}$, $99^{2}$, $121^{2}$
\item $(h,k)=(1,3)$ : $1^{2}$, $11^{4}$, $22^{2}$, $110^{2}$, $869^{2}$
\item $(h,k)=(1,4)$ : $1^{2}$, $121^{2}$, $902^{2}$
\item $(h,k)=(1,5)$ : $1^{2}$, $55^{2}$, $308^{2}$, $660^{2}$
\item $(h,k)=(1,6)$ : $1^{2}$, $11^{2}$, $44^{2}$, $88^{22}$
\item $(h,k)=(1,7)$ : $1^{2}$, $1023^{2}$
\item $(h,k)=(1,8)$ : $1^{2}$, $55^{2}$, $77^{4}$, $165^{2}$, $649^{2}$
\item $(h,k)=(1,9)$ : $1^{2}$, $132^{4}$, $198^{4}$, $363^{2}$
\item $(h,k)=(1,10)$ : $1^{2}$, $3^{682}$
\item $(h,k)=(2,3)$ : $1^{2}$, $1023^{2}$
\item $(h,k)=(2,4)$ : $1^{2}$, $11^{2}$, $44^{2}$, $66^{2}$, $198^{2}$, $231^{2}$, $473^{2}$
\item $(h,k)=(2,5)$ : $1^{2}$, $65^{22}$, $66^{2}$, $242^{2}$
\item $(h,k)=(2,6)$ : $1^{2}$, $1023^{2}$
\item $(h,k)=(2,7)$ : $1^{2}$, $11^{2}$, $44^{2}$, $45^{22}$, $88^{2}$, $99^{2}$, $121^{2}$, $165^{2}$
\item $(h,k)=(2,8)$ : $1^{2}$, $1023^{2}$
\item $(h,k)=(2,9)$ : $1^{2}$, $3^{682}$
\item $(h,k)=(2,10)$ : $1^{2}$, $132^{4}$, $198^{4}$, $363^{2}$
\item $(h,k)=(3,4)$ : $1^{2}$, $22^{2}$, $66^{2}$, $935^{2}$
\item $(h,k)=(3,5)$ : $1^{2}$, $44^{2}$, $231^{2}$, $748^{2}$
\item $(h,k)=(3,6)$ : $1^{2}$, $11^{2}$, $68^{22}$, $264^{2}$
\item $(h,k)=(3,7)$ : $1^{2}$, $11^{2}$, $33^{2}$, $55^{2}$, $66^{2}$, $77^{2}$, $132^{2}$, $198^{2}$, $451^{2}$
\item $(h,k)=(3,8)$ : $1^{2}$, $3^{682}$
\item $(h,k)=(3,9)$ : $1^{2}$, $1023^{2}$
\item $(h,k)=(3,10)$ : $1^{2}$, $55^{2}$, $77^{4}$, $165^{2}$, $649^{2}$
\item $(h,k)=(4,5)$ : $1^{2}$, $220^{2}$, $803^{2}$
\item $(h,k)=(4,6)$ : $1^{2}$, $22^{2}$, $44^{2}$, $176^{2}$, $781^{2}$
\item $(h,k)=(4,7)$ : $1^{2}$, $3^{682}$
\item $(h,k)=(4,8)$ : $1^{2}$, $11^{2}$, $33^{2}$, $55^{2}$, $66^{2}$, $77^{2}$, $132^{2}$, $198^{2}$, $451^{2}$
\item $(h,k)=(4,9)$ : $1^{2}$, $11^{2}$, $44^{2}$, $45^{22}$, $88^{2}$, $99^{2}$, $121^{2}$, $165^{2}$
\item $(h,k)=(4,10)$ : $1^{2}$, $1023^{2}$
\item $(h,k)=(5,6)$ : $1^{2}$, $3^{682}$
\item $(h,k)=(5,7)$ : $1^{2}$, $22^{2}$, $44^{2}$, $176^{2}$, $781^{2}$
\item $(h,k)=(5,8)$ : $1^{2}$, $11^{2}$, $68^{22}$, $264^{2}$
\item $(h,k)=(5,9)$ : $1^{2}$, $1023^{2}$
\item $(h,k)=(5,10)$ : $1^{2}$, $11^{2}$, $44^{2}$, $88^{22}$
\item $(h,k)=(6,7)$ : $1^{2}$, $220^{2}$, $803^{2}$
\item $(h,k)=(6,8)$ : $1^{2}$, $44^{2}$, $231^{2}$, $748^{2}$
\item $(h,k)=(6,9)$ : $1^{2}$, $65^{22}$, $66^{2}$, $242^{2}$
\item $(h,k)=(6,10)$ : $1^{2}$, $55^{2}$, $308^{2}$, $660^{2}$
\item $(h,k)=(7,8)$ : $1^{2}$, $22^{2}$, $66^{2}$, $935^{2}$
\item $(h,k)=(7,9)$ : $1^{2}$, $11^{2}$, $44^{2}$, $66^{2}$, $198^{2}$, $231^{2}$, $473^{2}$
\item $(h,k)=(7,10)$ : $1^{2}$, $121^{2}$, $902^{2}$
\item $(h,k)=(8,9)$ : $1^{2}$, $1023^{2}$
\item $(h,k)=(8,10)$ : $1^{2}$, $11^{4}$, $22^{2}$, $110^{2}$, $869^{2}$
\item $(h,k)=(9,10)$ : $1^{2}$, $11^{2}$, $22^{2}$, $33^{4}$, $44^{8}$, $55^{6}$, $66^{6}$, $77^{2}$, $88^{2}$, $99^{2}$, $121^{2}$
\end{itemize}

\section{Results for $n=12$}
\begin{itemize}
\item $(h,k)=(1,5)$ : $1^{16}$, $3^{40}$, $7^{24}$, $8^{12}$, $11^{12}$, $12^{8}$, $16^{12}$, $19^{12}$, $21^{24}$, $27^{8}$, $35^{24}$, $43^{24}$, $57^{8}$
\item $(h,k)=(1,7)$ : $1^{64}$, $3^{28}$, $4^{36}$, $9^{8}$, $12^{8}$, $13^{12}$, $14^{24}$, $16^{24}$, $24^{18}$, $28^{12}$, $29^{12}$, $31^{12}$, $32^{12}$, $35^{12}$, $39^{12}$
\item $(h,k)=(1,11)$ : $1^{4}$, $3^{1364}$
\item $(h,k)=(5,7)$ : $1^{4}$, $3^{1364}$
\item $(h,k)=(5,11)$ : $1^{64}$, $3^{28}$, $4^{36}$, $9^{8}$, $12^{8}$, $13^{12}$, $14^{24}$, $16^{24}$, $24^{18}$, $28^{12}$, $29^{12}$, $31^{12}$, $32^{12}$, $35^{12}$, $39^{12}$
\item $(h,k)=(7,11)$ : $1^{16}$, $3^{40}$, $7^{24}$, $8^{12}$, $11^{12}$, $12^{8}$, $16^{12}$, $19^{12}$, $21^{24}$, $27^{8}$, $35^{24}$, $43^{24}$, $57^{8}$
\end{itemize}

\section{Results for $n=13$}
\begin{itemize}
\item $(h,k)=(1,2)$ : $1^{2}$, $13^{4}$, $16^{26}$, $39^{6}$, $52^{4}$, $65^{12}$, $78^{8}$, $91^{6}$, $104^{12}$, $117^{6}$, $130^{4}$, $143^{2}$, $156^{2}$, $169^{2}$, $182^{8}$, $234^{2}$
\item $(h,k)=(1,3)$ : $1^{2}$, $143^{2}$, $208^{2}$, $3744^{2}$
\item $(h,k)=(1,4)$ : $1^{2}$, $247^{2}$, $3848^{2}$
\item $(h,k)=(1,5)$ : $1^{2}$, $39^{2}$, $4056^{2}$
\item $(h,k)=(1,6)$ : $1^{2}$, $13^{2}$, $4082^{2}$
\item $(h,k)=(1,7)$ : $1^{2}$, $315^{26}$
\item $(h,k)=(1,8)$ : $1^{2}$, $13^{2}$, $4082^{2}$
\item $(h,k)=(1,9)$ : $1^{2}$, $91^{2}$, $182^{2}$, $3822^{2}$
\item $(h,k)=(1,10)$ : $1^{2}$, $13^{2}$, $104^{2}$, $1924^{2}$, $2054^{2}$
\item $(h,k)=(1,11)$ : $1^{2}$, $39^{2}$, $195^{6}$, $234^{8}$, $273^{2}$, $312^{6}$, $351^{2}$, $429^{2}$, $546^{2}$
\item $(h,k)=(1,12)$ : $1^{2}$, $3^{2730}$
\item $(h,k)=(2,3)$ : $1^{2}$, $26^{2}$, $283^{26}$, $390^{2}$
\item $(h,k)=(2,4)$ : $1^{2}$, $78^{2}$, $117^{2}$, $169^{2}$, $403^{2}$, $3328^{2}$
\item $(h,k)=(2,5)$ : $1^{2}$, $195^{2}$, $416^{2}$, $780^{2}$, $2704^{2}$
\item $(h,k)=(2,6)$ : $1^{2}$, $39^{2}$, $1391^{2}$, $2665^{2}$
\item $(h,k)=(2,7)$ : $1^{2}$, $52^{2}$, $4043^{2}$
\item $(h,k)=(2,8)$ : $1^{2}$, $39^{2}$, $1638^{2}$, $2418^{2}$
\item $(h,k)=(2,9)$ : $1^{2}$, $117^{2}$, $143^{2}$, $174^{26}$, $299^{2}$, $1274^{2}$
\item $(h,k)=(2,10)$ : $1^{2}$, $273^{2}$, $286^{2}$, $3536^{2}$
\item $(h,k)=(2,11)$ : $1^{2}$, $3^{2730}$
\item $(h,k)=(2,12)$ : $1^{2}$, $39^{2}$, $195^{6}$, $234^{8}$, $273^{2}$, $312^{6}$, $351^{2}$, $429^{2}$, $546^{2}$
\item $(h,k)=(3,4)$ : $1^{2}$, $13^{2}$, $39^{2}$, $4043^{2}$
\item $(h,k)=(3,5)$ : $1^{2}$, $13^{4}$, $26^{2}$, $39^{2}$, $52^{2}$, $117^{2}$, $3835^{2}$
\item $(h,k)=(3,6)$ : $1^{2}$, $13^{2}$, $169^{2}$, $3913^{2}$
\item $(h,k)=(3,7)$ : $1^{2}$, $169^{2}$, $364^{2}$, $572^{2}$, $2990^{2}$
\item $(h,k)=(3,8)$ : $1^{2}$, $13^{2}$, $39^{4}$, $169^{2}$, $598^{2}$, $3237^{2}$
\item $(h,k)=(3,9)$ : $1^{2}$, $13^{4}$, $221^{2}$, $299^{2}$, $3549^{2}$
\item $(h,k)=(3,10)$ : $1^{2}$, $3^{2730}$
\item $(h,k)=(3,11)$ : $1^{2}$, $273^{2}$, $286^{2}$, $3536^{2}$
\item $(h,k)=(3,12)$ : $1^{2}$, $13^{2}$, $104^{2}$, $1924^{2}$, $2054^{2}$
\item $(h,k)=(4,5)$ : $1^{2}$, $286^{2}$, $338^{2}$, $3471^{2}$
\item $(h,k)=(4,6)$ : $1^{2}$, $39^{2}$, $4056^{2}$
\item $(h,k)=(4,7)$ : $1^{2}$, $13^{2}$, $104^{2}$, $273^{2}$, $3705^{2}$
\item $(h,k)=(4,8)$ : $1^{2}$, $13^{4}$, $26^{2}$, $39^{2}$, $403^{2}$, $988^{2}$, $1131^{2}$, $1482^{2}$
\item $(h,k)=(4,9)$ : $1^{2}$, $3^{2730}$
\item $(h,k)=(4,10)$ : $1^{2}$, $13^{4}$, $221^{2}$, $299^{2}$, $3549^{2}$
\item $(h,k)=(4,11)$ : $1^{2}$, $117^{2}$, $143^{2}$, $174^{26}$, $299^{2}$, $1274^{2}$
\item $(h,k)=(4,12)$ : $1^{2}$, $91^{2}$, $182^{2}$, $3822^{2}$
\item $(h,k)=(5,6)$ : $1^{2}$, $923^{2}$, $3172^{2}$
\item $(h,k)=(5,7)$ : $1^{2}$, $52^{2}$, $4043^{2}$
\item $(h,k)=(5,8)$ : $1^{2}$, $3^{2730}$
\item $(h,k)=(5,9)$ : $1^{2}$, $13^{4}$, $26^{2}$, $39^{2}$, $403^{2}$, $988^{2}$, $1131^{2}$, $1482^{2}$
\item $(h,k)=(5,10)$ : $1^{2}$, $13^{2}$, $39^{4}$, $169^{2}$, $598^{2}$, $3237^{2}$
\item $(h,k)=(5,11)$ : $1^{2}$, $39^{2}$, $1638^{2}$, $2418^{2}$
\item $(h,k)=(5,12)$ : $1^{2}$, $13^{2}$, $4082^{2}$
\item $(h,k)=(6,7)$ : $1^{2}$, $3^{2730}$
\item $(h,k)=(6,8)$ : $1^{2}$, $52^{2}$, $4043^{2}$
\item $(h,k)=(6,9)$ : $1^{2}$, $13^{2}$, $104^{2}$, $273^{2}$, $3705^{2}$
\item $(h,k)=(6,10)$ : $1^{2}$, $169^{2}$, $364^{2}$, $572^{2}$, $2990^{2}$
\item $(h,k)=(6,11)$ : $1^{2}$, $52^{2}$, $4043^{2}$
\item $(h,k)=(6,12)$ : $1^{2}$, $315^{26}$
\item $(h,k)=(7,8)$ : $1^{2}$, $923^{2}$, $3172^{2}$
\item $(h,k)=(7,9)$ : $1^{2}$, $39^{2}$, $4056^{2}$
\item $(h,k)=(7,10)$ : $1^{2}$, $13^{2}$, $169^{2}$, $3913^{2}$
\item $(h,k)=(7,11)$ : $1^{2}$, $39^{2}$, $1391^{2}$, $2665^{2}$
\item $(h,k)=(7,12)$ : $1^{2}$, $13^{2}$, $4082^{2}$
\item $(h,k)=(8,9)$ : $1^{2}$, $286^{2}$, $338^{2}$, $3471^{2}$
\item $(h,k)=(8,10)$ : $1^{2}$, $13^{4}$, $26^{2}$, $39^{2}$, $52^{2}$, $117^{2}$, $3835^{2}$
\item $(h,k)=(8,11)$ : $1^{2}$, $195^{2}$, $416^{2}$, $780^{2}$, $2704^{2}$
\item $(h,k)=(8,12)$ : $1^{2}$, $39^{2}$, $4056^{2}$
\item $(h,k)=(9,10)$ : $1^{2}$, $13^{2}$, $39^{2}$, $4043^{2}$
\item $(h,k)=(9,11)$ : $1^{2}$, $78^{2}$, $117^{2}$, $169^{2}$, $403^{2}$, $3328^{2}$
\item $(h,k)=(9,12)$ : $1^{2}$, $247^{2}$, $3848^{2}$
\item $(h,k)=(10,11)$ : $1^{2}$, $26^{2}$, $283^{26}$, $390^{2}$
\item $(h,k)=(10,12)$ : $1^{2}$, $143^{2}$, $208^{2}$, $3744^{2}$
\item $(h,k)=(11,12)$ : $1^{2}$, $13^{4}$, $16^{26}$, $39^{6}$, $52^{4}$, $65^{12}$, $78^{8}$, $91^{6}$, $104^{12}$, $117^{6}$, $130^{4}$, $143^{2}$, $156^{2}$, $169^{2}$, $182^{8}$, $234^{2}$
\end{itemize}

\section{Results for $n=14$}
\begin{itemize}
\item $(h,k)=(1,3)$ : $1^{4}$, $10^{14}$, $14^{4}$, $23^{14}$, $24^{14}$, $25^{28}$, $27^{14}$, $57^{14}$, $63^{2}$, $69^{14}$, $127^{14}$, $385^{4}$, $2310^{4}$
\item $(h,k)=(1,5)$ : $1^{4}$, $6^{28}$, $14^{18}$, $15^{14}$, $17^{14}$, $21^{14}$, $52^{14}$, $56^{4}$, $57^{14}$, $58^{14}$, $60^{14}$, $63^{2}$, $77^{4}$, $89^{14}$, $427^{4}$, $2107^{4}$
\item $(h,k)=(1,9)$ : $1^{4}$, $4^{14}$, $7^{2}$, $14^{2}$, $20^{14}$, $21^{4}$, $27^{28}$, $29^{14}$, $41^{14}$, $51^{14}$, $56^{4}$, $58^{14}$, $98^{4}$, $130^{14}$, $2555^{4}$
\item $(h,k)=(1,11)$ : $1^{4}$, $3^{14}$, $5^{28}$, $10^{14}$, $14^{4}$, $23^{14}$, $40^{14}$, $57^{14}$, $60^{14}$, $63^{6}$, $77^{4}$, $112^{14}$, $238^{4}$, $679^{4}$, $1890^{4}$
\item $(h,k)=(1,13)$ : $1^{4}$, $3^{5460}$
\item $(h,k)=(3,5)$ : $1^{4}$, $3^{14}$, $7^{4}$, $9^{14}$, $10^{14}$, $12^{28}$, $14^{6}$, $17^{14}$, $21^{4}$, $35^{14}$, $49^{2}$, $52^{14}$, $235^{14}$, $2674^{4}$
\item $(h,k)=(3,9)$ : $1^{4}$, $5^{14}$, $9^{14}$, $11^{14}$, $19^{14}$, $42^{14}$, $48^{14}$, $63^{2}$, $67^{14}$, $90^{14}$, $96^{14}$, $122^{28}$, $1855^{4}$
\item $(h,k)=(3,11)$ : $1^{4}$, $3^{5460}$
\item $(h,k)=(3,13)$ : $1^{4}$, $3^{14}$, $5^{28}$, $10^{14}$, $14^{4}$, $23^{14}$, $40^{14}$, $57^{14}$, $60^{14}$, $63^{6}$, $77^{4}$, $112^{14}$, $238^{4}$, $679^{4}$, $1890^{4}$
\item $(h,k)=(5,9)$ : $1^{4}$, $3^{5460}$
\item $(h,k)=(5,11)$ : $1^{4}$, $5^{14}$, $9^{14}$, $11^{14}$, $19^{14}$, $42^{14}$, $48^{14}$, $63^{2}$, $67^{14}$, $90^{14}$, $96^{14}$, $122^{28}$, $1855^{4}$
\item $(h,k)=(5,13)$ : $1^{4}$, $4^{14}$, $7^{2}$, $14^{2}$, $20^{14}$, $21^{4}$, $27^{28}$, $29^{14}$, $41^{14}$, $51^{14}$, $56^{4}$, $58^{14}$, $98^{4}$, $130^{14}$, $2555^{4}$
\item $(h,k)=(9,11)$ : $1^{4}$, $3^{14}$, $7^{4}$, $9^{14}$, $10^{14}$, $12^{28}$, $14^{6}$, $17^{14}$, $21^{4}$, $35^{14}$, $49^{2}$, $52^{14}$, $235^{14}$, $2674^{4}$
\item $(h,k)=(9,13)$ : $1^{4}$, $6^{28}$, $14^{18}$, $15^{14}$, $17^{14}$, $21^{14}$, $52^{14}$, $56^{4}$, $57^{14}$, $58^{14}$, $60^{14}$, $63^{2}$, $77^{4}$, $89^{14}$, $427^{4}$, $2107^{4}$
\item $(h,k)=(11,13)$ : $1^{4}$, $10^{14}$, $14^{4}$, $23^{14}$, $24^{14}$, $25^{28}$, $27^{14}$, $57^{14}$, $63^{2}$, $69^{14}$, $127^{14}$, $385^{4}$, $2310^{4}$
\end{itemize}

\section{Results for $n=15$}
\begin{itemize}
\item $(h,k)=(1,2)$ : $1^{2}$, $3^{2}$, $5^{8}$, $10^{8}$, $11^{90}$, $18^{50}$, $20^{30}$, $25^{6}$, $30^{24}$, $33^{10}$, $35^{36}$, $36^{20}$, $39^{10}$, $40^{42}$, $45^{18}$, $55^{6}$, $60^{18}$, $65^{6}$, $70^{6}$, $75^{16}$, $80^{6}$, $90^{8}$, $100^{6}$, $105^{16}$, $120^{14}$, $125^{6}$, $130^{6}$, $150^{8}$, $165^{2}$, $175^{6}$, $180^{2}$, $195^{2}$, $210^{2}$, $225^{2}$, $240^{2}$, $255^{4}$, $270^{2}$, $285^{4}$, $300^{2}$, $330^{2}$, $345^{2}$, $360^{2}$, $375^{2}$, $510^{2}$, $525^{2}$, $555^{2}$
\item $(h,k)=(1,4)$ : $1^{8}$, $3^{10}$, $5^{6}$, $10^{6}$, $20^{12}$, $45^{6}$, $66^{30}$, $160^{6}$, $300^{6}$, $395^{6}$, $470^{6}$, $930^{2}$, $1070^{6}$, $1075^{6}$, $1245^{6}$
\item $(h,k)=(1,7)$ : $1^{8}$, $5^{2}$, $10^{2}$, $165^{2}$, $175^{30}$, $182^{30}$, $560^{6}$, $695^{6}$, $2360^{6}$
\item $(h,k)=(1,8)$ : $1^{2}$, $3^{2}$, $15^{2}$, $30^{6}$, $75^{14}$, $150^{210}$
\item $(h,k)=(1,11)$ : $1^{32}$, $3^{22}$, $5^{30}$, $6^{20}$, $9^{10}$, $10^{30}$, $13^{30}$, $15^{10}$, $24^{10}$, $28^{30}$, $30^{10}$, $63^{10}$, $117^{10}$, $120^{10}$, $158^{30}$, $162^{30}$, $168^{10}$, $185^{6}$, $258^{10}$, $309^{10}$, $399^{10}$, $504^{10}$
\item $(h,k)=(1,13)$ : $1^{8}$, $5^{6}$, $7^{30}$, $9^{30}$, $10^{42}$, $13^{30}$, $14^{120}$, $15^{42}$, $16^{60}$, $20^{24}$, $25^{24}$, $30^{44}$, $35^{30}$, $38^{30}$, $40^{36}$, $45^{24}$, $50^{12}$, $55^{6}$, $60^{24}$, $70^{36}$, $75^{18}$, $80^{24}$, $85^{18}$, $90^{6}$, $95^{18}$, $105^{6}$, $125^{6}$, $130^{6}$, $135^{12}$, $140^{12}$, $145^{12}$, $155^{6}$, $165^{6}$
\item $(h,k)=(1,14)$ : $1^{2}$, $3^{10922}$
\item $(h,k)=(2,4)$ : $1^{2}$, $3^{2}$, $15^{6}$, $35^{6}$, $54^{10}$, $104^{30}$, $4800^{6}$
\item $(h,k)=(2,7)$ : $1^{32}$, $3^{22}$, $6^{30}$, $9^{10}$, $15^{10}$, $21^{10}$, $23^{30}$, $28^{30}$, $36^{10}$, $42^{10}$, $51^{10}$, $60^{2}$, $104^{30}$, $117^{10}$, $147^{10}$, $171^{30}$, $255^{2}$, $303^{10}$, $471^{10}$, $486^{10}$, $510^{10}$
\item $(h,k)=(2,8)$ : $1^{8}$, $3^{10}$, $5^{12}$, $15^{2}$, $30^{2}$, $70^{6}$, $80^{30}$, $90^{2}$, $225^{6}$, $270^{10}$, $350^{6}$, $355^{30}$, $555^{2}$, $885^{6}$, $1060^{6}$
\item $(h,k)=(2,11)$ : $1^{8}$, $5^{2}$, $10^{2}$, $14^{30}$, $35^{6}$, $80^{6}$, $134^{30}$, $280^{6}$, $735^{6}$, $1295^{6}$, $2290^{6}$
\item $(h,k)=(2,13)$ : $1^{2}$, $3^{10922}$
\item $(h,k)=(2,14)$ : $1^{8}$, $5^{6}$, $7^{30}$, $9^{30}$, $10^{42}$, $13^{30}$, $14^{120}$, $15^{42}$, $16^{60}$, $20^{24}$, $25^{24}$, $30^{44}$, $35^{30}$, $38^{30}$, $40^{36}$, $45^{24}$, $50^{12}$, $55^{6}$, $60^{24}$, $70^{36}$, $75^{18}$, $80^{24}$, $85^{18}$, $90^{6}$, $95^{18}$, $105^{6}$, $125^{6}$, $130^{6}$, $135^{12}$, $140^{12}$, $145^{12}$, $155^{6}$, $165^{6}$
\item $(h,k)=(4,7)$ : $1^{8}$, $5^{6}$, $10^{6}$, $15^{34}$, $20^{6}$, $34^{30}$, $35^{12}$, $65^{6}$, $70^{6}$, $265^{6}$, $320^{6}$, $381^{10}$, $435^{6}$, $730^{6}$, $1195^{6}$, $1385^{6}$
\item $(h,k)=(4,8)$ : $1^{2}$, $3^{2}$, $5^{14}$, $10^{2}$, $15^{12}$, $16^{30}$, $153^{10}$, $215^{6}$, $520^{6}$, $1320^{6}$, $1350^{2}$, $1875^{2}$, $5850^{2}$
\item $(h,k)=(4,11)$ : $1^{2}$, $3^{10922}$
\item $(h,k)=(4,13)$ : $1^{8}$, $5^{2}$, $10^{2}$, $14^{30}$, $35^{6}$, $80^{6}$, $134^{30}$, $280^{6}$, $735^{6}$, $1295^{6}$, $2290^{6}$
\item $(h,k)=(4,14)$ : $1^{32}$, $3^{22}$, $5^{30}$, $6^{20}$, $9^{10}$, $10^{30}$, $13^{30}$, $15^{10}$, $24^{10}$, $28^{30}$, $30^{10}$, $63^{10}$, $117^{10}$, $120^{10}$, $158^{30}$, $162^{30}$, $168^{10}$, $185^{6}$, $258^{10}$, $309^{10}$, $399^{10}$, $504^{10}$
\item $(h,k)=(7,8)$ : $1^{2}$, $3^{10922}$
\item $(h,k)=(7,11)$ : $1^{2}$, $3^{2}$, $5^{14}$, $10^{2}$, $15^{12}$, $16^{30}$, $153^{10}$, $215^{6}$, $520^{6}$, $1320^{6}$, $1350^{2}$, $1875^{2}$, $5850^{2}$
\item $(h,k)=(7,13)$ : $1^{8}$, $3^{10}$, $5^{12}$, $15^{2}$, $30^{2}$, $70^{6}$, $80^{30}$, $90^{2}$, $225^{6}$, $270^{10}$, $350^{6}$, $355^{30}$, $555^{2}$, $885^{6}$, $1060^{6}$
\item $(h,k)=(7,14)$ : $1^{2}$, $3^{2}$, $15^{2}$, $30^{6}$, $75^{14}$, $150^{210}$
\item $(h,k)=(8,11)$ : $1^{8}$, $5^{6}$, $10^{6}$, $15^{34}$, $20^{6}$, $34^{30}$, $35^{12}$, $65^{6}$, $70^{6}$, $265^{6}$, $320^{6}$, $381^{10}$, $435^{6}$, $730^{6}$, $1195^{6}$, $1385^{6}$
\item $(h,k)=(8,13)$ : $1^{32}$, $3^{22}$, $6^{30}$, $9^{10}$, $15^{10}$, $21^{10}$, $23^{30}$, $28^{30}$, $36^{10}$, $42^{10}$, $51^{10}$, $60^{2}$, $104^{30}$, $117^{10}$, $147^{10}$, $171^{30}$, $255^{2}$, $303^{10}$, $471^{10}$, $486^{10}$, $510^{10}$
\item $(h,k)=(8,14)$ : $1^{8}$, $5^{2}$, $10^{2}$, $165^{2}$, $175^{30}$, $182^{30}$, $560^{6}$, $695^{6}$, $2360^{6}$
\item $(h,k)=(11,13)$ : $1^{2}$, $3^{2}$, $15^{6}$, $35^{6}$, $54^{10}$, $104^{30}$, $4800^{6}$
\item $(h,k)=(11,14)$ : $1^{8}$, $3^{10}$, $5^{6}$, $10^{6}$, $20^{12}$, $45^{6}$, $66^{30}$, $160^{6}$, $300^{6}$, $395^{6}$, $470^{6}$, $930^{2}$, $1070^{6}$, $1075^{6}$, $1245^{6}$
\item $(h,k)=(13,14)$ : $1^{2}$, $3^{2}$, $5^{8}$, $10^{8}$, $11^{90}$, $18^{50}$, $20^{30}$, $25^{6}$, $30^{24}$, $33^{10}$, $35^{36}$, $36^{20}$, $39^{10}$, $40^{42}$, $45^{18}$, $55^{6}$, $60^{18}$, $65^{6}$, $70^{6}$, $75^{16}$, $80^{6}$, $90^{8}$, $100^{6}$, $105^{16}$, $120^{14}$, $125^{6}$, $130^{6}$, $150^{8}$, $165^{2}$, $175^{6}$, $180^{2}$, $195^{2}$, $210^{2}$, $225^{2}$, $240^{2}$, $255^{4}$, $270^{2}$, $285^{4}$, $300^{2}$, $330^{2}$, $345^{2}$, $360^{2}$, $375^{2}$, $510^{2}$, $525^{2}$, $555^{2}$
\end{itemize}

\input{R16.tex}
\section{Results for $n=17$}
\begin{itemize}
\item $(h,k)=(1,2)$ : $1^{2}$, $17^{2}$, $18^{136}$, $22^{68}$, $25^{34}$, $27^{34}$, $32^{34}$, $34^{36}$, $51^{6}$, $52^{34}$, $68^{10}$, $85^{20}$, $102^{34}$, $119^{12}$, $136^{46}$, $153^{2}$, $170^{12}$, $204^{40}$, $221^{10}$, $238^{18}$, $255^{8}$, $272^{6}$, $289^{2}$, $306^{16}$, $323^{18}$, $340^{14}$, $374^{12}$, $391^{12}$, $408^{8}$, $425^{8}$, $442^{4}$, $459^{2}$, $476^{2}$, $510^{4}$, $527^{6}$, $544^{8}$, $578^{10}$, $595^{4}$, $612^{4}$, $646^{6}$, $663^{2}$, $680^{2}$, $697^{8}$, $731^{2}$, $748^{2}$, $782^{2}$, $799^{4}$, $884^{6}$, $952^{2}$, $986^{2}$, $1003^{2}$
\item $(h,k)=(1,3)$ : $1^{2}$, $68^{2}$, $102^{2}$, $510^{2}$, $64855^{2}$
\item $(h,k)=(1,4)$ : $1^{2}$, $136^{2}$, $3876^{2}$, $7548^{2}$, $53975^{2}$
\item $(h,k)=(1,5)$ : $1^{2}$, $17^{2}$, $34^{2}$, $153^{2}$, $221^{2}$, $1326^{2}$, $3672^{2}$, $15062^{2}$, $45050^{2}$
\item $(h,k)=(1,6)$ : $1^{2}$, $17^{2}$, $21^{34}$, $51^{2}$, $963^{34}$, $1615^{2}$, $3043^{2}$, $14008^{2}$, $30073^{2}$
\item $(h,k)=(1,7)$ : $1^{2}$, $35^{34}$, $238^{2}$, $340^{2}$, $9860^{2}$, $11271^{2}$, $43231^{2}$
\item $(h,k)=(1,8)$ : $1^{2}$, $459^{2}$, $65076^{2}$
\item $(h,k)=(1,9)$ : $1^{2}$, $65535^{2}$
\item $(h,k)=(1,10)$ : $1^{2}$, $34^{4}$, $204^{2}$, $32198^{2}$, $33065^{2}$
\item $(h,k)=(1,11)$ : $1^{2}$, $34^{2}$, $51^{2}$, $238^{2}$, $1904^{2}$, $3724^{34}$
\item $(h,k)=(1,12)$ : $1^{2}$, $65^{34}$, $64430^{2}$
\item $(h,k)=(1,13)$ : $1^{2}$, $17^{2}$, $68^{2}$, $238^{2}$, $391^{2}$, $3315^{2}$, $61506^{2}$
\item $(h,k)=(1,14)$ : $1^{2}$, $136^{2}$, $272^{2}$, $30345^{2}$, $34782^{2}$
\item $(h,k)=(1,15)$ : $1^{2}$, $51^{2}$, $102^{36}$, $153^{2}$, $306^{22}$, $357^{4}$, $408^{16}$, $510^{8}$, $612^{14}$, $663^{6}$, $714^{4}$, $816^{6}$, $918^{10}$, $969^{6}$, $1020^{2}$, $1122^{4}$, $1173^{10}$, $1224^{2}$, $1530^{2}$, $1581^{2}$, $1734^{4}$, $1938^{2}$, $2091^{2}$, $2244^{2}$, $2346^{2}$, $2652^{6}$, $2958^{2}$
\item $(h,k)=(1,16)$ : $1^{2}$, $3^{43690}$
\item $(h,k)=(2,3)$ : $1^{2}$, $34^{2}$, $136^{2}$, $476^{2}$, $1462^{2}$, $2941^{2}$, $5610^{2}$, $18292^{2}$, $36584^{2}$
\item $(h,k)=(2,4)$ : $1^{2}$, $17^{38}$, $34^{2}$, $153^{2}$, $5576^{2}$, $8415^{2}$, $17680^{2}$, $33354^{2}$
\item $(h,k)=(2,5)$ : $1^{2}$, $510^{2}$, $663^{2}$, $2329^{2}$, $2788^{2}$, $59245^{2}$
\item $(h,k)=(2,6)$ : $1^{2}$, $17^{2}$, $34^{2}$, $51^{2}$, $102^{2}$, $1258^{2}$, $1632^{2}$, $16626^{2}$, $45815^{2}$
\item $(h,k)=(2,7)$ : $1^{2}$, $34^{4}$, $119^{2}$, $11730^{2}$, $20672^{2}$, $32946^{2}$
\item $(h,k)=(2,8)$ : $1^{2}$, $1139^{2}$, $4505^{2}$, $59891^{2}$
\item $(h,k)=(2,9)$ : $1^{2}$, $119^{2}$, $884^{2}$, $16762^{2}$, $47770^{2}$
\item $(h,k)=(2,10)$ : $1^{2}$, $17^{2}$, $969^{2}$, $1224^{2}$, $2159^{2}$, $2805^{2}$, $14552^{2}$, $19873^{2}$, $23936^{2}$
\item $(h,k)=(2,11)$ : $1^{2}$, $34^{2}$, $527^{2}$, $10149^{2}$, $14450^{2}$, $40375^{2}$
\item $(h,k)=(2,12)$ : $1^{2}$, $17^{2}$, $85^{2}$, $160^{34}$, $2516^{2}$, $3842^{2}$, $5151^{2}$, $9724^{2}$, $41480^{2}$
\item $(h,k)=(2,13)$ : $1^{2}$, $289^{2}$, $2941^{2}$, $6273^{2}$, $15232^{2}$, $40800^{2}$
\item $(h,k)=(2,14)$ : $1^{2}$, $34^{2}$, $391^{2}$, $658^{34}$, $53924^{2}$
\item $(h,k)=(2,15)$ : $1^{2}$, $3^{43690}$
\item $(h,k)=(2,16)$ : $1^{2}$, $51^{2}$, $102^{36}$, $153^{2}$, $306^{22}$, $357^{4}$, $408^{16}$, $510^{8}$, $612^{14}$, $663^{6}$, $714^{4}$, $816^{6}$, $918^{10}$, $969^{6}$, $1020^{2}$, $1122^{4}$, $1173^{10}$, $1224^{2}$, $1530^{2}$, $1581^{2}$, $1734^{4}$, $1938^{2}$, $2091^{2}$, $2244^{2}$, $2346^{2}$, $2652^{6}$, $2958^{2}$
\item $(h,k)=(3,4)$ : $1^{2}$, $17^{2}$, $42^{34}$, $748^{2}$, $64056^{2}$
\item $(h,k)=(3,5)$ : $1^{2}$, $12886^{2}$, $52649^{2}$
\item $(h,k)=(3,6)$ : $1^{2}$, $17^{2}$, $51^{2}$, $119^{2}$, $272^{2}$, $425^{2}$, $816^{2}$, $5899^{2}$, $14093^{2}$, $43843^{2}$
\item $(h,k)=(3,7)$ : $1^{2}$, $1309^{2}$, $3536^{2}$, $12665^{2}$, $17578^{2}$, $30447^{2}$
\item $(h,k)=(3,8)$ : $1^{2}$, $323^{2}$, $65212^{2}$
\item $(h,k)=(3,9)$ : $1^{2}$, $17^{2}$, $1462^{2}$, $64056^{2}$
\item $(h,k)=(3,10)$ : $1^{2}$, $34^{2}$, $51^{4}$, $68^{2}$, $136^{2}$, $187^{2}$, $986^{2}$, $7548^{2}$, $8041^{2}$, $10438^{2}$, $12087^{2}$, $25908^{2}$
\item $(h,k)=(3,11)$ : $1^{2}$, $34^{2}$, $119^{2}$, $255^{2}$, $476^{2}$, $493^{2}$, $6273^{2}$, $57885^{2}$
\item $(h,k)=(3,12)$ : $1^{2}$, $102^{2}$, $119^{2}$, $323^{2}$, $13175^{2}$, $51816^{2}$
\item $(h,k)=(3,13)$ : $1^{2}$, $34^{2}$, $85^{2}$, $1292^{2}$, $1921^{2}$, $4046^{2}$, $58157^{2}$
\item $(h,k)=(3,14)$ : $1^{2}$, $3^{43690}$
\item $(h,k)=(3,15)$ : $1^{2}$, $34^{2}$, $391^{2}$, $658^{34}$, $53924^{2}$
\item $(h,k)=(3,16)$ : $1^{2}$, $136^{2}$, $272^{2}$, $30345^{2}$, $34782^{2}$
\item $(h,k)=(4,5)$ : $1^{2}$, $17^{2}$, $34^{2}$, $1853^{2}$, $3587^{2}$, $60044^{2}$
\item $(h,k)=(4,6)$ : $1^{2}$, $136^{2}$, $5270^{2}$, $5287^{2}$, $10098^{2}$, $21352^{2}$, $23392^{2}$
\item $(h,k)=(4,7)$ : $1^{2}$, $85^{2}$, $306^{2}$, $561^{2}$, $1054^{2}$, $2686^{2}$, $13345^{2}$, $47498^{2}$
\item $(h,k)=(4,8)$ : $1^{2}$, $17^{2}$, $80^{34}$, $102^{2}$, $221^{2}$, $692^{34}$, $731^{2}$, $816^{2}$, $1511^{34}$, $2771^{2}$, $8636^{2}$, $13430^{2}$
\item $(h,k)=(4,9)$ : $1^{2}$, $17^{2}$, $51^{2}$, $570^{34}$, $55777^{2}$
\item $(h,k)=(4,10)$ : $1^{2}$, $17^{2}$, $187^{2}$, $2210^{2}$, $13498^{2}$, $49623^{2}$
\item $(h,k)=(4,11)$ : $1^{2}$, $340^{2}$, $765^{2}$, $1377^{2}$, $2873^{2}$, $13362^{2}$, $46818^{2}$
\item $(h,k)=(4,12)$ : $1^{2}$, $16864^{2}$, $48671^{2}$
\item $(h,k)=(4,13)$ : $1^{2}$, $3^{43690}$
\item $(h,k)=(4,14)$ : $1^{2}$, $34^{2}$, $85^{2}$, $1292^{2}$, $1921^{2}$, $4046^{2}$, $58157^{2}$
\item $(h,k)=(4,15)$ : $1^{2}$, $289^{2}$, $2941^{2}$, $6273^{2}$, $15232^{2}$, $40800^{2}$
\item $(h,k)=(4,16)$ : $1^{2}$, $17^{2}$, $68^{2}$, $238^{2}$, $391^{2}$, $3315^{2}$, $61506^{2}$
\item $(h,k)=(5,6)$ : $1^{2}$, $17^{2}$, $51^{2}$, $629^{2}$, $64838^{2}$
\item $(h,k)=(5,7)$ : $1^{2}$, $17^{2}$, $357^{2}$, $1258^{2}$, $63903^{2}$
\item $(h,k)=(5,8)$ : $1^{2}$, $51^{2}$, $612^{2}$, $3196^{2}$, $12971^{2}$, $48705^{2}$
\item $(h,k)=(5,9)$ : $1^{2}$, $85^{2}$, $1530^{2}$, $4012^{2}$, $7038^{2}$, $52870^{2}$
\item $(h,k)=(5,10)$ : $1^{2}$, $51^{2}$, $119^{2}$, $153^{2}$, $238^{2}$, $323^{2}$, $986^{2}$, $1275^{2}$, $1649^{2}$, $3009^{2}$, $8925^{2}$, $20978^{2}$, $27829^{2}$
\item $(h,k)=(5,11)$ : $1^{2}$, $85^{2}$, $340^{2}$, $2193^{2}$, $62917^{2}$
\item $(h,k)=(5,12)$ : $1^{2}$, $3^{43690}$
\item $(h,k)=(5,13)$ : $1^{2}$, $16864^{2}$, $48671^{2}$
\item $(h,k)=(5,14)$ : $1^{2}$, $102^{2}$, $119^{2}$, $323^{2}$, $13175^{2}$, $51816^{2}$
\item $(h,k)=(5,15)$ : $1^{2}$, $17^{2}$, $85^{2}$, $160^{34}$, $2516^{2}$, $3842^{2}$, $5151^{2}$, $9724^{2}$, $41480^{2}$
\item $(h,k)=(5,16)$ : $1^{2}$, $65^{34}$, $64430^{2}$
\item $(h,k)=(6,7)$ : $1^{2}$, $34^{2}$, $680^{2}$, $1079^{34}$, $46478^{2}$
\item $(h,k)=(6,8)$ : $1^{2}$, $17^{2}$, $34^{2}$, $102^{2}$, $136^{2}$, $459^{2}$, $1156^{2}$, $1224^{2}$, $12648^{2}$, $16116^{2}$, $33643^{2}$
\item $(h,k)=(6,9)$ : $1^{2}$, $1360^{2}$, $8211^{2}$, $16048^{2}$, $39916^{2}$
\item $(h,k)=(6,10)$ : $1^{2}$, $17^{2}$, $663^{2}$, $2482^{2}$, $62373^{2}$
\item $(h,k)=(6,11)$ : $1^{2}$, $3^{43690}$
\item $(h,k)=(6,12)$ : $1^{2}$, $85^{2}$, $340^{2}$, $2193^{2}$, $62917^{2}$
\item $(h,k)=(6,13)$ : $1^{2}$, $340^{2}$, $765^{2}$, $1377^{2}$, $2873^{2}$, $13362^{2}$, $46818^{2}$
\item $(h,k)=(6,14)$ : $1^{2}$, $34^{2}$, $119^{2}$, $255^{2}$, $476^{2}$, $493^{2}$, $6273^{2}$, $57885^{2}$
\item $(h,k)=(6,15)$ : $1^{2}$, $34^{2}$, $527^{2}$, $10149^{2}$, $14450^{2}$, $40375^{2}$
\item $(h,k)=(6,16)$ : $1^{2}$, $34^{2}$, $51^{2}$, $238^{2}$, $1904^{2}$, $3724^{34}$
\item $(h,k)=(7,8)$ : $1^{2}$, $102^{2}$, $65433^{2}$
\item $(h,k)=(7,9)$ : $1^{2}$, $17^{34}$, $51^{2}$, $306^{2}$, $1156^{2}$, $3468^{2}$, $4726^{2}$, $55539^{2}$
\item $(h,k)=(7,10)$ : $1^{2}$, $3^{43690}$
\item $(h,k)=(7,11)$ : $1^{2}$, $17^{2}$, $663^{2}$, $2482^{2}$, $62373^{2}$
\item $(h,k)=(7,12)$ : $1^{2}$, $51^{2}$, $119^{2}$, $153^{2}$, $238^{2}$, $323^{2}$, $986^{2}$, $1275^{2}$, $1649^{2}$, $3009^{2}$, $8925^{2}$, $20978^{2}$, $27829^{2}$
\item $(h,k)=(7,13)$ : $1^{2}$, $17^{2}$, $187^{2}$, $2210^{2}$, $13498^{2}$, $49623^{2}$
\item $(h,k)=(7,14)$ : $1^{2}$, $34^{2}$, $51^{4}$, $68^{2}$, $136^{2}$, $187^{2}$, $986^{2}$, $7548^{2}$, $8041^{2}$, $10438^{2}$, $12087^{2}$, $25908^{2}$
\item $(h,k)=(7,15)$ : $1^{2}$, $17^{2}$, $969^{2}$, $1224^{2}$, $2159^{2}$, $2805^{2}$, $14552^{2}$, $19873^{2}$, $23936^{2}$
\item $(h,k)=(7,16)$ : $1^{2}$, $34^{4}$, $204^{2}$, $32198^{2}$, $33065^{2}$
\item $(h,k)=(8,9)$ : $1^{2}$, $3^{43690}$
\item $(h,k)=(8,10)$ : $1^{2}$, $17^{34}$, $51^{2}$, $306^{2}$, $1156^{2}$, $3468^{2}$, $4726^{2}$, $55539^{2}$
\item $(h,k)=(8,11)$ : $1^{2}$, $1360^{2}$, $8211^{2}$, $16048^{2}$, $39916^{2}$
\item $(h,k)=(8,12)$ : $1^{2}$, $85^{2}$, $1530^{2}$, $4012^{2}$, $7038^{2}$, $52870^{2}$
\item $(h,k)=(8,13)$ : $1^{2}$, $17^{2}$, $51^{2}$, $570^{34}$, $55777^{2}$
\item $(h,k)=(8,14)$ : $1^{2}$, $17^{2}$, $1462^{2}$, $64056^{2}$
\item $(h,k)=(8,15)$ : $1^{2}$, $119^{2}$, $884^{2}$, $16762^{2}$, $47770^{2}$
\item $(h,k)=(8,16)$ : $1^{2}$, $65535^{2}$
\item $(h,k)=(9,10)$ : $1^{2}$, $102^{2}$, $65433^{2}$
\item $(h,k)=(9,11)$ : $1^{2}$, $17^{2}$, $34^{2}$, $102^{2}$, $136^{2}$, $459^{2}$, $1156^{2}$, $1224^{2}$, $12648^{2}$, $16116^{2}$, $33643^{2}$
\item $(h,k)=(9,12)$ : $1^{2}$, $51^{2}$, $612^{2}$, $3196^{2}$, $12971^{2}$, $48705^{2}$
\item $(h,k)=(9,13)$ : $1^{2}$, $17^{2}$, $80^{34}$, $102^{2}$, $221^{2}$, $692^{34}$, $731^{2}$, $816^{2}$, $1511^{34}$, $2771^{2}$, $8636^{2}$, $13430^{2}$
\item $(h,k)=(9,14)$ : $1^{2}$, $323^{2}$, $65212^{2}$
\item $(h,k)=(9,15)$ : $1^{2}$, $1139^{2}$, $4505^{2}$, $59891^{2}$
\item $(h,k)=(9,16)$ : $1^{2}$, $459^{2}$, $65076^{2}$
\item $(h,k)=(10,11)$ : $1^{2}$, $34^{2}$, $680^{2}$, $1079^{34}$, $46478^{2}$
\item $(h,k)=(10,12)$ : $1^{2}$, $17^{2}$, $357^{2}$, $1258^{2}$, $63903^{2}$
\item $(h,k)=(10,13)$ : $1^{2}$, $85^{2}$, $306^{2}$, $561^{2}$, $1054^{2}$, $2686^{2}$, $13345^{2}$, $47498^{2}$
\item $(h,k)=(10,14)$ : $1^{2}$, $1309^{2}$, $3536^{2}$, $12665^{2}$, $17578^{2}$, $30447^{2}$
\item $(h,k)=(10,15)$ : $1^{2}$, $34^{4}$, $119^{2}$, $11730^{2}$, $20672^{2}$, $32946^{2}$
\item $(h,k)=(10,16)$ : $1^{2}$, $35^{34}$, $238^{2}$, $340^{2}$, $9860^{2}$, $11271^{2}$, $43231^{2}$
\item $(h,k)=(11,12)$ : $1^{2}$, $17^{2}$, $51^{2}$, $629^{2}$, $64838^{2}$
\item $(h,k)=(11,13)$ : $1^{2}$, $136^{2}$, $5270^{2}$, $5287^{2}$, $10098^{2}$, $21352^{2}$, $23392^{2}$
\item $(h,k)=(11,14)$ : $1^{2}$, $17^{2}$, $51^{2}$, $119^{2}$, $272^{2}$, $425^{2}$, $816^{2}$, $5899^{2}$, $14093^{2}$, $43843^{2}$
\item $(h,k)=(11,15)$ : $1^{2}$, $17^{2}$, $34^{2}$, $51^{2}$, $102^{2}$, $1258^{2}$, $1632^{2}$, $16626^{2}$, $45815^{2}$
\item $(h,k)=(11,16)$ : $1^{2}$, $17^{2}$, $21^{34}$, $51^{2}$, $963^{34}$, $1615^{2}$, $3043^{2}$, $14008^{2}$, $30073^{2}$
\item $(h,k)=(12,13)$ : $1^{2}$, $17^{2}$, $34^{2}$, $1853^{2}$, $3587^{2}$, $60044^{2}$
\item $(h,k)=(12,14)$ : $1^{2}$, $12886^{2}$, $52649^{2}$
\item $(h,k)=(12,15)$ : $1^{2}$, $510^{2}$, $663^{2}$, $2329^{2}$, $2788^{2}$, $59245^{2}$
\item $(h,k)=(12,16)$ : $1^{2}$, $17^{2}$, $34^{2}$, $153^{2}$, $221^{2}$, $1326^{2}$, $3672^{2}$, $15062^{2}$, $45050^{2}$
\item $(h,k)=(13,14)$ : $1^{2}$, $17^{2}$, $42^{34}$, $748^{2}$, $64056^{2}$
\item $(h,k)=(13,15)$ : $1^{2}$, $17^{38}$, $34^{2}$, $153^{2}$, $5576^{2}$, $8415^{2}$, $17680^{2}$, $33354^{2}$
\item $(h,k)=(13,16)$ : $1^{2}$, $136^{2}$, $3876^{2}$, $7548^{2}$, $53975^{2}$
\item $(h,k)=(14,15)$ : $1^{2}$, $34^{2}$, $136^{2}$, $476^{2}$, $1462^{2}$, $2941^{2}$, $5610^{2}$, $18292^{2}$, $36584^{2}$
\item $(h,k)=(14,16)$ : $1^{2}$, $68^{2}$, $102^{2}$, $510^{2}$, $64855^{2}$
\item $(h,k)=(15,16)$ : $1^{2}$, $17^{2}$, $18^{136}$, $22^{68}$, $25^{34}$, $27^{34}$, $32^{34}$, $34^{36}$, $51^{6}$, $52^{34}$, $68^{10}$, $85^{20}$, $102^{34}$, $119^{12}$, $136^{46}$, $153^{2}$, $170^{12}$, $204^{40}$, $221^{10}$, $238^{18}$, $255^{8}$, $272^{6}$, $289^{2}$, $306^{16}$, $323^{18}$, $340^{14}$, $374^{12}$, $391^{12}$, $408^{8}$, $425^{8}$, $442^{4}$, $459^{2}$, $476^{2}$, $510^{4}$, $527^{6}$, $544^{8}$, $578^{10}$, $595^{4}$, $612^{4}$, $646^{6}$, $663^{2}$, $680^{2}$, $697^{8}$, $731^{2}$, $748^{2}$, $782^{2}$, $799^{4}$, $884^{6}$, $952^{2}$, $986^{2}$, $1003^{2}$
\end{itemize}

\input{R18.tex}
\input{R19.tex}
\input{R20.tex}
\section{Results for $n=21$}
\begin{itemize}
\item $(h,k)=(1,2)$ : $1^{2}$, $3^{2}$, $7^{8}$, $9^{28}$, $11^{84}$, $14^{2}$, $15^{42}$, $21^{6}$, $22^{252}$, $24^{98}$, $28^{24}$, $30^{532}$, $32^{84}$, $34^{42}$, $36^{112}$, $42^{16}$, $44^{42}$, $45^{126}$, $46^{84}$, $48^{84}$, $49^{60}$, $56^{228}$, $57^{42}$, $60^{140}$, $63^{58}$, $69^{168}$, $70^{120}$, $72^{112}$, $75^{28}$, $77^{42}$, $78^{140}$, $84^{168}$, $91^{60}$, $92^{42}$, $96^{42}$, $98^{192}$, $102^{42}$, $105^{20}$, $106^{42}$, $108^{28}$, $110^{84}$, $112^{84}$, $120^{70}$, $126^{222}$, $129^{42}$, $133^{90}$, $138^{28}$, $142^{42}$, $147^{80}$, $154^{90}$, $161^{42}$, $162^{28}$, $168^{132}$, $175^{60}$, $177^{28}$, $182^{84}$, $189^{30}$, $192^{42}$, $195^{28}$, $196^{6}$, $198^{42}$, $204^{42}$, $210^{142}$, $217^{12}$, $224^{42}$, $231^{122}$, $238^{12}$, $240^{14}$, $245^{36}$, $252^{96}$, $259^{18}$, $273^{20}$, $280^{12}$, $287^{12}$, $291^{14}$, $294^{98}$, $308^{12}$, $315^{124}$, $318^{14}$, $330^{14}$, $336^{50}$, $357^{46}$, $360^{14}$, $364^{6}$, $366^{14}$, $372^{28}$, $378^{60}$, $385^{12}$, $399^{58}$, $406^{12}$, $413^{12}$, $420^{54}$, $434^{6}$, $448^{12}$, $462^{14}$, $476^{12}$, $483^{24}$, $497^{6}$, $504^{48}$, $511^{18}$, $525^{20}$, $546^{26}$, $552^{14}$, $560^{12}$, $567^{12}$, $574^{12}$, $588^{32}$, $602^{30}$, $609^{14}$, $623^{6}$, $630^{14}$, $644^{6}$, $651^{24}$, $658^{6}$, $672^{60}$, $679^{12}$, $686^{18}$, $693^{12}$, $707^{12}$, $714^{14}$, $728^{12}$, $735^{12}$, $742^{18}$, $749^{6}$, $756^{12}$, $777^{8}$, $798^{24}$, $805^{6}$, $812^{6}$, $819^{6}$, $833^{6}$, $840^{22}$, $861^{14}$, $868^{6}$, $882^{34}$, $903^{30}$, $924^{14}$, $931^{6}$, $945^{6}$, $966^{12}$, $987^{10}$, $1008^{24}$, $1015^{6}$, $1029^{4}$, $1050^{20}$, $1071^{12}$, $1092^{16}$, $1113^{6}$, $1127^{6}$, $1134^{4}$, $1141^{6}$, $1155^{4}$, $1162^{6}$, $1176^{20}$, $1204^{6}$, $1218^{16}$, $1260^{10}$, $1281^{8}$, $1302^{14}$, $1323^{2}$, $1344^{8}$, $1358^{6}$, $1365^{8}$, $1386^{10}$, $1428^{8}$, $1449^{6}$, $1470^{14}$, $1484^{6}$, $1491^{4}$, $1512^{18}$, $1554^{2}$, $1575^{6}$, $1596^{6}$, $1617^{8}$, $1638^{8}$, $1659^{2}$, $1680^{8}$, $1722^{6}$, $1743^{2}$, $1764^{10}$, $1848^{2}$, $1869^{4}$, $1890^{2}$, $1932^{2}$, $1953^{4}$, $1974^{2}$, $1995^{2}$, $2037^{2}$, $2058^{4}$, $2079^{2}$, $2121^{4}$, $2142^{2}$, $2163^{4}$, $2184^{8}$, $2205^{2}$, $2226^{2}$, $2247^{2}$, $2310^{2}$, $2331^{2}$, $2352^{8}$, $2394^{2}$, $2415^{2}$, $2436^{2}$, $2478^{2}$, $2499^{2}$, $2520^{2}$, $2604^{10}$, $2688^{2}$, $2730^{2}$, $2772^{2}$, $2793^{6}$, $2814^{2}$, $2835^{4}$, $2877^{2}$, $2898^{2}$, $2919^{2}$, $2982^{2}$, $3045^{2}$, $3066^{2}$, $3192^{4}$, $3234^{8}$, $3339^{2}$, $3402^{2}$, $3465^{2}$, $3507^{2}$, $3570^{2}$, $3612^{4}$, $3738^{2}$, $4326^{2}$, $4704^{2}$, $4998^{2}$
\item $(h,k)=(1,4)$ : $1^{8}$, $7^{6}$, $57^{42}$, $63^{2}$, $574^{6}$, $693^{6}$, $1477^{6}$, $1603^{6}$, $1722^{2}$, $1827^{2}$, $5061^{2}$, $11130^{6}$, $11259^{14}$, $18207^{2}$, $21658^{6}$, $34076^{6}$, $40026^{2}$, $43890^{6}$, $49483^{6}$, $135961^{6}$
\item $(h,k)=(1,5)$ : $1^{2}$, $3^{2}$, $21^{16}$, $28^{6}$, $42^{2}$, $63^{2}$, $567^{2}$, $1587^{14}$, $1785^{2}$, $1852^{42}$, $36981^{6}$, $88470^{14}$, $265629^{2}$
\item $(h,k)=(1,8)$ : $1^{128}$, $3^{170}$, $4^{42}$, $6^{70}$, $8^{42}$, $9^{14}$, $10^{84}$, $11^{42}$, $12^{28}$, $15^{28}$, $18^{56}$, $21^{2}$, $24^{14}$, $27^{28}$, $30^{14}$, $36^{14}$, $38^{42}$, $39^{28}$, $42^{14}$, $46^{42}$, $63^{14}$, $70^{6}$, $78^{14}$, $81^{14}$, $84^{14}$, $88^{42}$, $106^{42}$, $111^{14}$, $123^{42}$, $130^{42}$, $138^{14}$, $149^{42}$, $150^{14}$, $162^{42}$, $174^{14}$, $254^{42}$, $288^{14}$, $300^{14}$, $329^{6}$, $437^{42}$, $447^{14}$, $531^{14}$, $540^{42}$, $543^{14}$, $612^{14}$, $645^{14}$, $651^{14}$, $684^{14}$, $719^{42}$, $747^{14}$, $1059^{14}$, $1107^{14}$, $1116^{14}$, $1120^{6}$, $1232^{6}$, $1242^{14}$, $1277^{42}$, $1338^{14}$, $1686^{14}$, $2433^{14}$, $2542^{42}$, $2580^{14}$, $2669^{42}$, $2711^{42}$, $2742^{42}$, $2763^{14}$, $2804^{42}$, $2889^{14}$, $3885^{14}$, $4515^{14}$, $5091^{14}$, $5226^{14}$, $5571^{14}$, $7035^{14}$, $7239^{14}$, $7668^{14}$, $8148^{14}$, $8223^{14}$, $8259^{14}$
\item $(h,k)=(1,10)$ : $1^{8}$, $7^{6}$, $14^{6}$, $35^{6}$, $63^{2}$, $343^{6}$, $3038^{6}$, $3528^{6}$, $5691^{6}$, $12660^{14}$, $15862^{6}$, $44394^{2}$, $130172^{6}$, $146475^{6}$
\item $(h,k)=(1,11)$ : $1^{2}$, $3^{2}$, $21^{2}$, $28^{12}$, $63^{2}$, $126^{48}$, $168^{96}$, $504^{4116}$
\item $(h,k)=(1,13)$ : $1^{8}$, $3^{42}$, $14^{6}$, $21^{8}$, $84^{2}$, $455^{6}$, $812^{6}$, $5117^{6}$, $6489^{6}$, $45976^{6}$, $95781^{6}$, $144823^{6}$, $149940^{2}$
\item $(h,k)=(1,16)$ : $1^{8}$, $7^{2}$, $14^{2}$, $21^{10}$, $30^{14}$, $42^{2}$, $84^{6}$, $119^{6}$, $273^{6}$, $841^{42}$, $1120^{6}$, $1820^{6}$, $3227^{6}$, $3560^{42}$, $5822^{42}$, $6909^{2}$, $8953^{6}$, $10220^{6}$, $14203^{6}$, $14651^{6}$, $45647^{6}$, $99960^{2}$, $141897^{6}$
\item $(h,k)=(1,17)$ : $1^{2}$, $3^{2}$, $7^{6}$, $14^{6}$, $63^{2}$, $499611^{2}$, $548835^{2}$
\item $(h,k)=(1,19)$ : $1^{8}$, $7^{18}$, $8^{42}$, $9^{84}$, $12^{168}$, $13^{756}$, $14^{6}$, $16^{84}$, $18^{126}$, $19^{42}$, $20^{336}$, $21^{26}$, $22^{294}$, $24^{210}$, $26^{378}$, $28^{96}$, $29^{42}$, $30^{84}$, $31^{42}$, $35^{78}$, $38^{168}$, $42^{90}$, $44^{84}$, $46^{126}$, $47^{84}$, $48^{84}$, $49^{210}$, $50^{84}$, $55^{84}$, $56^{336}$, $58^{42}$, $61^{42}$, $63^{296}$, $70^{228}$, $71^{84}$, $73^{84}$, $74^{42}$, $75^{42}$, $76^{42}$, $77^{150}$, $79^{42}$, $81^{42}$, $84^{762}$, $91^{234}$, $92^{42}$, $93^{42}$, $97^{42}$, $98^{294}$, $103^{42}$, $105^{108}$, $112^{216}$, $114^{42}$, $115^{42}$, $116^{42}$, $117^{126}$, $119^{102}$, $120^{42}$, $126^{288}$, $128^{42}$, $133^{126}$, $134^{42}$, $137^{42}$, $140^{48}$, $141^{42}$, $147^{30}$, $154^{150}$, $161^{66}$, $168^{90}$, $171^{42}$, $175^{60}$, $182^{120}$, $184^{42}$, $189^{6}$, $196^{84}$, $203^{78}$, $210^{84}$, $217^{84}$, $224^{72}$, $231^{42}$, $238^{54}$, $245^{90}$, $252^{96}$, $259^{54}$, $273^{66}$, $280^{54}$, $287^{24}$, $294^{60}$, $301^{24}$, $308^{72}$, $315^{18}$, $322^{90}$, $329^{12}$, $336^{48}$, $343^{18}$, $350^{78}$, $357^{36}$, $364^{18}$, $371^{12}$, $385^{48}$, $392^{6}$, $399^{30}$, $406^{30}$, $413^{42}$, $420^{66}$, $427^{78}$, $434^{24}$, $441^{24}$, $448^{36}$, $455^{6}$, $462^{42}$, $476^{24}$, $483^{12}$, $490^{18}$, $497^{18}$, $504^{42}$, $511^{12}$, $518^{12}$, $525^{36}$, $532^{12}$, $539^{30}$, $553^{6}$, $560^{18}$, $567^{12}$, $574^{6}$, $581^{6}$, $588^{6}$, $595^{6}$, $602^{18}$, $609^{18}$, $616^{12}$, $623^{12}$, $630^{60}$, $637^{6}$, $644^{30}$, $651^{6}$, $658^{24}$, $672^{12}$, $686^{36}$, $693^{18}$, $700^{6}$, $735^{12}$, $742^{18}$, $756^{24}$, $763^{24}$, $770^{6}$, $777^{12}$, $791^{6}$, $798^{18}$, $805^{6}$, $812^{6}$, $819^{24}$, $826^{12}$, $840^{6}$, $854^{18}$, $875^{6}$, $882^{24}$, $889^{6}$, $910^{18}$, $917^{18}$, $924^{6}$, $931^{6}$, $945^{6}$, $952^{12}$, $966^{24}$, $980^{6}$, $987^{6}$, $1001^{12}$, $1008^{24}$, $1015^{6}$, $1036^{6}$, $1050^{6}$, $1057^{6}$, $1071^{12}$, $1085^{12}$, $1099^{6}$, $1106^{6}$, $1113^{6}$, $1127^{6}$, $1141^{6}$, $1148^{12}$, $1190^{6}$, $1211^{6}$, $1218^{12}$, $1232^{12}$, $1239^{6}$, $1267^{6}$, $1288^{6}$, $1295^{6}$, $1302^{6}$, $1323^{6}$, $1337^{6}$, $1400^{6}$, $1421^{6}$, $1435^{6}$, $1484^{6}$, $1491^{6}$, $1505^{18}$, $1540^{6}$, $1568^{12}$, $1596^{6}$, $1617^{6}$, $1673^{6}$
\item $(h,k)=(1,20)$ : $1^{2}$, $3^{699050}$
\item $(h,k)=(2,4)$ : $1^{2}$, $3^{2}$, $14^{2}$, $21^{2}$, $49^{2}$, $168^{2}$, $2688^{2}$, $15330^{2}$, $198135^{2}$, $277389^{6}$
\item $(h,k)=(2,5)$ : $1^{8}$, $3^{42}$, $7^{6}$, $21^{2}$, $42^{6}$, $245^{6}$, $1099^{6}$, $1344^{6}$, $1806^{6}$, $1869^{2}$, $3387^{42}$, $12460^{6}$, $33747^{2}$, $35707^{6}$, $36253^{6}$, $75285^{6}$, $149667^{6}$
\item $(h,k)=(2,8)$ : $1^{8}$, $2^{84}$, $7^{8}$, $10^{42}$, $14^{8}$, $21^{4}$, $259^{6}$, $336^{6}$, $420^{6}$, $1645^{6}$, $1680^{6}$, $2100^{6}$, $3955^{6}$, $12348^{2}$, $62062^{6}$, $87556^{6}$, $132804^{2}$, $140987^{6}$
\item $(h,k)=(2,10)$ : $1^{2}$, $3^{2}$, $63^{2}$, $462^{2}$, $32256^{2}$, $53340^{6}$, $855771^{2}$
\item $(h,k)=(2,11)$ : $1^{8}$, $7^{6}$, $13^{42}$, $14^{2}$, $35^{6}$, $49^{2}$, $63^{6}$, $91^{6}$, $347^{42}$, $588^{6}$, $1323^{6}$, $3752^{6}$, $3768^{42}$, $25977^{6}$, $31920^{2}$, $32949^{2}$, $35826^{6}$, $112203^{6}$, $119119^{6}$
\item $(h,k)=(2,13)$ : $1^{2}$, $3^{58}$, $63^{2}$, $189^{2}$, $301^{6}$, $336^{6}$, $756^{2}$, $8449^{6}$, $21588^{6}$, $74613^{2}$, $293615^{6}$
\item $(h,k)=(2,16)$ : $1^{128}$, $3^{170}$, $6^{42}$, $9^{56}$, $10^{42}$, $12^{14}$, $13^{42}$, $15^{42}$, $16^{42}$, $18^{14}$, $21^{14}$, $22^{42}$, $24^{14}$, $32^{42}$, $33^{14}$, $39^{14}$, $42^{14}$, $43^{42}$, $45^{14}$, $48^{14}$, $51^{14}$, $54^{14}$, $68^{42}$, $99^{14}$, $140^{42}$, $141^{14}$, $153^{14}$, $159^{14}$, $201^{14}$, $255^{14}$, $309^{14}$, $320^{42}$, $360^{14}$, $462^{2}$, $552^{42}$, $733^{42}$, $898^{42}$, $1151^{42}$, $1176^{2}$, $1885^{42}$, $2029^{42}$, $2139^{14}$, $2154^{14}$, $2217^{14}$, $2517^{14}$, $2664^{42}$, $2683^{42}$, $2767^{42}$, $3045^{14}$, $3090^{14}$, $3273^{14}$, $4527^{14}$, $4542^{14}$, $5025^{14}$, $5187^{14}$, $5742^{14}$, $6549^{14}$, $6636^{2}$, $7899^{14}$, $7905^{14}$, $7950^{14}$, $8097^{14}$, $8193^{14}$, $8277^{14}$
\item $(h,k)=(2,17)$ : $1^{8}$, $7^{12}$, $21^{6}$, $63^{14}$, $168^{2}$, $217^{6}$, $308^{6}$, $665^{6}$, $882^{2}$, $2049^{14}$, $2114^{6}$, $3297^{6}$, $3738^{2}$, $6755^{6}$, $15526^{6}$, $21826^{6}$, $24752^{6}$, $46529^{6}$, $52507^{6}$, $124614^{2}$, $126931^{6}$
\item $(h,k)=(2,19)$ : $1^{2}$, $3^{699050}$
\item $(h,k)=(2,20)$ : $1^{8}$, $7^{18}$, $8^{42}$, $9^{84}$, $12^{168}$, $13^{756}$, $14^{6}$, $16^{84}$, $18^{126}$, $19^{42}$, $20^{336}$, $21^{26}$, $22^{294}$, $24^{210}$, $26^{378}$, $28^{96}$, $29^{42}$, $30^{84}$, $31^{42}$, $35^{78}$, $38^{168}$, $42^{90}$, $44^{84}$, $46^{126}$, $47^{84}$, $48^{84}$, $49^{210}$, $50^{84}$, $55^{84}$, $56^{336}$, $58^{42}$, $61^{42}$, $63^{296}$, $70^{228}$, $71^{84}$, $73^{84}$, $74^{42}$, $75^{42}$, $76^{42}$, $77^{150}$, $79^{42}$, $81^{42}$, $84^{762}$, $91^{234}$, $92^{42}$, $93^{42}$, $97^{42}$, $98^{294}$, $103^{42}$, $105^{108}$, $112^{216}$, $114^{42}$, $115^{42}$, $116^{42}$, $117^{126}$, $119^{102}$, $120^{42}$, $126^{288}$, $128^{42}$, $133^{126}$, $134^{42}$, $137^{42}$, $140^{48}$, $141^{42}$, $147^{30}$, $154^{150}$, $161^{66}$, $168^{90}$, $171^{42}$, $175^{60}$, $182^{120}$, $184^{42}$, $189^{6}$, $196^{84}$, $203^{78}$, $210^{84}$, $217^{84}$, $224^{72}$, $231^{42}$, $238^{54}$, $245^{90}$, $252^{96}$, $259^{54}$, $273^{66}$, $280^{54}$, $287^{24}$, $294^{60}$, $301^{24}$, $308^{72}$, $315^{18}$, $322^{90}$, $329^{12}$, $336^{48}$, $343^{18}$, $350^{78}$, $357^{36}$, $364^{18}$, $371^{12}$, $385^{48}$, $392^{6}$, $399^{30}$, $406^{30}$, $413^{42}$, $420^{66}$, $427^{78}$, $434^{24}$, $441^{24}$, $448^{36}$, $455^{6}$, $462^{42}$, $476^{24}$, $483^{12}$, $490^{18}$, $497^{18}$, $504^{42}$, $511^{12}$, $518^{12}$, $525^{36}$, $532^{12}$, $539^{30}$, $553^{6}$, $560^{18}$, $567^{12}$, $574^{6}$, $581^{6}$, $588^{6}$, $595^{6}$, $602^{18}$, $609^{18}$, $616^{12}$, $623^{12}$, $630^{60}$, $637^{6}$, $644^{30}$, $651^{6}$, $658^{24}$, $672^{12}$, $686^{36}$, $693^{18}$, $700^{6}$, $735^{12}$, $742^{18}$, $756^{24}$, $763^{24}$, $770^{6}$, $777^{12}$, $791^{6}$, $798^{18}$, $805^{6}$, $812^{6}$, $819^{24}$, $826^{12}$, $840^{6}$, $854^{18}$, $875^{6}$, $882^{24}$, $889^{6}$, $910^{18}$, $917^{18}$, $924^{6}$, $931^{6}$, $945^{6}$, $952^{12}$, $966^{24}$, $980^{6}$, $987^{6}$, $1001^{12}$, $1008^{24}$, $1015^{6}$, $1036^{6}$, $1050^{6}$, $1057^{6}$, $1071^{12}$, $1085^{12}$, $1099^{6}$, $1106^{6}$, $1113^{6}$, $1127^{6}$, $1141^{6}$, $1148^{12}$, $1190^{6}$, $1211^{6}$, $1218^{12}$, $1232^{12}$, $1239^{6}$, $1267^{6}$, $1288^{6}$, $1295^{6}$, $1302^{6}$, $1323^{6}$, $1337^{6}$, $1400^{6}$, $1421^{6}$, $1435^{6}$, $1484^{6}$, $1491^{6}$, $1505^{18}$, $1540^{6}$, $1568^{12}$, $1596^{6}$, $1617^{6}$, $1673^{6}$
\item $(h,k)=(4,5)$ : $1^{2}$, $3^{2}$, $9^{14}$, $21^{14}$, $63^{2}$, $9555^{2}$, $20790^{2}$, $37583^{6}$, $100268^{6}$, $107037^{6}$, $283290^{2}$
\item $(h,k)=(4,8)$ : $1^{2}$, $3^{2}$, $14^{6}$, $28^{6}$, $63^{2}$, $2954^{6}$, $3528^{2}$, $4494^{2}$, $13139^{6}$, $15456^{14}$, $61866^{2}$, $62286^{2}$, $253246^{6}$
\item $(h,k)=(4,10)$ : $1^{8}$, $3^{42}$, $21^{12}$, $63^{2}$, $84^{6}$, $120^{14}$, $248^{42}$, $451^{42}$, $1463^{42}$, $69027^{2}$, $80157^{2}$, $136479^{6}$, $147735^{6}$
\item $(h,k)=(4,11)$ : $1^{128}$, $3^{254}$, $4^{42}$, $5^{42}$, $6^{84}$, $7^{6}$, $8^{42}$, $9^{140}$, $12^{42}$, $15^{42}$, $18^{14}$, $19^{42}$, $21^{56}$, $27^{42}$, $33^{28}$, $36^{14}$, $51^{14}$, $57^{14}$, $60^{84}$, $84^{14}$, $87^{14}$, $111^{14}$, $112^{42}$, $126^{14}$, $127^{42}$, $144^{14}$, $167^{42}$, $219^{14}$, $297^{14}$, $315^{14}$, $330^{28}$, $387^{14}$, $399^{14}$, $450^{42}$, $582^{14}$, $714^{6}$, $741^{14}$, $873^{14}$, $1134^{42}$, $1302^{42}$, $1434^{14}$, $1484^{42}$, $1702^{42}$, $1980^{14}$, $2196^{14}$, $2274^{42}$, $2520^{42}$, $2643^{42}$, $2775^{14}$, $3288^{14}$, $3453^{14}$, $3684^{14}$, $4005^{14}$, $4128^{14}$, $4218^{14}$, $7293^{14}$, $7467^{14}$, $7476^{14}$, $7758^{14}$, $7842^{14}$, $7878^{14}$, $8202^{14}$, $8214^{14}$, $8295^{14}$
\item $(h,k)=(4,13)$ : $1^{8}$, $14^{6}$, $21^{2}$, $28^{12}$, $38^{42}$, $63^{2}$, $105^{6}$, $154^{6}$, $210^{6}$, $231^{6}$, $243^{14}$, $273^{2}$, $315^{6}$, $322^{6}$, $826^{6}$, $1652^{6}$, $1746^{14}$, $3822^{6}$, $4456^{42}$, $10724^{6}$, $12124^{6}$, $14966^{6}$, $16879^{42}$, $58170^{2}$, $59479^{6}$, $70763^{6}$
\item $(h,k)=(4,16)$ : $1^{8}$, $7^{6}$, $14^{2}$, $21^{8}$, $28^{6}$, $42^{6}$, $49^{2}$, $77^{6}$, $112^{6}$, $147^{2}$, $259^{6}$, $329^{6}$, $1491^{6}$, $1589^{6}$, $1743^{6}$, $2541^{6}$, $2723^{6}$, $4284^{6}$, $4753^{6}$, $30597^{2}$, $119259^{2}$, $139223^{6}$, $140273^{6}$
\item $(h,k)=(4,17)$ : $1^{2}$, $3^{699050}$
\item $(h,k)=(4,19)$ : $1^{8}$, $7^{12}$, $21^{6}$, $63^{14}$, $168^{2}$, $217^{6}$, $308^{6}$, $665^{6}$, $882^{2}$, $2049^{14}$, $2114^{6}$, $3297^{6}$, $3738^{2}$, $6755^{6}$, $15526^{6}$, $21826^{6}$, $24752^{6}$, $46529^{6}$, $52507^{6}$, $124614^{2}$, $126931^{6}$
\item $(h,k)=(4,20)$ : $1^{2}$, $3^{2}$, $7^{6}$, $14^{6}$, $63^{2}$, $499611^{2}$, $548835^{2}$
\item $(h,k)=(5,8)$ : $1^{8}$, $63^{2}$, $147^{6}$, $168^{6}$, $189^{6}$, $210^{2}$, $231^{6}$, $245^{12}$, $259^{6}$, $336^{2}$, $686^{6}$, $2898^{2}$, $11620^{6}$, $31059^{2}$, $111657^{2}$, $137984^{6}$, $149009^{6}$
\item $(h,k)=(5,10)$ : $1^{2}$, $3^{2}$, $14^{2}$, $28^{6}$, $49^{2}$, $378^{2}$, $385^{6}$, $861^{2}$, $1260^{2}$, $2580^{14}$, $6468^{2}$, $11494^{6}$, $60900^{2}$, $100695^{2}$, $105000^{6}$, $509166^{2}$
\item $(h,k)=(5,11)$ : $1^{8}$, $21^{2}$, $28^{6}$, $45^{42}$, $49^{18}$, $63^{2}$, $77^{6}$, $496^{42}$, $1022^{6}$, $2688^{2}$, $5461^{42}$, $12684^{2}$, $19502^{6}$, $22267^{6}$, $22792^{6}$, $47229^{6}$, $60025^{6}$, $66234^{2}$, $107191^{6}$
\item $(h,k)=(5,13)$ : $1^{2}$, $3^{2}$, $7^{2}$, $14^{8}$, $21^{4}$, $35^{12}$, $398^{42}$, $427^{6}$, $567^{2}$, $1260^{2}$, $1522^{42}$, $2961^{2}$, $4431^{6}$, $8589^{2}$, $48510^{2}$, $68892^{14}$, $70658^{6}$, $95088^{2}$, $142170^{2}$
\item $(h,k)=(5,16)$ : $1^{2}$, $3^{699050}$
\item $(h,k)=(5,17)$ : $1^{8}$, $7^{6}$, $14^{2}$, $21^{8}$, $28^{6}$, $42^{6}$, $49^{2}$, $77^{6}$, $112^{6}$, $147^{2}$, $259^{6}$, $329^{6}$, $1491^{6}$, $1589^{6}$, $1743^{6}$, $2541^{6}$, $2723^{6}$, $4284^{6}$, $4753^{6}$, $30597^{2}$, $119259^{2}$, $139223^{6}$, $140273^{6}$
\item $(h,k)=(5,19)$ : $1^{128}$, $3^{170}$, $6^{42}$, $9^{56}$, $10^{42}$, $12^{14}$, $13^{42}$, $15^{42}$, $16^{42}$, $18^{14}$, $21^{14}$, $22^{42}$, $24^{14}$, $32^{42}$, $33^{14}$, $39^{14}$, $42^{14}$, $43^{42}$, $45^{14}$, $48^{14}$, $51^{14}$, $54^{14}$, $68^{42}$, $99^{14}$, $140^{42}$, $141^{14}$, $153^{14}$, $159^{14}$, $201^{14}$, $255^{14}$, $309^{14}$, $320^{42}$, $360^{14}$, $462^{2}$, $552^{42}$, $733^{42}$, $898^{42}$, $1151^{42}$, $1176^{2}$, $1885^{42}$, $2029^{42}$, $2139^{14}$, $2154^{14}$, $2217^{14}$, $2517^{14}$, $2664^{42}$, $2683^{42}$, $2767^{42}$, $3045^{14}$, $3090^{14}$, $3273^{14}$, $4527^{14}$, $4542^{14}$, $5025^{14}$, $5187^{14}$, $5742^{14}$, $6549^{14}$, $6636^{2}$, $7899^{14}$, $7905^{14}$, $7950^{14}$, $8097^{14}$, $8193^{14}$, $8277^{14}$
\item $(h,k)=(5,20)$ : $1^{8}$, $7^{2}$, $14^{2}$, $21^{10}$, $30^{14}$, $42^{2}$, $84^{6}$, $119^{6}$, $273^{6}$, $841^{42}$, $1120^{6}$, $1820^{6}$, $3227^{6}$, $3560^{42}$, $5822^{42}$, $6909^{2}$, $8953^{6}$, $10220^{6}$, $14203^{6}$, $14651^{6}$, $45647^{6}$, $99960^{2}$, $141897^{6}$
\item $(h,k)=(8,10)$ : $1^{2}$, $3^{58}$, $42^{2}$, $63^{2}$, $840^{2}$, $7665^{2}$, $45969^{6}$, $65730^{2}$, $836241^{2}$
\item $(h,k)=(8,11)$ : $1^{8}$, $2^{84}$, $7^{18}$, $21^{6}$, $63^{2}$, $77^{6}$, $85^{42}$, $1285^{42}$, $1414^{6}$, $1876^{6}$, $2094^{14}$, $4550^{6}$, $8673^{6}$, $27804^{2}$, $30576^{6}$, $45157^{6}$, $51506^{6}$, $58562^{6}$, $87423^{6}$, $107625^{2}$
\item $(h,k)=(8,13)$ : $1^{2}$, $3^{699050}$
\item $(h,k)=(8,16)$ : $1^{2}$, $3^{2}$, $7^{2}$, $14^{8}$, $21^{4}$, $35^{12}$, $398^{42}$, $427^{6}$, $567^{2}$, $1260^{2}$, $1522^{42}$, $2961^{2}$, $4431^{6}$, $8589^{2}$, $48510^{2}$, $68892^{14}$, $70658^{6}$, $95088^{2}$, $142170^{2}$
\item $(h,k)=(8,17)$ : $1^{8}$, $14^{6}$, $21^{2}$, $28^{12}$, $38^{42}$, $63^{2}$, $105^{6}$, $154^{6}$, $210^{6}$, $231^{6}$, $243^{14}$, $273^{2}$, $315^{6}$, $322^{6}$, $826^{6}$, $1652^{6}$, $1746^{14}$, $3822^{6}$, $4456^{42}$, $10724^{6}$, $12124^{6}$, $14966^{6}$, $16879^{42}$, $58170^{2}$, $59479^{6}$, $70763^{6}$
\item $(h,k)=(8,19)$ : $1^{2}$, $3^{58}$, $63^{2}$, $189^{2}$, $301^{6}$, $336^{6}$, $756^{2}$, $8449^{6}$, $21588^{6}$, $74613^{2}$, $293615^{6}$
\item $(h,k)=(8,20)$ : $1^{8}$, $3^{42}$, $14^{6}$, $21^{8}$, $84^{2}$, $455^{6}$, $812^{6}$, $5117^{6}$, $6489^{6}$, $45976^{6}$, $95781^{6}$, $144823^{6}$, $149940^{2}$
\item $(h,k)=(10,11)$ : $1^{2}$, $3^{699050}$
\item $(h,k)=(10,13)$ : $1^{8}$, $2^{84}$, $7^{18}$, $21^{6}$, $63^{2}$, $77^{6}$, $85^{42}$, $1285^{42}$, $1414^{6}$, $1876^{6}$, $2094^{14}$, $4550^{6}$, $8673^{6}$, $27804^{2}$, $30576^{6}$, $45157^{6}$, $51506^{6}$, $58562^{6}$, $87423^{6}$, $107625^{2}$
\item $(h,k)=(10,16)$ : $1^{8}$, $21^{2}$, $28^{6}$, $45^{42}$, $49^{18}$, $63^{2}$, $77^{6}$, $496^{42}$, $1022^{6}$, $2688^{2}$, $5461^{42}$, $12684^{2}$, $19502^{6}$, $22267^{6}$, $22792^{6}$, $47229^{6}$, $60025^{6}$, $66234^{2}$, $107191^{6}$
\item $(h,k)=(10,17)$ : $1^{128}$, $3^{254}$, $4^{42}$, $5^{42}$, $6^{84}$, $7^{6}$, $8^{42}$, $9^{140}$, $12^{42}$, $15^{42}$, $18^{14}$, $19^{42}$, $21^{56}$, $27^{42}$, $33^{28}$, $36^{14}$, $51^{14}$, $57^{14}$, $60^{84}$, $84^{14}$, $87^{14}$, $111^{14}$, $112^{42}$, $126^{14}$, $127^{42}$, $144^{14}$, $167^{42}$, $219^{14}$, $297^{14}$, $315^{14}$, $330^{28}$, $387^{14}$, $399^{14}$, $450^{42}$, $582^{14}$, $714^{6}$, $741^{14}$, $873^{14}$, $1134^{42}$, $1302^{42}$, $1434^{14}$, $1484^{42}$, $1702^{42}$, $1980^{14}$, $2196^{14}$, $2274^{42}$, $2520^{42}$, $2643^{42}$, $2775^{14}$, $3288^{14}$, $3453^{14}$, $3684^{14}$, $4005^{14}$, $4128^{14}$, $4218^{14}$, $7293^{14}$, $7467^{14}$, $7476^{14}$, $7758^{14}$, $7842^{14}$, $7878^{14}$, $8202^{14}$, $8214^{14}$, $8295^{14}$
\item $(h,k)=(10,19)$ : $1^{8}$, $7^{6}$, $13^{42}$, $14^{2}$, $35^{6}$, $49^{2}$, $63^{6}$, $91^{6}$, $347^{42}$, $588^{6}$, $1323^{6}$, $3752^{6}$, $3768^{42}$, $25977^{6}$, $31920^{2}$, $32949^{2}$, $35826^{6}$, $112203^{6}$, $119119^{6}$
\item $(h,k)=(10,20)$ : $1^{2}$, $3^{2}$, $21^{2}$, $28^{12}$, $63^{2}$, $126^{48}$, $168^{96}$, $504^{4116}$
\item $(h,k)=(11,13)$ : $1^{2}$, $3^{58}$, $42^{2}$, $63^{2}$, $840^{2}$, $7665^{2}$, $45969^{6}$, $65730^{2}$, $836241^{2}$
\item $(h,k)=(11,16)$ : $1^{2}$, $3^{2}$, $14^{2}$, $28^{6}$, $49^{2}$, $378^{2}$, $385^{6}$, $861^{2}$, $1260^{2}$, $2580^{14}$, $6468^{2}$, $11494^{6}$, $60900^{2}$, $100695^{2}$, $105000^{6}$, $509166^{2}$
\item $(h,k)=(11,17)$ : $1^{8}$, $3^{42}$, $21^{12}$, $63^{2}$, $84^{6}$, $120^{14}$, $248^{42}$, $451^{42}$, $1463^{42}$, $69027^{2}$, $80157^{2}$, $136479^{6}$, $147735^{6}$
\item $(h,k)=(11,19)$ : $1^{2}$, $3^{2}$, $63^{2}$, $462^{2}$, $32256^{2}$, $53340^{6}$, $855771^{2}$
\item $(h,k)=(11,20)$ : $1^{8}$, $7^{6}$, $14^{6}$, $35^{6}$, $63^{2}$, $343^{6}$, $3038^{6}$, $3528^{6}$, $5691^{6}$, $12660^{14}$, $15862^{6}$, $44394^{2}$, $130172^{6}$, $146475^{6}$
\item $(h,k)=(13,16)$ : $1^{8}$, $63^{2}$, $147^{6}$, $168^{6}$, $189^{6}$, $210^{2}$, $231^{6}$, $245^{12}$, $259^{6}$, $336^{2}$, $686^{6}$, $2898^{2}$, $11620^{6}$, $31059^{2}$, $111657^{2}$, $137984^{6}$, $149009^{6}$
\item $(h,k)=(13,17)$ : $1^{2}$, $3^{2}$, $14^{6}$, $28^{6}$, $63^{2}$, $2954^{6}$, $3528^{2}$, $4494^{2}$, $13139^{6}$, $15456^{14}$, $61866^{2}$, $62286^{2}$, $253246^{6}$
\item $(h,k)=(13,19)$ : $1^{8}$, $2^{84}$, $7^{8}$, $10^{42}$, $14^{8}$, $21^{4}$, $259^{6}$, $336^{6}$, $420^{6}$, $1645^{6}$, $1680^{6}$, $2100^{6}$, $3955^{6}$, $12348^{2}$, $62062^{6}$, $87556^{6}$, $132804^{2}$, $140987^{6}$
\item $(h,k)=(13,20)$ : $1^{128}$, $3^{170}$, $4^{42}$, $6^{70}$, $8^{42}$, $9^{14}$, $10^{84}$, $11^{42}$, $12^{28}$, $15^{28}$, $18^{56}$, $21^{2}$, $24^{14}$, $27^{28}$, $30^{14}$, $36^{14}$, $38^{42}$, $39^{28}$, $42^{14}$, $46^{42}$, $63^{14}$, $70^{6}$, $78^{14}$, $81^{14}$, $84^{14}$, $88^{42}$, $106^{42}$, $111^{14}$, $123^{42}$, $130^{42}$, $138^{14}$, $149^{42}$, $150^{14}$, $162^{42}$, $174^{14}$, $254^{42}$, $288^{14}$, $300^{14}$, $329^{6}$, $437^{42}$, $447^{14}$, $531^{14}$, $540^{42}$, $543^{14}$, $612^{14}$, $645^{14}$, $651^{14}$, $684^{14}$, $719^{42}$, $747^{14}$, $1059^{14}$, $1107^{14}$, $1116^{14}$, $1120^{6}$, $1232^{6}$, $1242^{14}$, $1277^{42}$, $1338^{14}$, $1686^{14}$, $2433^{14}$, $2542^{42}$, $2580^{14}$, $2669^{42}$, $2711^{42}$, $2742^{42}$, $2763^{14}$, $2804^{42}$, $2889^{14}$, $3885^{14}$, $4515^{14}$, $5091^{14}$, $5226^{14}$, $5571^{14}$, $7035^{14}$, $7239^{14}$, $7668^{14}$, $8148^{14}$, $8223^{14}$, $8259^{14}$
\item $(h,k)=(16,17)$ : $1^{2}$, $3^{2}$, $9^{14}$, $21^{14}$, $63^{2}$, $9555^{2}$, $20790^{2}$, $37583^{6}$, $100268^{6}$, $107037^{6}$, $283290^{2}$
\item $(h,k)=(16,19)$ : $1^{8}$, $3^{42}$, $7^{6}$, $21^{2}$, $42^{6}$, $245^{6}$, $1099^{6}$, $1344^{6}$, $1806^{6}$, $1869^{2}$, $3387^{42}$, $12460^{6}$, $33747^{2}$, $35707^{6}$, $36253^{6}$, $75285^{6}$, $149667^{6}$
\item $(h,k)=(16,20)$ : $1^{2}$, $3^{2}$, $21^{16}$, $28^{6}$, $42^{2}$, $63^{2}$, $567^{2}$, $1587^{14}$, $1785^{2}$, $1852^{42}$, $36981^{6}$, $88470^{14}$, $265629^{2}$
\item $(h,k)=(17,19)$ : $1^{2}$, $3^{2}$, $14^{2}$, $21^{2}$, $49^{2}$, $168^{2}$, $2688^{2}$, $15330^{2}$, $198135^{2}$, $277389^{6}$
\item $(h,k)=(17,20)$ : $1^{8}$, $7^{6}$, $57^{42}$, $63^{2}$, $574^{6}$, $693^{6}$, $1477^{6}$, $1603^{6}$, $1722^{2}$, $1827^{2}$, $5061^{2}$, $11130^{6}$, $11259^{14}$, $18207^{2}$, $21658^{6}$, $34076^{6}$, $40026^{2}$, $43890^{6}$, $49483^{6}$, $135961^{6}$
\item $(h,k)=(19,20)$ : $1^{2}$, $3^{2}$, $7^{8}$, $9^{28}$, $11^{84}$, $14^{2}$, $15^{42}$, $21^{6}$, $22^{252}$, $24^{98}$, $28^{24}$, $30^{532}$, $32^{84}$, $34^{42}$, $36^{112}$, $42^{16}$, $44^{42}$, $45^{126}$, $46^{84}$, $48^{84}$, $49^{60}$, $56^{228}$, $57^{42}$, $60^{140}$, $63^{58}$, $69^{168}$, $70^{120}$, $72^{112}$, $75^{28}$, $77^{42}$, $78^{140}$, $84^{168}$, $91^{60}$, $92^{42}$, $96^{42}$, $98^{192}$, $102^{42}$, $105^{20}$, $106^{42}$, $108^{28}$, $110^{84}$, $112^{84}$, $120^{70}$, $126^{222}$, $129^{42}$, $133^{90}$, $138^{28}$, $142^{42}$, $147^{80}$, $154^{90}$, $161^{42}$, $162^{28}$, $168^{132}$, $175^{60}$, $177^{28}$, $182^{84}$, $189^{30}$, $192^{42}$, $195^{28}$, $196^{6}$, $198^{42}$, $204^{42}$, $210^{142}$, $217^{12}$, $224^{42}$, $231^{122}$, $238^{12}$, $240^{14}$, $245^{36}$, $252^{96}$, $259^{18}$, $273^{20}$, $280^{12}$, $287^{12}$, $291^{14}$, $294^{98}$, $308^{12}$, $315^{124}$, $318^{14}$, $330^{14}$, $336^{50}$, $357^{46}$, $360^{14}$, $364^{6}$, $366^{14}$, $372^{28}$, $378^{60}$, $385^{12}$, $399^{58}$, $406^{12}$, $413^{12}$, $420^{54}$, $434^{6}$, $448^{12}$, $462^{14}$, $476^{12}$, $483^{24}$, $497^{6}$, $504^{48}$, $511^{18}$, $525^{20}$, $546^{26}$, $552^{14}$, $560^{12}$, $567^{12}$, $574^{12}$, $588^{32}$, $602^{30}$, $609^{14}$, $623^{6}$, $630^{14}$, $644^{6}$, $651^{24}$, $658^{6}$, $672^{60}$, $679^{12}$, $686^{18}$, $693^{12}$, $707^{12}$, $714^{14}$, $728^{12}$, $735^{12}$, $742^{18}$, $749^{6}$, $756^{12}$, $777^{8}$, $798^{24}$, $805^{6}$, $812^{6}$, $819^{6}$, $833^{6}$, $840^{22}$, $861^{14}$, $868^{6}$, $882^{34}$, $903^{30}$, $924^{14}$, $931^{6}$, $945^{6}$, $966^{12}$, $987^{10}$, $1008^{24}$, $1015^{6}$, $1029^{4}$, $1050^{20}$, $1071^{12}$, $1092^{16}$, $1113^{6}$, $1127^{6}$, $1134^{4}$, $1141^{6}$, $1155^{4}$, $1162^{6}$, $1176^{20}$, $1204^{6}$, $1218^{16}$, $1260^{10}$, $1281^{8}$, $1302^{14}$, $1323^{2}$, $1344^{8}$, $1358^{6}$, $1365^{8}$, $1386^{10}$, $1428^{8}$, $1449^{6}$, $1470^{14}$, $1484^{6}$, $1491^{4}$, $1512^{18}$, $1554^{2}$, $1575^{6}$, $1596^{6}$, $1617^{8}$, $1638^{8}$, $1659^{2}$, $1680^{8}$, $1722^{6}$, $1743^{2}$, $1764^{10}$, $1848^{2}$, $1869^{4}$, $1890^{2}$, $1932^{2}$, $1953^{4}$, $1974^{2}$, $1995^{2}$, $2037^{2}$, $2058^{4}$, $2079^{2}$, $2121^{4}$, $2142^{2}$, $2163^{4}$, $2184^{8}$, $2205^{2}$, $2226^{2}$, $2247^{2}$, $2310^{2}$, $2331^{2}$, $2352^{8}$, $2394^{2}$, $2415^{2}$, $2436^{2}$, $2478^{2}$, $2499^{2}$, $2520^{2}$, $2604^{10}$, $2688^{2}$, $2730^{2}$, $2772^{2}$, $2793^{6}$, $2814^{2}$, $2835^{4}$, $2877^{2}$, $2898^{2}$, $2919^{2}$, $2982^{2}$, $3045^{2}$, $3066^{2}$, $3192^{4}$, $3234^{8}$, $3339^{2}$, $3402^{2}$, $3465^{2}$, $3507^{2}$, $3570^{2}$, $3612^{4}$, $3738^{2}$, $4326^{2}$, $4704^{2}$, $4998^{2}$
\end{itemize}

\input{R22.tex}
\input{R23.tex}
\input{R24.tex}
\section{Results for $n=25$}
\begin{itemize}
\item $(h,k)=(1,2)$ : $1^{2}$, $5^{2}$, $10^{2}$, $15^{560}$, $17^{950}$, $19^{350}$, $20^{130}$, $25^{6}$, $26^{100}$, $30^{400}$, $35^{30}$, $37^{50}$, $39^{50}$, $41^{50}$, $44^{800}$, $45^{590}$, $46^{650}$, $50^{866}$, $52^{400}$, $54^{100}$, $57^{300}$, $59^{200}$, $60^{590}$, $63^{200}$, $65^{130}$, $66^{350}$, $70^{320}$, $71^{150}$, $72^{50}$, $74^{500}$, $75^{180}$, $76^{200}$, $78^{100}$, $80^{380}$, $81^{200}$, $82^{200}$, $84^{200}$, $90^{700}$, $91^{200}$, $93^{50}$, $97^{300}$, $98^{200}$, $100^{118}$, $105^{40}$, $106^{400}$, $108^{250}$, $110^{290}$, $112^{200}$, $115^{10}$, $117^{500}$, $118^{50}$, $120^{480}$, $124^{150}$, $125^{224}$, $127^{100}$, $130^{510}$, $140^{440}$, $145^{160}$, $149^{150}$, $150^{196}$, $155^{40}$, $160^{100}$, $164^{50}$, $166^{100}$, $170^{20}$, $173^{100}$, $175^{86}$, $176^{100}$, $179^{100}$, $180^{310}$, $182^{100}$, $184^{150}$, $185^{10}$, $188^{150}$, $190^{120}$, $195^{50}$, $200^{340}$, $205^{50}$, $210^{170}$, $215^{100}$, $220^{320}$, $225^{380}$, $228^{50}$, $230^{80}$, $236^{150}$, $238^{50}$, $240^{130}$, $245^{290}$, $246^{50}$, $250^{270}$, $255^{160}$, $256^{50}$, $260^{40}$, $265^{50}$, $270^{150}$, $273^{50}$, $274^{50}$, $275^{308}$, $280^{30}$, $288^{100}$, $290^{60}$, $295^{10}$, $296^{50}$, $299^{100}$, $300^{360}$, $315^{80}$, $319^{50}$, $320^{150}$, $325^{40}$, $330^{160}$, $335^{50}$, $340^{20}$, $345^{110}$, $350^{808}$, $353^{50}$, $354^{50}$, $358^{50}$, $360^{100}$, $362^{50}$, $365^{20}$, $369^{50}$, $370^{170}$, $375^{58}$, $380^{60}$, $385^{60}$, $387^{50}$, $390^{90}$, $399^{50}$, $400^{260}$, $405^{40}$, $410^{20}$, $415^{80}$, $420^{10}$, $440^{20}$, $445^{40}$, $450^{398}$, $460^{20}$, $465^{40}$, $470^{90}$, $475^{100}$, $480^{20}$, $482^{50}$, $490^{70}$, $500^{326}$, $510^{60}$, $515^{10}$, $520^{30}$, $525^{66}$, $535^{40}$, $540^{20}$, $544^{50}$, $545^{70}$, $550^{220}$, $555^{40}$, $559^{50}$, $560^{110}$, $565^{60}$, $570^{80}$, $575^{214}$, $580^{80}$, $598^{50}$, $600^{264}$, $610^{40}$, $615^{10}$, $625^{32}$, $630^{40}$, $645^{20}$, $650^{182}$, $655^{70}$, $670^{30}$, $675^{132}$, $680^{20}$, $690^{40}$, $700^{100}$, $710^{20}$, $720^{20}$, $725^{150}$, $730^{20}$, $750^{228}$, $760^{20}$, $770^{30}$, $775^{112}$, $790^{40}$, $795^{20}$, $800^{140}$, $805^{10}$, $820^{30}$, $825^{50}$, $830^{60}$, $840^{20}$, $850^{106}$, $860^{20}$, $865^{10}$, $870^{50}$, $875^{66}$, $880^{20}$, $885^{40}$, $890^{50}$, $900^{146}$, $920^{50}$, $925^{42}$, $935^{10}$, $945^{30}$, $950^{206}$, $955^{30}$, $960^{40}$, $965^{20}$, $975^{30}$, $980^{10}$, $985^{20}$, $1000^{136}$, $1010^{20}$, $1020^{50}$, $1025^{62}$, $1035^{20}$, $1045^{20}$, $1050^{166}$, $1075^{4}$, $1080^{10}$, $1085^{10}$, $1100^{110}$, $1125^{110}$, $1140^{40}$, $1150^{168}$, $1165^{10}$, $1170^{10}$, $1175^{40}$, $1180^{20}$, $1185^{20}$, $1200^{50}$, $1210^{10}$, $1215^{10}$, $1220^{20}$, $1225^{64}$, $1250^{116}$, $1260^{10}$, $1270^{20}$, $1275^{46}$, $1290^{10}$, $1295^{30}$, $1300^{120}$, $1305^{30}$, $1310^{20}$, $1325^{86}$, $1340^{30}$, $1350^{128}$, $1370^{10}$, $1375^{42}$, $1380^{10}$, $1400^{152}$, $1425^{16}$, $1430^{10}$, $1445^{10}$, $1450^{108}$, $1460^{20}$, $1465^{10}$, $1470^{20}$, $1475^{18}$, $1480^{10}$, $1500^{58}$, $1525^{10}$, $1545^{10}$, $1550^{116}$, $1555^{10}$, $1560^{10}$, $1570^{10}$, $1575^{36}$, $1580^{30}$, $1585^{10}$, $1600^{60}$, $1605^{10}$, $1615^{20}$, $1625^{68}$, $1650^{44}$, $1660^{10}$, $1670^{30}$, $1675^{12}$, $1700^{98}$, $1725^{32}$, $1750^{52}$, $1760^{10}$, $1765^{10}$, $1775^{26}$, $1790^{20}$, $1800^{60}$, $1810^{10}$, $1815^{10}$, $1825^{32}$, $1830^{10}$, $1835^{10}$, $1850^{94}$, $1875^{24}$, $1880^{20}$, $1895^{10}$, $1900^{64}$, $1905^{10}$, $1910^{10}$, $1915^{10}$, $1925^{6}$, $1945^{10}$, $1950^{90}$, $1955^{10}$, $1975^{12}$, $2000^{46}$, $2025^{26}$, $2030^{10}$, $2040^{10}$, $2050^{130}$, $2075^{20}$, $2100^{94}$, $2125^{14}$, $2140^{10}$, $2150^{52}$, $2175^{36}$, $2200^{80}$, $2225^{18}$, $2230^{10}$, $2250^{26}$, $2275^{18}$, $2300^{32}$, $2305^{10}$, $2310^{10}$, $2325^{24}$, $2340^{10}$, $2350^{56}$, $2375^{20}$, $2395^{10}$, $2400^{62}$, $2410^{10}$, $2420^{10}$, $2450^{58}$, $2475^{38}$, $2500^{26}$, $2525^{26}$, $2550^{20}$, $2575^{32}$, $2585^{10}$, $2600^{54}$, $2625^{10}$, $2650^{30}$, $2675^{10}$, $2680^{10}$, $2690^{10}$, $2700^{30}$, $2750^{22}$, $2775^{8}$, $2800^{48}$, $2825^{16}$, $2850^{56}$, $2870^{10}$, $2875^{6}$, $2900^{20}$, $2925^{28}$, $2950^{56}$, $2965^{10}$, $2975^{12}$, $3000^{60}$, $3025^{14}$, $3050^{14}$, $3075^{26}$, $3100^{46}$, $3125^{8}$, $3150^{46}$, $3175^{10}$, $3200^{20}$, $3225^{14}$, $3230^{10}$, $3250^{32}$, $3275^{6}$, $3300^{20}$, $3325^{10}$, $3345^{10}$, $3350^{38}$, $3375^{18}$, $3400^{22}$, $3425^{16}$, $3450^{30}$, $3475^{4}$, $3500^{24}$, $3525^{10}$, $3550^{28}$, $3575^{14}$, $3600^{16}$, $3625^{6}$, $3650^{30}$, $3675^{6}$, $3700^{32}$, $3725^{28}$, $3750^{22}$, $3775^{4}$, $3800^{18}$, $3825^{6}$, $3850^{30}$, $3875^{4}$, $3900^{30}$, $3950^{22}$, $3975^{10}$, $4000^{20}$, $4025^{12}$, $4050^{44}$, $4075^{6}$, $4100^{18}$, $4125^{16}$, $4150^{30}$, $4175^{8}$, $4200^{28}$, $4250^{20}$, $4275^{8}$, $4300^{16}$, $4325^{6}$, $4350^{16}$, $4375^{4}$, $4400^{16}$, $4425^{4}$, $4450^{22}$, $4475^{6}$, $4500^{22}$, $4525^{6}$, $4550^{20}$, $4575^{4}$, $4600^{12}$, $4625^{4}$, $4650^{22}$, $4675^{2}$, $4700^{4}$, $4750^{22}$, $4800^{22}$, $4825^{4}$, $4850^{22}$, $4875^{8}$, $4900^{12}$, $4925^{8}$, $4950^{16}$, $4975^{10}$, $5025^{8}$, $5050^{14}$, $5075^{4}$, $5100^{12}$, $5125^{8}$, $5150^{12}$, $5175^{8}$, $5200^{10}$, $5225^{10}$, $5250^{26}$, $5275^{2}$, $5300^{22}$, $5325^{18}$, $5350^{12}$, $5375^{6}$, $5400^{8}$, $5450^{8}$, $5475^{8}$, $5500^{14}$, $5550^{20}$, $5575^{4}$, $5600^{10}$, $5625^{12}$, $5650^{16}$, $5675^{12}$, $5700^{32}$, $5725^{10}$, $5750^{16}$, $5775^{14}$, $5800^{8}$, $5850^{12}$, $5875^{4}$, $5900^{12}$, $5925^{6}$, $5950^{24}$, $5975^{6}$, $6000^{14}$, $6025^{6}$, $6050^{4}$, $6075^{6}$, $6100^{10}$, $6125^{10}$, $6150^{10}$, $6175^{4}$, $6200^{14}$, $6225^{8}$, $6250^{4}$, $6275^{6}$, $6300^{8}$, $6325^{8}$, $6350^{6}$, $6375^{8}$, $6400^{18}$, $6450^{8}$, $6500^{8}$, $6525^{4}$, $6550^{20}$, $6625^{6}$, $6650^{6}$, $6700^{12}$, $6725^{8}$, $6750^{4}$, $6800^{4}$, $6825^{10}$, $6850^{14}$, $6875^{4}$, $6900^{2}$, $6925^{4}$, $6950^{18}$, $6975^{8}$, $7000^{22}$, $7025^{6}$, $7050^{2}$, $7075^{4}$, $7100^{6}$, $7125^{2}$, $7150^{2}$, $7200^{8}$, $7225^{4}$, $7250^{6}$, $7275^{4}$, $7300^{6}$, $7325^{4}$, $7350^{6}$, $7375^{6}$, $7400^{4}$, $7425^{4}$, $7450^{2}$, $7475^{2}$, $7500^{4}$, $7525^{2}$, $7550^{10}$, $7575^{6}$, $7600^{8}$, $7625^{4}$, $7650^{4}$, $7675^{6}$, $7700^{8}$, $7725^{4}$, $7750^{8}$, $7775^{8}$, $7800^{2}$, $7825^{4}$, $7850^{2}$, $7875^{4}$, $7900^{2}$, $7925^{2}$, $7950^{6}$, $8000^{4}$, $8025^{2}$, $8050^{12}$, $8075^{2}$, $8100^{2}$, $8125^{4}$, $8150^{4}$, $8175^{10}$, $8200^{10}$, $8225^{4}$, $8250^{4}$, $8275^{4}$, $8300^{4}$, $8350^{6}$, $8375^{4}$, $8425^{6}$, $8450^{6}$, $8475^{4}$, $8500^{8}$, $8525^{8}$, $8550^{4}$, $8575^{4}$, $8600^{6}$, $8650^{4}$, $8675^{8}$, $8700^{6}$, $8725^{4}$, $8750^{10}$, $8775^{2}$, $8800^{2}$, $8875^{4}$, $8900^{2}$, $8925^{4}$, $8950^{2}$, $8975^{6}$, $9000^{2}$, $9025^{4}$, $9050^{4}$, $9075^{6}$, $9100^{18}$, $9150^{4}$, $9200^{6}$, $9225^{4}$, $9275^{2}$, $9300^{12}$, $9325^{4}$, $9350^{2}$, $9375^{6}$, $9400^{2}$, $9425^{2}$, $9450^{2}$, $9475^{4}$, $9525^{2}$, $9550^{2}$, $9650^{2}$, $9675^{2}$, $9725^{4}$, $9750^{2}$, $9775^{4}$, $9825^{2}$, $9850^{8}$, $9875^{8}$, $9950^{6}$, $9975^{2}$, $10000^{6}$, $10100^{2}$, $10125^{4}$, $10150^{2}$, $10175^{2}$, $10200^{4}$, $10225^{2}$, $10250^{2}$, $10275^{10}$, $10300^{2}$, $10350^{4}$, $10375^{2}$, $10400^{4}$, $10425^{2}$, $10450^{2}$, $10500^{2}$, $10525^{2}$, $10550^{2}$, $10575^{2}$, $10650^{2}$, $10675^{2}$, $10700^{6}$, $10725^{4}$, $10800^{2}$, $10825^{2}$, $10850^{4}$, $10900^{4}$, $10925^{2}$, $11000^{4}$, $11025^{2}$, $11050^{2}$, $11075^{2}$, $11125^{2}$, $11200^{4}$, $11225^{4}$, $11250^{2}$, $11325^{2}$, $11400^{4}$, $11500^{2}$, $11525^{2}$, $11550^{6}$, $11625^{2}$, $11650^{6}$, $11675^{2}$, $11700^{2}$, $11750^{4}$, $11800^{2}$, $11825^{2}$, $11925^{2}$, $11950^{2}$, $12000^{2}$, $12025^{4}$, $12050^{6}$, $12100^{4}$, $12150^{2}$, $12175^{2}$, $12200^{2}$, $12225^{4}$, $12250^{2}$, $12400^{4}$, $12450^{6}$, $12525^{2}$, $12575^{4}$, $12700^{6}$, $12750^{2}$, $12800^{4}$, $12850^{2}$, $12900^{2}$, $12950^{2}$, $13000^{4}$, $13050^{2}$, $13075^{2}$, $13150^{6}$, $13175^{2}$, $13200^{6}$, $13225^{2}$, $13275^{2}$, $13325^{2}$, $13350^{2}$, $13375^{2}$, $13450^{2}$, $13525^{2}$, $13625^{2}$, $13650^{2}$, $13900^{2}$, $13925^{2}$, $14025^{2}$, $14050^{2}$, $14175^{2}$, $14225^{2}$, $14250^{2}$, $14450^{2}$, $14650^{2}$, $14700^{2}$, $14725^{4}$, $14750^{2}$, $14800^{2}$, $14925^{2}$, $14950^{2}$, $15050^{2}$, $15125^{2}$, $15250^{4}$, $15400^{2}$, $15600^{2}$, $15650^{2}$, $15825^{2}$, $15875^{2}$, $16225^{2}$, $16325^{2}$, $16350^{2}$, $16375^{2}$, $16500^{2}$, $16550^{2}$, $16575^{2}$, $16650^{2}$, $16900^{2}$, $17075^{2}$, $17600^{2}$, $17650^{2}$, $17700^{2}$, $18400^{2}$, $18725^{2}$, $19150^{2}$, $19275^{2}$, $19600^{2}$, $20300^{2}$, $20625^{2}$, $20875^{2}$, $22000^{2}$, $22675^{2}$
\item $(h,k)=(1,3)$ : $1^{2}$, $15^{12}$, $50^{2}$, $5900^{2}$, $8625^{2}$, $26975^{2}$, $36850^{2}$, $2051535^{10}$, $6441050^{2}$
\item $(h,k)=(1,4)$ : $1^{2}$, $3^{10}$, $8915^{10}$, $9550^{2}$, $15675^{2}$, $262925^{2}$, $784845^{10}$, $12520250^{2}$
\item $(h,k)=(1,6)$ : $1^{32}$, $10^{10}$, $15^{20}$, $20^{30}$, $25^{10}$, $35^{60}$, $40^{10}$, $50^{2}$, $80^{50}$, $90^{10}$, $105^{10}$, $160^{10}$, $425^{2}$, $565^{10}$, $655^{10}$, $670^{10}$, $900^{2}$, $2155^{10}$, $2198^{50}$, $2615^{10}$, $4950^{10}$, $6267^{50}$, $6816^{50}$, $14100^{10}$, $15835^{10}$, $25305^{10}$, $28709^{50}$, $64405^{10}$, $78320^{10}$, $81700^{10}$, $86005^{10}$, $110825^{2}$, $158760^{10}$, $160755^{10}$, $165780^{10}$, $201585^{10}$, $213535^{10}$, $249870^{10}$, $295630^{10}$, $376710^{10}$, $427820^{10}$, $484195^{10}$
\item $(h,k)=(1,7)$ : $1^{2}$, $5^{2}$, $10^{2}$, $475^{2}$, $900^{4}$, $1250^{2}$, $10675^{2}$, $121025^{2}$, $592350^{2}$, $2183650^{2}$, $13865975^{2}$
\item $(h,k)=(1,8)$ : $1^{2}$, $5^{10}$, $15^{2}$, $225^{2}$, $275^{2}$, $475^{2}$, $7160^{10}$, $53225^{2}$, $16687175^{2}$
\item $(h,k)=(1,9)$ : $1^{2}$, $3^{10}$, $164^{50}$, $5050^{2}$, $2161525^{2}$, $14606525^{2}$
\item $(h,k)=(1,11)$ : $1^{32}$, $5^{10}$, $10^{50}$, $15^{20}$, $20^{10}$, $45^{50}$, $60^{10}$, $61^{50}$, $70^{10}$, $85^{10}$, $135^{10}$, $150^{2}$, $175^{12}$, $200^{2}$, $320^{10}$, $335^{20}$, $340^{10}$, $372^{50}$, $385^{10}$, $550^{10}$, $895^{10}$, $1530^{10}$, $1750^{2}$, $2390^{10}$, $3945^{10}$, $4060^{50}$, $4795^{10}$, $5295^{10}$, $5320^{10}$, $7720^{10}$, $15270^{10}$, $18050^{10}$, $22895^{10}$, $28211^{50}$, $35355^{10}$, $37135^{10}$, $44706^{50}$, $46315^{10}$, $62720^{10}$, $69070^{10}$, $71617^{50}$, $87339^{50}$, $117665^{10}$, $144945^{10}$, $146530^{10}$, $378370^{10}$, $464895^{10}$, $470940^{10}$, $539950^{2}$
\item $(h,k)=(1,12)$ : $1^{2}$, $5^{2}$, $10^{2}$, $75^{2}$, $200^{2}$, $3425^{2}$, $13435^{50}$, $28780^{10}$, $324425^{2}$, $338400^{2}$, $364850^{2}$, $2206205^{10}$, $4235025^{2}$
\item $(h,k)=(1,13)$ : $1^{2}$, $15^{2}$, $60^{10}$, $120^{150}$, $900^{1202}$, $1800^{18030}$
\item $(h,k)=(1,14)$ : $1^{2}$, $3^{10}$, $32280^{10}$, $659725^{2}$, $5772600^{2}$, $10183475^{2}$
\item $(h,k)=(1,16)$ : $1^{32}$, $10^{10}$, $21^{50}$, $25^{2}$, $45^{20}$, $50^{10}$, $107^{50}$, $130^{10}$, $245^{10}$, $345^{10}$, $542^{50}$, $790^{10}$, $1330^{50}$, $1360^{10}$, $1825^{2}$, $2665^{10}$, $3280^{10}$, $3880^{10}$, $9067^{50}$, $10525^{10}$, $12635^{10}$, $14235^{10}$, $22240^{10}$, $23330^{10}$, $51230^{10}$, $53880^{10}$, $74055^{10}$, $80125^{10}$, $95945^{10}$, $124885^{10}$, $163865^{10}$, $205895^{10}$, $217810^{10}$, $239495^{10}$, $247930^{10}$, $280535^{10}$, $302350^{10}$, $420930^{10}$, $536920^{10}$, $540375^{2}$
\item $(h,k)=(1,17)$ : $1^{2}$, $5^{2}$, $10^{2}$, $235^{10}$, $6025^{2}$, $16770000^{2}$
\item $(h,k)=(1,18)$ : $1^{2}$, $10^{10}$, $15^{2}$, $100^{2}$, $575^{2}$, $1125^{2}$, $2275^{2}$, $716225^{2}$, $1144025^{2}$, $7008625^{2}$, $7904200^{2}$
\item $(h,k)=(1,19)$ : $1^{2}$, $3^{10}$, $30^{10}$, $55^{10}$, $4975^{2}$, $6150^{2}$, $10350^{2}$, $367083^{50}$, $3231775^{2}$, $4346450^{2}$
\item $(h,k)=(1,21)$ : $1^{32}$, $5^{10}$, $7^{50}$, $8^{50}$, $10^{10}$, $15^{10}$, $20^{10}$, $25^{2}$, $28^{50}$, $75^{10}$, $175^{2}$, $200^{2}$, $250^{10}$, $314^{50}$, $530^{10}$, $740^{10}$, $1630^{10}$, $1814^{50}$, $4094^{50}$, $5710^{10}$, $8300^{10}$, $9025^{2}$, $19536^{50}$, $23248^{50}$, $29975^{2}$, $45180^{10}$, $45685^{10}$, $61640^{10}$, $71670^{10}$, $79520^{50}$, $81225^{10}$, $82450^{50}$, $146885^{10}$, $148450^{2}$, $158915^{10}$, $292775^{10}$, $313870^{10}$, $354275^{2}$, $415800^{10}$, $540990^{10}$
\item $(h,k)=(1,22)$ : $1^{2}$, $5^{2}$, $10^{2}$, $125^{2}$, $295^{10}$, $1533775^{2}$, $15241825^{2}$
\item $(h,k)=(1,23)$ : $1^{2}$, $15^{2}$, $60^{30}$, $90^{200}$, $123^{50}$, $150^{706}$, $180^{280}$, $198^{200}$, $210^{70}$, $213^{150}$, $225^{160}$, $240^{150}$, $243^{200}$, $246^{200}$, $300^{4}$, $315^{40}$, $345^{10}$, $351^{200}$, $360^{200}$, $372^{150}$, $375^{20}$, $420^{200}$, $435^{40}$, $450^{8}$, $510^{20}$, $519^{100}$, $525^{30}$, $540^{140}$, $552^{150}$, $570^{40}$, $600^{76}$, $630^{80}$, $645^{100}$, $660^{280}$, $675^{124}$, $690^{80}$, $714^{50}$, $720^{40}$, $735^{60}$, $738^{50}$, $750^{72}$, $765^{80}$, $768^{50}$, $822^{50}$, $825^{144}$, $870^{10}$, $897^{100}$, $900^{112}$, $945^{80}$, $957^{50}$, $960^{100}$, $975^{34}$, $990^{40}$, $1020^{20}$, $1035^{60}$, $1050^{248}$, $1080^{60}$, $1095^{20}$, $1125^{4}$, $1140^{60}$, $1200^{44}$, $1215^{40}$, $1245^{40}$, $1335^{40}$, $1350^{138}$, $1425^{14}$, $1470^{10}$, $1500^{52}$, $1530^{60}$, $1545^{10}$, $1605^{40}$, $1632^{50}$, $1635^{20}$, $1650^{40}$, $1680^{70}$, $1695^{60}$, $1710^{50}$, $1725^{20}$, $1800^{80}$, $1845^{10}$, $1875^{8}$, $2025^{76}$, $2070^{10}$, $2100^{20}$, $2130^{20}$, $2175^{24}$, $2250^{90}$, $2310^{20}$, $2325^{60}$, $2385^{20}$, $2400^{36}$, $2475^{34}$, $2550^{18}$, $2580^{20}$, $2595^{10}$, $2610^{40}$, $2625^{18}$, $2670^{50}$, $2700^{50}$, $2760^{20}$, $2775^{20}$, $2805^{10}$, $2850^{52}$, $3000^{78}$, $3060^{30}$, $3075^{26}$, $3150^{24}$, $3240^{10}$, $3300^{42}$, $3375^{60}$, $3420^{20}$, $3450^{30}$, $3495^{10}$, $3510^{10}$, $3525^{26}$, $3555^{10}$, $3600^{22}$, $3645^{10}$, $3750^{30}$, $3810^{10}$, $3825^{20}$, $3900^{104}$, $3915^{30}$, $3930^{20}$, $3975^{40}$, $4050^{38}$, $4110^{10}$, $4125^{16}$, $4200^{48}$, $4275^{8}$, $4350^{50}$, $4380^{20}$, $4395^{10}$, $4425^{2}$, $4500^{16}$, $4650^{14}$, $4680^{10}$, $4725^{4}$, $4800^{22}$, $4815^{10}$, $4875^{2}$, $4980^{10}$, $5025^{4}$, $5100^{30}$, $5175^{20}$, $5250^{20}$, $5295^{10}$, $5325^{10}$, $5400^{16}$, $5430^{10}$, $5475^{16}$, $5505^{10}$, $5550^{40}$, $5640^{10}$, $5685^{10}$, $5700^{10}$, $5730^{10}$, $5745^{10}$, $5850^{52}$, $5925^{8}$, $6000^{22}$, $6075^{4}$, $6150^{40}$, $6225^{10}$, $6300^{32}$, $6450^{20}$, $6525^{24}$, $6600^{12}$, $6675^{6}$, $6750^{14}$, $6900^{10}$, $6915^{10}$, $6930^{10}$, $7050^{20}$, $7125^{8}$, $7185^{10}$, $7200^{12}$, $7350^{20}$, $7425^{22}$, $7500^{12}$, $7575^{4}$, $7650^{12}$, $7725^{26}$, $7755^{10}$, $7800^{16}$, $7875^{8}$, $7950^{16}$, $8040^{10}$, $8100^{12}$, $8250^{4}$, $8325^{6}$, $8400^{12}$, $8550^{20}$, $8625^{6}$, $8775^{16}$, $8850^{32}$, $8925^{4}$, $9000^{26}$, $9075^{4}$, $9150^{8}$, $9300^{20}$, $9375^{4}$, $9450^{12}$, $9525^{8}$, $9600^{8}$, $9690^{10}$, $9750^{4}$, $9900^{8}$, $10050^{6}$, $10125^{10}$, $10275^{8}$, $10350^{6}$, $10500^{16}$, $10575^{4}$, $10650^{10}$, $10725^{6}$, $10875^{6}$, $10950^{8}$, $11100^{8}$, $11175^{14}$, $11250^{20}$, $11400^{10}$, $11475^{2}$, $11550^{12}$, $11700^{20}$, $11850^{8}$, $12000^{4}$, $12075^{10}$, $12150^{14}$, $12300^{4}$, $12375^{8}$, $12450^{18}$, $12525^{2}$, $12600^{6}$, $12750^{12}$, $12825^{4}$, $12900^{8}$, $12975^{4}$, $13050^{10}$, $13200^{16}$, $13275^{4}$, $13350^{8}$, $13650^{4}$, $13800^{4}$, $13950^{12}$, $14025^{2}$, $14250^{10}$, $14400^{14}$, $14550^{20}$, $14700^{8}$, $14775^{4}$, $14850^{2}$, $15075^{2}$, $15150^{10}$, $15300^{6}$, $15375^{4}$, $15525^{2}$, $15750^{8}$, $15900^{8}$, $15975^{4}$, $16050^{8}$, $16200^{8}$, $16425^{4}$, $16500^{6}$, $16650^{12}$, $16725^{4}$, $16800^{2}$, $16950^{4}$, $17025^{2}$, $17100^{18}$, $17175^{4}$, $17250^{8}$, $17325^{8}$, $17400^{4}$, $17550^{4}$, $17625^{4}$, $17700^{8}$, $17775^{2}$, $17850^{6}$, $17925^{2}$, $18000^{6}$, $18150^{4}$, $18225^{4}$, $18300^{2}$, $18375^{6}$, $18450^{2}$, $18525^{4}$, $18600^{4}$, $18825^{2}$, $18975^{2}$, $19050^{4}$, $19125^{2}$, $19200^{8}$, $19350^{2}$, $19500^{8}$, $19575^{2}$, $19650^{6}$, $19950^{2}$, $20100^{2}$, $20400^{4}$, $20475^{6}$, $20550^{2}$, $20625^{2}$, $20850^{4}$, $20925^{2}$, $21000^{6}$, $21075^{4}$, $21225^{2}$, $21300^{2}$, $21450^{2}$, $21675^{4}$, $21975^{4}$, $22125^{2}$, $22275^{2}$, $22500^{2}$, $22650^{4}$, $22800^{4}$, $23025^{2}$, $23100^{6}$, $23250^{4}$, $23325^{2}$, $23850^{2}$, $24075^{2}$, $24150^{2}$, $24375^{2}$, $24450^{2}$, $24525^{4}$, $24825^{2}$, $24900^{2}$, $25050^{2}$, $25275^{4}$, $25350^{4}$, $25575^{2}$, $25650^{2}$, $25800^{2}$, $25950^{2}$, $26100^{4}$, $26250^{2}$, $26400^{2}$, $26700^{2}$, $26850^{2}$, $26925^{6}$, $27000^{2}$, $27225^{4}$, $27300^{14}$, $27450^{2}$, $27600^{2}$, $27900^{2}$, $27975^{2}$, $28050^{2}$, $28125^{2}$, $28200^{2}$, $28425^{2}$, $28650^{2}$, $29025^{2}$, $29175^{4}$, $29325^{4}$, $30000^{4}$, $30375^{2}$, $30525^{2}$, $30750^{2}$, $30825^{4}$, $31350^{2}$, $31725^{2}$, $32175^{2}$, $32550^{4}$, $33600^{2}$, $33675^{2}$, $34200^{2}$, $34875^{2}$, $35400^{2}$, $35475^{2}$, $36675^{2}$, $37350^{2}$, $37725^{2}$, $38100^{4}$, $38400^{2}$, $38550^{2}$, $38850^{2}$, $39000^{2}$, $39225^{2}$, $39450^{2}$, $39600^{2}$, $39825^{2}$, $40050^{2}$, $40125^{2}$, $40350^{2}$, $41775^{2}$, $42075^{2}$, $42525^{2}$, $42750^{2}$, $43350^{2}$, $44100^{2}$, $44175^{2}$, $44250^{2}$, $44850^{2}$, $45375^{2}$, $46950^{2}$, $47625^{2}$, $49125^{2}$, $50700^{2}$, $52800^{2}$, $52950^{2}$, $57450^{2}$, $60900^{2}$, $62625^{2}$
\item $(h,k)=(1,24)$ : $1^{2}$, $3^{11184810}$
\item $(h,k)=(2,3)$ : $1^{2}$, $3^{10}$, $5^{50}$, $50^{2}$, $1450^{2}$, $4025^{2}$, $9700^{2}$, $37275^{2}$, $79375^{2}$, $1489850^{2}$, $2798975^{2}$, $12356375^{2}$
\item $(h,k)=(2,4)$ : $1^{2}$, $15^{2}$, $50^{2}$, $135^{10}$, $300^{2}$, $525^{2}$, $1190^{10}$, $1325^{2}$, $2725^{2}$, $2840^{10}$, $26350^{2}$, $84625^{2}$, $191475^{2}$, $216575^{2}$, $326475^{2}$, $662000^{2}$, $675975^{2}$, $14567975^{2}$
\item $(h,k)=(2,6)$ : $1^{2}$, $5^{2}$, $10^{2}$, $25^{10}$, $35^{10}$, $100^{2}$, $3000^{2}$, $17133^{50}$, $111925^{2}$, $173250^{2}$, $16060300^{2}$
\item $(h,k)=(2,7)$ : $1^{32}$, $5^{20}$, $8^{50}$, $10^{10}$, $15^{10}$, $30^{10}$, $65^{10}$, $160^{10}$, $234^{50}$, $235^{10}$, $285^{10}$, $455^{10}$, $860^{10}$, $955^{10}$, $1107^{50}$, $1155^{10}$, $1455^{10}$, $1475^{2}$, $5390^{10}$, $6375^{2}$, $8770^{10}$, $12690^{50}$, $18555^{10}$, $19800^{10}$, $22070^{10}$, $23015^{10}$, $25675^{2}$, $32977^{50}$, $59185^{10}$, $67190^{10}$, $85650^{2}$, $89325^{10}$, $90025^{10}$, $170725^{10}$, $262405^{10}$, $341180^{10}$, $364390^{10}$, $423025^{2}$, $462200^{10}$, $468080^{10}$, $533925^{10}$
\item $(h,k)=(2,8)$ : $1^{2}$, $3^{10}$, $750^{2}$, $7625^{10}$, $27650^{2}$, $1867875^{2}$, $14842800^{2}$
\item $(h,k)=(2,9)$ : $1^{2}$, $15^{2}$, $125^{2}$, $275^{2}$, $325^{2}$, $350625^{2}$, $7793525^{2}$, $8632325^{2}$
\item $(h,k)=(2,11)$ : $1^{2}$, $5^{2}$, $10^{2}$, $350^{2}$, $1150^{2}$, $1825^{2}$, $2315^{10}$, $61325^{2}$, $135725^{2}$, $16565250^{2}$
\item $(h,k)=(2,12)$ : $1^{32}$, $5^{10}$, $15^{10}$, $40^{10}$, $60^{10}$, $100^{2}$, $125^{10}$, $210^{10}$, $225^{10}$, $245^{20}$, $295^{10}$, $325^{2}$, $752^{50}$, $1615^{10}$, $2035^{10}$, $2070^{10}$, $2490^{10}$, $2515^{10}$, $2670^{10}$, $2825^{2}$, $4102^{50}$, $5640^{10}$, $6072^{50}$, $8484^{50}$, $8515^{10}$, $10335^{10}$, $10670^{10}$, $10900^{2}$, $12240^{10}$, $16580^{10}$, $21840^{10}$, $33028^{50}$, $46000^{2}$, $51040^{50}$, $51540^{10}$, $53740^{10}$, $56056^{50}$, $73480^{10}$, $87965^{10}$, $118630^{10}$, $165345^{10}$, $189585^{10}$, $191615^{10}$, $265470^{10}$, $307420^{10}$, $404705^{10}$, $439150^{10}$, $482075^{2}$
\item $(h,k)=(2,13)$ : $1^{2}$, $3^{10}$, $17050^{2}$, $20025^{2}$, $30575^{2}$, $82675^{2}$, $153450^{2}$, $232650^{2}$, $338840^{50}$, $7769775^{2}$
\item $(h,k)=(2,14)$ : $1^{2}$, $15^{2}$, $75^{2}$, $300^{2}$, $525^{2}$, $11950^{2}$, $27250^{2}$, $130550^{2}$, $503000^{2}$, $1377300^{2}$, $2945250^{10}$
\item $(h,k)=(2,16)$ : $1^{2}$, $5^{2}$, $10^{2}$, $15^{10}$, $37950^{2}$, $87450^{2}$, $94200^{2}$, $208650^{2}$, $406975^{2}$, $15941900^{2}$
\item $(h,k)=(2,17)$ : $1^{32}$, $5^{40}$, $6^{50}$, $7^{50}$, $15^{10}$, $40^{10}$, $65^{10}$, $150^{2}$, $200^{10}$, $235^{20}$, $250^{2}$, $475^{2}$, $675^{10}$, $889^{50}$, $935^{10}$, $1146^{50}$, $1400^{50}$, $1800^{10}$, $3536^{50}$, $3655^{10}$, $4220^{50}$, $5815^{10}$, $16895^{10}$, $19789^{50}$, $47500^{10}$, $65250^{50}$, $81815^{10}$, $86770^{10}$, $127060^{10}$, $136220^{10}$, $182485^{10}$, $231685^{10}$, $259975^{10}$, $269800^{10}$, $270130^{10}$, $518460^{10}$, $523295^{10}$, $541350^{2}$
\item $(h,k)=(2,18)$ : $1^{2}$, $3^{10}$, $50^{2}$, $3750^{2}$, $23700^{2}$, $45350^{2}$, $122931^{50}$, $13631075^{2}$
\item $(h,k)=(2,19)$ : $1^{2}$, $15^{12}$, $65^{10}$, $400^{2}$, $1400^{2}$, $1925^{2}$, $3200^{2}$, $3925^{2}$, $37925^{50}$, $473425^{2}$, $15344400^{2}$
\item $(h,k)=(2,21)$ : $1^{2}$, $5^{2}$, $10^{2}$, $3800^{10}$, $223825^{2}$, $16534375^{2}$
\item $(h,k)=(2,22)$ : $1^{32}$, $5^{20}$, $10^{20}$, $15^{10}$, $20^{10}$, $25^{10}$, $40^{10}$, $45^{50}$, $90^{10}$, $105^{10}$, $135^{10}$, $177^{50}$, $290^{10}$, $310^{10}$, $315^{10}$, $325^{10}$, $520^{10}$, $660^{10}$, $670^{10}$, $685^{10}$, $770^{10}$, $1085^{10}$, $2201^{50}$, $3645^{10}$, $9265^{10}$, $10405^{10}$, $11510^{10}$, $27388^{50}$, $29150^{2}$, $44700^{2}$, $97125^{2}$, $105798^{50}$, $126220^{10}$, $177255^{10}$, $226670^{10}$, $250180^{10}$, $272900^{2}$, $286490^{10}$, $402515^{10}$, $537275^{10}$, $541100^{10}$
\item $(h,k)=(2,23)$ : $1^{2}$, $3^{11184810}$
\item $(h,k)=(2,24)$ : $1^{2}$, $15^{2}$, $60^{30}$, $90^{200}$, $123^{50}$, $150^{706}$, $180^{280}$, $198^{200}$, $210^{70}$, $213^{150}$, $225^{160}$, $240^{150}$, $243^{200}$, $246^{200}$, $300^{4}$, $315^{40}$, $345^{10}$, $351^{200}$, $360^{200}$, $372^{150}$, $375^{20}$, $420^{200}$, $435^{40}$, $450^{8}$, $510^{20}$, $519^{100}$, $525^{30}$, $540^{140}$, $552^{150}$, $570^{40}$, $600^{76}$, $630^{80}$, $645^{100}$, $660^{280}$, $675^{124}$, $690^{80}$, $714^{50}$, $720^{40}$, $735^{60}$, $738^{50}$, $750^{72}$, $765^{80}$, $768^{50}$, $822^{50}$, $825^{144}$, $870^{10}$, $897^{100}$, $900^{112}$, $945^{80}$, $957^{50}$, $960^{100}$, $975^{34}$, $990^{40}$, $1020^{20}$, $1035^{60}$, $1050^{248}$, $1080^{60}$, $1095^{20}$, $1125^{4}$, $1140^{60}$, $1200^{44}$, $1215^{40}$, $1245^{40}$, $1335^{40}$, $1350^{138}$, $1425^{14}$, $1470^{10}$, $1500^{52}$, $1530^{60}$, $1545^{10}$, $1605^{40}$, $1632^{50}$, $1635^{20}$, $1650^{40}$, $1680^{70}$, $1695^{60}$, $1710^{50}$, $1725^{20}$, $1800^{80}$, $1845^{10}$, $1875^{8}$, $2025^{76}$, $2070^{10}$, $2100^{20}$, $2130^{20}$, $2175^{24}$, $2250^{90}$, $2310^{20}$, $2325^{60}$, $2385^{20}$, $2400^{36}$, $2475^{34}$, $2550^{18}$, $2580^{20}$, $2595^{10}$, $2610^{40}$, $2625^{18}$, $2670^{50}$, $2700^{50}$, $2760^{20}$, $2775^{20}$, $2805^{10}$, $2850^{52}$, $3000^{78}$, $3060^{30}$, $3075^{26}$, $3150^{24}$, $3240^{10}$, $3300^{42}$, $3375^{60}$, $3420^{20}$, $3450^{30}$, $3495^{10}$, $3510^{10}$, $3525^{26}$, $3555^{10}$, $3600^{22}$, $3645^{10}$, $3750^{30}$, $3810^{10}$, $3825^{20}$, $3900^{104}$, $3915^{30}$, $3930^{20}$, $3975^{40}$, $4050^{38}$, $4110^{10}$, $4125^{16}$, $4200^{48}$, $4275^{8}$, $4350^{50}$, $4380^{20}$, $4395^{10}$, $4425^{2}$, $4500^{16}$, $4650^{14}$, $4680^{10}$, $4725^{4}$, $4800^{22}$, $4815^{10}$, $4875^{2}$, $4980^{10}$, $5025^{4}$, $5100^{30}$, $5175^{20}$, $5250^{20}$, $5295^{10}$, $5325^{10}$, $5400^{16}$, $5430^{10}$, $5475^{16}$, $5505^{10}$, $5550^{40}$, $5640^{10}$, $5685^{10}$, $5700^{10}$, $5730^{10}$, $5745^{10}$, $5850^{52}$, $5925^{8}$, $6000^{22}$, $6075^{4}$, $6150^{40}$, $6225^{10}$, $6300^{32}$, $6450^{20}$, $6525^{24}$, $6600^{12}$, $6675^{6}$, $6750^{14}$, $6900^{10}$, $6915^{10}$, $6930^{10}$, $7050^{20}$, $7125^{8}$, $7185^{10}$, $7200^{12}$, $7350^{20}$, $7425^{22}$, $7500^{12}$, $7575^{4}$, $7650^{12}$, $7725^{26}$, $7755^{10}$, $7800^{16}$, $7875^{8}$, $7950^{16}$, $8040^{10}$, $8100^{12}$, $8250^{4}$, $8325^{6}$, $8400^{12}$, $8550^{20}$, $8625^{6}$, $8775^{16}$, $8850^{32}$, $8925^{4}$, $9000^{26}$, $9075^{4}$, $9150^{8}$, $9300^{20}$, $9375^{4}$, $9450^{12}$, $9525^{8}$, $9600^{8}$, $9690^{10}$, $9750^{4}$, $9900^{8}$, $10050^{6}$, $10125^{10}$, $10275^{8}$, $10350^{6}$, $10500^{16}$, $10575^{4}$, $10650^{10}$, $10725^{6}$, $10875^{6}$, $10950^{8}$, $11100^{8}$, $11175^{14}$, $11250^{20}$, $11400^{10}$, $11475^{2}$, $11550^{12}$, $11700^{20}$, $11850^{8}$, $12000^{4}$, $12075^{10}$, $12150^{14}$, $12300^{4}$, $12375^{8}$, $12450^{18}$, $12525^{2}$, $12600^{6}$, $12750^{12}$, $12825^{4}$, $12900^{8}$, $12975^{4}$, $13050^{10}$, $13200^{16}$, $13275^{4}$, $13350^{8}$, $13650^{4}$, $13800^{4}$, $13950^{12}$, $14025^{2}$, $14250^{10}$, $14400^{14}$, $14550^{20}$, $14700^{8}$, $14775^{4}$, $14850^{2}$, $15075^{2}$, $15150^{10}$, $15300^{6}$, $15375^{4}$, $15525^{2}$, $15750^{8}$, $15900^{8}$, $15975^{4}$, $16050^{8}$, $16200^{8}$, $16425^{4}$, $16500^{6}$, $16650^{12}$, $16725^{4}$, $16800^{2}$, $16950^{4}$, $17025^{2}$, $17100^{18}$, $17175^{4}$, $17250^{8}$, $17325^{8}$, $17400^{4}$, $17550^{4}$, $17625^{4}$, $17700^{8}$, $17775^{2}$, $17850^{6}$, $17925^{2}$, $18000^{6}$, $18150^{4}$, $18225^{4}$, $18300^{2}$, $18375^{6}$, $18450^{2}$, $18525^{4}$, $18600^{4}$, $18825^{2}$, $18975^{2}$, $19050^{4}$, $19125^{2}$, $19200^{8}$, $19350^{2}$, $19500^{8}$, $19575^{2}$, $19650^{6}$, $19950^{2}$, $20100^{2}$, $20400^{4}$, $20475^{6}$, $20550^{2}$, $20625^{2}$, $20850^{4}$, $20925^{2}$, $21000^{6}$, $21075^{4}$, $21225^{2}$, $21300^{2}$, $21450^{2}$, $21675^{4}$, $21975^{4}$, $22125^{2}$, $22275^{2}$, $22500^{2}$, $22650^{4}$, $22800^{4}$, $23025^{2}$, $23100^{6}$, $23250^{4}$, $23325^{2}$, $23850^{2}$, $24075^{2}$, $24150^{2}$, $24375^{2}$, $24450^{2}$, $24525^{4}$, $24825^{2}$, $24900^{2}$, $25050^{2}$, $25275^{4}$, $25350^{4}$, $25575^{2}$, $25650^{2}$, $25800^{2}$, $25950^{2}$, $26100^{4}$, $26250^{2}$, $26400^{2}$, $26700^{2}$, $26850^{2}$, $26925^{6}$, $27000^{2}$, $27225^{4}$, $27300^{14}$, $27450^{2}$, $27600^{2}$, $27900^{2}$, $27975^{2}$, $28050^{2}$, $28125^{2}$, $28200^{2}$, $28425^{2}$, $28650^{2}$, $29025^{2}$, $29175^{4}$, $29325^{4}$, $30000^{4}$, $30375^{2}$, $30525^{2}$, $30750^{2}$, $30825^{4}$, $31350^{2}$, $31725^{2}$, $32175^{2}$, $32550^{4}$, $33600^{2}$, $33675^{2}$, $34200^{2}$, $34875^{2}$, $35400^{2}$, $35475^{2}$, $36675^{2}$, $37350^{2}$, $37725^{2}$, $38100^{4}$, $38400^{2}$, $38550^{2}$, $38850^{2}$, $39000^{2}$, $39225^{2}$, $39450^{2}$, $39600^{2}$, $39825^{2}$, $40050^{2}$, $40125^{2}$, $40350^{2}$, $41775^{2}$, $42075^{2}$, $42525^{2}$, $42750^{2}$, $43350^{2}$, $44100^{2}$, $44175^{2}$, $44250^{2}$, $44850^{2}$, $45375^{2}$, $46950^{2}$, $47625^{2}$, $49125^{2}$, $50700^{2}$, $52800^{2}$, $52950^{2}$, $57450^{2}$, $60900^{2}$, $62625^{2}$
\item $(h,k)=(3,4)$ : $1^{2}$, $5^{2}$, $10^{2}$, $100^{2}$, $175^{2}$, $650^{2}$, $1250^{2}$, $1355^{10}$, $1444^{50}$, $8010^{10}$, $311375^{2}$, $1085650^{2}$, $3059015^{10}$
\item $(h,k)=(3,6)$ : $1^{2}$, $15^{2}$, $50^{2}$, $125^{2}$, $850^{2}$, $1150^{2}$, $1475^{2}$, $2400^{2}$, $4450^{2}$, $6050^{2}$, $8175^{2}$, $22670^{50}$, $137225^{2}$, $319425^{2}$, $375375^{2}$, $464075^{2}$, $832250^{2}$, $1263650^{2}$, $3262500^{2}$, $9531225^{2}$
\item $(h,k)=(3,7)$ : $1^{2}$, $3^{10}$, $10^{10}$, $4350^{2}$, $48800^{2}$, $63935^{10}$, $261725^{2}$, $1982200^{2}$, $2829925^{2}$, $11330475^{2}$
\item $(h,k)=(3,8)$ : $1^{32}$, $2^{300}$, $5^{30}$, $8^{50}$, $9^{50}$, $10^{20}$, $20^{10}$, $22^{50}$, $45^{10}$, $50^{12}$, $65^{10}$, $70^{10}$, $96^{50}$, $175^{2}$, $305^{10}$, $320^{10}$, $475^{10}$, $570^{10}$, $605^{10}$, $1000^{10}$, $3015^{10}$, $6565^{10}$, $7160^{10}$, $17918^{50}$, $32475^{2}$, $38335^{10}$, $100540^{50}$, $106668^{50}$, $448305^{10}$, $509425^{2}$, $533240^{10}$, $539735^{10}$, $540735^{10}$
\item $(h,k)=(3,9)$ : $1^{2}$, $5^{2}$, $10^{2}$, $38^{50}$, $75^{10}$, $29975^{2}$, $86330^{10}$, $236975^{2}$, $453773^{50}$, $1366800^{2}$, $3366150^{2}$
\item $(h,k)=(3,11)$ : $1^{2}$, $15^{2}$, $200^{10}$, $850^{2}$, $2720^{10}$, $12175^{2}$, $16749575^{2}$
\item $(h,k)=(3,12)$ : $1^{2}$, $3^{10}$, $5^{10}$, $25^{2}$, $50^{2}$, $150^{2}$, $3165^{10}$, $162850^{2}$, $16598275^{2}$
\item $(h,k)=(3,13)$ : $1^{32}$, $4^{50}$, $5^{60}$, $10^{10}$, $16^{50}$, $20^{10}$, $22^{50}$, $65^{10}$, $70^{10}$, $100^{10}$, $130^{10}$, $235^{10}$, $269^{50}$, $305^{10}$, $400^{2}$, $713^{50}$, $840^{10}$, $865^{10}$, $1310^{10}$, $2030^{10}$, $2050^{10}$, $2357^{50}$, $2805^{10}$, $2820^{10}$, $2915^{10}$, $3015^{10}$, $3930^{10}$, $5110^{10}$, $5472^{50}$, $6515^{10}$, $8390^{10}$, $15150^{10}$, $24745^{50}$, $27525^{10}$, $29730^{10}$, $40055^{10}$, $64025^{10}$, $66215^{10}$, $105805^{10}$, $125085^{10}$, $196150^{10}$, $198025^{10}$, $204200^{2}$, $222170^{10}$, $256310^{10}$, $299745^{10}$, $309790^{10}$, $314370^{10}$, $333980^{10}$, $498840^{10}$
\item $(h,k)=(3,14)$ : $1^{2}$, $5^{2}$, $10^{2}$, $150^{2}$, $195^{10}$, $250^{2}$, $1075^{2}$, $1400^{2}$, $1925^{2}$, $5050^{2}$, $5725^{2}$, $8150^{2}$, $16752500^{2}$
\item $(h,k)=(3,16)$ : $1^{2}$, $15^{2}$, $2600^{2}$, $76575^{2}$, $667921^{50}$
\item $(h,k)=(3,17)$ : $1^{2}$, $3^{210}$, $25^{2}$, $100^{2}$, $110^{50}$, $600^{2}$, $5550^{2}$, $12250^{2}$, $165102^{50}$, $783325^{2}$, $11844750^{2}$
\item $(h,k)=(3,18)$ : $1^{32}$, $5^{60}$, $10^{50}$, $15^{10}$, $20^{10}$, $25^{10}$, $65^{10}$, $105^{10}$, $230^{10}$, $325^{2}$, $360^{10}$, $415^{10}$, $470^{10}$, $654^{50}$, $750^{10}$, $845^{10}$, $2325^{10}$, $3075^{10}$, $3225^{10}$, $3925^{10}$, $4000^{10}$, $4790^{10}$, $5100^{2}$, $5400^{2}$, $5891^{50}$, $6215^{10}$, $7137^{50}$, $8030^{10}$, $9466^{50}$, $10475^{10}$, $15620^{10}$, $21310^{10}$, $25335^{10}$, $34555^{10}$, $37485^{10}$, $49220^{10}$, $49235^{10}$, $57407^{50}$, $66335^{10}$, $98634^{50}$, $196765^{10}$, $359735^{10}$, $444620^{10}$, $468105^{10}$, $491250^{2}$, $541320^{10}$
\item $(h,k)=(3,19)$ : $1^{2}$, $5^{2}$, $10^{2}$, $575^{2}$, $3450^{2}$, $121725^{2}$, $143225^{2}$, $3889125^{2}$, $12619100^{2}$
\item $(h,k)=(3,21)$ : $1^{2}$, $15^{2}$, $200^{2}$, $1125^{2}$, $2975^{2}$, $39805^{10}$, $533625^{2}$, $16040250^{2}$
\item $(h,k)=(3,22)$ : $1^{2}$, $3^{11184810}$
\item $(h,k)=(3,23)$ : $1^{32}$, $5^{20}$, $10^{20}$, $15^{10}$, $20^{10}$, $25^{10}$, $40^{10}$, $45^{50}$, $90^{10}$, $105^{10}$, $135^{10}$, $177^{50}$, $290^{10}$, $310^{10}$, $315^{10}$, $325^{10}$, $520^{10}$, $660^{10}$, $670^{10}$, $685^{10}$, $770^{10}$, $1085^{10}$, $2201^{50}$, $3645^{10}$, $9265^{10}$, $10405^{10}$, $11510^{10}$, $27388^{50}$, $29150^{2}$, $44700^{2}$, $97125^{2}$, $105798^{50}$, $126220^{10}$, $177255^{10}$, $226670^{10}$, $250180^{10}$, $272900^{2}$, $286490^{10}$, $402515^{10}$, $537275^{10}$, $541100^{10}$
\item $(h,k)=(3,24)$ : $1^{2}$, $5^{2}$, $10^{2}$, $125^{2}$, $295^{10}$, $1533775^{2}$, $15241825^{2}$
\item $(h,k)=(4,6)$ : $1^{2}$, $3^{10}$, $75^{2}$, $250^{10}$, $1750^{2}$, $45700^{2}$, $2243375^{10}$, $5511550^{2}$
\item $(h,k)=(4,7)$ : $1^{2}$, $5^{10}$, $15^{2}$, $50^{2}$, $100^{2}$, $7350^{2}$, $9650^{2}$, $25675^{2}$, $29775^{2}$, $362100^{2}$, $501715^{10}$, $2587275^{2}$, $3798325^{2}$, $7448300^{2}$
\item $(h,k)=(4,8)$ : $1^{2}$, $5^{2}$, $10^{2}$, $25^{2}$, $100^{4}$, $125^{2}$, $275^{2}$, $425^{2}$, $428^{50}$, $600^{2}$, $2050^{2}$, $2865^{10}$, $34702^{50}$, $68925^{2}$, $72350^{2}$, $120325^{10}$, $2522425^{2}$, $2866750^{2}$, $9748850^{2}$
\item $(h,k)=(4,9)$ : $1^{32}$, $5^{90}$, $15^{10}$, $30^{10}$, $48^{50}$, $50^{10}$, $70^{10}$, $100^{10}$, $270^{10}$, $305^{20}$, $390^{10}$, $930^{10}$, $995^{10}$, $2190^{10}$, $2195^{10}$, $3125^{2}$, $3179^{50}$, $7105^{10}$, $8760^{10}$, $8930^{10}$, $9695^{10}$, $12500^{2}$, $25220^{10}$, $39870^{10}$, $45521^{50}$, $62840^{10}$, $102890^{10}$, $115685^{10}$, $154225^{2}$, $166310^{10}$, $212520^{10}$, $237180^{10}$, $296270^{10}$, $334280^{10}$, $372250^{2}$, $410560^{10}$, $425780^{10}$, $531495^{10}$
\item $(h,k)=(4,11)$ : $1^{2}$, $3^{10}$, $25^{12}$, $125^{2}$, $1000^{2}$, $3355185^{10}$
\item $(h,k)=(4,12)$ : $1^{2}$, $15^{2}$, $60^{10}$, $6675^{2}$, $51900^{2}$, $118800^{2}$, $16599525^{2}$
\item $(h,k)=(4,13)$ : $1^{2}$, $5^{2}$, $10^{2}$, $50^{2}$, $500^{2}$, $550^{2}$, $5275^{2}$, $137685^{10}$, $143450^{2}$, $15938950^{2}$
\item $(h,k)=(4,14)$ : $1^{32}$, $3^{50}$, $5^{30}$, $10^{10}$, $15^{60}$, $25^{20}$, $55^{10}$, $200^{50}$, $305^{10}$, $345^{10}$, $435^{10}$, $611^{50}$, $1680^{10}$, $2130^{10}$, $3150^{2}$, $3575^{10}$, $3865^{10}$, $4410^{10}$, $7765^{10}$, $14165^{10}$, $19150^{10}$, $68850^{10}$, $87135^{10}$, $108297^{50}$, $141750^{2}$, $155750^{2}$, $192990^{10}$, $241550^{2}$, $260740^{10}$, $454385^{10}$, $513885^{10}$, $527185^{10}$, $538230^{10}$
\item $(h,k)=(4,16)$ : $1^{2}$, $3^{10}$, $75^{2}$, $125^{4}$, $12950^{2}$, $27775^{10}$, $758225^{2}$, $4202225^{2}$, $11664600^{2}$
\item $(h,k)=(4,17)$ : $1^{2}$, $15^{2}$, $85^{10}$, $175^{6}$, $1400^{2}$, $1550^{2}$, $5650^{2}$, $16275^{2}$, $7783525^{2}$, $8967850^{2}$
\item $(h,k)=(4,18)$ : $1^{2}$, $5^{2}$, $10^{2}$, $100^{2}$, $1243900^{2}$, $2040525^{2}$, $3538300^{2}$, $9954375^{2}$
\item $(h,k)=(4,19)$ : $1^{32}$, $5^{10}$, $8^{50}$, $10^{50}$, $15^{10}$, $25^{10}$, $30^{10}$, $40^{10}$, $75^{10}$, $160^{10}$, $220^{10}$, $232^{50}$, $390^{10}$, $625^{2}$, $775^{10}$, $1095^{10}$, $1110^{10}$, $2025^{2}$, $2045^{10}$, $3385^{10}$, $4066^{50}$, $5105^{10}$, $5510^{10}$, $8945^{10}$, $11940^{10}$, $13173^{50}$, $17553^{50}$, $20440^{10}$, $47635^{10}$, $52465^{10}$, $83060^{10}$, $106935^{10}$, $108610^{10}$, $124735^{10}$, $159485^{10}$, $197110^{10}$, $359400^{10}$, $386365^{10}$, $420965^{10}$, $430850^{10}$, $532870^{10}$, $539525^{2}$
\item $(h,k)=(4,21)$ : $1^{2}$, $3^{11184810}$
\item $(h,k)=(4,22)$ : $1^{2}$, $15^{2}$, $200^{2}$, $1125^{2}$, $2975^{2}$, $39805^{10}$, $533625^{2}$, $16040250^{2}$
\item $(h,k)=(4,23)$ : $1^{2}$, $5^{2}$, $10^{2}$, $3800^{10}$, $223825^{2}$, $16534375^{2}$
\item $(h,k)=(4,24)$ : $1^{32}$, $5^{10}$, $7^{50}$, $8^{50}$, $10^{10}$, $15^{10}$, $20^{10}$, $25^{2}$, $28^{50}$, $75^{10}$, $175^{2}$, $200^{2}$, $250^{10}$, $314^{50}$, $530^{10}$, $740^{10}$, $1630^{10}$, $1814^{50}$, $4094^{50}$, $5710^{10}$, $8300^{10}$, $9025^{2}$, $19536^{50}$, $23248^{50}$, $29975^{2}$, $45180^{10}$, $45685^{10}$, $61640^{10}$, $71670^{10}$, $79520^{50}$, $81225^{10}$, $82450^{50}$, $146885^{10}$, $148450^{2}$, $158915^{10}$, $292775^{10}$, $313870^{10}$, $354275^{2}$, $415800^{10}$, $540990^{10}$
\item $(h,k)=(6,7)$ : $1^{2}$, $5^{2}$, $10^{2}$, $14650^{2}$, $14750^{2}$, $39250^{2}$, $117530^{10}$, $169100^{2}$, $249875^{2}$, $697825^{2}$, $1113600^{2}$, $2778100^{10}$
\item $(h,k)=(6,8)$ : $1^{2}$, $15^{2}$, $115^{10}$, $225^{2}$, $950^{2}$, $2425^{2}$, $15100^{2}$, $202525^{2}$, $219800^{2}$, $685675^{2}$, $1970550^{2}$, $13679375^{2}$
\item $(h,k)=(6,9)$ : $1^{2}$, $3^{10}$, $316^{50}$, $12525^{2}$, $19475^{2}$, $1077000^{2}$, $7377250^{2}$, $8283050^{2}$
\item $(h,k)=(6,11)$ : $1^{32}$, $5^{10}$, $7^{50}$, $10^{20}$, $15^{80}$, $25^{10}$, $45^{10}$, $50^{10}$, $85^{10}$, $106^{50}$, $130^{10}$, $140^{10}$, $144^{50}$, $205^{10}$, $217^{50}$, $235^{10}$, $290^{20}$, $320^{20}$, $390^{50}$, $480^{10}$, $960^{10}$, $1405^{10}$, $1470^{10}$, $1575^{2}$, $1970^{10}$, $2460^{10}$, $3129^{50}$, $3950^{2}$, $4320^{10}$, $7150^{2}$, $7455^{10}$, $13425^{2}$, $23360^{50}$, $24364^{50}$, $31850^{10}$, $36025^{10}$, $42950^{10}$, $45630^{10}$, $53190^{10}$, $55745^{10}$, $60940^{20}$, $66435^{10}$, $67852^{50}$, $68000^{2}$, $73688^{50}$, $104265^{10}$, $149990^{10}$, $218110^{10}$, $315005^{10}$, $431325^{2}$, $480925^{10}$, $539270^{10}$
\item $(h,k)=(6,12)$ : $1^{2}$, $5^{2}$, $10^{2}$, $25^{2}$, $45^{10}$, $75^{2}$, $2525^{2}$, $4225^{2}$, $212710^{10}$, $361375^{2}$, $428350^{2}$, $2457200^{2}$, $4399925^{2}$, $8059725^{2}$
\item $(h,k)=(6,13)$ : $1^{2}$, $15^{2}$, $25^{2}$, $50^{2}$, $365^{10}$, $2625^{2}$, $441950^{2}$, $5203925^{2}$, $11126800^{2}$
\item $(h,k)=(6,14)$ : $1^{2}$, $3^{10}$, $20^{10}$, $60^{10}$, $100^{2}$, $200^{2}$, $625^{2}$, $16775875^{2}$
\item $(h,k)=(6,16)$ : $1^{32}$, $5^{60}$, $10^{20}$, $15^{10}$, $25^{2}$, $35^{10}$, $40^{10}$, $50^{10}$, $320^{10}$, $505^{10}$, $645^{10}$, $685^{10}$, $715^{10}$, $880^{10}$, $900^{2}$, $1018^{50}$, $1795^{10}$, $2425^{50}$, $2655^{10}$, $5200^{10}$, $5625^{10}$, $7216^{50}$, $9910^{10}$, $11880^{10}$, $14771^{50}$, $16290^{10}$, $27600^{2}$, $30004^{50}$, $30300^{10}$, $34637^{50}$, $39627^{50}$, $41439^{50}$, $73660^{10}$, $97540^{10}$, $122370^{10}$, $172345^{10}$, $176120^{10}$, $300415^{10}$, $316775^{10}$, $318040^{10}$, $361910^{10}$, $467280^{10}$
\item $(h,k)=(6,17)$ : $1^{2}$, $5^{2}$, $10^{2}$, $150^{2}$, $1450^{2}$, $48050^{2}$, $112975^{2}$, $530100^{2}$, $561475^{2}$, $624040^{10}$, $2480560^{10}$
\item $(h,k)=(6,18)$ : $1^{2}$, $15^{2}$, $118800^{2}$, $1248675^{2}$, $15409725^{2}$
\item $(h,k)=(6,19)$ : $1^{2}$, $3^{11184810}$
\item $(h,k)=(6,21)$ : $1^{32}$, $5^{10}$, $8^{50}$, $10^{50}$, $15^{10}$, $25^{10}$, $30^{10}$, $40^{10}$, $75^{10}$, $160^{10}$, $220^{10}$, $232^{50}$, $390^{10}$, $625^{2}$, $775^{10}$, $1095^{10}$, $1110^{10}$, $2025^{2}$, $2045^{10}$, $3385^{10}$, $4066^{50}$, $5105^{10}$, $5510^{10}$, $8945^{10}$, $11940^{10}$, $13173^{50}$, $17553^{50}$, $20440^{10}$, $47635^{10}$, $52465^{10}$, $83060^{10}$, $106935^{10}$, $108610^{10}$, $124735^{10}$, $159485^{10}$, $197110^{10}$, $359400^{10}$, $386365^{10}$, $420965^{10}$, $430850^{10}$, $532870^{10}$, $539525^{2}$
\item $(h,k)=(6,22)$ : $1^{2}$, $5^{2}$, $10^{2}$, $575^{2}$, $3450^{2}$, $121725^{2}$, $143225^{2}$, $3889125^{2}$, $12619100^{2}$
\item $(h,k)=(6,23)$ : $1^{2}$, $15^{12}$, $65^{10}$, $400^{2}$, $1400^{2}$, $1925^{2}$, $3200^{2}$, $3925^{2}$, $37925^{50}$, $473425^{2}$, $15344400^{2}$
\item $(h,k)=(6,24)$ : $1^{2}$, $3^{10}$, $30^{10}$, $55^{10}$, $4975^{2}$, $6150^{2}$, $10350^{2}$, $367083^{50}$, $3231775^{2}$, $4346450^{2}$
\item $(h,k)=(7,8)$ : $1^{2}$, $3^{10}$, $50^{2}$, $27180^{10}$, $47000^{2}$, $109925^{2}$, $770225^{2}$, $15714100^{2}$
\item $(h,k)=(7,9)$ : $1^{2}$, $15^{2}$, $25^{2}$, $28600^{2}$, $34790^{10}$, $40405^{10}$, $83575^{2}$, $945050^{2}$, $5051750^{2}$, $10292225^{2}$
\item $(h,k)=(7,11)$ : $1^{2}$, $5^{2}$, $10^{2}$, $25^{2}$, $240^{10}$, $450^{2}$, $650^{2}$, $750^{2}$, $3075^{2}$, $253500^{2}$, $721200^{10}$, $12911550^{2}$
\item $(h,k)=(7,12)$ : $1^{32}$, $5^{20}$, $10^{20}$, $15^{10}$, $20^{10}$, $35^{10}$, $40^{10}$, $45^{10}$, $50^{20}$, $55^{10}$, $140^{10}$, $168^{50}$, $190^{10}$, $355^{10}$, $375^{10}$, $420^{10}$, $610^{10}$, $655^{10}$, $830^{10}$, $1025^{2}$, $1210^{10}$, $1265^{10}$, $1678^{50}$, $2552^{50}$, $2884^{50}$, $3335^{10}$, $4780^{10}$, $5050^{2}$, $6160^{10}$, $9810^{10}$, $13945^{10}$, $15135^{10}$, $16975^{10}$, $26500^{10}$, $45400^{10}$, $46405^{10}$, $46750^{10}$, $67665^{10}$, $73950^{10}$, $95570^{10}$, $100265^{10}$, $140210^{10}$, $203040^{10}$, $228150^{2}$, $307900^{2}$, $315150^{10}$, $347355^{10}$, $348965^{10}$, $378415^{10}$, $444735^{10}$, $453700^{10}$
\item $(h,k)=(7,13)$ : $1^{2}$, $3^{10}$, $2175^{2}$, $29650^{2}$, $16745375^{2}$
\item $(h,k)=(7,14)$ : $1^{2}$, $15^{2}$, $25^{4}$, $50^{2}$, $80^{10}$, $300^{2}$, $450^{2}$, $1450^{2}$, $1950^{2}$, $6500^{2}$, $7250^{2}$, $47495^{10}$, $285425^{2}$, $346925^{2}$, $476125^{2}$, $861975^{2}$, $3575225^{2}$, $10975650^{2}$
\item $(h,k)=(7,16)$ : $1^{2}$, $5^{2}$, $10^{2}$, $75^{2}$, $95^{10}$, $200^{2}$, $275^{2}$, $2525^{2}$, $3525^{2}$, $3950^{2}$, $12300^{2}$, $94050^{2}$, $16659825^{2}$
\item $(h,k)=(7,17)$ : $1^{32}$, $3^{50}$, $5^{10}$, $6^{50}$, $15^{10}$, $20^{30}$, $25^{2}$, $29^{50}$, $35^{20}$, $85^{10}$, $90^{10}$, $95^{10}$, $100^{10}$, $120^{50}$, $195^{10}$, $320^{10}$, $490^{50}$, $510^{10}$, $900^{10}$, $1465^{20}$, $1600^{2}$, $2695^{10}$, $2888^{50}$, $5697^{50}$, $13714^{50}$, $14810^{10}$, $20795^{10}$, $45490^{10}$, $49885^{10}$, $55331^{50}$, $82705^{10}$, $101025^{10}$, $120375^{2}$, $134605^{10}$, $135720^{10}$, $167295^{10}$, $177890^{10}$, $247955^{10}$, $264910^{10}$, $292320^{10}$, $327870^{10}$, $347925^{2}$, $360175^{10}$, $438545^{10}$
\item $(h,k)=(7,18)$ : $1^{2}$, $3^{11184810}$
\item $(h,k)=(7,19)$ : $1^{2}$, $15^{2}$, $118800^{2}$, $1248675^{2}$, $15409725^{2}$
\item $(h,k)=(7,21)$ : $1^{2}$, $5^{2}$, $10^{2}$, $100^{2}$, $1243900^{2}$, $2040525^{2}$, $3538300^{2}$, $9954375^{2}$
\item $(h,k)=(7,22)$ : $1^{32}$, $5^{60}$, $10^{50}$, $15^{10}$, $20^{10}$, $25^{10}$, $65^{10}$, $105^{10}$, $230^{10}$, $325^{2}$, $360^{10}$, $415^{10}$, $470^{10}$, $654^{50}$, $750^{10}$, $845^{10}$, $2325^{10}$, $3075^{10}$, $3225^{10}$, $3925^{10}$, $4000^{10}$, $4790^{10}$, $5100^{2}$, $5400^{2}$, $5891^{50}$, $6215^{10}$, $7137^{50}$, $8030^{10}$, $9466^{50}$, $10475^{10}$, $15620^{10}$, $21310^{10}$, $25335^{10}$, $34555^{10}$, $37485^{10}$, $49220^{10}$, $49235^{10}$, $57407^{50}$, $66335^{10}$, $98634^{50}$, $196765^{10}$, $359735^{10}$, $444620^{10}$, $468105^{10}$, $491250^{2}$, $541320^{10}$
\item $(h,k)=(7,23)$ : $1^{2}$, $3^{10}$, $50^{2}$, $3750^{2}$, $23700^{2}$, $45350^{2}$, $122931^{50}$, $13631075^{2}$
\item $(h,k)=(7,24)$ : $1^{2}$, $10^{10}$, $15^{2}$, $100^{2}$, $575^{2}$, $1125^{2}$, $2275^{2}$, $716225^{2}$, $1144025^{2}$, $7008625^{2}$, $7904200^{2}$
\item $(h,k)=(8,9)$ : $1^{2}$, $5^{2}$, $10^{2}$, $55^{10}$, $625^{2}$, $2975^{2}$, $5042^{50}$, $6475^{2}$, $8875^{2}$, $12500^{10}$, $36325^{2}$, $807700^{2}$, $2180850^{2}$, $3036750^{2}$, $10507800^{2}$
\item $(h,k)=(8,11)$ : $1^{2}$, $15^{2}$, $150^{2}$, $2575^{2}$, $36775^{2}$, $75290^{10}$, $337875^{2}$, $719975^{2}$, $15303400^{2}$
\item $(h,k)=(8,12)$ : $1^{2}$, $3^{10}$, $5^{10}$, $25^{4}$, $8275^{2}$, $12025^{2}$, $49375^{2}$, $169600^{2}$, $647275^{2}$, $3178115^{10}$
\item $(h,k)=(8,13)$ : $1^{32}$, $5^{20}$, $15^{10}$, $20^{10}$, $25^{10}$, $30^{10}$, $31^{50}$, $65^{50}$, $85^{10}$, $200^{2}$, $210^{10}$, $215^{10}$, $239^{50}$, $295^{10}$, $350^{10}$, $580^{10}$, $1080^{10}$, $1735^{50}$, $2070^{10}$, $2325^{10}$, $3585^{10}$, $4535^{10}$, $5780^{10}$, $5985^{10}$, $11421^{50}$, $11430^{10}$, $14130^{10}$, $21022^{50}$, $32945^{10}$, $33225^{2}$, $54178^{50}$, $68360^{10}$, $110625^{2}$, $146515^{10}$, $200315^{10}$, $213290^{10}$, $214950^{10}$, $249100^{10}$, $267260^{10}$, $392775^{2}$, $394045^{10}$, $470660^{10}$, $494425^{10}$
\item $(h,k)=(8,14)$ : $1^{2}$, $5^{2}$, $10^{2}$, $50^{2}$, $4000^{2}$, $7450^{2}$, $219525^{2}$, $515650^{2}$, $1931150^{2}$, $14099375^{2}$
\item $(h,k)=(8,16)$ : $1^{2}$, $15^{2}$, $55^{10}$, $100^{2}$, $775^{2}$, $850^{2}$, $2755^{10}$, $9250^{2}$, $14730^{10}$, $35775^{2}$, $142000^{2}$, $197850^{2}$, $565225^{2}$, $690325^{2}$, $1343375^{2}$, $1713550^{2}$, $11990425^{2}$
\item $(h,k)=(8,17)$ : $1^{2}$, $3^{11184810}$
\item $(h,k)=(8,18)$ : $1^{32}$, $3^{50}$, $5^{10}$, $6^{50}$, $15^{10}$, $20^{30}$, $25^{2}$, $29^{50}$, $35^{20}$, $85^{10}$, $90^{10}$, $95^{10}$, $100^{10}$, $120^{50}$, $195^{10}$, $320^{10}$, $490^{50}$, $510^{10}$, $900^{10}$, $1465^{20}$, $1600^{2}$, $2695^{10}$, $2888^{50}$, $5697^{50}$, $13714^{50}$, $14810^{10}$, $20795^{10}$, $45490^{10}$, $49885^{10}$, $55331^{50}$, $82705^{10}$, $101025^{10}$, $120375^{2}$, $134605^{10}$, $135720^{10}$, $167295^{10}$, $177890^{10}$, $247955^{10}$, $264910^{10}$, $292320^{10}$, $327870^{10}$, $347925^{2}$, $360175^{10}$, $438545^{10}$
\item $(h,k)=(8,19)$ : $1^{2}$, $5^{2}$, $10^{2}$, $150^{2}$, $1450^{2}$, $48050^{2}$, $112975^{2}$, $530100^{2}$, $561475^{2}$, $624040^{10}$, $2480560^{10}$
\item $(h,k)=(8,21)$ : $1^{2}$, $15^{2}$, $85^{10}$, $175^{6}$, $1400^{2}$, $1550^{2}$, $5650^{2}$, $16275^{2}$, $7783525^{2}$, $8967850^{2}$
\item $(h,k)=(8,22)$ : $1^{2}$, $3^{210}$, $25^{2}$, $100^{2}$, $110^{50}$, $600^{2}$, $5550^{2}$, $12250^{2}$, $165102^{50}$, $783325^{2}$, $11844750^{2}$
\item $(h,k)=(8,23)$ : $1^{32}$, $5^{40}$, $6^{50}$, $7^{50}$, $15^{10}$, $40^{10}$, $65^{10}$, $150^{2}$, $200^{10}$, $235^{20}$, $250^{2}$, $475^{2}$, $675^{10}$, $889^{50}$, $935^{10}$, $1146^{50}$, $1400^{50}$, $1800^{10}$, $3536^{50}$, $3655^{10}$, $4220^{50}$, $5815^{10}$, $16895^{10}$, $19789^{50}$, $47500^{10}$, $65250^{50}$, $81815^{10}$, $86770^{10}$, $127060^{10}$, $136220^{10}$, $182485^{10}$, $231685^{10}$, $259975^{10}$, $269800^{10}$, $270130^{10}$, $518460^{10}$, $523295^{10}$, $541350^{2}$
\item $(h,k)=(8,24)$ : $1^{2}$, $5^{2}$, $10^{2}$, $235^{10}$, $6025^{2}$, $16770000^{2}$
\item $(h,k)=(9,11)$ : $1^{2}$, $3^{10}$, $25^{2}$, $125^{2}$, $300^{2}$, $15600^{2}$, $377950^{2}$, $1205755^{10}$, $10354425^{2}$
\item $(h,k)=(9,12)$ : $1^{2}$, $15^{2}$, $250^{2}$, $615^{10}$, $1075^{2}$, $25045^{10}$, $140675^{2}$, $145150^{2}$, $245765^{10}$, $325275^{2}$, $14807650^{2}$
\item $(h,k)=(9,13)$ : $1^{2}$, $5^{2}$, $10^{2}$, $25^{10}$, $50^{2}$, $2875^{2}$, $7425^{2}$, $16766725^{2}$
\item $(h,k)=(9,14)$ : $1^{32}$, $5^{30}$, $10^{30}$, $30^{10}$, $75^{10}$, $115^{20}$, $265^{10}$, $285^{10}$, $485^{10}$, $555^{10}$, $1027^{50}$, $1325^{2}$, $2920^{50}$, $2973^{50}$, $3150^{2}$, $5200^{2}$, $7390^{10}$, $7799^{50}$, $10022^{50}$, $15364^{50}$, $20755^{10}$, $32590^{10}$, $34201^{50}$, $40425^{10}$, $45550^{2}$, $60433^{50}$, $198735^{10}$, $235725^{10}$, $253860^{10}$, $260460^{10}$, $265860^{10}$, $286915^{10}$, $339675^{2}$, $456645^{10}$, $541435^{10}$
\item $(h,k)=(9,16)$ : $1^{2}$, $3^{11184810}$
\item $(h,k)=(9,17)$ : $1^{2}$, $15^{2}$, $55^{10}$, $100^{2}$, $775^{2}$, $850^{2}$, $2755^{10}$, $9250^{2}$, $14730^{10}$, $35775^{2}$, $142000^{2}$, $197850^{2}$, $565225^{2}$, $690325^{2}$, $1343375^{2}$, $1713550^{2}$, $11990425^{2}$
\item $(h,k)=(9,18)$ : $1^{2}$, $5^{2}$, $10^{2}$, $75^{2}$, $95^{10}$, $200^{2}$, $275^{2}$, $2525^{2}$, $3525^{2}$, $3950^{2}$, $12300^{2}$, $94050^{2}$, $16659825^{2}$
\item $(h,k)=(9,19)$ : $1^{32}$, $5^{60}$, $10^{20}$, $15^{10}$, $25^{2}$, $35^{10}$, $40^{10}$, $50^{10}$, $320^{10}$, $505^{10}$, $645^{10}$, $685^{10}$, $715^{10}$, $880^{10}$, $900^{2}$, $1018^{50}$, $1795^{10}$, $2425^{50}$, $2655^{10}$, $5200^{10}$, $5625^{10}$, $7216^{50}$, $9910^{10}$, $11880^{10}$, $14771^{50}$, $16290^{10}$, $27600^{2}$, $30004^{50}$, $30300^{10}$, $34637^{50}$, $39627^{50}$, $41439^{50}$, $73660^{10}$, $97540^{10}$, $122370^{10}$, $172345^{10}$, $176120^{10}$, $300415^{10}$, $316775^{10}$, $318040^{10}$, $361910^{10}$, $467280^{10}$
\item $(h,k)=(9,21)$ : $1^{2}$, $3^{10}$, $75^{2}$, $125^{4}$, $12950^{2}$, $27775^{10}$, $758225^{2}$, $4202225^{2}$, $11664600^{2}$
\item $(h,k)=(9,22)$ : $1^{2}$, $15^{2}$, $2600^{2}$, $76575^{2}$, $667921^{50}$
\item $(h,k)=(9,23)$ : $1^{2}$, $5^{2}$, $10^{2}$, $15^{10}$, $37950^{2}$, $87450^{2}$, $94200^{2}$, $208650^{2}$, $406975^{2}$, $15941900^{2}$
\item $(h,k)=(9,24)$ : $1^{32}$, $10^{10}$, $21^{50}$, $25^{2}$, $45^{20}$, $50^{10}$, $107^{50}$, $130^{10}$, $245^{10}$, $345^{10}$, $542^{50}$, $790^{10}$, $1330^{50}$, $1360^{10}$, $1825^{2}$, $2665^{10}$, $3280^{10}$, $3880^{10}$, $9067^{50}$, $10525^{10}$, $12635^{10}$, $14235^{10}$, $22240^{10}$, $23330^{10}$, $51230^{10}$, $53880^{10}$, $74055^{10}$, $80125^{10}$, $95945^{10}$, $124885^{10}$, $163865^{10}$, $205895^{10}$, $217810^{10}$, $239495^{10}$, $247930^{10}$, $280535^{10}$, $302350^{10}$, $420930^{10}$, $536920^{10}$, $540375^{2}$
\item $(h,k)=(11,12)$ : $1^{2}$, $5^{2}$, $10^{2}$, $25^{2}$, $109^{50}$, $275^{2}$, $300^{2}$, $1175^{2}$, $777080^{10}$, $1014375^{2}$, $11872925^{2}$
\item $(h,k)=(11,13)$ : $1^{2}$, $15^{2}$, $20^{10}$, $75^{2}$, $130^{10}$, $280^{10}$, $675^{2}$, $78470^{10}$, $89375^{2}$, $163600^{2}$, $16128975^{2}$
\item $(h,k)=(11,14)$ : $1^{2}$, $3^{11184810}$
\item $(h,k)=(11,16)$ : $1^{32}$, $5^{30}$, $10^{30}$, $30^{10}$, $75^{10}$, $115^{20}$, $265^{10}$, $285^{10}$, $485^{10}$, $555^{10}$, $1027^{50}$, $1325^{2}$, $2920^{50}$, $2973^{50}$, $3150^{2}$, $5200^{2}$, $7390^{10}$, $7799^{50}$, $10022^{50}$, $15364^{50}$, $20755^{10}$, $32590^{10}$, $34201^{50}$, $40425^{10}$, $45550^{2}$, $60433^{50}$, $198735^{10}$, $235725^{10}$, $253860^{10}$, $260460^{10}$, $265860^{10}$, $286915^{10}$, $339675^{2}$, $456645^{10}$, $541435^{10}$
\item $(h,k)=(11,17)$ : $1^{2}$, $5^{2}$, $10^{2}$, $50^{2}$, $4000^{2}$, $7450^{2}$, $219525^{2}$, $515650^{2}$, $1931150^{2}$, $14099375^{2}$
\item $(h,k)=(11,18)$ : $1^{2}$, $15^{2}$, $25^{4}$, $50^{2}$, $80^{10}$, $300^{2}$, $450^{2}$, $1450^{2}$, $1950^{2}$, $6500^{2}$, $7250^{2}$, $47495^{10}$, $285425^{2}$, $346925^{2}$, $476125^{2}$, $861975^{2}$, $3575225^{2}$, $10975650^{2}$
\item $(h,k)=(11,19)$ : $1^{2}$, $3^{10}$, $20^{10}$, $60^{10}$, $100^{2}$, $200^{2}$, $625^{2}$, $16775875^{2}$
\item $(h,k)=(11,21)$ : $1^{32}$, $3^{50}$, $5^{30}$, $10^{10}$, $15^{60}$, $25^{20}$, $55^{10}$, $200^{50}$, $305^{10}$, $345^{10}$, $435^{10}$, $611^{50}$, $1680^{10}$, $2130^{10}$, $3150^{2}$, $3575^{10}$, $3865^{10}$, $4410^{10}$, $7765^{10}$, $14165^{10}$, $19150^{10}$, $68850^{10}$, $87135^{10}$, $108297^{50}$, $141750^{2}$, $155750^{2}$, $192990^{10}$, $241550^{2}$, $260740^{10}$, $454385^{10}$, $513885^{10}$, $527185^{10}$, $538230^{10}$
\item $(h,k)=(11,22)$ : $1^{2}$, $5^{2}$, $10^{2}$, $150^{2}$, $195^{10}$, $250^{2}$, $1075^{2}$, $1400^{2}$, $1925^{2}$, $5050^{2}$, $5725^{2}$, $8150^{2}$, $16752500^{2}$
\item $(h,k)=(11,23)$ : $1^{2}$, $15^{2}$, $75^{2}$, $300^{2}$, $525^{2}$, $11950^{2}$, $27250^{2}$, $130550^{2}$, $503000^{2}$, $1377300^{2}$, $2945250^{10}$
\item $(h,k)=(11,24)$ : $1^{2}$, $3^{10}$, $32280^{10}$, $659725^{2}$, $5772600^{2}$, $10183475^{2}$
\item $(h,k)=(12,13)$ : $1^{2}$, $3^{11184810}$
\item $(h,k)=(12,14)$ : $1^{2}$, $15^{2}$, $20^{10}$, $75^{2}$, $130^{10}$, $280^{10}$, $675^{2}$, $78470^{10}$, $89375^{2}$, $163600^{2}$, $16128975^{2}$
\item $(h,k)=(12,16)$ : $1^{2}$, $5^{2}$, $10^{2}$, $25^{10}$, $50^{2}$, $2875^{2}$, $7425^{2}$, $16766725^{2}$
\item $(h,k)=(12,17)$ : $1^{32}$, $5^{20}$, $15^{10}$, $20^{10}$, $25^{10}$, $30^{10}$, $31^{50}$, $65^{50}$, $85^{10}$, $200^{2}$, $210^{10}$, $215^{10}$, $239^{50}$, $295^{10}$, $350^{10}$, $580^{10}$, $1080^{10}$, $1735^{50}$, $2070^{10}$, $2325^{10}$, $3585^{10}$, $4535^{10}$, $5780^{10}$, $5985^{10}$, $11421^{50}$, $11430^{10}$, $14130^{10}$, $21022^{50}$, $32945^{10}$, $33225^{2}$, $54178^{50}$, $68360^{10}$, $110625^{2}$, $146515^{10}$, $200315^{10}$, $213290^{10}$, $214950^{10}$, $249100^{10}$, $267260^{10}$, $392775^{2}$, $394045^{10}$, $470660^{10}$, $494425^{10}$
\item $(h,k)=(12,18)$ : $1^{2}$, $3^{10}$, $2175^{2}$, $29650^{2}$, $16745375^{2}$
\item $(h,k)=(12,19)$ : $1^{2}$, $15^{2}$, $25^{2}$, $50^{2}$, $365^{10}$, $2625^{2}$, $441950^{2}$, $5203925^{2}$, $11126800^{2}$
\item $(h,k)=(12,21)$ : $1^{2}$, $5^{2}$, $10^{2}$, $50^{2}$, $500^{2}$, $550^{2}$, $5275^{2}$, $137685^{10}$, $143450^{2}$, $15938950^{2}$
\item $(h,k)=(12,22)$ : $1^{32}$, $4^{50}$, $5^{60}$, $10^{10}$, $16^{50}$, $20^{10}$, $22^{50}$, $65^{10}$, $70^{10}$, $100^{10}$, $130^{10}$, $235^{10}$, $269^{50}$, $305^{10}$, $400^{2}$, $713^{50}$, $840^{10}$, $865^{10}$, $1310^{10}$, $2030^{10}$, $2050^{10}$, $2357^{50}$, $2805^{10}$, $2820^{10}$, $2915^{10}$, $3015^{10}$, $3930^{10}$, $5110^{10}$, $5472^{50}$, $6515^{10}$, $8390^{10}$, $15150^{10}$, $24745^{50}$, $27525^{10}$, $29730^{10}$, $40055^{10}$, $64025^{10}$, $66215^{10}$, $105805^{10}$, $125085^{10}$, $196150^{10}$, $198025^{10}$, $204200^{2}$, $222170^{10}$, $256310^{10}$, $299745^{10}$, $309790^{10}$, $314370^{10}$, $333980^{10}$, $498840^{10}$
\item $(h,k)=(12,23)$ : $1^{2}$, $3^{10}$, $17050^{2}$, $20025^{2}$, $30575^{2}$, $82675^{2}$, $153450^{2}$, $232650^{2}$, $338840^{50}$, $7769775^{2}$
\item $(h,k)=(12,24)$ : $1^{2}$, $15^{2}$, $60^{10}$, $120^{150}$, $900^{1202}$, $1800^{18030}$
\item $(h,k)=(13,14)$ : $1^{2}$, $5^{2}$, $10^{2}$, $25^{2}$, $109^{50}$, $275^{2}$, $300^{2}$, $1175^{2}$, $777080^{10}$, $1014375^{2}$, $11872925^{2}$
\item $(h,k)=(13,16)$ : $1^{2}$, $15^{2}$, $250^{2}$, $615^{10}$, $1075^{2}$, $25045^{10}$, $140675^{2}$, $145150^{2}$, $245765^{10}$, $325275^{2}$, $14807650^{2}$
\item $(h,k)=(13,17)$ : $1^{2}$, $3^{10}$, $5^{10}$, $25^{4}$, $8275^{2}$, $12025^{2}$, $49375^{2}$, $169600^{2}$, $647275^{2}$, $3178115^{10}$
\item $(h,k)=(13,18)$ : $1^{32}$, $5^{20}$, $10^{20}$, $15^{10}$, $20^{10}$, $35^{10}$, $40^{10}$, $45^{10}$, $50^{20}$, $55^{10}$, $140^{10}$, $168^{50}$, $190^{10}$, $355^{10}$, $375^{10}$, $420^{10}$, $610^{10}$, $655^{10}$, $830^{10}$, $1025^{2}$, $1210^{10}$, $1265^{10}$, $1678^{50}$, $2552^{50}$, $2884^{50}$, $3335^{10}$, $4780^{10}$, $5050^{2}$, $6160^{10}$, $9810^{10}$, $13945^{10}$, $15135^{10}$, $16975^{10}$, $26500^{10}$, $45400^{10}$, $46405^{10}$, $46750^{10}$, $67665^{10}$, $73950^{10}$, $95570^{10}$, $100265^{10}$, $140210^{10}$, $203040^{10}$, $228150^{2}$, $307900^{2}$, $315150^{10}$, $347355^{10}$, $348965^{10}$, $378415^{10}$, $444735^{10}$, $453700^{10}$
\item $(h,k)=(13,19)$ : $1^{2}$, $5^{2}$, $10^{2}$, $25^{2}$, $45^{10}$, $75^{2}$, $2525^{2}$, $4225^{2}$, $212710^{10}$, $361375^{2}$, $428350^{2}$, $2457200^{2}$, $4399925^{2}$, $8059725^{2}$
\item $(h,k)=(13,21)$ : $1^{2}$, $15^{2}$, $60^{10}$, $6675^{2}$, $51900^{2}$, $118800^{2}$, $16599525^{2}$
\item $(h,k)=(13,22)$ : $1^{2}$, $3^{10}$, $5^{10}$, $25^{2}$, $50^{2}$, $150^{2}$, $3165^{10}$, $162850^{2}$, $16598275^{2}$
\item $(h,k)=(13,23)$ : $1^{32}$, $5^{10}$, $15^{10}$, $40^{10}$, $60^{10}$, $100^{2}$, $125^{10}$, $210^{10}$, $225^{10}$, $245^{20}$, $295^{10}$, $325^{2}$, $752^{50}$, $1615^{10}$, $2035^{10}$, $2070^{10}$, $2490^{10}$, $2515^{10}$, $2670^{10}$, $2825^{2}$, $4102^{50}$, $5640^{10}$, $6072^{50}$, $8484^{50}$, $8515^{10}$, $10335^{10}$, $10670^{10}$, $10900^{2}$, $12240^{10}$, $16580^{10}$, $21840^{10}$, $33028^{50}$, $46000^{2}$, $51040^{50}$, $51540^{10}$, $53740^{10}$, $56056^{50}$, $73480^{10}$, $87965^{10}$, $118630^{10}$, $165345^{10}$, $189585^{10}$, $191615^{10}$, $265470^{10}$, $307420^{10}$, $404705^{10}$, $439150^{10}$, $482075^{2}$
\item $(h,k)=(13,24)$ : $1^{2}$, $5^{2}$, $10^{2}$, $75^{2}$, $200^{2}$, $3425^{2}$, $13435^{50}$, $28780^{10}$, $324425^{2}$, $338400^{2}$, $364850^{2}$, $2206205^{10}$, $4235025^{2}$
\item $(h,k)=(14,16)$ : $1^{2}$, $3^{10}$, $25^{2}$, $125^{2}$, $300^{2}$, $15600^{2}$, $377950^{2}$, $1205755^{10}$, $10354425^{2}$
\item $(h,k)=(14,17)$ : $1^{2}$, $15^{2}$, $150^{2}$, $2575^{2}$, $36775^{2}$, $75290^{10}$, $337875^{2}$, $719975^{2}$, $15303400^{2}$
\item $(h,k)=(14,18)$ : $1^{2}$, $5^{2}$, $10^{2}$, $25^{2}$, $240^{10}$, $450^{2}$, $650^{2}$, $750^{2}$, $3075^{2}$, $253500^{2}$, $721200^{10}$, $12911550^{2}$
\item $(h,k)=(14,19)$ : $1^{32}$, $5^{10}$, $7^{50}$, $10^{20}$, $15^{80}$, $25^{10}$, $45^{10}$, $50^{10}$, $85^{10}$, $106^{50}$, $130^{10}$, $140^{10}$, $144^{50}$, $205^{10}$, $217^{50}$, $235^{10}$, $290^{20}$, $320^{20}$, $390^{50}$, $480^{10}$, $960^{10}$, $1405^{10}$, $1470^{10}$, $1575^{2}$, $1970^{10}$, $2460^{10}$, $3129^{50}$, $3950^{2}$, $4320^{10}$, $7150^{2}$, $7455^{10}$, $13425^{2}$, $23360^{50}$, $24364^{50}$, $31850^{10}$, $36025^{10}$, $42950^{10}$, $45630^{10}$, $53190^{10}$, $55745^{10}$, $60940^{20}$, $66435^{10}$, $67852^{50}$, $68000^{2}$, $73688^{50}$, $104265^{10}$, $149990^{10}$, $218110^{10}$, $315005^{10}$, $431325^{2}$, $480925^{10}$, $539270^{10}$
\item $(h,k)=(14,21)$ : $1^{2}$, $3^{10}$, $25^{12}$, $125^{2}$, $1000^{2}$, $3355185^{10}$
\item $(h,k)=(14,22)$ : $1^{2}$, $15^{2}$, $200^{10}$, $850^{2}$, $2720^{10}$, $12175^{2}$, $16749575^{2}$
\item $(h,k)=(14,23)$ : $1^{2}$, $5^{2}$, $10^{2}$, $350^{2}$, $1150^{2}$, $1825^{2}$, $2315^{10}$, $61325^{2}$, $135725^{2}$, $16565250^{2}$
\item $(h,k)=(14,24)$ : $1^{32}$, $5^{10}$, $10^{50}$, $15^{20}$, $20^{10}$, $45^{50}$, $60^{10}$, $61^{50}$, $70^{10}$, $85^{10}$, $135^{10}$, $150^{2}$, $175^{12}$, $200^{2}$, $320^{10}$, $335^{20}$, $340^{10}$, $372^{50}$, $385^{10}$, $550^{10}$, $895^{10}$, $1530^{10}$, $1750^{2}$, $2390^{10}$, $3945^{10}$, $4060^{50}$, $4795^{10}$, $5295^{10}$, $5320^{10}$, $7720^{10}$, $15270^{10}$, $18050^{10}$, $22895^{10}$, $28211^{50}$, $35355^{10}$, $37135^{10}$, $44706^{50}$, $46315^{10}$, $62720^{10}$, $69070^{10}$, $71617^{50}$, $87339^{50}$, $117665^{10}$, $144945^{10}$, $146530^{10}$, $378370^{10}$, $464895^{10}$, $470940^{10}$, $539950^{2}$
\item $(h,k)=(16,17)$ : $1^{2}$, $5^{2}$, $10^{2}$, $55^{10}$, $625^{2}$, $2975^{2}$, $5042^{50}$, $6475^{2}$, $8875^{2}$, $12500^{10}$, $36325^{2}$, $807700^{2}$, $2180850^{2}$, $3036750^{2}$, $10507800^{2}$
\item $(h,k)=(16,18)$ : $1^{2}$, $15^{2}$, $25^{2}$, $28600^{2}$, $34790^{10}$, $40405^{10}$, $83575^{2}$, $945050^{2}$, $5051750^{2}$, $10292225^{2}$
\item $(h,k)=(16,19)$ : $1^{2}$, $3^{10}$, $316^{50}$, $12525^{2}$, $19475^{2}$, $1077000^{2}$, $7377250^{2}$, $8283050^{2}$
\item $(h,k)=(16,21)$ : $1^{32}$, $5^{90}$, $15^{10}$, $30^{10}$, $48^{50}$, $50^{10}$, $70^{10}$, $100^{10}$, $270^{10}$, $305^{20}$, $390^{10}$, $930^{10}$, $995^{10}$, $2190^{10}$, $2195^{10}$, $3125^{2}$, $3179^{50}$, $7105^{10}$, $8760^{10}$, $8930^{10}$, $9695^{10}$, $12500^{2}$, $25220^{10}$, $39870^{10}$, $45521^{50}$, $62840^{10}$, $102890^{10}$, $115685^{10}$, $154225^{2}$, $166310^{10}$, $212520^{10}$, $237180^{10}$, $296270^{10}$, $334280^{10}$, $372250^{2}$, $410560^{10}$, $425780^{10}$, $531495^{10}$
\item $(h,k)=(16,22)$ : $1^{2}$, $5^{2}$, $10^{2}$, $38^{50}$, $75^{10}$, $29975^{2}$, $86330^{10}$, $236975^{2}$, $453773^{50}$, $1366800^{2}$, $3366150^{2}$
\item $(h,k)=(16,23)$ : $1^{2}$, $15^{2}$, $125^{2}$, $275^{2}$, $325^{2}$, $350625^{2}$, $7793525^{2}$, $8632325^{2}$
\item $(h,k)=(16,24)$ : $1^{2}$, $3^{10}$, $164^{50}$, $5050^{2}$, $2161525^{2}$, $14606525^{2}$
\item $(h,k)=(17,18)$ : $1^{2}$, $3^{10}$, $50^{2}$, $27180^{10}$, $47000^{2}$, $109925^{2}$, $770225^{2}$, $15714100^{2}$
\item $(h,k)=(17,19)$ : $1^{2}$, $15^{2}$, $115^{10}$, $225^{2}$, $950^{2}$, $2425^{2}$, $15100^{2}$, $202525^{2}$, $219800^{2}$, $685675^{2}$, $1970550^{2}$, $13679375^{2}$
\item $(h,k)=(17,21)$ : $1^{2}$, $5^{2}$, $10^{2}$, $25^{2}$, $100^{4}$, $125^{2}$, $275^{2}$, $425^{2}$, $428^{50}$, $600^{2}$, $2050^{2}$, $2865^{10}$, $34702^{50}$, $68925^{2}$, $72350^{2}$, $120325^{10}$, $2522425^{2}$, $2866750^{2}$, $9748850^{2}$
\item $(h,k)=(17,22)$ : $1^{32}$, $2^{300}$, $5^{30}$, $8^{50}$, $9^{50}$, $10^{20}$, $20^{10}$, $22^{50}$, $45^{10}$, $50^{12}$, $65^{10}$, $70^{10}$, $96^{50}$, $175^{2}$, $305^{10}$, $320^{10}$, $475^{10}$, $570^{10}$, $605^{10}$, $1000^{10}$, $3015^{10}$, $6565^{10}$, $7160^{10}$, $17918^{50}$, $32475^{2}$, $38335^{10}$, $100540^{50}$, $106668^{50}$, $448305^{10}$, $509425^{2}$, $533240^{10}$, $539735^{10}$, $540735^{10}$
\item $(h,k)=(17,23)$ : $1^{2}$, $3^{10}$, $750^{2}$, $7625^{10}$, $27650^{2}$, $1867875^{2}$, $14842800^{2}$
\item $(h,k)=(17,24)$ : $1^{2}$, $5^{10}$, $15^{2}$, $225^{2}$, $275^{2}$, $475^{2}$, $7160^{10}$, $53225^{2}$, $16687175^{2}$
\item $(h,k)=(18,19)$ : $1^{2}$, $5^{2}$, $10^{2}$, $14650^{2}$, $14750^{2}$, $39250^{2}$, $117530^{10}$, $169100^{2}$, $249875^{2}$, $697825^{2}$, $1113600^{2}$, $2778100^{10}$
\item $(h,k)=(18,21)$ : $1^{2}$, $5^{10}$, $15^{2}$, $50^{2}$, $100^{2}$, $7350^{2}$, $9650^{2}$, $25675^{2}$, $29775^{2}$, $362100^{2}$, $501715^{10}$, $2587275^{2}$, $3798325^{2}$, $7448300^{2}$
\item $(h,k)=(18,22)$ : $1^{2}$, $3^{10}$, $10^{10}$, $4350^{2}$, $48800^{2}$, $63935^{10}$, $261725^{2}$, $1982200^{2}$, $2829925^{2}$, $11330475^{2}$
\item $(h,k)=(18,23)$ : $1^{32}$, $5^{20}$, $8^{50}$, $10^{10}$, $15^{10}$, $30^{10}$, $65^{10}$, $160^{10}$, $234^{50}$, $235^{10}$, $285^{10}$, $455^{10}$, $860^{10}$, $955^{10}$, $1107^{50}$, $1155^{10}$, $1455^{10}$, $1475^{2}$, $5390^{10}$, $6375^{2}$, $8770^{10}$, $12690^{50}$, $18555^{10}$, $19800^{10}$, $22070^{10}$, $23015^{10}$, $25675^{2}$, $32977^{50}$, $59185^{10}$, $67190^{10}$, $85650^{2}$, $89325^{10}$, $90025^{10}$, $170725^{10}$, $262405^{10}$, $341180^{10}$, $364390^{10}$, $423025^{2}$, $462200^{10}$, $468080^{10}$, $533925^{10}$
\item $(h,k)=(18,24)$ : $1^{2}$, $5^{2}$, $10^{2}$, $475^{2}$, $900^{4}$, $1250^{2}$, $10675^{2}$, $121025^{2}$, $592350^{2}$, $2183650^{2}$, $13865975^{2}$
\item $(h,k)=(19,21)$ : $1^{2}$, $3^{10}$, $75^{2}$, $250^{10}$, $1750^{2}$, $45700^{2}$, $2243375^{10}$, $5511550^{2}$
\item $(h,k)=(19,22)$ : $1^{2}$, $15^{2}$, $50^{2}$, $125^{2}$, $850^{2}$, $1150^{2}$, $1475^{2}$, $2400^{2}$, $4450^{2}$, $6050^{2}$, $8175^{2}$, $22670^{50}$, $137225^{2}$, $319425^{2}$, $375375^{2}$, $464075^{2}$, $832250^{2}$, $1263650^{2}$, $3262500^{2}$, $9531225^{2}$
\item $(h,k)=(19,23)$ : $1^{2}$, $5^{2}$, $10^{2}$, $25^{10}$, $35^{10}$, $100^{2}$, $3000^{2}$, $17133^{50}$, $111925^{2}$, $173250^{2}$, $16060300^{2}$
\item $(h,k)=(19,24)$ : $1^{32}$, $10^{10}$, $15^{20}$, $20^{30}$, $25^{10}$, $35^{60}$, $40^{10}$, $50^{2}$, $80^{50}$, $90^{10}$, $105^{10}$, $160^{10}$, $425^{2}$, $565^{10}$, $655^{10}$, $670^{10}$, $900^{2}$, $2155^{10}$, $2198^{50}$, $2615^{10}$, $4950^{10}$, $6267^{50}$, $6816^{50}$, $14100^{10}$, $15835^{10}$, $25305^{10}$, $28709^{50}$, $64405^{10}$, $78320^{10}$, $81700^{10}$, $86005^{10}$, $110825^{2}$, $158760^{10}$, $160755^{10}$, $165780^{10}$, $201585^{10}$, $213535^{10}$, $249870^{10}$, $295630^{10}$, $376710^{10}$, $427820^{10}$, $484195^{10}$
\item $(h,k)=(21,22)$ : $1^{2}$, $5^{2}$, $10^{2}$, $100^{2}$, $175^{2}$, $650^{2}$, $1250^{2}$, $1355^{10}$, $1444^{50}$, $8010^{10}$, $311375^{2}$, $1085650^{2}$, $3059015^{10}$
\item $(h,k)=(21,23)$ : $1^{2}$, $15^{2}$, $50^{2}$, $135^{10}$, $300^{2}$, $525^{2}$, $1190^{10}$, $1325^{2}$, $2725^{2}$, $2840^{10}$, $26350^{2}$, $84625^{2}$, $191475^{2}$, $216575^{2}$, $326475^{2}$, $662000^{2}$, $675975^{2}$, $14567975^{2}$
\item $(h,k)=(21,24)$ : $1^{2}$, $3^{10}$, $8915^{10}$, $9550^{2}$, $15675^{2}$, $262925^{2}$, $784845^{10}$, $12520250^{2}$
\item $(h,k)=(22,23)$ : $1^{2}$, $3^{10}$, $5^{50}$, $50^{2}$, $1450^{2}$, $4025^{2}$, $9700^{2}$, $37275^{2}$, $79375^{2}$, $1489850^{2}$, $2798975^{2}$, $12356375^{2}$
\item $(h,k)=(22,24)$ : $1^{2}$, $15^{12}$, $50^{2}$, $5900^{2}$, $8625^{2}$, $26975^{2}$, $36850^{2}$, $2051535^{10}$, $6441050^{2}$
\item $(h,k)=(23,24)$ : $1^{2}$, $5^{2}$, $10^{2}$, $15^{560}$, $17^{950}$, $19^{350}$, $20^{130}$, $25^{6}$, $26^{100}$, $30^{400}$, $35^{30}$, $37^{50}$, $39^{50}$, $41^{50}$, $44^{800}$, $45^{590}$, $46^{650}$, $50^{866}$, $52^{400}$, $54^{100}$, $57^{300}$, $59^{200}$, $60^{590}$, $63^{200}$, $65^{130}$, $66^{350}$, $70^{320}$, $71^{150}$, $72^{50}$, $74^{500}$, $75^{180}$, $76^{200}$, $78^{100}$, $80^{380}$, $81^{200}$, $82^{200}$, $84^{200}$, $90^{700}$, $91^{200}$, $93^{50}$, $97^{300}$, $98^{200}$, $100^{118}$, $105^{40}$, $106^{400}$, $108^{250}$, $110^{290}$, $112^{200}$, $115^{10}$, $117^{500}$, $118^{50}$, $120^{480}$, $124^{150}$, $125^{224}$, $127^{100}$, $130^{510}$, $140^{440}$, $145^{160}$, $149^{150}$, $150^{196}$, $155^{40}$, $160^{100}$, $164^{50}$, $166^{100}$, $170^{20}$, $173^{100}$, $175^{86}$, $176^{100}$, $179^{100}$, $180^{310}$, $182^{100}$, $184^{150}$, $185^{10}$, $188^{150}$, $190^{120}$, $195^{50}$, $200^{340}$, $205^{50}$, $210^{170}$, $215^{100}$, $220^{320}$, $225^{380}$, $228^{50}$, $230^{80}$, $236^{150}$, $238^{50}$, $240^{130}$, $245^{290}$, $246^{50}$, $250^{270}$, $255^{160}$, $256^{50}$, $260^{40}$, $265^{50}$, $270^{150}$, $273^{50}$, $274^{50}$, $275^{308}$, $280^{30}$, $288^{100}$, $290^{60}$, $295^{10}$, $296^{50}$, $299^{100}$, $300^{360}$, $315^{80}$, $319^{50}$, $320^{150}$, $325^{40}$, $330^{160}$, $335^{50}$, $340^{20}$, $345^{110}$, $350^{808}$, $353^{50}$, $354^{50}$, $358^{50}$, $360^{100}$, $362^{50}$, $365^{20}$, $369^{50}$, $370^{170}$, $375^{58}$, $380^{60}$, $385^{60}$, $387^{50}$, $390^{90}$, $399^{50}$, $400^{260}$, $405^{40}$, $410^{20}$, $415^{80}$, $420^{10}$, $440^{20}$, $445^{40}$, $450^{398}$, $460^{20}$, $465^{40}$, $470^{90}$, $475^{100}$, $480^{20}$, $482^{50}$, $490^{70}$, $500^{326}$, $510^{60}$, $515^{10}$, $520^{30}$, $525^{66}$, $535^{40}$, $540^{20}$, $544^{50}$, $545^{70}$, $550^{220}$, $555^{40}$, $559^{50}$, $560^{110}$, $565^{60}$, $570^{80}$, $575^{214}$, $580^{80}$, $598^{50}$, $600^{264}$, $610^{40}$, $615^{10}$, $625^{32}$, $630^{40}$, $645^{20}$, $650^{182}$, $655^{70}$, $670^{30}$, $675^{132}$, $680^{20}$, $690^{40}$, $700^{100}$, $710^{20}$, $720^{20}$, $725^{150}$, $730^{20}$, $750^{228}$, $760^{20}$, $770^{30}$, $775^{112}$, $790^{40}$, $795^{20}$, $800^{140}$, $805^{10}$, $820^{30}$, $825^{50}$, $830^{60}$, $840^{20}$, $850^{106}$, $860^{20}$, $865^{10}$, $870^{50}$, $875^{66}$, $880^{20}$, $885^{40}$, $890^{50}$, $900^{146}$, $920^{50}$, $925^{42}$, $935^{10}$, $945^{30}$, $950^{206}$, $955^{30}$, $960^{40}$, $965^{20}$, $975^{30}$, $980^{10}$, $985^{20}$, $1000^{136}$, $1010^{20}$, $1020^{50}$, $1025^{62}$, $1035^{20}$, $1045^{20}$, $1050^{166}$, $1075^{4}$, $1080^{10}$, $1085^{10}$, $1100^{110}$, $1125^{110}$, $1140^{40}$, $1150^{168}$, $1165^{10}$, $1170^{10}$, $1175^{40}$, $1180^{20}$, $1185^{20}$, $1200^{50}$, $1210^{10}$, $1215^{10}$, $1220^{20}$, $1225^{64}$, $1250^{116}$, $1260^{10}$, $1270^{20}$, $1275^{46}$, $1290^{10}$, $1295^{30}$, $1300^{120}$, $1305^{30}$, $1310^{20}$, $1325^{86}$, $1340^{30}$, $1350^{128}$, $1370^{10}$, $1375^{42}$, $1380^{10}$, $1400^{152}$, $1425^{16}$, $1430^{10}$, $1445^{10}$, $1450^{108}$, $1460^{20}$, $1465^{10}$, $1470^{20}$, $1475^{18}$, $1480^{10}$, $1500^{58}$, $1525^{10}$, $1545^{10}$, $1550^{116}$, $1555^{10}$, $1560^{10}$, $1570^{10}$, $1575^{36}$, $1580^{30}$, $1585^{10}$, $1600^{60}$, $1605^{10}$, $1615^{20}$, $1625^{68}$, $1650^{44}$, $1660^{10}$, $1670^{30}$, $1675^{12}$, $1700^{98}$, $1725^{32}$, $1750^{52}$, $1760^{10}$, $1765^{10}$, $1775^{26}$, $1790^{20}$, $1800^{60}$, $1810^{10}$, $1815^{10}$, $1825^{32}$, $1830^{10}$, $1835^{10}$, $1850^{94}$, $1875^{24}$, $1880^{20}$, $1895^{10}$, $1900^{64}$, $1905^{10}$, $1910^{10}$, $1915^{10}$, $1925^{6}$, $1945^{10}$, $1950^{90}$, $1955^{10}$, $1975^{12}$, $2000^{46}$, $2025^{26}$, $2030^{10}$, $2040^{10}$, $2050^{130}$, $2075^{20}$, $2100^{94}$, $2125^{14}$, $2140^{10}$, $2150^{52}$, $2175^{36}$, $2200^{80}$, $2225^{18}$, $2230^{10}$, $2250^{26}$, $2275^{18}$, $2300^{32}$, $2305^{10}$, $2310^{10}$, $2325^{24}$, $2340^{10}$, $2350^{56}$, $2375^{20}$, $2395^{10}$, $2400^{62}$, $2410^{10}$, $2420^{10}$, $2450^{58}$, $2475^{38}$, $2500^{26}$, $2525^{26}$, $2550^{20}$, $2575^{32}$, $2585^{10}$, $2600^{54}$, $2625^{10}$, $2650^{30}$, $2675^{10}$, $2680^{10}$, $2690^{10}$, $2700^{30}$, $2750^{22}$, $2775^{8}$, $2800^{48}$, $2825^{16}$, $2850^{56}$, $2870^{10}$, $2875^{6}$, $2900^{20}$, $2925^{28}$, $2950^{56}$, $2965^{10}$, $2975^{12}$, $3000^{60}$, $3025^{14}$, $3050^{14}$, $3075^{26}$, $3100^{46}$, $3125^{8}$, $3150^{46}$, $3175^{10}$, $3200^{20}$, $3225^{14}$, $3230^{10}$, $3250^{32}$, $3275^{6}$, $3300^{20}$, $3325^{10}$, $3345^{10}$, $3350^{38}$, $3375^{18}$, $3400^{22}$, $3425^{16}$, $3450^{30}$, $3475^{4}$, $3500^{24}$, $3525^{10}$, $3550^{28}$, $3575^{14}$, $3600^{16}$, $3625^{6}$, $3650^{30}$, $3675^{6}$, $3700^{32}$, $3725^{28}$, $3750^{22}$, $3775^{4}$, $3800^{18}$, $3825^{6}$, $3850^{30}$, $3875^{4}$, $3900^{30}$, $3950^{22}$, $3975^{10}$, $4000^{20}$, $4025^{12}$, $4050^{44}$, $4075^{6}$, $4100^{18}$, $4125^{16}$, $4150^{30}$, $4175^{8}$, $4200^{28}$, $4250^{20}$, $4275^{8}$, $4300^{16}$, $4325^{6}$, $4350^{16}$, $4375^{4}$, $4400^{16}$, $4425^{4}$, $4450^{22}$, $4475^{6}$, $4500^{22}$, $4525^{6}$, $4550^{20}$, $4575^{4}$, $4600^{12}$, $4625^{4}$, $4650^{22}$, $4675^{2}$, $4700^{4}$, $4750^{22}$, $4800^{22}$, $4825^{4}$, $4850^{22}$, $4875^{8}$, $4900^{12}$, $4925^{8}$, $4950^{16}$, $4975^{10}$, $5025^{8}$, $5050^{14}$, $5075^{4}$, $5100^{12}$, $5125^{8}$, $5150^{12}$, $5175^{8}$, $5200^{10}$, $5225^{10}$, $5250^{26}$, $5275^{2}$, $5300^{22}$, $5325^{18}$, $5350^{12}$, $5375^{6}$, $5400^{8}$, $5450^{8}$, $5475^{8}$, $5500^{14}$, $5550^{20}$, $5575^{4}$, $5600^{10}$, $5625^{12}$, $5650^{16}$, $5675^{12}$, $5700^{32}$, $5725^{10}$, $5750^{16}$, $5775^{14}$, $5800^{8}$, $5850^{12}$, $5875^{4}$, $5900^{12}$, $5925^{6}$, $5950^{24}$, $5975^{6}$, $6000^{14}$, $6025^{6}$, $6050^{4}$, $6075^{6}$, $6100^{10}$, $6125^{10}$, $6150^{10}$, $6175^{4}$, $6200^{14}$, $6225^{8}$, $6250^{4}$, $6275^{6}$, $6300^{8}$, $6325^{8}$, $6350^{6}$, $6375^{8}$, $6400^{18}$, $6450^{8}$, $6500^{8}$, $6525^{4}$, $6550^{20}$, $6625^{6}$, $6650^{6}$, $6700^{12}$, $6725^{8}$, $6750^{4}$, $6800^{4}$, $6825^{10}$, $6850^{14}$, $6875^{4}$, $6900^{2}$, $6925^{4}$, $6950^{18}$, $6975^{8}$, $7000^{22}$, $7025^{6}$, $7050^{2}$, $7075^{4}$, $7100^{6}$, $7125^{2}$, $7150^{2}$, $7200^{8}$, $7225^{4}$, $7250^{6}$, $7275^{4}$, $7300^{6}$, $7325^{4}$, $7350^{6}$, $7375^{6}$, $7400^{4}$, $7425^{4}$, $7450^{2}$, $7475^{2}$, $7500^{4}$, $7525^{2}$, $7550^{10}$, $7575^{6}$, $7600^{8}$, $7625^{4}$, $7650^{4}$, $7675^{6}$, $7700^{8}$, $7725^{4}$, $7750^{8}$, $7775^{8}$, $7800^{2}$, $7825^{4}$, $7850^{2}$, $7875^{4}$, $7900^{2}$, $7925^{2}$, $7950^{6}$, $8000^{4}$, $8025^{2}$, $8050^{12}$, $8075^{2}$, $8100^{2}$, $8125^{4}$, $8150^{4}$, $8175^{10}$, $8200^{10}$, $8225^{4}$, $8250^{4}$, $8275^{4}$, $8300^{4}$, $8350^{6}$, $8375^{4}$, $8425^{6}$, $8450^{6}$, $8475^{4}$, $8500^{8}$, $8525^{8}$, $8550^{4}$, $8575^{4}$, $8600^{6}$, $8650^{4}$, $8675^{8}$, $8700^{6}$, $8725^{4}$, $8750^{10}$, $8775^{2}$, $8800^{2}$, $8875^{4}$, $8900^{2}$, $8925^{4}$, $8950^{2}$, $8975^{6}$, $9000^{2}$, $9025^{4}$, $9050^{4}$, $9075^{6}$, $9100^{18}$, $9150^{4}$, $9200^{6}$, $9225^{4}$, $9275^{2}$, $9300^{12}$, $9325^{4}$, $9350^{2}$, $9375^{6}$, $9400^{2}$, $9425^{2}$, $9450^{2}$, $9475^{4}$, $9525^{2}$, $9550^{2}$, $9650^{2}$, $9675^{2}$, $9725^{4}$, $9750^{2}$, $9775^{4}$, $9825^{2}$, $9850^{8}$, $9875^{8}$, $9950^{6}$, $9975^{2}$, $10000^{6}$, $10100^{2}$, $10125^{4}$, $10150^{2}$, $10175^{2}$, $10200^{4}$, $10225^{2}$, $10250^{2}$, $10275^{10}$, $10300^{2}$, $10350^{4}$, $10375^{2}$, $10400^{4}$, $10425^{2}$, $10450^{2}$, $10500^{2}$, $10525^{2}$, $10550^{2}$, $10575^{2}$, $10650^{2}$, $10675^{2}$, $10700^{6}$, $10725^{4}$, $10800^{2}$, $10825^{2}$, $10850^{4}$, $10900^{4}$, $10925^{2}$, $11000^{4}$, $11025^{2}$, $11050^{2}$, $11075^{2}$, $11125^{2}$, $11200^{4}$, $11225^{4}$, $11250^{2}$, $11325^{2}$, $11400^{4}$, $11500^{2}$, $11525^{2}$, $11550^{6}$, $11625^{2}$, $11650^{6}$, $11675^{2}$, $11700^{2}$, $11750^{4}$, $11800^{2}$, $11825^{2}$, $11925^{2}$, $11950^{2}$, $12000^{2}$, $12025^{4}$, $12050^{6}$, $12100^{4}$, $12150^{2}$, $12175^{2}$, $12200^{2}$, $12225^{4}$, $12250^{2}$, $12400^{4}$, $12450^{6}$, $12525^{2}$, $12575^{4}$, $12700^{6}$, $12750^{2}$, $12800^{4}$, $12850^{2}$, $12900^{2}$, $12950^{2}$, $13000^{4}$, $13050^{2}$, $13075^{2}$, $13150^{6}$, $13175^{2}$, $13200^{6}$, $13225^{2}$, $13275^{2}$, $13325^{2}$, $13350^{2}$, $13375^{2}$, $13450^{2}$, $13525^{2}$, $13625^{2}$, $13650^{2}$, $13900^{2}$, $13925^{2}$, $14025^{2}$, $14050^{2}$, $14175^{2}$, $14225^{2}$, $14250^{2}$, $14450^{2}$, $14650^{2}$, $14700^{2}$, $14725^{4}$, $14750^{2}$, $14800^{2}$, $14925^{2}$, $14950^{2}$, $15050^{2}$, $15125^{2}$, $15250^{4}$, $15400^{2}$, $15600^{2}$, $15650^{2}$, $15825^{2}$, $15875^{2}$, $16225^{2}$, $16325^{2}$, $16350^{2}$, $16375^{2}$, $16500^{2}$, $16550^{2}$, $16575^{2}$, $16650^{2}$, $16900^{2}$, $17075^{2}$, $17600^{2}$, $17650^{2}$, $17700^{2}$, $18400^{2}$, $18725^{2}$, $19150^{2}$, $19275^{2}$, $19600^{2}$, $20300^{2}$, $20625^{2}$, $20875^{2}$, $22000^{2}$, $22675^{2}$
\end{itemize}



\end{document}

