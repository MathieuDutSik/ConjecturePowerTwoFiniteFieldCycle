\documentclass[12pt]{article}
\usepackage{amsfonts, amsmath, latexsym, epic, eepic, pifont}
\usepackage{epsf}
\title{Sequence lists}
\usepackage{vmargin}
\setpapersize{custom}{21cm}{29.7cm}
\setmarginsrb{0.7cm}{2cm}{0.7cm}{3.5cm}{0pt}{0pt}{0pt}{0pt}
%marge gauche, marge haut, marge droite, marge bas.


\begin{document}
\newcommand{\RR}{\ensuremath{\mathbb{R}}}
\newcommand{\NN}{\ensuremath{\mathbb{N}}}
\newcommand{\QQ}{\ensuremath{\mathbb{Q}}}
\newcommand{\CC}{\ensuremath{\mathbb{C}}}
\newcommand{\ZZ}{\ensuremath{\mathbb{Z}}}
\newcommand{\TT}{\ensuremath{\mathbb{T}}}
\newcommand{\FF}{\ensuremath{\mathbb{F}}}
\newtheorem{prop}{Proposition}
\newtheorem{theorem}{Theorem}
\newtheorem{cor}{Corollary}
\newtheorem{lem}{Lemma}
\newtheorem{conjecture}{Conjecture}
\newtheorem{claim}{Claim}
\newtheorem{remark}{Remark}
\newtheorem{definition}{Definition}
\newtheorem{proposition}{Proposition}
\newcommand{\qed}{\hfill $\Box$ }
\newcommand{\proof}{\noindent{\bf Proof.}\ \ }


\maketitle


\begin{abstract}
We consider here some sequences arising from diagram geometry.
\end{abstract}


\section{Definition and basic properties}
Consider the Galois field $\FF_{2^n}$ with $2^n$ elements.
If $h$ is prime with $n$, then the function
\begin{equation*}
\begin{array}{rcl}
\phi_k:\FF_{2^n} &\rightarrow &\FF_{2^n}\\
x&\mapsto x^{2^h-1}
\end{array}
\end{equation*}
is bijective.

As a consequence the function
\begin{equation*}
\begin{array}{rcl}
\psi_{h,k}:\FF_{2^n} &\rightarrow &\FF_{2^n}\\
x&\mapsto (1+x^{\frac{2^h-1}{2^k-1}})^{\frac{2^k-1}{2^h-1}})-1
\end{array}
\end{equation*}
is bijective.

Of interest is to study the cycle structure of $\psi_{h,k}$ for $h$ and $k$ being primes with $n$, $1\leq h,k \leq n-1$.
This structure show up in the works of Antonio Pasini and Yoshiara Satoshi.





\section{Results for $n=5$}
\begin{itemize}
\item $(h,k)=(1,2)$ : $1^{2}$, $5^{2}$, $10^{2}$
\item $(h,k)=(1,3)$ : $1^{2}$, $15^{2}$
\item $(h,k)=(1,4)$ : $1^{2}$, $3^{10}$
\item $(h,k)=(2,3)$ : $1^{2}$, $3^{10}$
\item $(h,k)=(2,4)$ : $1^{2}$, $15^{2}$
\item $(h,k)=(3,4)$ : $1^{2}$, $5^{2}$, $10^{2}$
\end{itemize}

\section{Results for $n=6$}
\begin{itemize}
\item $(h,k)=(1,5)$ : $1^{4}$, $3^{20}$
\end{itemize}

\section{Results for $n=7$}
\begin{itemize}
\item $(h,k)=(1,2)$ : $1^{2}$, $7^{2}$, $14^{2}$, $21^{4}$
\item $(h,k)=(1,3)$ : $1^{2}$, $63^{2}$
\item $(h,k)=(1,4)$ : $1^{2}$, $63^{2}$
\item $(h,k)=(1,5)$ : $1^{2}$, $63^{2}$
\item $(h,k)=(1,6)$ : $1^{2}$, $3^{42}$
\item $(h,k)=(2,3)$ : $1^{2}$, $63^{2}$
\item $(h,k)=(2,4)$ : $1^{2}$, $14^{2}$, $49^{2}$
\item $(h,k)=(2,5)$ : $1^{2}$, $3^{42}$
\item $(h,k)=(2,6)$ : $1^{2}$, $63^{2}$
\item $(h,k)=(3,4)$ : $1^{2}$, $3^{42}$
\item $(h,k)=(3,5)$ : $1^{2}$, $14^{2}$, $49^{2}$
\item $(h,k)=(3,6)$ : $1^{2}$, $63^{2}$
\item $(h,k)=(4,5)$ : $1^{2}$, $63^{2}$
\item $(h,k)=(4,6)$ : $1^{2}$, $63^{2}$
\item $(h,k)=(5,6)$ : $1^{2}$, $7^{2}$, $14^{2}$, $21^{4}$
\end{itemize}

\section{Results for $n=8$}
\begin{itemize}
\item $(h,k)=(1,3)$ : $1^{4}$, $3^{4}$, $9^{16}$, $24^{4}$
\item $(h,k)=(1,5)$ : $1^{16}$, $4^{20}$, $7^{8}$, $8^{2}$, $11^{8}$
\item $(h,k)=(1,7)$ : $1^{4}$, $3^{84}$
\item $(h,k)=(3,5)$ : $1^{4}$, $3^{84}$
\item $(h,k)=(3,7)$ : $1^{16}$, $4^{20}$, $7^{8}$, $8^{2}$, $11^{8}$
\item $(h,k)=(5,7)$ : $1^{4}$, $3^{4}$, $9^{16}$, $24^{4}$
\end{itemize}

\section{Results for $n=9$}
\begin{itemize}
\item $(h,k)=(1,2)$ : $1^{2}$, $3^{2}$, $12^{6}$, $18^{2}$, $27^{2}$, $36^{2}$, $45^{6}$
\item $(h,k)=(1,4)$ : $1^{8}$, $3^{6}$, $9^{2}$, $10^{18}$, $36^{8}$
\item $(h,k)=(1,5)$ : $1^{2}$, $3^{2}$, $36^{2}$, $72^{6}$
\item $(h,k)=(1,7)$ : $1^{8}$, $3^{6}$, $6^{24}$, $9^{8}$, $12^{6}$, $15^{6}$, $18^{6}$
\item $(h,k)=(1,8)$ : $1^{2}$, $3^{170}$
\item $(h,k)=(2,4)$ : $1^{2}$, $3^{2}$, $60^{6}$, $72^{2}$
\item $(h,k)=(2,5)$ : $1^{8}$, $3^{6}$, $33^{6}$, $36^{8}$
\item $(h,k)=(2,7)$ : $1^{2}$, $3^{170}$
\item $(h,k)=(2,8)$ : $1^{8}$, $3^{6}$, $6^{24}$, $9^{8}$, $12^{6}$, $15^{6}$, $18^{6}$
\item $(h,k)=(4,5)$ : $1^{2}$, $3^{170}$
\item $(h,k)=(4,7)$ : $1^{8}$, $3^{6}$, $33^{6}$, $36^{8}$
\item $(h,k)=(4,8)$ : $1^{2}$, $3^{2}$, $36^{2}$, $72^{6}$
\item $(h,k)=(5,7)$ : $1^{2}$, $3^{2}$, $60^{6}$, $72^{2}$
\item $(h,k)=(5,8)$ : $1^{8}$, $3^{6}$, $9^{2}$, $10^{18}$, $36^{8}$
\item $(h,k)=(7,8)$ : $1^{2}$, $3^{2}$, $12^{6}$, $18^{2}$, $27^{2}$, $36^{2}$, $45^{6}$
\end{itemize}

\section{Results for $n=10$}
\begin{itemize}
\item $(h,k)=(1,3)$ : $1^{4}$, $7^{10}$, $10^{4}$, $12^{10}$, $14^{10}$, $15^{2}$, $18^{20}$, $65^{4}$
\item $(h,k)=(1,7)$ : $1^{4}$, $3^{10}$, $5^{2}$, $10^{2}$, $12^{20}$, $24^{20}$, $30^{2}$, $45^{4}$
\item $(h,k)=(1,9)$ : $1^{4}$, $3^{340}$
\item $(h,k)=(3,7)$ : $1^{4}$, $3^{340}$
\item $(h,k)=(3,9)$ : $1^{4}$, $3^{10}$, $5^{2}$, $10^{2}$, $12^{20}$, $24^{20}$, $30^{2}$, $45^{4}$
\item $(h,k)=(7,9)$ : $1^{4}$, $7^{10}$, $10^{4}$, $12^{10}$, $14^{10}$, $15^{2}$, $18^{20}$, $65^{4}$
\end{itemize}

\section{Results for $n=11$}
\begin{itemize}
\item $(h,k)=(1,2)$ : $1^{2}$, $11^{2}$, $22^{2}$, $33^{4}$, $44^{8}$, $55^{6}$, $66^{6}$, $77^{2}$, $88^{2}$, $99^{2}$, $121^{2}$
\item $(h,k)=(1,3)$ : $1^{2}$, $11^{4}$, $22^{2}$, $110^{2}$, $869^{2}$
\item $(h,k)=(1,4)$ : $1^{2}$, $121^{2}$, $902^{2}$
\item $(h,k)=(1,5)$ : $1^{2}$, $55^{2}$, $308^{2}$, $660^{2}$
\item $(h,k)=(1,6)$ : $1^{2}$, $11^{2}$, $44^{2}$, $88^{22}$
\item $(h,k)=(1,7)$ : $1^{2}$, $1023^{2}$
\item $(h,k)=(1,8)$ : $1^{2}$, $55^{2}$, $77^{4}$, $165^{2}$, $649^{2}$
\item $(h,k)=(1,9)$ : $1^{2}$, $132^{4}$, $198^{4}$, $363^{2}$
\item $(h,k)=(1,10)$ : $1^{2}$, $3^{682}$
\item $(h,k)=(2,3)$ : $1^{2}$, $1023^{2}$
\item $(h,k)=(2,4)$ : $1^{2}$, $11^{2}$, $44^{2}$, $66^{2}$, $198^{2}$, $231^{2}$, $473^{2}$
\item $(h,k)=(2,5)$ : $1^{2}$, $65^{22}$, $66^{2}$, $242^{2}$
\item $(h,k)=(2,6)$ : $1^{2}$, $1023^{2}$
\item $(h,k)=(2,7)$ : $1^{2}$, $11^{2}$, $44^{2}$, $45^{22}$, $88^{2}$, $99^{2}$, $121^{2}$, $165^{2}$
\item $(h,k)=(2,8)$ : $1^{2}$, $1023^{2}$
\item $(h,k)=(2,9)$ : $1^{2}$, $3^{682}$
\item $(h,k)=(2,10)$ : $1^{2}$, $132^{4}$, $198^{4}$, $363^{2}$
\item $(h,k)=(3,4)$ : $1^{2}$, $22^{2}$, $66^{2}$, $935^{2}$
\item $(h,k)=(3,5)$ : $1^{2}$, $44^{2}$, $231^{2}$, $748^{2}$
\item $(h,k)=(3,6)$ : $1^{2}$, $11^{2}$, $68^{22}$, $264^{2}$
\item $(h,k)=(3,7)$ : $1^{2}$, $11^{2}$, $33^{2}$, $55^{2}$, $66^{2}$, $77^{2}$, $132^{2}$, $198^{2}$, $451^{2}$
\item $(h,k)=(3,8)$ : $1^{2}$, $3^{682}$
\item $(h,k)=(3,9)$ : $1^{2}$, $1023^{2}$
\item $(h,k)=(3,10)$ : $1^{2}$, $55^{2}$, $77^{4}$, $165^{2}$, $649^{2}$
\item $(h,k)=(4,5)$ : $1^{2}$, $220^{2}$, $803^{2}$
\item $(h,k)=(4,6)$ : $1^{2}$, $22^{2}$, $44^{2}$, $176^{2}$, $781^{2}$
\item $(h,k)=(4,7)$ : $1^{2}$, $3^{682}$
\item $(h,k)=(4,8)$ : $1^{2}$, $11^{2}$, $33^{2}$, $55^{2}$, $66^{2}$, $77^{2}$, $132^{2}$, $198^{2}$, $451^{2}$
\item $(h,k)=(4,9)$ : $1^{2}$, $11^{2}$, $44^{2}$, $45^{22}$, $88^{2}$, $99^{2}$, $121^{2}$, $165^{2}$
\item $(h,k)=(4,10)$ : $1^{2}$, $1023^{2}$
\item $(h,k)=(5,6)$ : $1^{2}$, $3^{682}$
\item $(h,k)=(5,7)$ : $1^{2}$, $22^{2}$, $44^{2}$, $176^{2}$, $781^{2}$
\item $(h,k)=(5,8)$ : $1^{2}$, $11^{2}$, $68^{22}$, $264^{2}$
\item $(h,k)=(5,9)$ : $1^{2}$, $1023^{2}$
\item $(h,k)=(5,10)$ : $1^{2}$, $11^{2}$, $44^{2}$, $88^{22}$
\item $(h,k)=(6,7)$ : $1^{2}$, $220^{2}$, $803^{2}$
\item $(h,k)=(6,8)$ : $1^{2}$, $44^{2}$, $231^{2}$, $748^{2}$
\item $(h,k)=(6,9)$ : $1^{2}$, $65^{22}$, $66^{2}$, $242^{2}$
\item $(h,k)=(6,10)$ : $1^{2}$, $55^{2}$, $308^{2}$, $660^{2}$
\item $(h,k)=(7,8)$ : $1^{2}$, $22^{2}$, $66^{2}$, $935^{2}$
\item $(h,k)=(7,9)$ : $1^{2}$, $11^{2}$, $44^{2}$, $66^{2}$, $198^{2}$, $231^{2}$, $473^{2}$
\item $(h,k)=(7,10)$ : $1^{2}$, $121^{2}$, $902^{2}$
\item $(h,k)=(8,9)$ : $1^{2}$, $1023^{2}$
\item $(h,k)=(8,10)$ : $1^{2}$, $11^{4}$, $22^{2}$, $110^{2}$, $869^{2}$
\item $(h,k)=(9,10)$ : $1^{2}$, $11^{2}$, $22^{2}$, $33^{4}$, $44^{8}$, $55^{6}$, $66^{6}$, $77^{2}$, $88^{2}$, $99^{2}$, $121^{2}$
\end{itemize}

\section{Results for $n=12$}
\begin{itemize}
\item $(h,k)=(1,5)$ : $1^{16}$, $3^{40}$, $7^{24}$, $8^{12}$, $11^{12}$, $12^{8}$, $16^{12}$, $19^{12}$, $21^{24}$, $27^{8}$, $35^{24}$, $43^{24}$, $57^{8}$
\item $(h,k)=(1,7)$ : $1^{64}$, $3^{28}$, $4^{36}$, $9^{8}$, $12^{8}$, $13^{12}$, $14^{24}$, $16^{24}$, $24^{18}$, $28^{12}$, $29^{12}$, $31^{12}$, $32^{12}$, $35^{12}$, $39^{12}$
\item $(h,k)=(1,11)$ : $1^{4}$, $3^{1364}$
\item $(h,k)=(5,7)$ : $1^{4}$, $3^{1364}$
\item $(h,k)=(5,11)$ : $1^{64}$, $3^{28}$, $4^{36}$, $9^{8}$, $12^{8}$, $13^{12}$, $14^{24}$, $16^{24}$, $24^{18}$, $28^{12}$, $29^{12}$, $31^{12}$, $32^{12}$, $35^{12}$, $39^{12}$
\item $(h,k)=(7,11)$ : $1^{16}$, $3^{40}$, $7^{24}$, $8^{12}$, $11^{12}$, $12^{8}$, $16^{12}$, $19^{12}$, $21^{24}$, $27^{8}$, $35^{24}$, $43^{24}$, $57^{8}$
\end{itemize}

\section{Results for $n=13$}
\begin{itemize}
\item $(h,k)=(1,2)$ : $1^{2}$, $13^{4}$, $16^{26}$, $39^{6}$, $52^{4}$, $65^{12}$, $78^{8}$, $91^{6}$, $104^{12}$, $117^{6}$, $130^{4}$, $143^{2}$, $156^{2}$, $169^{2}$, $182^{8}$, $234^{2}$
\item $(h,k)=(1,3)$ : $1^{2}$, $143^{2}$, $208^{2}$, $3744^{2}$
\item $(h,k)=(1,4)$ : $1^{2}$, $247^{2}$, $3848^{2}$
\item $(h,k)=(1,5)$ : $1^{2}$, $39^{2}$, $4056^{2}$
\item $(h,k)=(1,6)$ : $1^{2}$, $13^{2}$, $4082^{2}$
\item $(h,k)=(1,7)$ : $1^{2}$, $315^{26}$
\item $(h,k)=(1,8)$ : $1^{2}$, $13^{2}$, $4082^{2}$
\item $(h,k)=(1,9)$ : $1^{2}$, $91^{2}$, $182^{2}$, $3822^{2}$
\item $(h,k)=(1,10)$ : $1^{2}$, $13^{2}$, $104^{2}$, $1924^{2}$, $2054^{2}$
\item $(h,k)=(1,11)$ : $1^{2}$, $39^{2}$, $195^{6}$, $234^{8}$, $273^{2}$, $312^{6}$, $351^{2}$, $429^{2}$, $546^{2}$
\item $(h,k)=(1,12)$ : $1^{2}$, $3^{2730}$
\item $(h,k)=(2,3)$ : $1^{2}$, $26^{2}$, $283^{26}$, $390^{2}$
\item $(h,k)=(2,4)$ : $1^{2}$, $78^{2}$, $117^{2}$, $169^{2}$, $403^{2}$, $3328^{2}$
\item $(h,k)=(2,5)$ : $1^{2}$, $195^{2}$, $416^{2}$, $780^{2}$, $2704^{2}$
\item $(h,k)=(2,6)$ : $1^{2}$, $39^{2}$, $1391^{2}$, $2665^{2}$
\item $(h,k)=(2,7)$ : $1^{2}$, $52^{2}$, $4043^{2}$
\item $(h,k)=(2,8)$ : $1^{2}$, $39^{2}$, $1638^{2}$, $2418^{2}$
\item $(h,k)=(2,9)$ : $1^{2}$, $117^{2}$, $143^{2}$, $174^{26}$, $299^{2}$, $1274^{2}$
\item $(h,k)=(2,10)$ : $1^{2}$, $273^{2}$, $286^{2}$, $3536^{2}$
\item $(h,k)=(2,11)$ : $1^{2}$, $3^{2730}$
\item $(h,k)=(2,12)$ : $1^{2}$, $39^{2}$, $195^{6}$, $234^{8}$, $273^{2}$, $312^{6}$, $351^{2}$, $429^{2}$, $546^{2}$
\item $(h,k)=(3,4)$ : $1^{2}$, $13^{2}$, $39^{2}$, $4043^{2}$
\item $(h,k)=(3,5)$ : $1^{2}$, $13^{4}$, $26^{2}$, $39^{2}$, $52^{2}$, $117^{2}$, $3835^{2}$
\item $(h,k)=(3,6)$ : $1^{2}$, $13^{2}$, $169^{2}$, $3913^{2}$
\item $(h,k)=(3,7)$ : $1^{2}$, $169^{2}$, $364^{2}$, $572^{2}$, $2990^{2}$
\item $(h,k)=(3,8)$ : $1^{2}$, $13^{2}$, $39^{4}$, $169^{2}$, $598^{2}$, $3237^{2}$
\item $(h,k)=(3,9)$ : $1^{2}$, $13^{4}$, $221^{2}$, $299^{2}$, $3549^{2}$
\item $(h,k)=(3,10)$ : $1^{2}$, $3^{2730}$
\item $(h,k)=(3,11)$ : $1^{2}$, $273^{2}$, $286^{2}$, $3536^{2}$
\item $(h,k)=(3,12)$ : $1^{2}$, $13^{2}$, $104^{2}$, $1924^{2}$, $2054^{2}$
\item $(h,k)=(4,5)$ : $1^{2}$, $286^{2}$, $338^{2}$, $3471^{2}$
\item $(h,k)=(4,6)$ : $1^{2}$, $39^{2}$, $4056^{2}$
\item $(h,k)=(4,7)$ : $1^{2}$, $13^{2}$, $104^{2}$, $273^{2}$, $3705^{2}$
\item $(h,k)=(4,8)$ : $1^{2}$, $13^{4}$, $26^{2}$, $39^{2}$, $403^{2}$, $988^{2}$, $1131^{2}$, $1482^{2}$
\item $(h,k)=(4,9)$ : $1^{2}$, $3^{2730}$
\item $(h,k)=(4,10)$ : $1^{2}$, $13^{4}$, $221^{2}$, $299^{2}$, $3549^{2}$
\item $(h,k)=(4,11)$ : $1^{2}$, $117^{2}$, $143^{2}$, $174^{26}$, $299^{2}$, $1274^{2}$
\item $(h,k)=(4,12)$ : $1^{2}$, $91^{2}$, $182^{2}$, $3822^{2}$
\item $(h,k)=(5,6)$ : $1^{2}$, $923^{2}$, $3172^{2}$
\item $(h,k)=(5,7)$ : $1^{2}$, $52^{2}$, $4043^{2}$
\item $(h,k)=(5,8)$ : $1^{2}$, $3^{2730}$
\item $(h,k)=(5,9)$ : $1^{2}$, $13^{4}$, $26^{2}$, $39^{2}$, $403^{2}$, $988^{2}$, $1131^{2}$, $1482^{2}$
\item $(h,k)=(5,10)$ : $1^{2}$, $13^{2}$, $39^{4}$, $169^{2}$, $598^{2}$, $3237^{2}$
\item $(h,k)=(5,11)$ : $1^{2}$, $39^{2}$, $1638^{2}$, $2418^{2}$
\item $(h,k)=(5,12)$ : $1^{2}$, $13^{2}$, $4082^{2}$
\item $(h,k)=(6,7)$ : $1^{2}$, $3^{2730}$
\item $(h,k)=(6,8)$ : $1^{2}$, $52^{2}$, $4043^{2}$
\item $(h,k)=(6,9)$ : $1^{2}$, $13^{2}$, $104^{2}$, $273^{2}$, $3705^{2}$
\item $(h,k)=(6,10)$ : $1^{2}$, $169^{2}$, $364^{2}$, $572^{2}$, $2990^{2}$
\item $(h,k)=(6,11)$ : $1^{2}$, $52^{2}$, $4043^{2}$
\item $(h,k)=(6,12)$ : $1^{2}$, $315^{26}$
\item $(h,k)=(7,8)$ : $1^{2}$, $923^{2}$, $3172^{2}$
\item $(h,k)=(7,9)$ : $1^{2}$, $39^{2}$, $4056^{2}$
\item $(h,k)=(7,10)$ : $1^{2}$, $13^{2}$, $169^{2}$, $3913^{2}$
\item $(h,k)=(7,11)$ : $1^{2}$, $39^{2}$, $1391^{2}$, $2665^{2}$
\item $(h,k)=(7,12)$ : $1^{2}$, $13^{2}$, $4082^{2}$
\item $(h,k)=(8,9)$ : $1^{2}$, $286^{2}$, $338^{2}$, $3471^{2}$
\item $(h,k)=(8,10)$ : $1^{2}$, $13^{4}$, $26^{2}$, $39^{2}$, $52^{2}$, $117^{2}$, $3835^{2}$
\item $(h,k)=(8,11)$ : $1^{2}$, $195^{2}$, $416^{2}$, $780^{2}$, $2704^{2}$
\item $(h,k)=(8,12)$ : $1^{2}$, $39^{2}$, $4056^{2}$
\item $(h,k)=(9,10)$ : $1^{2}$, $13^{2}$, $39^{2}$, $4043^{2}$
\item $(h,k)=(9,11)$ : $1^{2}$, $78^{2}$, $117^{2}$, $169^{2}$, $403^{2}$, $3328^{2}$
\item $(h,k)=(9,12)$ : $1^{2}$, $247^{2}$, $3848^{2}$
\item $(h,k)=(10,11)$ : $1^{2}$, $26^{2}$, $283^{26}$, $390^{2}$
\item $(h,k)=(10,12)$ : $1^{2}$, $143^{2}$, $208^{2}$, $3744^{2}$
\item $(h,k)=(11,12)$ : $1^{2}$, $13^{4}$, $16^{26}$, $39^{6}$, $52^{4}$, $65^{12}$, $78^{8}$, $91^{6}$, $104^{12}$, $117^{6}$, $130^{4}$, $143^{2}$, $156^{2}$, $169^{2}$, $182^{8}$, $234^{2}$
\end{itemize}

\section{Results for $n=14$}
\begin{itemize}
\item $(h,k)=(1,3)$ : $1^{4}$, $10^{14}$, $14^{4}$, $23^{14}$, $24^{14}$, $25^{28}$, $27^{14}$, $57^{14}$, $63^{2}$, $69^{14}$, $127^{14}$, $385^{4}$, $2310^{4}$
\item $(h,k)=(1,5)$ : $1^{4}$, $6^{28}$, $14^{18}$, $15^{14}$, $17^{14}$, $21^{14}$, $52^{14}$, $56^{4}$, $57^{14}$, $58^{14}$, $60^{14}$, $63^{2}$, $77^{4}$, $89^{14}$, $427^{4}$, $2107^{4}$
\item $(h,k)=(1,9)$ : $1^{4}$, $4^{14}$, $7^{2}$, $14^{2}$, $20^{14}$, $21^{4}$, $27^{28}$, $29^{14}$, $41^{14}$, $51^{14}$, $56^{4}$, $58^{14}$, $98^{4}$, $130^{14}$, $2555^{4}$
\item $(h,k)=(1,11)$ : $1^{4}$, $3^{14}$, $5^{28}$, $10^{14}$, $14^{4}$, $23^{14}$, $40^{14}$, $57^{14}$, $60^{14}$, $63^{6}$, $77^{4}$, $112^{14}$, $238^{4}$, $679^{4}$, $1890^{4}$
\item $(h,k)=(1,13)$ : $1^{4}$, $3^{5460}$
\item $(h,k)=(3,5)$ : $1^{4}$, $3^{14}$, $7^{4}$, $9^{14}$, $10^{14}$, $12^{28}$, $14^{6}$, $17^{14}$, $21^{4}$, $35^{14}$, $49^{2}$, $52^{14}$, $235^{14}$, $2674^{4}$
\item $(h,k)=(3,9)$ : $1^{4}$, $5^{14}$, $9^{14}$, $11^{14}$, $19^{14}$, $42^{14}$, $48^{14}$, $63^{2}$, $67^{14}$, $90^{14}$, $96^{14}$, $122^{28}$, $1855^{4}$
\item $(h,k)=(3,11)$ : $1^{4}$, $3^{5460}$
\item $(h,k)=(3,13)$ : $1^{4}$, $3^{14}$, $5^{28}$, $10^{14}$, $14^{4}$, $23^{14}$, $40^{14}$, $57^{14}$, $60^{14}$, $63^{6}$, $77^{4}$, $112^{14}$, $238^{4}$, $679^{4}$, $1890^{4}$
\item $(h,k)=(5,9)$ : $1^{4}$, $3^{5460}$
\item $(h,k)=(5,11)$ : $1^{4}$, $5^{14}$, $9^{14}$, $11^{14}$, $19^{14}$, $42^{14}$, $48^{14}$, $63^{2}$, $67^{14}$, $90^{14}$, $96^{14}$, $122^{28}$, $1855^{4}$
\item $(h,k)=(5,13)$ : $1^{4}$, $4^{14}$, $7^{2}$, $14^{2}$, $20^{14}$, $21^{4}$, $27^{28}$, $29^{14}$, $41^{14}$, $51^{14}$, $56^{4}$, $58^{14}$, $98^{4}$, $130^{14}$, $2555^{4}$
\item $(h,k)=(9,11)$ : $1^{4}$, $3^{14}$, $7^{4}$, $9^{14}$, $10^{14}$, $12^{28}$, $14^{6}$, $17^{14}$, $21^{4}$, $35^{14}$, $49^{2}$, $52^{14}$, $235^{14}$, $2674^{4}$
\item $(h,k)=(9,13)$ : $1^{4}$, $6^{28}$, $14^{18}$, $15^{14}$, $17^{14}$, $21^{14}$, $52^{14}$, $56^{4}$, $57^{14}$, $58^{14}$, $60^{14}$, $63^{2}$, $77^{4}$, $89^{14}$, $427^{4}$, $2107^{4}$
\item $(h,k)=(11,13)$ : $1^{4}$, $10^{14}$, $14^{4}$, $23^{14}$, $24^{14}$, $25^{28}$, $27^{14}$, $57^{14}$, $63^{2}$, $69^{14}$, $127^{14}$, $385^{4}$, $2310^{4}$
\end{itemize}

\section{Results for $n=15$}
\begin{itemize}
\item $(h,k)=(1,2)$ : $1^{2}$, $3^{2}$, $5^{8}$, $10^{8}$, $11^{90}$, $18^{50}$, $20^{30}$, $25^{6}$, $30^{24}$, $33^{10}$, $35^{36}$, $36^{20}$, $39^{10}$, $40^{42}$, $45^{18}$, $55^{6}$, $60^{18}$, $65^{6}$, $70^{6}$, $75^{16}$, $80^{6}$, $90^{8}$, $100^{6}$, $105^{16}$, $120^{14}$, $125^{6}$, $130^{6}$, $150^{8}$, $165^{2}$, $175^{6}$, $180^{2}$, $195^{2}$, $210^{2}$, $225^{2}$, $240^{2}$, $255^{4}$, $270^{2}$, $285^{4}$, $300^{2}$, $330^{2}$, $345^{2}$, $360^{2}$, $375^{2}$, $510^{2}$, $525^{2}$, $555^{2}$
\item $(h,k)=(1,4)$ : $1^{8}$, $3^{10}$, $5^{6}$, $10^{6}$, $20^{12}$, $45^{6}$, $66^{30}$, $160^{6}$, $300^{6}$, $395^{6}$, $470^{6}$, $930^{2}$, $1070^{6}$, $1075^{6}$, $1245^{6}$
\item $(h,k)=(1,7)$ : $1^{8}$, $5^{2}$, $10^{2}$, $165^{2}$, $175^{30}$, $182^{30}$, $560^{6}$, $695^{6}$, $2360^{6}$
\item $(h,k)=(1,8)$ : $1^{2}$, $3^{2}$, $15^{2}$, $30^{6}$, $75^{14}$, $150^{210}$
\item $(h,k)=(1,11)$ : $1^{32}$, $3^{22}$, $5^{30}$, $6^{20}$, $9^{10}$, $10^{30}$, $13^{30}$, $15^{10}$, $24^{10}$, $28^{30}$, $30^{10}$, $63^{10}$, $117^{10}$, $120^{10}$, $158^{30}$, $162^{30}$, $168^{10}$, $185^{6}$, $258^{10}$, $309^{10}$, $399^{10}$, $504^{10}$
\item $(h,k)=(1,13)$ : $1^{8}$, $5^{6}$, $7^{30}$, $9^{30}$, $10^{42}$, $13^{30}$, $14^{120}$, $15^{42}$, $16^{60}$, $20^{24}$, $25^{24}$, $30^{44}$, $35^{30}$, $38^{30}$, $40^{36}$, $45^{24}$, $50^{12}$, $55^{6}$, $60^{24}$, $70^{36}$, $75^{18}$, $80^{24}$, $85^{18}$, $90^{6}$, $95^{18}$, $105^{6}$, $125^{6}$, $130^{6}$, $135^{12}$, $140^{12}$, $145^{12}$, $155^{6}$, $165^{6}$
\item $(h,k)=(1,14)$ : $1^{2}$, $3^{10922}$
\item $(h,k)=(2,4)$ : $1^{2}$, $3^{2}$, $15^{6}$, $35^{6}$, $54^{10}$, $104^{30}$, $4800^{6}$
\item $(h,k)=(2,7)$ : $1^{32}$, $3^{22}$, $6^{30}$, $9^{10}$, $15^{10}$, $21^{10}$, $23^{30}$, $28^{30}$, $36^{10}$, $42^{10}$, $51^{10}$, $60^{2}$, $104^{30}$, $117^{10}$, $147^{10}$, $171^{30}$, $255^{2}$, $303^{10}$, $471^{10}$, $486^{10}$, $510^{10}$
\item $(h,k)=(2,8)$ : $1^{8}$, $3^{10}$, $5^{12}$, $15^{2}$, $30^{2}$, $70^{6}$, $80^{30}$, $90^{2}$, $225^{6}$, $270^{10}$, $350^{6}$, $355^{30}$, $555^{2}$, $885^{6}$, $1060^{6}$
\item $(h,k)=(2,11)$ : $1^{8}$, $5^{2}$, $10^{2}$, $14^{30}$, $35^{6}$, $80^{6}$, $134^{30}$, $280^{6}$, $735^{6}$, $1295^{6}$, $2290^{6}$
\item $(h,k)=(2,13)$ : $1^{2}$, $3^{10922}$
\item $(h,k)=(2,14)$ : $1^{8}$, $5^{6}$, $7^{30}$, $9^{30}$, $10^{42}$, $13^{30}$, $14^{120}$, $15^{42}$, $16^{60}$, $20^{24}$, $25^{24}$, $30^{44}$, $35^{30}$, $38^{30}$, $40^{36}$, $45^{24}$, $50^{12}$, $55^{6}$, $60^{24}$, $70^{36}$, $75^{18}$, $80^{24}$, $85^{18}$, $90^{6}$, $95^{18}$, $105^{6}$, $125^{6}$, $130^{6}$, $135^{12}$, $140^{12}$, $145^{12}$, $155^{6}$, $165^{6}$
\item $(h,k)=(4,7)$ : $1^{8}$, $5^{6}$, $10^{6}$, $15^{34}$, $20^{6}$, $34^{30}$, $35^{12}$, $65^{6}$, $70^{6}$, $265^{6}$, $320^{6}$, $381^{10}$, $435^{6}$, $730^{6}$, $1195^{6}$, $1385^{6}$
\item $(h,k)=(4,8)$ : $1^{2}$, $3^{2}$, $5^{14}$, $10^{2}$, $15^{12}$, $16^{30}$, $153^{10}$, $215^{6}$, $520^{6}$, $1320^{6}$, $1350^{2}$, $1875^{2}$, $5850^{2}$
\item $(h,k)=(4,11)$ : $1^{2}$, $3^{10922}$
\item $(h,k)=(4,13)$ : $1^{8}$, $5^{2}$, $10^{2}$, $14^{30}$, $35^{6}$, $80^{6}$, $134^{30}$, $280^{6}$, $735^{6}$, $1295^{6}$, $2290^{6}$
\item $(h,k)=(4,14)$ : $1^{32}$, $3^{22}$, $5^{30}$, $6^{20}$, $9^{10}$, $10^{30}$, $13^{30}$, $15^{10}$, $24^{10}$, $28^{30}$, $30^{10}$, $63^{10}$, $117^{10}$, $120^{10}$, $158^{30}$, $162^{30}$, $168^{10}$, $185^{6}$, $258^{10}$, $309^{10}$, $399^{10}$, $504^{10}$
\item $(h,k)=(7,8)$ : $1^{2}$, $3^{10922}$
\item $(h,k)=(7,11)$ : $1^{2}$, $3^{2}$, $5^{14}$, $10^{2}$, $15^{12}$, $16^{30}$, $153^{10}$, $215^{6}$, $520^{6}$, $1320^{6}$, $1350^{2}$, $1875^{2}$, $5850^{2}$
\item $(h,k)=(7,13)$ : $1^{8}$, $3^{10}$, $5^{12}$, $15^{2}$, $30^{2}$, $70^{6}$, $80^{30}$, $90^{2}$, $225^{6}$, $270^{10}$, $350^{6}$, $355^{30}$, $555^{2}$, $885^{6}$, $1060^{6}$
\item $(h,k)=(7,14)$ : $1^{2}$, $3^{2}$, $15^{2}$, $30^{6}$, $75^{14}$, $150^{210}$
\item $(h,k)=(8,11)$ : $1^{8}$, $5^{6}$, $10^{6}$, $15^{34}$, $20^{6}$, $34^{30}$, $35^{12}$, $65^{6}$, $70^{6}$, $265^{6}$, $320^{6}$, $381^{10}$, $435^{6}$, $730^{6}$, $1195^{6}$, $1385^{6}$
\item $(h,k)=(8,13)$ : $1^{32}$, $3^{22}$, $6^{30}$, $9^{10}$, $15^{10}$, $21^{10}$, $23^{30}$, $28^{30}$, $36^{10}$, $42^{10}$, $51^{10}$, $60^{2}$, $104^{30}$, $117^{10}$, $147^{10}$, $171^{30}$, $255^{2}$, $303^{10}$, $471^{10}$, $486^{10}$, $510^{10}$
\item $(h,k)=(8,14)$ : $1^{8}$, $5^{2}$, $10^{2}$, $165^{2}$, $175^{30}$, $182^{30}$, $560^{6}$, $695^{6}$, $2360^{6}$
\item $(h,k)=(11,13)$ : $1^{2}$, $3^{2}$, $15^{6}$, $35^{6}$, $54^{10}$, $104^{30}$, $4800^{6}$
\item $(h,k)=(11,14)$ : $1^{8}$, $3^{10}$, $5^{6}$, $10^{6}$, $20^{12}$, $45^{6}$, $66^{30}$, $160^{6}$, $300^{6}$, $395^{6}$, $470^{6}$, $930^{2}$, $1070^{6}$, $1075^{6}$, $1245^{6}$
\item $(h,k)=(13,14)$ : $1^{2}$, $3^{2}$, $5^{8}$, $10^{8}$, $11^{90}$, $18^{50}$, $20^{30}$, $25^{6}$, $30^{24}$, $33^{10}$, $35^{36}$, $36^{20}$, $39^{10}$, $40^{42}$, $45^{18}$, $55^{6}$, $60^{18}$, $65^{6}$, $70^{6}$, $75^{16}$, $80^{6}$, $90^{8}$, $100^{6}$, $105^{16}$, $120^{14}$, $125^{6}$, $130^{6}$, $150^{8}$, $165^{2}$, $175^{6}$, $180^{2}$, $195^{2}$, $210^{2}$, $225^{2}$, $240^{2}$, $255^{4}$, $270^{2}$, $285^{4}$, $300^{2}$, $330^{2}$, $345^{2}$, $360^{2}$, $375^{2}$, $510^{2}$, $525^{2}$, $555^{2}$
\end{itemize}

\section{Results for $n=16$}
\begin{itemize}
\item $(h,k)=(1,3)$ : $1^{4}$, $3^{4}$, $9^{16}$, $24^{4}$, $32^{16}$, $34^{16}$, $39^{16}$, $40^{16}$, $47^{16}$, $77^{16}$, $83^{16}$, $125^{16}$, $165^{16}$, $216^{32}$, $220^{16}$, $269^{16}$, $270^{16}$, $287^{16}$, $394^{16}$, $548^{16}$, $578^{16}$, $3520^{2}$
\item $(h,k)=(1,5)$ : $1^{16}$, $4^{20}$, $7^{8}$, $8^{2}$, $11^{8}$, $20^{8}$, $32^{40}$, $39^{16}$, $48^{8}$, $51^{16}$, $64^{2}$, $80^{8}$, $84^{16}$, $88^{8}$, $100^{16}$, $137^{32}$, $155^{16}$, $211^{32}$, $226^{16}$, $252^{16}$, $263^{16}$, $269^{16}$, $336^{2}$, $405^{16}$, $415^{16}$, $416^{16}$, $461^{16}$
\item $(h,k)=(1,7)$ : $1^{4}$, $3^{84}$, $6^{16}$, $16^{2}$, $20^{16}$, $32^{2}$, $50^{16}$, $56^{4}$, $60^{16}$, $78^{16}$, $113^{16}$, $118^{16}$, $127^{16}$, $134^{16}$, $155^{16}$, $165^{16}$, $228^{16}$, $253^{16}$, $256^{4}$, $391^{16}$, $451^{16}$, $689^{16}$, $722^{16}$, $1888^{2}$
\item $(h,k)=(1,9)$ : $1^{256}$, $3^{64}$, $4^{168}$, $5^{32}$, $6^{64}$, $7^{96}$, $8^{84}$, $12^{24}$, $13^{32}$, $15^{32}$, $16^{64}$, $17^{32}$, $19^{32}$, $21^{32}$, $22^{32}$, $24^{16}$, $26^{64}$, $27^{32}$, $28^{8}$, $30^{32}$, $31^{32}$, $32^{6}$, $34^{16}$, $44^{8}$, $48^{32}$, $50^{16}$, $51^{16}$, $52^{16}$, $54^{32}$, $57^{32}$, $62^{16}$, $64^{48}$, $70^{16}$, $76^{16}$, $77^{32}$, $80^{20}$, $83^{16}$, $107^{16}$, $113^{16}$, $115^{16}$, $116^{16}$, $120^{16}$, $121^{16}$, $122^{16}$, $124^{16}$, $131^{48}$, $135^{48}$, $139^{16}$, $143^{16}$
\item $(h,k)=(1,11)$ : $1^{4}$, $3^{4}$, $4^{8}$, $9^{16}$, $15^{16}$, $24^{4}$, $31^{16}$, $32^{16}$, $43^{16}$, $75^{16}$, $98^{16}$, $137^{16}$, $193^{16}$, $277^{16}$, $279^{16}$, $329^{16}$, $335^{16}$, $387^{16}$, $393^{16}$, $557^{16}$, $897^{16}$
\item $(h,k)=(1,13)$ : $1^{16}$, $4^{52}$, $7^{24}$, $8^{14}$, $11^{8}$, $12^{16}$, $16^{2}$, $24^{32}$, $30^{16}$, $44^{16}$, $56^{16}$, $76^{16}$, $80^{16}$, $87^{16}$, $89^{16}$, $107^{16}$, $136^{16}$, $144^{16}$, $159^{16}$, $170^{16}$, $200^{16}$, $205^{16}$, $232^{4}$, $260^{8}$, $307^{16}$, $350^{16}$, $416^{16}$, $423^{16}$, $730^{16}$
\item $(h,k)=(1,15)$ : $1^{4}$, $3^{21844}$
\item $(h,k)=(3,5)$ : $1^{4}$, $3^{84}$, $13^{16}$, $15^{16}$, $16^{4}$, $17^{16}$, $24^{4}$, $29^{16}$, $40^{4}$, $66^{16}$, $69^{16}$, $84^{8}$, $96^{2}$, $116^{16}$, $176^{16}$, $180^{16}$, $184^{16}$, $185^{16}$, $192^{2}$, $233^{16}$, $251^{16}$, $293^{16}$, $504^{16}$, $791^{16}$, $3440^{4}$
\item $(h,k)=(3,7)$ : $1^{16}$, $4^{28}$, $7^{8}$, $8^{2}$, $11^{8}$, $12^{8}$, $16^{2}$, $30^{16}$, $32^{32}$, $37^{16}$, $54^{16}$, $88^{8}$, $91^{16}$, $103^{16}$, $132^{16}$, $141^{16}$, $148^{8}$, $151^{16}$, $152^{4}$, $163^{16}$, $176^{2}$, $188^{8}$, $200^{8}$, $281^{16}$, $288^{8}$, $313^{16}$, $323^{16}$, $327^{16}$, $373^{16}$, $465^{16}$, $506^{16}$
\item $(h,k)=(3,9)$ : $1^{4}$, $3^{4}$, $9^{16}$, $16^{8}$, $24^{12}$, $32^{4}$, $50^{16}$, $52^{16}$, $53^{16}$, $113^{16}$, $114^{16}$, $118^{16}$, $124^{16}$, $136^{8}$, $226^{16}$, $241^{16}$, $243^{16}$, $247^{16}$, $260^{16}$, $266^{16}$, $287^{16}$, $360^{16}$, $420^{8}$, $457^{16}$, $491^{16}$, $576^{2}$
\item $(h,k)=(3,11)$ : $1^{256}$, $3^{128}$, $4^{72}$, $5^{32}$, $6^{64}$, $7^{32}$, $8^{24}$, $9^{16}$, $10^{32}$, $13^{64}$, $14^{80}$, $16^{64}$, $17^{32}$, $19^{32}$, $20^{80}$, $21^{32}$, $22^{32}$, $23^{32}$, $24^{32}$, $26^{16}$, $28^{32}$, $30^{32}$, $32^{40}$, $34^{64}$, $38^{32}$, $43^{48}$, $44^{8}$, $45^{16}$, $46^{16}$, $48^{64}$, $51^{16}$, $53^{32}$, $54^{16}$, $55^{16}$, $56^{36}$, $60^{8}$, $65^{16}$, $70^{32}$, $80^{16}$, $82^{16}$, $84^{16}$, $86^{16}$, $94^{16}$, $105^{32}$, $106^{16}$, $110^{16}$, $112^{2}$, $117^{16}$, $127^{32}$, $131^{32}$, $135^{16}$, $139^{16}$, $143^{16}$
\item $(h,k)=(3,13)$ : $1^{4}$, $3^{21844}$
\item $(h,k)=(3,15)$ : $1^{16}$, $4^{52}$, $7^{24}$, $8^{14}$, $11^{8}$, $12^{16}$, $16^{2}$, $24^{32}$, $30^{16}$, $44^{16}$, $56^{16}$, $76^{16}$, $80^{16}$, $87^{16}$, $89^{16}$, $107^{16}$, $136^{16}$, $144^{16}$, $159^{16}$, $170^{16}$, $200^{16}$, $205^{16}$, $232^{4}$, $260^{8}$, $307^{16}$, $350^{16}$, $416^{16}$, $423^{16}$, $730^{16}$
\item $(h,k)=(5,7)$ : $1^{4}$, $3^{4}$, $4^{16}$, $9^{16}$, $16^{4}$, $24^{4}$, $36^{16}$, $45^{16}$, $46^{16}$, $53^{16}$, $85^{16}$, $137^{16}$, $141^{16}$, $196^{16}$, $211^{16}$, $227^{16}$, $235^{16}$, $239^{16}$, $355^{16}$, $427^{16}$, $490^{16}$, $535^{16}$, $612^{8}$, $1232^{4}$
\item $(h,k)=(5,9)$ : $1^{16}$, $3^{32}$, $4^{28}$, $7^{8}$, $8^{2}$, $11^{8}$, $16^{24}$, $19^{32}$, $33^{16}$, $48^{2}$, $52^{16}$, $59^{32}$, $62^{16}$, $79^{16}$, $116^{8}$, $117^{16}$, $122^{16}$, $139^{32}$, $170^{16}$, $192^{2}$, $231^{16}$, $271^{16}$, $277^{16}$, $333^{16}$, $389^{16}$, $683^{16}$, $707^{16}$
\item $(h,k)=(5,11)$ : $1^{4}$, $3^{21844}$
\item $(h,k)=(5,13)$ : $1^{256}$, $3^{128}$, $4^{72}$, $5^{32}$, $6^{64}$, $7^{32}$, $8^{24}$, $9^{16}$, $10^{32}$, $13^{64}$, $14^{80}$, $16^{64}$, $17^{32}$, $19^{32}$, $20^{80}$, $21^{32}$, $22^{32}$, $23^{32}$, $24^{32}$, $26^{16}$, $28^{32}$, $30^{32}$, $32^{40}$, $34^{64}$, $38^{32}$, $43^{48}$, $44^{8}$, $45^{16}$, $46^{16}$, $48^{64}$, $51^{16}$, $53^{32}$, $54^{16}$, $55^{16}$, $56^{36}$, $60^{8}$, $65^{16}$, $70^{32}$, $80^{16}$, $82^{16}$, $84^{16}$, $86^{16}$, $94^{16}$, $105^{32}$, $106^{16}$, $110^{16}$, $112^{2}$, $117^{16}$, $127^{32}$, $131^{32}$, $135^{16}$, $139^{16}$, $143^{16}$
\item $(h,k)=(5,15)$ : $1^{4}$, $3^{4}$, $4^{8}$, $9^{16}$, $15^{16}$, $24^{4}$, $31^{16}$, $32^{16}$, $43^{16}$, $75^{16}$, $98^{16}$, $137^{16}$, $193^{16}$, $277^{16}$, $279^{16}$, $329^{16}$, $335^{16}$, $387^{16}$, $393^{16}$, $557^{16}$, $897^{16}$
\item $(h,k)=(7,9)$ : $1^{4}$, $3^{21844}$
\item $(h,k)=(7,11)$ : $1^{16}$, $3^{32}$, $4^{28}$, $7^{8}$, $8^{2}$, $11^{8}$, $16^{24}$, $19^{32}$, $33^{16}$, $48^{2}$, $52^{16}$, $59^{32}$, $62^{16}$, $79^{16}$, $116^{8}$, $117^{16}$, $122^{16}$, $139^{32}$, $170^{16}$, $192^{2}$, $231^{16}$, $271^{16}$, $277^{16}$, $333^{16}$, $389^{16}$, $683^{16}$, $707^{16}$
\item $(h,k)=(7,13)$ : $1^{4}$, $3^{4}$, $9^{16}$, $16^{8}$, $24^{12}$, $32^{4}$, $50^{16}$, $52^{16}$, $53^{16}$, $113^{16}$, $114^{16}$, $118^{16}$, $124^{16}$, $136^{8}$, $226^{16}$, $241^{16}$, $243^{16}$, $247^{16}$, $260^{16}$, $266^{16}$, $287^{16}$, $360^{16}$, $420^{8}$, $457^{16}$, $491^{16}$, $576^{2}$
\item $(h,k)=(7,15)$ : $1^{256}$, $3^{64}$, $4^{168}$, $5^{32}$, $6^{64}$, $7^{96}$, $8^{84}$, $12^{24}$, $13^{32}$, $15^{32}$, $16^{64}$, $17^{32}$, $19^{32}$, $21^{32}$, $22^{32}$, $24^{16}$, $26^{64}$, $27^{32}$, $28^{8}$, $30^{32}$, $31^{32}$, $32^{6}$, $34^{16}$, $44^{8}$, $48^{32}$, $50^{16}$, $51^{16}$, $52^{16}$, $54^{32}$, $57^{32}$, $62^{16}$, $64^{48}$, $70^{16}$, $76^{16}$, $77^{32}$, $80^{20}$, $83^{16}$, $107^{16}$, $113^{16}$, $115^{16}$, $116^{16}$, $120^{16}$, $121^{16}$, $122^{16}$, $124^{16}$, $131^{48}$, $135^{48}$, $139^{16}$, $143^{16}$
\item $(h,k)=(9,11)$ : $1^{4}$, $3^{4}$, $4^{16}$, $9^{16}$, $16^{4}$, $24^{4}$, $36^{16}$, $45^{16}$, $46^{16}$, $53^{16}$, $85^{16}$, $137^{16}$, $141^{16}$, $196^{16}$, $211^{16}$, $227^{16}$, $235^{16}$, $239^{16}$, $355^{16}$, $427^{16}$, $490^{16}$, $535^{16}$, $612^{8}$, $1232^{4}$
\item $(h,k)=(9,13)$ : $1^{16}$, $4^{28}$, $7^{8}$, $8^{2}$, $11^{8}$, $12^{8}$, $16^{2}$, $30^{16}$, $32^{32}$, $37^{16}$, $54^{16}$, $88^{8}$, $91^{16}$, $103^{16}$, $132^{16}$, $141^{16}$, $148^{8}$, $151^{16}$, $152^{4}$, $163^{16}$, $176^{2}$, $188^{8}$, $200^{8}$, $281^{16}$, $288^{8}$, $313^{16}$, $323^{16}$, $327^{16}$, $373^{16}$, $465^{16}$, $506^{16}$
\item $(h,k)=(9,15)$ : $1^{4}$, $3^{84}$, $6^{16}$, $16^{2}$, $20^{16}$, $32^{2}$, $50^{16}$, $56^{4}$, $60^{16}$, $78^{16}$, $113^{16}$, $118^{16}$, $127^{16}$, $134^{16}$, $155^{16}$, $165^{16}$, $228^{16}$, $253^{16}$, $256^{4}$, $391^{16}$, $451^{16}$, $689^{16}$, $722^{16}$, $1888^{2}$
\item $(h,k)=(11,13)$ : $1^{4}$, $3^{84}$, $13^{16}$, $15^{16}$, $16^{4}$, $17^{16}$, $24^{4}$, $29^{16}$, $40^{4}$, $66^{16}$, $69^{16}$, $84^{8}$, $96^{2}$, $116^{16}$, $176^{16}$, $180^{16}$, $184^{16}$, $185^{16}$, $192^{2}$, $233^{16}$, $251^{16}$, $293^{16}$, $504^{16}$, $791^{16}$, $3440^{4}$
\item $(h,k)=(11,15)$ : $1^{16}$, $4^{20}$, $7^{8}$, $8^{2}$, $11^{8}$, $20^{8}$, $32^{40}$, $39^{16}$, $48^{8}$, $51^{16}$, $64^{2}$, $80^{8}$, $84^{16}$, $88^{8}$, $100^{16}$, $137^{32}$, $155^{16}$, $211^{32}$, $226^{16}$, $252^{16}$, $263^{16}$, $269^{16}$, $336^{2}$, $405^{16}$, $415^{16}$, $416^{16}$, $461^{16}$
\item $(h,k)=(13,15)$ : $1^{4}$, $3^{4}$, $9^{16}$, $24^{4}$, $32^{16}$, $34^{16}$, $39^{16}$, $40^{16}$, $47^{16}$, $77^{16}$, $83^{16}$, $125^{16}$, $165^{16}$, $216^{32}$, $220^{16}$, $269^{16}$, $270^{16}$, $287^{16}$, $394^{16}$, $548^{16}$, $578^{16}$, $3520^{2}$
\end{itemize}

\section{Results for $n=17$}
\begin{itemize}
\item $(h,k)=(1,2)$ : $1^{2}$, $17^{2}$, $18^{136}$, $22^{68}$, $25^{34}$, $27^{34}$, $32^{34}$, $34^{36}$, $51^{6}$, $52^{34}$, $68^{10}$, $85^{20}$, $102^{34}$, $119^{12}$, $136^{46}$, $153^{2}$, $170^{12}$, $204^{40}$, $221^{10}$, $238^{18}$, $255^{8}$, $272^{6}$, $289^{2}$, $306^{16}$, $323^{18}$, $340^{14}$, $374^{12}$, $391^{12}$, $408^{8}$, $425^{8}$, $442^{4}$, $459^{2}$, $476^{2}$, $510^{4}$, $527^{6}$, $544^{8}$, $578^{10}$, $595^{4}$, $612^{4}$, $646^{6}$, $663^{2}$, $680^{2}$, $697^{8}$, $731^{2}$, $748^{2}$, $782^{2}$, $799^{4}$, $884^{6}$, $952^{2}$, $986^{2}$, $1003^{2}$
\item $(h,k)=(1,3)$ : $1^{2}$, $68^{2}$, $102^{2}$, $510^{2}$, $64855^{2}$
\item $(h,k)=(1,4)$ : $1^{2}$, $136^{2}$, $3876^{2}$, $7548^{2}$, $53975^{2}$
\item $(h,k)=(1,5)$ : $1^{2}$, $17^{2}$, $34^{2}$, $153^{2}$, $221^{2}$, $1326^{2}$, $3672^{2}$, $15062^{2}$, $45050^{2}$
\item $(h,k)=(1,6)$ : $1^{2}$, $17^{2}$, $21^{34}$, $51^{2}$, $963^{34}$, $1615^{2}$, $3043^{2}$, $14008^{2}$, $30073^{2}$
\item $(h,k)=(1,7)$ : $1^{2}$, $35^{34}$, $238^{2}$, $340^{2}$, $9860^{2}$, $11271^{2}$, $43231^{2}$
\item $(h,k)=(1,8)$ : $1^{2}$, $459^{2}$, $65076^{2}$
\item $(h,k)=(1,9)$ : $1^{2}$, $65535^{2}$
\item $(h,k)=(1,10)$ : $1^{2}$, $34^{4}$, $204^{2}$, $32198^{2}$, $33065^{2}$
\item $(h,k)=(1,11)$ : $1^{2}$, $34^{2}$, $51^{2}$, $238^{2}$, $1904^{2}$, $3724^{34}$
\item $(h,k)=(1,12)$ : $1^{2}$, $65^{34}$, $64430^{2}$
\item $(h,k)=(1,13)$ : $1^{2}$, $17^{2}$, $68^{2}$, $238^{2}$, $391^{2}$, $3315^{2}$, $61506^{2}$
\item $(h,k)=(1,14)$ : $1^{2}$, $136^{2}$, $272^{2}$, $30345^{2}$, $34782^{2}$
\item $(h,k)=(1,15)$ : $1^{2}$, $51^{2}$, $102^{36}$, $153^{2}$, $306^{22}$, $357^{4}$, $408^{16}$, $510^{8}$, $612^{14}$, $663^{6}$, $714^{4}$, $816^{6}$, $918^{10}$, $969^{6}$, $1020^{2}$, $1122^{4}$, $1173^{10}$, $1224^{2}$, $1530^{2}$, $1581^{2}$, $1734^{4}$, $1938^{2}$, $2091^{2}$, $2244^{2}$, $2346^{2}$, $2652^{6}$, $2958^{2}$
\item $(h,k)=(1,16)$ : $1^{2}$, $3^{43690}$
\item $(h,k)=(2,3)$ : $1^{2}$, $34^{2}$, $136^{2}$, $476^{2}$, $1462^{2}$, $2941^{2}$, $5610^{2}$, $18292^{2}$, $36584^{2}$
\item $(h,k)=(2,4)$ : $1^{2}$, $17^{38}$, $34^{2}$, $153^{2}$, $5576^{2}$, $8415^{2}$, $17680^{2}$, $33354^{2}$
\item $(h,k)=(2,5)$ : $1^{2}$, $510^{2}$, $663^{2}$, $2329^{2}$, $2788^{2}$, $59245^{2}$
\item $(h,k)=(2,6)$ : $1^{2}$, $17^{2}$, $34^{2}$, $51^{2}$, $102^{2}$, $1258^{2}$, $1632^{2}$, $16626^{2}$, $45815^{2}$
\item $(h,k)=(2,7)$ : $1^{2}$, $34^{4}$, $119^{2}$, $11730^{2}$, $20672^{2}$, $32946^{2}$
\item $(h,k)=(2,8)$ : $1^{2}$, $1139^{2}$, $4505^{2}$, $59891^{2}$
\item $(h,k)=(2,9)$ : $1^{2}$, $119^{2}$, $884^{2}$, $16762^{2}$, $47770^{2}$
\item $(h,k)=(2,10)$ : $1^{2}$, $17^{2}$, $969^{2}$, $1224^{2}$, $2159^{2}$, $2805^{2}$, $14552^{2}$, $19873^{2}$, $23936^{2}$
\item $(h,k)=(2,11)$ : $1^{2}$, $34^{2}$, $527^{2}$, $10149^{2}$, $14450^{2}$, $40375^{2}$
\item $(h,k)=(2,12)$ : $1^{2}$, $17^{2}$, $85^{2}$, $160^{34}$, $2516^{2}$, $3842^{2}$, $5151^{2}$, $9724^{2}$, $41480^{2}$
\item $(h,k)=(2,13)$ : $1^{2}$, $289^{2}$, $2941^{2}$, $6273^{2}$, $15232^{2}$, $40800^{2}$
\item $(h,k)=(2,14)$ : $1^{2}$, $34^{2}$, $391^{2}$, $658^{34}$, $53924^{2}$
\item $(h,k)=(2,15)$ : $1^{2}$, $3^{43690}$
\item $(h,k)=(2,16)$ : $1^{2}$, $51^{2}$, $102^{36}$, $153^{2}$, $306^{22}$, $357^{4}$, $408^{16}$, $510^{8}$, $612^{14}$, $663^{6}$, $714^{4}$, $816^{6}$, $918^{10}$, $969^{6}$, $1020^{2}$, $1122^{4}$, $1173^{10}$, $1224^{2}$, $1530^{2}$, $1581^{2}$, $1734^{4}$, $1938^{2}$, $2091^{2}$, $2244^{2}$, $2346^{2}$, $2652^{6}$, $2958^{2}$
\item $(h,k)=(3,4)$ : $1^{2}$, $17^{2}$, $42^{34}$, $748^{2}$, $64056^{2}$
\item $(h,k)=(3,5)$ : $1^{2}$, $12886^{2}$, $52649^{2}$
\item $(h,k)=(3,6)$ : $1^{2}$, $17^{2}$, $51^{2}$, $119^{2}$, $272^{2}$, $425^{2}$, $816^{2}$, $5899^{2}$, $14093^{2}$, $43843^{2}$
\item $(h,k)=(3,7)$ : $1^{2}$, $1309^{2}$, $3536^{2}$, $12665^{2}$, $17578^{2}$, $30447^{2}$
\item $(h,k)=(3,8)$ : $1^{2}$, $323^{2}$, $65212^{2}$
\item $(h,k)=(3,9)$ : $1^{2}$, $17^{2}$, $1462^{2}$, $64056^{2}$
\item $(h,k)=(3,10)$ : $1^{2}$, $34^{2}$, $51^{4}$, $68^{2}$, $136^{2}$, $187^{2}$, $986^{2}$, $7548^{2}$, $8041^{2}$, $10438^{2}$, $12087^{2}$, $25908^{2}$
\item $(h,k)=(3,11)$ : $1^{2}$, $34^{2}$, $119^{2}$, $255^{2}$, $476^{2}$, $493^{2}$, $6273^{2}$, $57885^{2}$
\item $(h,k)=(3,12)$ : $1^{2}$, $102^{2}$, $119^{2}$, $323^{2}$, $13175^{2}$, $51816^{2}$
\item $(h,k)=(3,13)$ : $1^{2}$, $34^{2}$, $85^{2}$, $1292^{2}$, $1921^{2}$, $4046^{2}$, $58157^{2}$
\item $(h,k)=(3,14)$ : $1^{2}$, $3^{43690}$
\item $(h,k)=(3,15)$ : $1^{2}$, $34^{2}$, $391^{2}$, $658^{34}$, $53924^{2}$
\item $(h,k)=(3,16)$ : $1^{2}$, $136^{2}$, $272^{2}$, $30345^{2}$, $34782^{2}$
\item $(h,k)=(4,5)$ : $1^{2}$, $17^{2}$, $34^{2}$, $1853^{2}$, $3587^{2}$, $60044^{2}$
\item $(h,k)=(4,6)$ : $1^{2}$, $136^{2}$, $5270^{2}$, $5287^{2}$, $10098^{2}$, $21352^{2}$, $23392^{2}$
\item $(h,k)=(4,7)$ : $1^{2}$, $85^{2}$, $306^{2}$, $561^{2}$, $1054^{2}$, $2686^{2}$, $13345^{2}$, $47498^{2}$
\item $(h,k)=(4,8)$ : $1^{2}$, $17^{2}$, $80^{34}$, $102^{2}$, $221^{2}$, $692^{34}$, $731^{2}$, $816^{2}$, $1511^{34}$, $2771^{2}$, $8636^{2}$, $13430^{2}$
\item $(h,k)=(4,9)$ : $1^{2}$, $17^{2}$, $51^{2}$, $570^{34}$, $55777^{2}$
\item $(h,k)=(4,10)$ : $1^{2}$, $17^{2}$, $187^{2}$, $2210^{2}$, $13498^{2}$, $49623^{2}$
\item $(h,k)=(4,11)$ : $1^{2}$, $340^{2}$, $765^{2}$, $1377^{2}$, $2873^{2}$, $13362^{2}$, $46818^{2}$
\item $(h,k)=(4,12)$ : $1^{2}$, $16864^{2}$, $48671^{2}$
\item $(h,k)=(4,13)$ : $1^{2}$, $3^{43690}$
\item $(h,k)=(4,14)$ : $1^{2}$, $34^{2}$, $85^{2}$, $1292^{2}$, $1921^{2}$, $4046^{2}$, $58157^{2}$
\item $(h,k)=(4,15)$ : $1^{2}$, $289^{2}$, $2941^{2}$, $6273^{2}$, $15232^{2}$, $40800^{2}$
\item $(h,k)=(4,16)$ : $1^{2}$, $17^{2}$, $68^{2}$, $238^{2}$, $391^{2}$, $3315^{2}$, $61506^{2}$
\item $(h,k)=(5,6)$ : $1^{2}$, $17^{2}$, $51^{2}$, $629^{2}$, $64838^{2}$
\item $(h,k)=(5,7)$ : $1^{2}$, $17^{2}$, $357^{2}$, $1258^{2}$, $63903^{2}$
\item $(h,k)=(5,8)$ : $1^{2}$, $51^{2}$, $612^{2}$, $3196^{2}$, $12971^{2}$, $48705^{2}$
\item $(h,k)=(5,9)$ : $1^{2}$, $85^{2}$, $1530^{2}$, $4012^{2}$, $7038^{2}$, $52870^{2}$
\item $(h,k)=(5,10)$ : $1^{2}$, $51^{2}$, $119^{2}$, $153^{2}$, $238^{2}$, $323^{2}$, $986^{2}$, $1275^{2}$, $1649^{2}$, $3009^{2}$, $8925^{2}$, $20978^{2}$, $27829^{2}$
\item $(h,k)=(5,11)$ : $1^{2}$, $85^{2}$, $340^{2}$, $2193^{2}$, $62917^{2}$
\item $(h,k)=(5,12)$ : $1^{2}$, $3^{43690}$
\item $(h,k)=(5,13)$ : $1^{2}$, $16864^{2}$, $48671^{2}$
\item $(h,k)=(5,14)$ : $1^{2}$, $102^{2}$, $119^{2}$, $323^{2}$, $13175^{2}$, $51816^{2}$
\item $(h,k)=(5,15)$ : $1^{2}$, $17^{2}$, $85^{2}$, $160^{34}$, $2516^{2}$, $3842^{2}$, $5151^{2}$, $9724^{2}$, $41480^{2}$
\item $(h,k)=(5,16)$ : $1^{2}$, $65^{34}$, $64430^{2}$
\item $(h,k)=(6,7)$ : $1^{2}$, $34^{2}$, $680^{2}$, $1079^{34}$, $46478^{2}$
\item $(h,k)=(6,8)$ : $1^{2}$, $17^{2}$, $34^{2}$, $102^{2}$, $136^{2}$, $459^{2}$, $1156^{2}$, $1224^{2}$, $12648^{2}$, $16116^{2}$, $33643^{2}$
\item $(h,k)=(6,9)$ : $1^{2}$, $1360^{2}$, $8211^{2}$, $16048^{2}$, $39916^{2}$
\item $(h,k)=(6,10)$ : $1^{2}$, $17^{2}$, $663^{2}$, $2482^{2}$, $62373^{2}$
\item $(h,k)=(6,11)$ : $1^{2}$, $3^{43690}$
\item $(h,k)=(6,12)$ : $1^{2}$, $85^{2}$, $340^{2}$, $2193^{2}$, $62917^{2}$
\item $(h,k)=(6,13)$ : $1^{2}$, $340^{2}$, $765^{2}$, $1377^{2}$, $2873^{2}$, $13362^{2}$, $46818^{2}$
\item $(h,k)=(6,14)$ : $1^{2}$, $34^{2}$, $119^{2}$, $255^{2}$, $476^{2}$, $493^{2}$, $6273^{2}$, $57885^{2}$
\item $(h,k)=(6,15)$ : $1^{2}$, $34^{2}$, $527^{2}$, $10149^{2}$, $14450^{2}$, $40375^{2}$
\item $(h,k)=(6,16)$ : $1^{2}$, $34^{2}$, $51^{2}$, $238^{2}$, $1904^{2}$, $3724^{34}$
\item $(h,k)=(7,8)$ : $1^{2}$, $102^{2}$, $65433^{2}$
\item $(h,k)=(7,9)$ : $1^{2}$, $17^{34}$, $51^{2}$, $306^{2}$, $1156^{2}$, $3468^{2}$, $4726^{2}$, $55539^{2}$
\item $(h,k)=(7,10)$ : $1^{2}$, $3^{43690}$
\item $(h,k)=(7,11)$ : $1^{2}$, $17^{2}$, $663^{2}$, $2482^{2}$, $62373^{2}$
\item $(h,k)=(7,12)$ : $1^{2}$, $51^{2}$, $119^{2}$, $153^{2}$, $238^{2}$, $323^{2}$, $986^{2}$, $1275^{2}$, $1649^{2}$, $3009^{2}$, $8925^{2}$, $20978^{2}$, $27829^{2}$
\item $(h,k)=(7,13)$ : $1^{2}$, $17^{2}$, $187^{2}$, $2210^{2}$, $13498^{2}$, $49623^{2}$
\item $(h,k)=(7,14)$ : $1^{2}$, $34^{2}$, $51^{4}$, $68^{2}$, $136^{2}$, $187^{2}$, $986^{2}$, $7548^{2}$, $8041^{2}$, $10438^{2}$, $12087^{2}$, $25908^{2}$
\item $(h,k)=(7,15)$ : $1^{2}$, $17^{2}$, $969^{2}$, $1224^{2}$, $2159^{2}$, $2805^{2}$, $14552^{2}$, $19873^{2}$, $23936^{2}$
\item $(h,k)=(7,16)$ : $1^{2}$, $34^{4}$, $204^{2}$, $32198^{2}$, $33065^{2}$
\item $(h,k)=(8,9)$ : $1^{2}$, $3^{43690}$
\item $(h,k)=(8,10)$ : $1^{2}$, $17^{34}$, $51^{2}$, $306^{2}$, $1156^{2}$, $3468^{2}$, $4726^{2}$, $55539^{2}$
\item $(h,k)=(8,11)$ : $1^{2}$, $1360^{2}$, $8211^{2}$, $16048^{2}$, $39916^{2}$
\item $(h,k)=(8,12)$ : $1^{2}$, $85^{2}$, $1530^{2}$, $4012^{2}$, $7038^{2}$, $52870^{2}$
\item $(h,k)=(8,13)$ : $1^{2}$, $17^{2}$, $51^{2}$, $570^{34}$, $55777^{2}$
\item $(h,k)=(8,14)$ : $1^{2}$, $17^{2}$, $1462^{2}$, $64056^{2}$
\item $(h,k)=(8,15)$ : $1^{2}$, $119^{2}$, $884^{2}$, $16762^{2}$, $47770^{2}$
\item $(h,k)=(8,16)$ : $1^{2}$, $65535^{2}$
\item $(h,k)=(9,10)$ : $1^{2}$, $102^{2}$, $65433^{2}$
\item $(h,k)=(9,11)$ : $1^{2}$, $17^{2}$, $34^{2}$, $102^{2}$, $136^{2}$, $459^{2}$, $1156^{2}$, $1224^{2}$, $12648^{2}$, $16116^{2}$, $33643^{2}$
\item $(h,k)=(9,12)$ : $1^{2}$, $51^{2}$, $612^{2}$, $3196^{2}$, $12971^{2}$, $48705^{2}$
\item $(h,k)=(9,13)$ : $1^{2}$, $17^{2}$, $80^{34}$, $102^{2}$, $221^{2}$, $692^{34}$, $731^{2}$, $816^{2}$, $1511^{34}$, $2771^{2}$, $8636^{2}$, $13430^{2}$
\item $(h,k)=(9,14)$ : $1^{2}$, $323^{2}$, $65212^{2}$
\item $(h,k)=(9,15)$ : $1^{2}$, $1139^{2}$, $4505^{2}$, $59891^{2}$
\item $(h,k)=(9,16)$ : $1^{2}$, $459^{2}$, $65076^{2}$
\item $(h,k)=(10,11)$ : $1^{2}$, $34^{2}$, $680^{2}$, $1079^{34}$, $46478^{2}$
\item $(h,k)=(10,12)$ : $1^{2}$, $17^{2}$, $357^{2}$, $1258^{2}$, $63903^{2}$
\item $(h,k)=(10,13)$ : $1^{2}$, $85^{2}$, $306^{2}$, $561^{2}$, $1054^{2}$, $2686^{2}$, $13345^{2}$, $47498^{2}$
\item $(h,k)=(10,14)$ : $1^{2}$, $1309^{2}$, $3536^{2}$, $12665^{2}$, $17578^{2}$, $30447^{2}$
\item $(h,k)=(10,15)$ : $1^{2}$, $34^{4}$, $119^{2}$, $11730^{2}$, $20672^{2}$, $32946^{2}$
\item $(h,k)=(10,16)$ : $1^{2}$, $35^{34}$, $238^{2}$, $340^{2}$, $9860^{2}$, $11271^{2}$, $43231^{2}$
\item $(h,k)=(11,12)$ : $1^{2}$, $17^{2}$, $51^{2}$, $629^{2}$, $64838^{2}$
\item $(h,k)=(11,13)$ : $1^{2}$, $136^{2}$, $5270^{2}$, $5287^{2}$, $10098^{2}$, $21352^{2}$, $23392^{2}$
\item $(h,k)=(11,14)$ : $1^{2}$, $17^{2}$, $51^{2}$, $119^{2}$, $272^{2}$, $425^{2}$, $816^{2}$, $5899^{2}$, $14093^{2}$, $43843^{2}$
\item $(h,k)=(11,15)$ : $1^{2}$, $17^{2}$, $34^{2}$, $51^{2}$, $102^{2}$, $1258^{2}$, $1632^{2}$, $16626^{2}$, $45815^{2}$
\item $(h,k)=(11,16)$ : $1^{2}$, $17^{2}$, $21^{34}$, $51^{2}$, $963^{34}$, $1615^{2}$, $3043^{2}$, $14008^{2}$, $30073^{2}$
\item $(h,k)=(12,13)$ : $1^{2}$, $17^{2}$, $34^{2}$, $1853^{2}$, $3587^{2}$, $60044^{2}$
\item $(h,k)=(12,14)$ : $1^{2}$, $12886^{2}$, $52649^{2}$
\item $(h,k)=(12,15)$ : $1^{2}$, $510^{2}$, $663^{2}$, $2329^{2}$, $2788^{2}$, $59245^{2}$
\item $(h,k)=(12,16)$ : $1^{2}$, $17^{2}$, $34^{2}$, $153^{2}$, $221^{2}$, $1326^{2}$, $3672^{2}$, $15062^{2}$, $45050^{2}$
\item $(h,k)=(13,14)$ : $1^{2}$, $17^{2}$, $42^{34}$, $748^{2}$, $64056^{2}$
\item $(h,k)=(13,15)$ : $1^{2}$, $17^{38}$, $34^{2}$, $153^{2}$, $5576^{2}$, $8415^{2}$, $17680^{2}$, $33354^{2}$
\item $(h,k)=(13,16)$ : $1^{2}$, $136^{2}$, $3876^{2}$, $7548^{2}$, $53975^{2}$
\item $(h,k)=(14,15)$ : $1^{2}$, $34^{2}$, $136^{2}$, $476^{2}$, $1462^{2}$, $2941^{2}$, $5610^{2}$, $18292^{2}$, $36584^{2}$
\item $(h,k)=(14,16)$ : $1^{2}$, $68^{2}$, $102^{2}$, $510^{2}$, $64855^{2}$
\item $(h,k)=(15,16)$ : $1^{2}$, $17^{2}$, $18^{136}$, $22^{68}$, $25^{34}$, $27^{34}$, $32^{34}$, $34^{36}$, $51^{6}$, $52^{34}$, $68^{10}$, $85^{20}$, $102^{34}$, $119^{12}$, $136^{46}$, $153^{2}$, $170^{12}$, $204^{40}$, $221^{10}$, $238^{18}$, $255^{8}$, $272^{6}$, $289^{2}$, $306^{16}$, $323^{18}$, $340^{14}$, $374^{12}$, $391^{12}$, $408^{8}$, $425^{8}$, $442^{4}$, $459^{2}$, $476^{2}$, $510^{4}$, $527^{6}$, $544^{8}$, $578^{10}$, $595^{4}$, $612^{4}$, $646^{6}$, $663^{2}$, $680^{2}$, $697^{8}$, $731^{2}$, $748^{2}$, $782^{2}$, $799^{4}$, $884^{6}$, $952^{2}$, $986^{2}$, $1003^{2}$
\end{itemize}

\section{Results for $n=18$}
\begin{itemize}
\item $(h,k)=(1,5)$ : $1^{4}$, $3^{32}$, $6^{18}$, $14^{18}$, $25^{18}$, $29^{18}$, $34^{18}$, $36^{2}$, $37^{18}$, $38^{18}$, $47^{18}$, $67^{18}$, $72^{6}$, $78^{18}$, $104^{18}$, $107^{18}$, $110^{18}$, $123^{18}$, $150^{18}$, $155^{18}$, $160^{18}$, $209^{18}$, $215^{18}$, $241^{18}$, $245^{18}$, $251^{18}$, $270^{4}$, $288^{18}$, $314^{18}$, $374^{18}$, $381^{18}$, $483^{18}$, $491^{18}$, $594^{2}$, $14442^{12}$
\item $(h,k)=(1,7)$ : $1^{64}$, $3^{42}$, $5^{108}$, $6^{60}$, $8^{18}$, $9^{20}$, $11^{18}$, $12^{60}$, $15^{66}$, $18^{18}$, $24^{6}$, $26^{18}$, $27^{4}$, $28^{36}$, $31^{18}$, $32^{18}$, $33^{18}$, $35^{18}$, $36^{18}$, $38^{18}$, $42^{30}$, $46^{18}$, $48^{6}$, $49^{90}$, $57^{18}$, $59^{36}$, $65^{18}$, $68^{18}$, $70^{18}$, $75^{18}$, $79^{18}$, $85^{18}$, $91^{18}$, $93^{18}$, $96^{12}$, $99^{18}$, $118^{18}$, $135^{18}$, $168^{12}$, $207^{12}$, $252^{12}$, $462^{12}$, $614^{36}$, $627^{12}$, $675^{12}$, $960^{12}$, $1422^{12}$, $1584^{12}$, $1797^{12}$, $1869^{12}$, $2034^{12}$, $2043^{12}$, $2055^{12}$, $2106^{4}$
\item $(h,k)=(1,11)$ : $1^{4}$, $3^{20}$, $11^{18}$, $12^{24}$, $18^{2}$, $20^{54}$, $22^{18}$, $27^{2}$, $30^{18}$, $34^{18}$, $36^{2}$, $45^{18}$, $54^{18}$, $59^{18}$, $60^{18}$, $63^{4}$, $67^{18}$, $72^{18}$, $81^{18}$, $91^{36}$, $132^{18}$, $137^{18}$, $151^{18}$, $161^{18}$, $164^{18}$, $189^{12}$, $215^{18}$, $250^{18}$, $276^{18}$, $283^{18}$, $306^{18}$, $405^{18}$, $490^{18}$, $499^{18}$, $585^{18}$, $1083^{12}$, $8802^{4}$, $30987^{4}$
\item $(h,k)=(1,13)$ : $1^{64}$, $3^{42}$, $4^{18}$, $6^{48}$, $7^{36}$, $9^{14}$, $10^{18}$, $12^{12}$, $13^{54}$, $15^{30}$, $17^{72}$, $18^{30}$, $19^{36}$, $20^{18}$, $22^{18}$, $24^{12}$, $30^{36}$, $32^{36}$, $33^{30}$, $35^{18}$, $36^{50}$, $38^{18}$, $41^{18}$, $47^{18}$, $48^{36}$, $50^{18}$, $54^{22}$, $59^{36}$, $62^{18}$, $65^{18}$, $66^{12}$, $75^{12}$, $76^{36}$, $80^{18}$, $89^{18}$, $90^{12}$, $114^{18}$, $115^{18}$, $126^{12}$, $130^{18}$, $150^{18}$, $169^{18}$, $246^{36}$, $306^{4}$, $415^{36}$, $417^{12}$, $498^{12}$, $527^{36}$, $650^{36}$, $660^{36}$, $675^{36}$, $1368^{4}$, $1470^{12}$, $1809^{12}$, $1881^{12}$, $2001^{12}$
\item $(h,k)=(1,17)$ : $1^{4}$, $3^{87380}$
\item $(h,k)=(5,7)$ : $1^{4}$, $3^{20}$, $4^{18}$, $6^{18}$, $18^{18}$, $27^{4}$, $32^{18}$, $37^{18}$, $38^{18}$, $41^{18}$, $60^{6}$, $68^{18}$, $72^{20}$, $77^{18}$, $105^{18}$, $124^{18}$, $126^{18}$, $130^{18}$, $173^{18}$, $174^{18}$, $180^{18}$, $182^{18}$, $190^{18}$, $200^{18}$, $207^{18}$, $218^{18}$, $228^{12}$, $238^{18}$, $323^{18}$, $346^{18}$, $368^{18}$, $426^{18}$, $433^{18}$, $675^{4}$, $702^{4}$, $774^{4}$, $42120^{4}$
\item $(h,k)=(5,11)$ : $1^{64}$, $3^{96}$, $4^{36}$, $5^{18}$, $7^{18}$, $8^{18}$, $9^{36}$, $10^{36}$, $11^{18}$, $12^{36}$, $16^{18}$, $18^{16}$, $19^{18}$, $21^{18}$, $25^{18}$, $31^{18}$, $33^{24}$, $35^{18}$, $36^{8}$, $38^{18}$, $39^{24}$, $44^{18}$, $45^{12}$, $46^{18}$, $54^{24}$, $56^{18}$, $61^{18}$, $64^{36}$, $65^{18}$, $69^{12}$, $71^{18}$, $83^{18}$, $85^{36}$, $97^{18}$, $104^{18}$, $113^{18}$, $128^{18}$, $129^{18}$, $131^{18}$, $144^{12}$, $150^{12}$, $153^{36}$, $180^{2}$, $222^{36}$, $240^{12}$, $282^{12}$, $381^{12}$, $387^{12}$, $416^{72}$, $477^{12}$, $640^{36}$, $677^{36}$, $684^{36}$, $711^{4}$, $1323^{12}$, $1743^{12}$, $1809^{12}$, $2007^{12}$
\item $(h,k)=(5,13)$ : $1^{4}$, $3^{87380}$
\item $(h,k)=(5,17)$ : $1^{64}$, $3^{42}$, $4^{18}$, $6^{48}$, $7^{36}$, $9^{14}$, $10^{18}$, $12^{12}$, $13^{54}$, $15^{30}$, $17^{72}$, $18^{30}$, $19^{36}$, $20^{18}$, $22^{18}$, $24^{12}$, $30^{36}$, $32^{36}$, $33^{30}$, $35^{18}$, $36^{50}$, $38^{18}$, $41^{18}$, $47^{18}$, $48^{36}$, $50^{18}$, $54^{22}$, $59^{36}$, $62^{18}$, $65^{18}$, $66^{12}$, $75^{12}$, $76^{36}$, $80^{18}$, $89^{18}$, $90^{12}$, $114^{18}$, $115^{18}$, $126^{12}$, $130^{18}$, $150^{18}$, $169^{18}$, $246^{36}$, $306^{4}$, $415^{36}$, $417^{12}$, $498^{12}$, $527^{36}$, $650^{36}$, $660^{36}$, $675^{36}$, $1368^{4}$, $1470^{12}$, $1809^{12}$, $1881^{12}$, $2001^{12}$
\item $(h,k)=(7,11)$ : $1^{4}$, $3^{87380}$
\item $(h,k)=(7,13)$ : $1^{64}$, $3^{96}$, $4^{36}$, $5^{18}$, $7^{18}$, $8^{18}$, $9^{36}$, $10^{36}$, $11^{18}$, $12^{36}$, $16^{18}$, $18^{16}$, $19^{18}$, $21^{18}$, $25^{18}$, $31^{18}$, $33^{24}$, $35^{18}$, $36^{8}$, $38^{18}$, $39^{24}$, $44^{18}$, $45^{12}$, $46^{18}$, $54^{24}$, $56^{18}$, $61^{18}$, $64^{36}$, $65^{18}$, $69^{12}$, $71^{18}$, $83^{18}$, $85^{36}$, $97^{18}$, $104^{18}$, $113^{18}$, $128^{18}$, $129^{18}$, $131^{18}$, $144^{12}$, $150^{12}$, $153^{36}$, $180^{2}$, $222^{36}$, $240^{12}$, $282^{12}$, $381^{12}$, $387^{12}$, $416^{72}$, $477^{12}$, $640^{36}$, $677^{36}$, $684^{36}$, $711^{4}$, $1323^{12}$, $1743^{12}$, $1809^{12}$, $2007^{12}$
\item $(h,k)=(7,17)$ : $1^{4}$, $3^{20}$, $11^{18}$, $12^{24}$, $18^{2}$, $20^{54}$, $22^{18}$, $27^{2}$, $30^{18}$, $34^{18}$, $36^{2}$, $45^{18}$, $54^{18}$, $59^{18}$, $60^{18}$, $63^{4}$, $67^{18}$, $72^{18}$, $81^{18}$, $91^{36}$, $132^{18}$, $137^{18}$, $151^{18}$, $161^{18}$, $164^{18}$, $189^{12}$, $215^{18}$, $250^{18}$, $276^{18}$, $283^{18}$, $306^{18}$, $405^{18}$, $490^{18}$, $499^{18}$, $585^{18}$, $1083^{12}$, $8802^{4}$, $30987^{4}$
\item $(h,k)=(11,13)$ : $1^{4}$, $3^{20}$, $4^{18}$, $6^{18}$, $18^{18}$, $27^{4}$, $32^{18}$, $37^{18}$, $38^{18}$, $41^{18}$, $60^{6}$, $68^{18}$, $72^{20}$, $77^{18}$, $105^{18}$, $124^{18}$, $126^{18}$, $130^{18}$, $173^{18}$, $174^{18}$, $180^{18}$, $182^{18}$, $190^{18}$, $200^{18}$, $207^{18}$, $218^{18}$, $228^{12}$, $238^{18}$, $323^{18}$, $346^{18}$, $368^{18}$, $426^{18}$, $433^{18}$, $675^{4}$, $702^{4}$, $774^{4}$, $42120^{4}$
\item $(h,k)=(11,17)$ : $1^{64}$, $3^{42}$, $5^{108}$, $6^{60}$, $8^{18}$, $9^{20}$, $11^{18}$, $12^{60}$, $15^{66}$, $18^{18}$, $24^{6}$, $26^{18}$, $27^{4}$, $28^{36}$, $31^{18}$, $32^{18}$, $33^{18}$, $35^{18}$, $36^{18}$, $38^{18}$, $42^{30}$, $46^{18}$, $48^{6}$, $49^{90}$, $57^{18}$, $59^{36}$, $65^{18}$, $68^{18}$, $70^{18}$, $75^{18}$, $79^{18}$, $85^{18}$, $91^{18}$, $93^{18}$, $96^{12}$, $99^{18}$, $118^{18}$, $135^{18}$, $168^{12}$, $207^{12}$, $252^{12}$, $462^{12}$, $614^{36}$, $627^{12}$, $675^{12}$, $960^{12}$, $1422^{12}$, $1584^{12}$, $1797^{12}$, $1869^{12}$, $2034^{12}$, $2043^{12}$, $2055^{12}$, $2106^{4}$
\item $(h,k)=(13,17)$ : $1^{4}$, $3^{32}$, $6^{18}$, $14^{18}$, $25^{18}$, $29^{18}$, $34^{18}$, $36^{2}$, $37^{18}$, $38^{18}$, $47^{18}$, $67^{18}$, $72^{6}$, $78^{18}$, $104^{18}$, $107^{18}$, $110^{18}$, $123^{18}$, $150^{18}$, $155^{18}$, $160^{18}$, $209^{18}$, $215^{18}$, $241^{18}$, $245^{18}$, $251^{18}$, $270^{4}$, $288^{18}$, $314^{18}$, $374^{18}$, $381^{18}$, $483^{18}$, $491^{18}$, $594^{2}$, $14442^{12}$
\end{itemize}

\section{Results for $n=19$}
\begin{itemize}
\item $(h,k)=(1,2)$ : $1^{2}$, $14^{76}$, $15^{38}$, $18^{76}$, $19^{4}$, $20^{152}$, $27^{114}$, $31^{38}$, $34^{76}$, $38^{4}$, $41^{38}$, $42^{114}$, $53^{38}$, $54^{38}$, $57^{12}$, $70^{38}$, $73^{38}$, $76^{2}$, $80^{38}$, $95^{10}$, $114^{24}$, $133^{44}$, $152^{60}$, $171^{38}$, $190^{104}$, $209^{20}$, $228^{60}$, $247^{8}$, $266^{58}$, $285^{18}$, $304^{20}$, $323^{24}$, $342^{52}$, $361^{10}$, $380^{30}$, $399^{22}$, $418^{32}$, $437^{20}$, $456^{32}$, $475^{4}$, $494^{4}$, $513^{14}$, $532^{30}$, $551^{14}$, $570^{12}$, $608^{24}$, $627^{4}$, $646^{10}$, $665^{12}$, $684^{2}$, $703^{2}$, $722^{6}$, $741^{2}$, $760^{14}$, $779^{6}$, $798^{12}$, $817^{4}$, $836^{2}$, $855^{6}$, $874^{20}$, $912^{6}$, $950^{8}$, $969^{4}$, $1007^{4}$, $1026^{6}$, $1045^{2}$, $1064^{10}$, $1083^{6}$, $1102^{2}$, $1121^{4}$, $1159^{2}$, $1178^{2}$, $1197^{4}$, $1216^{4}$, $1235^{4}$, $1273^{4}$, $1311^{2}$, $1330^{8}$, $1349^{2}$, $1387^{2}$, $1444^{2}$, $1482^{2}$, $1501^{4}$, $1577^{2}$, $1615^{2}$, $1634^{4}$, $1653^{6}$, $1786^{2}$, $1824^{2}$, $1843^{2}$, $1862^{2}$, $1900^{2}$, $2014^{2}$, $2033^{4}$, $2128^{2}$, $2223^{2}$, $2261^{2}$, $2622^{2}$
\item $(h,k)=(1,3)$ : $1^{2}$, $2014^{2}$, $260129^{2}$
\item $(h,k)=(1,4)$ : $1^{2}$, $19^{2}$, $133^{2}$, $380^{2}$, $494^{2}$, $646^{2}$, $7581^{2}$, $51547^{2}$, $201343^{2}$
\item $(h,k)=(1,5)$ : $1^{2}$, $152^{2}$, $456^{2}$, $874^{2}$, $48925^{2}$, $211736^{2}$
\item $(h,k)=(1,6)$ : $1^{2}$, $4655^{2}$, $10678^{2}$, $31711^{2}$, $44973^{2}$, $170126^{2}$
\item $(h,k)=(1,7)$ : $1^{2}$, $19^{2}$, $152^{2}$, $190^{2}$, $90231^{2}$, $171551^{2}$
\item $(h,k)=(1,8)$ : $1^{2}$, $38^{4}$, $95^{4}$, $475^{2}$, $874^{2}$, $5719^{2}$, $9500^{2}$, $245309^{2}$
\item $(h,k)=(1,9)$ : $1^{2}$, $19^{2}$, $133^{2}$, $665^{2}$, $798^{2}$, $3097^{2}$, $257431^{2}$
\item $(h,k)=(1,10)$ : $1^{2}$, $37449^{14}$
\item $(h,k)=(1,11)$ : $1^{2}$, $38^{2}$, $76^{2}$, $95^{2}$, $261934^{2}$
\item $(h,k)=(1,12)$ : $1^{2}$, $228^{2}$, $1786^{2}$, $19247^{2}$, $28310^{2}$, $212572^{2}$
\item $(h,k)=(1,13)$ : $1^{2}$, $19^{2}$, $39^{38}$, $95^{2}$, $779^{38}$, $874^{2}$, $12103^{2}$, $48507^{2}$, $53941^{2}$, $64923^{2}$, $66139^{2}$
\item $(h,k)=(1,14)$ : $1^{2}$, $38^{2}$, $1216^{2}$, $28614^{2}$, $33079^{2}$, $199196^{2}$
\item $(h,k)=(1,15)$ : $1^{2}$, $57^{2}$, $760^{2}$, $29241^{2}$, $232085^{2}$
\item $(h,k)=(1,16)$ : $1^{2}$, $38^{2}$, $76^{2}$, $418^{2}$, $24966^{2}$, $236645^{2}$
\item $(h,k)=(1,17)$ : $1^{2}$, $54^{76}$, $57^{2}$, $60^{76}$, $123^{38}$, $126^{114}$, $171^{6}$, $219^{38}$, $240^{38}$, $342^{6}$, $399^{16}$, $456^{8}$, $513^{16}$, $570^{60}$, $684^{26}$, $741^{8}$, $798^{14}$, $969^{12}$, $1026^{36}$, $1140^{4}$, $1197^{6}$, $1254^{8}$, $1311^{6}$, $1368^{4}$, $1596^{6}$, $1653^{4}$, $1710^{4}$, $1824^{16}$, $1881^{4}$, $1938^{4}$, $1995^{4}$, $2109^{2}$, $2166^{2}$, $2280^{2}$, $2337^{4}$, $2394^{4}$, $2565^{2}$, $2622^{4}$, $2850^{2}$, $3021^{4}$, $3078^{2}$, $3135^{2}$, $3249^{4}$, $3591^{2}$, $3648^{2}$, $3705^{2}$, $3990^{2}$, $4503^{2}$, $4731^{2}$, $4902^{2}$, $4959^{4}$, $5358^{2}$, $5472^{2}$, $5529^{2}$, $6099^{2}$, $7866^{2}$
\item $(h,k)=(1,18)$ : $1^{2}$, $3^{174762}$
\item $(h,k)=(2,3)$ : $1^{2}$, $19^{2}$, $678^{38}$, $6574^{2}$, $242668^{2}$
\item $(h,k)=(2,4)$ : $1^{2}$, $8^{38}$, $19^{2}$, $57^{2}$, $152^{2}$, $190^{2}$, $266^{2}$, $1463^{2}$, $1786^{2}$, $2033^{2}$, $2413^{2}$, $3401^{2}$, $11191^{2}$, $32224^{2}$, $62510^{2}$, $67108^{2}$, $77178^{2}$
\item $(h,k)=(2,5)$ : $1^{2}$, $133^{2}$, $646^{2}$, $1300^{38}$, $92454^{2}$, $144210^{2}$
\item $(h,k)=(2,6)$ : $1^{2}$, $418^{2}$, $2033^{2}$, $2185^{2}$, $23560^{2}$, $82004^{2}$, $151943^{2}$
\item $(h,k)=(2,7)$ : $1^{2}$, $152^{2}$, $1102^{2}$, $4161^{2}$, $7714^{2}$, $123557^{2}$, $125457^{2}$
\item $(h,k)=(2,8)$ : $1^{2}$, $19^{2}$, $19931^{2}$, $72333^{2}$, $78261^{2}$, $91599^{2}$
\item $(h,k)=(2,9)$ : $1^{2}$, $95^{2}$, $513^{2}$, $106723^{2}$, $154812^{2}$
\item $(h,k)=(2,10)$ : $1^{2}$, $57^{2}$, $304^{2}$, $1159^{2}$, $1292^{2}$, $2907^{2}$, $35625^{2}$, $220799^{2}$
\item $(h,k)=(2,11)$ : $1^{2}$, $19^{4}$, $1292^{2}$, $23351^{2}$, $89338^{2}$, $148124^{2}$
\item $(h,k)=(2,12)$ : $1^{2}$, $19^{4}$, $38^{2}$, $76^{2}$, $133^{2}$, $182^{38}$, $209^{2}$, $589^{2}$, $257602^{2}$
\item $(h,k)=(2,13)$ : $1^{2}$, $38^{2}$, $57^{2}$, $1026^{2}$, $2565^{2}$, $79686^{2}$, $87191^{2}$, $91580^{2}$
\item $(h,k)=(2,14)$ : $1^{2}$, $627^{2}$, $5396^{2}$, $19000^{2}$, $54245^{2}$, $69578^{2}$, $113297^{2}$
\item $(h,k)=(2,15)$ : $1^{2}$, $19^{2}$, $380^{2}$, $1387^{2}$, $4997^{2}$, $33782^{2}$, $37791^{2}$, $61161^{2}$, $122626^{2}$
\item $(h,k)=(2,16)$ : $1^{2}$, $2052^{2}$, $5985^{2}$, $6232^{2}$, $9804^{2}$, $68913^{2}$, $169157^{2}$
\item $(h,k)=(2,17)$ : $1^{2}$, $3^{174762}$
\item $(h,k)=(2,18)$ : $1^{2}$, $54^{76}$, $57^{2}$, $60^{76}$, $123^{38}$, $126^{114}$, $171^{6}$, $219^{38}$, $240^{38}$, $342^{6}$, $399^{16}$, $456^{8}$, $513^{16}$, $570^{60}$, $684^{26}$, $741^{8}$, $798^{14}$, $969^{12}$, $1026^{36}$, $1140^{4}$, $1197^{6}$, $1254^{8}$, $1311^{6}$, $1368^{4}$, $1596^{6}$, $1653^{4}$, $1710^{4}$, $1824^{16}$, $1881^{4}$, $1938^{4}$, $1995^{4}$, $2109^{2}$, $2166^{2}$, $2280^{2}$, $2337^{4}$, $2394^{4}$, $2565^{2}$, $2622^{4}$, $2850^{2}$, $3021^{4}$, $3078^{2}$, $3135^{2}$, $3249^{4}$, $3591^{2}$, $3648^{2}$, $3705^{2}$, $3990^{2}$, $4503^{2}$, $4731^{2}$, $4902^{2}$, $4959^{4}$, $5358^{2}$, $5472^{2}$, $5529^{2}$, $6099^{2}$, $7866^{2}$
\item $(h,k)=(3,4)$ : $1^{2}$, $38^{2}$, $760^{2}$, $5529^{2}$, $255816^{2}$
\item $(h,k)=(3,5)$ : $1^{2}$, $95^{2}$, $247^{2}$, $703^{2}$, $261098^{2}$
\item $(h,k)=(3,6)$ : $1^{2}$, $38^{2}$, $95^{2}$, $513^{2}$, $760^{2}$, $17480^{2}$, $18259^{2}$, $224998^{2}$
\item $(h,k)=(3,7)$ : $1^{2}$, $38^{2}$, $7695^{2}$, $10843^{38}$, $22021^{2}$, $26372^{2}$
\item $(h,k)=(3,8)$ : $1^{2}$, $152^{2}$, $4256^{2}$, $257735^{2}$
\item $(h,k)=(3,9)$ : $1^{2}$, $133^{2}$, $342^{2}$, $1159^{2}$, $8820^{38}$, $21945^{2}$, $70984^{2}$
\item $(h,k)=(3,10)$ : $1^{2}$, $3781^{2}$, $5027^{38}$, $58178^{2}$, $104671^{2}$
\item $(h,k)=(3,11)$ : $1^{2}$, $19^{2}$, $38^{4}$, $285^{2}$, $646^{2}$, $1026^{2}$, $5054^{2}$, $7543^{2}$, $17670^{2}$, $229824^{2}$
\item $(h,k)=(3,12)$ : $1^{2}$, $38^{2}$, $1463^{2}$, $10963^{2}$, $14516^{2}$, $235163^{2}$
\item $(h,k)=(3,13)$ : $1^{2}$, $7^{38}$, $57^{2}$, $1254^{2}$, $12253^{38}$, $27892^{2}$
\item $(h,k)=(3,14)$ : $1^{2}$, $19^{2}$, $46^{38}$, $304^{2}$, $1064^{2}$, $2660^{2}$, $2983^{2}$, $254239^{2}$
\item $(h,k)=(3,15)$ : $1^{2}$, $262143^{2}$
\item $(h,k)=(3,16)$ : $1^{2}$, $3^{174762}$
\item $(h,k)=(3,17)$ : $1^{2}$, $2052^{2}$, $5985^{2}$, $6232^{2}$, $9804^{2}$, $68913^{2}$, $169157^{2}$
\item $(h,k)=(3,18)$ : $1^{2}$, $38^{2}$, $76^{2}$, $418^{2}$, $24966^{2}$, $236645^{2}$
\item $(h,k)=(4,5)$ : $1^{2}$, $228^{2}$, $528^{38}$, $17138^{2}$, $62415^{2}$, $172330^{2}$
\item $(h,k)=(4,6)$ : $1^{2}$, $798^{2}$, $931^{2}$, $1710^{2}$, $9253^{2}$, $10222^{2}$, $76703^{2}$, $162526^{2}$
\item $(h,k)=(4,7)$ : $1^{2}$, $456^{2}$, $5548^{2}$, $9443^{2}$, $77805^{2}$, $168891^{2}$
\item $(h,k)=(4,8)$ : $1^{2}$, $38^{4}$, $114^{4}$, $209^{2}$, $285^{2}$, $5111^{2}$, $9443^{2}$, $40223^{2}$, $47006^{2}$, $47842^{2}$, $111720^{2}$
\item $(h,k)=(4,9)$ : $1^{2}$, $19^{2}$, $38^{2}$, $57^{2}$, $190^{2}$, $17499^{2}$, $244340^{2}$
\item $(h,k)=(4,10)$ : $1^{2}$, $19^{4}$, $38^{2}$, $71763^{2}$, $190304^{2}$
\item $(h,k)=(4,11)$ : $1^{2}$, $38^{2}$, $4845^{2}$, $62662^{2}$, $194598^{2}$
\item $(h,k)=(4,12)$ : $1^{2}$, $95^{2}$, $133^{2}$, $323^{2}$, $2945^{2}$, $10963^{2}$, $247684^{2}$
\item $(h,k)=(4,13)$ : $1^{2}$, $15^{38}$, $19^{2}$, $252^{38}$, $589^{2}$, $1767^{2}$, $36024^{2}$, $218671^{2}$
\item $(h,k)=(4,14)$ : $1^{2}$, $57^{2}$, $171^{4}$, $1007^{2}$, $1615^{2}$, $9063^{2}$, $17366^{2}$, $27284^{2}$, $205409^{2}$
\item $(h,k)=(4,15)$ : $1^{2}$, $3^{174762}$
\item $(h,k)=(4,16)$ : $1^{2}$, $262143^{2}$
\item $(h,k)=(4,17)$ : $1^{2}$, $19^{2}$, $380^{2}$, $1387^{2}$, $4997^{2}$, $33782^{2}$, $37791^{2}$, $61161^{2}$, $122626^{2}$
\item $(h,k)=(4,18)$ : $1^{2}$, $57^{2}$, $760^{2}$, $29241^{2}$, $232085^{2}$
\item $(h,k)=(5,6)$ : $1^{2}$, $342^{2}$, $418^{2}$, $2375^{2}$, $9804^{2}$, $20444^{2}$, $31958^{2}$, $196802^{2}$
\item $(h,k)=(5,7)$ : $1^{2}$, $38^{2}$, $228^{2}$, $323^{2}$, $3192^{2}$, $5168^{2}$, $253194^{2}$
\item $(h,k)=(5,8)$ : $1^{2}$, $38^{2}$, $57^{2}$, $262048^{2}$
\item $(h,k)=(5,9)$ : $1^{2}$, $38^{2}$, $57^{2}$, $551^{2}$, $36062^{2}$, $225435^{2}$
\item $(h,k)=(5,10)$ : $1^{2}$, $19^{4}$, $38^{4}$, $114^{2}$, $190^{2}$, $247^{2}$, $817^{2}$, $893^{2}$, $2698^{2}$, $15048^{2}$, $44232^{2}$, $197790^{2}$
\item $(h,k)=(5,11)$ : $1^{2}$, $38^{2}$, $152^{2}$, $722^{2}$, $68172^{2}$, $193059^{2}$
\item $(h,k)=(5,12)$ : $1^{2}$, $19^{2}$, $76^{2}$, $171^{2}$, $342^{2}$, $608^{2}$, $1119^{38}$, $3667^{2}$, $10735^{2}$, $12806^{2}$, $52136^{2}$, $160322^{2}$
\item $(h,k)=(5,13)$ : $1^{2}$, $19^{2}$, $5795^{2}$, $11052^{38}$, $46341^{2}$
\item $(h,k)=(5,14)$ : $1^{2}$, $3^{174762}$
\item $(h,k)=(5,15)$ : $1^{2}$, $57^{2}$, $171^{4}$, $1007^{2}$, $1615^{2}$, $9063^{2}$, $17366^{2}$, $27284^{2}$, $205409^{2}$
\item $(h,k)=(5,16)$ : $1^{2}$, $19^{2}$, $46^{38}$, $304^{2}$, $1064^{2}$, $2660^{2}$, $2983^{2}$, $254239^{2}$
\item $(h,k)=(5,17)$ : $1^{2}$, $627^{2}$, $5396^{2}$, $19000^{2}$, $54245^{2}$, $69578^{2}$, $113297^{2}$
\item $(h,k)=(5,18)$ : $1^{2}$, $38^{2}$, $1216^{2}$, $28614^{2}$, $33079^{2}$, $199196^{2}$
\item $(h,k)=(6,7)$ : $1^{2}$, $38^{2}$, $95^{2}$, $152^{2}$, $380^{2}$, $475^{2}$, $11856^{2}$, $249147^{2}$
\item $(h,k)=(6,8)$ : $1^{2}$, $380^{2}$, $6650^{2}$, $13110^{2}$, $17955^{2}$, $24168^{2}$, $77273^{2}$, $122607^{2}$
\item $(h,k)=(6,9)$ : $1^{2}$, $209^{2}$, $228^{2}$, $1026^{2}$, $2831^{2}$, $257849^{2}$
\item $(h,k)=(6,10)$ : $1^{2}$, $7^{38}$, $38^{2}$, $41^{38}$, $133^{2}$, $37791^{2}$, $103645^{2}$, $119624^{2}$
\item $(h,k)=(6,11)$ : $1^{2}$, $38^{2}$, $95^{2}$, $22021^{2}$, $239989^{2}$
\item $(h,k)=(6,12)$ : $1^{2}$, $76^{2}$, $228^{2}$, $380^{2}$, $817^{2}$, $1121^{2}$, $1463^{2}$, $4180^{2}$, $253878^{2}$
\item $(h,k)=(6,13)$ : $1^{2}$, $3^{174762}$
\item $(h,k)=(6,14)$ : $1^{2}$, $19^{2}$, $5795^{2}$, $11052^{38}$, $46341^{2}$
\item $(h,k)=(6,15)$ : $1^{2}$, $15^{38}$, $19^{2}$, $252^{38}$, $589^{2}$, $1767^{2}$, $36024^{2}$, $218671^{2}$
\item $(h,k)=(6,16)$ : $1^{2}$, $7^{38}$, $57^{2}$, $1254^{2}$, $12253^{38}$, $27892^{2}$
\item $(h,k)=(6,17)$ : $1^{2}$, $38^{2}$, $57^{2}$, $1026^{2}$, $2565^{2}$, $79686^{2}$, $87191^{2}$, $91580^{2}$
\item $(h,k)=(6,18)$ : $1^{2}$, $19^{2}$, $39^{38}$, $95^{2}$, $779^{38}$, $874^{2}$, $12103^{2}$, $48507^{2}$, $53941^{2}$, $64923^{2}$, $66139^{2}$
\item $(h,k)=(7,8)$ : $1^{2}$, $114^{2}$, $1121^{2}$, $260908^{2}$
\item $(h,k)=(7,9)$ : $1^{2}$, $190^{2}$, $1045^{2}$, $260908^{2}$
\item $(h,k)=(7,10)$ : $1^{2}$, $19^{4}$, $57^{2}$, $3990^{2}$, $6365^{2}$, $24928^{2}$, $226765^{2}$
\item $(h,k)=(7,11)$ : $1^{2}$, $11^{38}$, $1843^{2}$, $3743^{2}$, $4040^{38}$, $36309^{2}$, $51319^{2}$, $91960^{2}$
\item $(h,k)=(7,12)$ : $1^{2}$, $3^{174762}$
\item $(h,k)=(7,13)$ : $1^{2}$, $76^{2}$, $228^{2}$, $380^{2}$, $817^{2}$, $1121^{2}$, $1463^{2}$, $4180^{2}$, $253878^{2}$
\item $(h,k)=(7,14)$ : $1^{2}$, $19^{2}$, $76^{2}$, $171^{2}$, $342^{2}$, $608^{2}$, $1119^{38}$, $3667^{2}$, $10735^{2}$, $12806^{2}$, $52136^{2}$, $160322^{2}$
\item $(h,k)=(7,15)$ : $1^{2}$, $95^{2}$, $133^{2}$, $323^{2}$, $2945^{2}$, $10963^{2}$, $247684^{2}$
\item $(h,k)=(7,16)$ : $1^{2}$, $38^{2}$, $1463^{2}$, $10963^{2}$, $14516^{2}$, $235163^{2}$
\item $(h,k)=(7,17)$ : $1^{2}$, $19^{4}$, $38^{2}$, $76^{2}$, $133^{2}$, $182^{38}$, $209^{2}$, $589^{2}$, $257602^{2}$
\item $(h,k)=(7,18)$ : $1^{2}$, $228^{2}$, $1786^{2}$, $19247^{2}$, $28310^{2}$, $212572^{2}$
\item $(h,k)=(8,9)$ : $1^{2}$, $152^{2}$, $1235^{2}$, $19722^{2}$, $108110^{2}$, $132924^{2}$
\item $(h,k)=(8,10)$ : $1^{2}$, $57^{2}$, $171^{2}$, $1121^{2}$, $4294^{2}$, $5092^{2}$, $14535^{2}$, $236873^{2}$
\item $(h,k)=(8,11)$ : $1^{2}$, $3^{174762}$
\item $(h,k)=(8,12)$ : $1^{2}$, $11^{38}$, $1843^{2}$, $3743^{2}$, $4040^{38}$, $36309^{2}$, $51319^{2}$, $91960^{2}$
\item $(h,k)=(8,13)$ : $1^{2}$, $38^{2}$, $95^{2}$, $22021^{2}$, $239989^{2}$
\item $(h,k)=(8,14)$ : $1^{2}$, $38^{2}$, $152^{2}$, $722^{2}$, $68172^{2}$, $193059^{2}$
\item $(h,k)=(8,15)$ : $1^{2}$, $38^{2}$, $4845^{2}$, $62662^{2}$, $194598^{2}$
\item $(h,k)=(8,16)$ : $1^{2}$, $19^{2}$, $38^{4}$, $285^{2}$, $646^{2}$, $1026^{2}$, $5054^{2}$, $7543^{2}$, $17670^{2}$, $229824^{2}$
\item $(h,k)=(8,17)$ : $1^{2}$, $19^{4}$, $1292^{2}$, $23351^{2}$, $89338^{2}$, $148124^{2}$
\item $(h,k)=(8,18)$ : $1^{2}$, $38^{2}$, $76^{2}$, $95^{2}$, $261934^{2}$
\item $(h,k)=(9,10)$ : $1^{2}$, $3^{174762}$
\item $(h,k)=(9,11)$ : $1^{2}$, $57^{2}$, $171^{2}$, $1121^{2}$, $4294^{2}$, $5092^{2}$, $14535^{2}$, $236873^{2}$
\item $(h,k)=(9,12)$ : $1^{2}$, $19^{4}$, $57^{2}$, $3990^{2}$, $6365^{2}$, $24928^{2}$, $226765^{2}$
\item $(h,k)=(9,13)$ : $1^{2}$, $7^{38}$, $38^{2}$, $41^{38}$, $133^{2}$, $37791^{2}$, $103645^{2}$, $119624^{2}$
\item $(h,k)=(9,14)$ : $1^{2}$, $19^{4}$, $38^{4}$, $114^{2}$, $190^{2}$, $247^{2}$, $817^{2}$, $893^{2}$, $2698^{2}$, $15048^{2}$, $44232^{2}$, $197790^{2}$
\item $(h,k)=(9,15)$ : $1^{2}$, $19^{4}$, $38^{2}$, $71763^{2}$, $190304^{2}$
\item $(h,k)=(9,16)$ : $1^{2}$, $3781^{2}$, $5027^{38}$, $58178^{2}$, $104671^{2}$
\item $(h,k)=(9,17)$ : $1^{2}$, $57^{2}$, $304^{2}$, $1159^{2}$, $1292^{2}$, $2907^{2}$, $35625^{2}$, $220799^{2}$
\item $(h,k)=(9,18)$ : $1^{2}$, $37449^{14}$
\item $(h,k)=(10,11)$ : $1^{2}$, $152^{2}$, $1235^{2}$, $19722^{2}$, $108110^{2}$, $132924^{2}$
\item $(h,k)=(10,12)$ : $1^{2}$, $190^{2}$, $1045^{2}$, $260908^{2}$
\item $(h,k)=(10,13)$ : $1^{2}$, $209^{2}$, $228^{2}$, $1026^{2}$, $2831^{2}$, $257849^{2}$
\item $(h,k)=(10,14)$ : $1^{2}$, $38^{2}$, $57^{2}$, $551^{2}$, $36062^{2}$, $225435^{2}$
\item $(h,k)=(10,15)$ : $1^{2}$, $19^{2}$, $38^{2}$, $57^{2}$, $190^{2}$, $17499^{2}$, $244340^{2}$
\item $(h,k)=(10,16)$ : $1^{2}$, $133^{2}$, $342^{2}$, $1159^{2}$, $8820^{38}$, $21945^{2}$, $70984^{2}$
\item $(h,k)=(10,17)$ : $1^{2}$, $95^{2}$, $513^{2}$, $106723^{2}$, $154812^{2}$
\item $(h,k)=(10,18)$ : $1^{2}$, $19^{2}$, $133^{2}$, $665^{2}$, $798^{2}$, $3097^{2}$, $257431^{2}$
\item $(h,k)=(11,12)$ : $1^{2}$, $114^{2}$, $1121^{2}$, $260908^{2}$
\item $(h,k)=(11,13)$ : $1^{2}$, $380^{2}$, $6650^{2}$, $13110^{2}$, $17955^{2}$, $24168^{2}$, $77273^{2}$, $122607^{2}$
\item $(h,k)=(11,14)$ : $1^{2}$, $38^{2}$, $57^{2}$, $262048^{2}$
\item $(h,k)=(11,15)$ : $1^{2}$, $38^{4}$, $114^{4}$, $209^{2}$, $285^{2}$, $5111^{2}$, $9443^{2}$, $40223^{2}$, $47006^{2}$, $47842^{2}$, $111720^{2}$
\item $(h,k)=(11,16)$ : $1^{2}$, $152^{2}$, $4256^{2}$, $257735^{2}$
\item $(h,k)=(11,17)$ : $1^{2}$, $19^{2}$, $19931^{2}$, $72333^{2}$, $78261^{2}$, $91599^{2}$
\item $(h,k)=(11,18)$ : $1^{2}$, $38^{4}$, $95^{4}$, $475^{2}$, $874^{2}$, $5719^{2}$, $9500^{2}$, $245309^{2}$
\item $(h,k)=(12,13)$ : $1^{2}$, $38^{2}$, $95^{2}$, $152^{2}$, $380^{2}$, $475^{2}$, $11856^{2}$, $249147^{2}$
\item $(h,k)=(12,14)$ : $1^{2}$, $38^{2}$, $228^{2}$, $323^{2}$, $3192^{2}$, $5168^{2}$, $253194^{2}$
\item $(h,k)=(12,15)$ : $1^{2}$, $456^{2}$, $5548^{2}$, $9443^{2}$, $77805^{2}$, $168891^{2}$
\item $(h,k)=(12,16)$ : $1^{2}$, $38^{2}$, $7695^{2}$, $10843^{38}$, $22021^{2}$, $26372^{2}$
\item $(h,k)=(12,17)$ : $1^{2}$, $152^{2}$, $1102^{2}$, $4161^{2}$, $7714^{2}$, $123557^{2}$, $125457^{2}$
\item $(h,k)=(12,18)$ : $1^{2}$, $19^{2}$, $152^{2}$, $190^{2}$, $90231^{2}$, $171551^{2}$
\item $(h,k)=(13,14)$ : $1^{2}$, $342^{2}$, $418^{2}$, $2375^{2}$, $9804^{2}$, $20444^{2}$, $31958^{2}$, $196802^{2}$
\item $(h,k)=(13,15)$ : $1^{2}$, $798^{2}$, $931^{2}$, $1710^{2}$, $9253^{2}$, $10222^{2}$, $76703^{2}$, $162526^{2}$
\item $(h,k)=(13,16)$ : $1^{2}$, $38^{2}$, $95^{2}$, $513^{2}$, $760^{2}$, $17480^{2}$, $18259^{2}$, $224998^{2}$
\item $(h,k)=(13,17)$ : $1^{2}$, $418^{2}$, $2033^{2}$, $2185^{2}$, $23560^{2}$, $82004^{2}$, $151943^{2}$
\item $(h,k)=(13,18)$ : $1^{2}$, $4655^{2}$, $10678^{2}$, $31711^{2}$, $44973^{2}$, $170126^{2}$
\item $(h,k)=(14,15)$ : $1^{2}$, $228^{2}$, $528^{38}$, $17138^{2}$, $62415^{2}$, $172330^{2}$
\item $(h,k)=(14,16)$ : $1^{2}$, $95^{2}$, $247^{2}$, $703^{2}$, $261098^{2}$
\item $(h,k)=(14,17)$ : $1^{2}$, $133^{2}$, $646^{2}$, $1300^{38}$, $92454^{2}$, $144210^{2}$
\item $(h,k)=(14,18)$ : $1^{2}$, $152^{2}$, $456^{2}$, $874^{2}$, $48925^{2}$, $211736^{2}$
\item $(h,k)=(15,16)$ : $1^{2}$, $38^{2}$, $760^{2}$, $5529^{2}$, $255816^{2}$
\item $(h,k)=(15,17)$ : $1^{2}$, $8^{38}$, $19^{2}$, $57^{2}$, $152^{2}$, $190^{2}$, $266^{2}$, $1463^{2}$, $1786^{2}$, $2033^{2}$, $2413^{2}$, $3401^{2}$, $11191^{2}$, $32224^{2}$, $62510^{2}$, $67108^{2}$, $77178^{2}$
\item $(h,k)=(15,18)$ : $1^{2}$, $19^{2}$, $133^{2}$, $380^{2}$, $494^{2}$, $646^{2}$, $7581^{2}$, $51547^{2}$, $201343^{2}$
\item $(h,k)=(16,17)$ : $1^{2}$, $19^{2}$, $678^{38}$, $6574^{2}$, $242668^{2}$
\item $(h,k)=(16,18)$ : $1^{2}$, $2014^{2}$, $260129^{2}$
\item $(h,k)=(17,18)$ : $1^{2}$, $14^{76}$, $15^{38}$, $18^{76}$, $19^{4}$, $20^{152}$, $27^{114}$, $31^{38}$, $34^{76}$, $38^{4}$, $41^{38}$, $42^{114}$, $53^{38}$, $54^{38}$, $57^{12}$, $70^{38}$, $73^{38}$, $76^{2}$, $80^{38}$, $95^{10}$, $114^{24}$, $133^{44}$, $152^{60}$, $171^{38}$, $190^{104}$, $209^{20}$, $228^{60}$, $247^{8}$, $266^{58}$, $285^{18}$, $304^{20}$, $323^{24}$, $342^{52}$, $361^{10}$, $380^{30}$, $399^{22}$, $418^{32}$, $437^{20}$, $456^{32}$, $475^{4}$, $494^{4}$, $513^{14}$, $532^{30}$, $551^{14}$, $570^{12}$, $608^{24}$, $627^{4}$, $646^{10}$, $665^{12}$, $684^{2}$, $703^{2}$, $722^{6}$, $741^{2}$, $760^{14}$, $779^{6}$, $798^{12}$, $817^{4}$, $836^{2}$, $855^{6}$, $874^{20}$, $912^{6}$, $950^{8}$, $969^{4}$, $1007^{4}$, $1026^{6}$, $1045^{2}$, $1064^{10}$, $1083^{6}$, $1102^{2}$, $1121^{4}$, $1159^{2}$, $1178^{2}$, $1197^{4}$, $1216^{4}$, $1235^{4}$, $1273^{4}$, $1311^{2}$, $1330^{8}$, $1349^{2}$, $1387^{2}$, $1444^{2}$, $1482^{2}$, $1501^{4}$, $1577^{2}$, $1615^{2}$, $1634^{4}$, $1653^{6}$, $1786^{2}$, $1824^{2}$, $1843^{2}$, $1862^{2}$, $1900^{2}$, $2014^{2}$, $2033^{4}$, $2128^{2}$, $2223^{2}$, $2261^{2}$, $2622^{2}$
\end{itemize}

\section{Results for $n=20$}
\begin{itemize}
\item $(h,k)=(1,3)$ : $1^{4}$, $3^{4}$, $7^{10}$, $10^{4}$, $12^{10}$, $14^{10}$, $15^{2}$, $18^{20}$, $20^{20}$, $23^{20}$, $39^{40}$, $42^{20}$, $43^{20}$, $45^{20}$, $52^{20}$, $65^{12}$, $94^{20}$, $104^{20}$, $116^{20}$, $124^{20}$, $132^{20}$, $150^{4}$, $191^{20}$, $196^{20}$, $217^{20}$, $220^{2}$, $258^{20}$, $272^{20}$, $291^{20}$, $309^{20}$, $357^{20}$, $387^{20}$, $407^{20}$, $488^{20}$, $602^{20}$, $669^{20}$, $825^{20}$, $854^{20}$, $868^{20}$, $871^{20}$, $883^{20}$, $947^{20}$, $987^{20}$, $990^{20}$, $1010^{20}$, $1011^{20}$, $1020^{20}$, $1102^{20}$, $1128^{20}$, $1180^{20}$, $1288^{20}$, $1315^{8}$, $1383^{20}$, $1407^{20}$, $1453^{20}$, $1514^{20}$, $1565^{20}$, $1999^{20}$, $2216^{20}$, $2333^{20}$, $3543^{20}$, $3947^{20}$, $4277^{20}$, $5655^{20}$
\item $(h,k)=(1,7)$ : $1^{4}$, $3^{14}$, $5^{2}$, $8^{20}$, $10^{2}$, $12^{20}$, $15^{8}$, $24^{20}$, $30^{2}$, $37^{20}$, $45^{4}$, $47^{20}$, $60^{8}$, $65^{8}$, $77^{20}$, $84^{20}$, $87^{20}$, $103^{20}$, $142^{20}$, $174^{20}$, $229^{20}$, $230^{4}$, $252^{20}$, $257^{20}$, $265^{20}$, $280^{2}$, $291^{20}$, $311^{20}$, $323^{20}$, $332^{20}$, $376^{20}$, $396^{20}$, $406^{20}$, $531^{20}$, $546^{20}$, $556^{20}$, $588^{20}$, $618^{20}$, $640^{4}$, $656^{20}$, $703^{20}$, $780^{20}$, $804^{20}$, $817^{20}$, $978^{20}$, $1058^{20}$, $1081^{20}$, $1111^{20}$, $1163^{20}$, $1258^{20}$, $1452^{20}$, $1508^{20}$, $1509^{20}$, $1665^{20}$, $1769^{20}$, $1841^{20}$, $2138^{20}$, $2170^{20}$, $2250^{20}$, $2299^{20}$, $2335^{20}$, $2696^{20}$, $3195^{20}$, $3488^{20}$, $3561^{20}$, $7980^{2}$
\item $(h,k)=(1,9)$ : $1^{16}$, $3^{340}$, $4^{20}$, $5^{8}$, $9^{20}$, $11^{20}$, $12^{20}$, $14^{40}$, $25^{8}$, $29^{20}$, $32^{20}$, $33^{20}$, $35^{8}$, $41^{20}$, $42^{20}$, $43^{20}$, $52^{20}$, $56^{20}$, $60^{40}$, $74^{20}$, $78^{20}$, $88^{20}$, $99^{20}$, $105^{20}$, $114^{20}$, $120^{24}$, $151^{20}$, $154^{20}$, $161^{20}$, $165^{20}$, $182^{20}$, $198^{20}$, $211^{20}$, $217^{20}$, $234^{20}$, $237^{20}$, $238^{20}$, $242^{20}$, $245^{8}$, $250^{20}$, $257^{20}$, $261^{20}$, $267^{20}$, $287^{20}$, $303^{20}$, $319^{20}$, $352^{20}$, $363^{20}$, $371^{20}$, $400^{20}$, $401^{20}$, $428^{20}$, $460^{8}$, $521^{20}$, $617^{20}$, $643^{20}$, $831^{20}$, $924^{40}$, $7005^{40}$, $9450^{8}$, $20810^{8}$, $34080^{8}$
\item $(h,k)=(1,11)$ : $1^{1024}$, $3^{444}$, $4^{840}$, $5^{220}$, $6^{380}$, $7^{160}$, $8^{200}$, $9^{160}$, $10^{292}$, $12^{180}$, $13^{80}$, $14^{60}$, $15^{60}$, $16^{80}$, $17^{80}$, $18^{140}$, $20^{130}$, $21^{40}$, $22^{140}$, $23^{80}$, $24^{20}$, $26^{80}$, $27^{80}$, $28^{120}$, $29^{40}$, $30^{36}$, $31^{120}$, $32^{30}$, $33^{40}$, $34^{40}$, $36^{120}$, $37^{40}$, $38^{60}$, $39^{120}$, $40^{6}$, $41^{40}$, $42^{40}$, $44^{40}$, $45^{80}$, $47^{40}$, $48^{80}$, $50^{40}$, $56^{60}$, $57^{80}$, $59^{40}$, $60^{40}$, $62^{80}$, $64^{80}$, $65^{40}$, $66^{80}$, $70^{24}$, $74^{40}$, $77^{40}$, $82^{40}$, $83^{80}$, $85^{40}$, $88^{60}$, $90^{20}$, $92^{20}$, $95^{20}$, $96^{20}$, $98^{20}$, $100^{42}$, $103^{80}$, $104^{40}$, $107^{40}$, $108^{60}$, $109^{40}$, $110^{20}$, $112^{60}$, $113^{40}$, $116^{40}$, $118^{20}$, $121^{60}$, $122^{40}$, $128^{50}$, $129^{40}$, $134^{20}$, $138^{20}$, $140^{20}$, $142^{80}$, $144^{20}$, $145^{20}$, $146^{20}$, $147^{20}$, $150^{40}$, $154^{60}$, $155^{40}$, $156^{80}$, $158^{40}$, $160^{100}$, $162^{80}$, $164^{20}$, $166^{20}$, $167^{40}$, $172^{10}$, $176^{20}$, $177^{20}$, $184^{20}$, $191^{20}$, $194^{60}$, $199^{40}$, $202^{20}$, $206^{20}$, $209^{40}$, $216^{40}$, $228^{40}$, $233^{20}$, $236^{40}$, $240^{8}$, $242^{40}$, $248^{60}$, $252^{40}$, $253^{40}$, $254^{40}$, $258^{20}$, $260^{40}$, $262^{40}$, $264^{30}$, $269^{40}$, $281^{20}$, $282^{20}$, $289^{20}$, $291^{20}$, $293^{40}$, $296^{20}$, $302^{20}$, $308^{20}$, $312^{40}$, $313^{20}$, $316^{20}$, $326^{20}$, $329^{20}$, $330^{20}$, $341^{20}$, $344^{20}$, $348^{20}$, $355^{20}$, $356^{10}$, $358^{20}$, $361^{20}$, $367^{20}$, $377^{20}$, $379^{20}$, $382^{20}$, $388^{20}$, $394^{20}$, $399^{20}$, $407^{20}$, $412^{20}$, $423^{20}$, $433^{20}$, $438^{20}$, $441^{40}$, $443^{20}$, $444^{10}$, $445^{40}$, $446^{20}$, $448^{40}$, $468^{20}$, $471^{40}$, $473^{20}$, $477^{20}$, $484^{20}$, $485^{20}$, $489^{20}$, $497^{20}$, $501^{20}$, $503^{20}$, $504^{20}$, $511^{20}$, $514^{20}$, $515^{20}$, $519^{20}$, $520^{20}$, $521^{20}$, $527^{20}$, $533^{20}$, $535^{20}$
\item $(h,k)=(1,13)$ : $1^{16}$, $7^{10}$, $10^{4}$, $12^{10}$, $14^{10}$, $15^{2}$, $16^{10}$, $18^{20}$, $22^{20}$, $23^{20}$, $30^{20}$, $39^{20}$, $40^{8}$, $48^{20}$, $49^{20}$, $50^{20}$, $56^{20}$, $59^{20}$, $62^{20}$, $64^{20}$, $65^{4}$, $66^{20}$, $70^{20}$, $73^{20}$, $80^{20}$, $82^{20}$, $84^{20}$, $85^{20}$, $88^{20}$, $91^{20}$, $95^{8}$, $100^{20}$, $109^{20}$, $113^{20}$, $115^{8}$, $116^{20}$, $117^{20}$, $127^{20}$, $140^{20}$, $143^{20}$, $145^{20}$, $146^{20}$, $174^{20}$, $177^{20}$, $179^{20}$, $188^{20}$, $190^{20}$, $195^{20}$, $205^{20}$, $215^{8}$, $222^{20}$, $225^{20}$, $227^{20}$, $274^{20}$, $278^{20}$, $285^{8}$, $313^{20}$, $335^{20}$, $341^{20}$, $359^{20}$, $366^{20}$, $414^{20}$, $524^{40}$, $647^{20}$, $685^{20}$, $805^{8}$, $1120^{8}$, $1326^{20}$, $1536^{40}$, $1585^{8}$, $4875^{8}$, $6700^{2}$, $9340^{8}$, $14790^{8}$, $16355^{8}$, $17565^{8}$, $27270^{8}$
\item $(h,k)=(1,17)$ : $1^{16}$, $3^{10}$, $5^{2}$, $7^{20}$, $10^{26}$, $12^{20}$, $13^{40}$, $20^{20}$, $21^{20}$, $24^{40}$, $25^{28}$, $30^{2}$, $40^{2}$, $44^{20}$, $45^{4}$, $48^{20}$, $49^{20}$, $50^{20}$, $52^{40}$, $55^{20}$, $57^{20}$, $60^{2}$, $65^{20}$, $66^{20}$, $75^{20}$, $76^{20}$, $85^{20}$, $87^{20}$, $91^{20}$, $103^{20}$, $116^{20}$, $123^{20}$, $140^{20}$, $148^{20}$, $150^{40}$, $151^{20}$, $169^{20}$, $171^{20}$, $174^{20}$, $178^{20}$, $180^{20}$, $184^{20}$, $185^{20}$, $218^{20}$, $260^{20}$, $273^{20}$, $292^{20}$, $309^{20}$, $326^{20}$, $328^{20}$, $337^{20}$, $390^{8}$, $439^{20}$, $455^{20}$, $487^{20}$, $625^{8}$, $739^{20}$, $808^{20}$, $970^{8}$, $1050^{8}$, $1785^{20}$, $6350^{8}$, $7095^{8}$, $8805^{8}$, $11500^{8}$, $16680^{8}$, $16865^{8}$, $34400^{8}$
\item $(h,k)=(1,19)$ : $1^{4}$, $3^{349524}$
\item $(h,k)=(3,7)$ : $1^{16}$, $3^{340}$, $4^{20}$, $14^{40}$, $15^{8}$, $21^{20}$, $34^{20}$, $37^{20}$, $45^{20}$, $47^{20}$, $48^{20}$, $51^{20}$, $61^{20}$, $66^{20}$, $72^{20}$, $76^{20}$, $79^{20}$, $80^{2}$, $101^{20}$, $106^{20}$, $113^{20}$, $114^{20}$, $118^{20}$, $119^{20}$, $127^{20}$, $131^{20}$, $134^{40}$, $136^{20}$, $148^{20}$, $151^{20}$, $159^{20}$, $161^{20}$, $164^{20}$, $177^{20}$, $185^{8}$, $187^{20}$, $202^{20}$, $214^{20}$, $217^{20}$, $235^{8}$, $237^{20}$, $242^{20}$, $250^{20}$, $251^{20}$, $279^{20}$, $280^{8}$, $294^{20}$, $311^{20}$, $327^{20}$, $338^{20}$, $340^{8}$, $346^{20}$, $354^{20}$, $377^{40}$, $390^{8}$, $430^{20}$, $440^{4}$, $466^{20}$, $471^{20}$, $514^{20}$, $526^{20}$, $849^{20}$, $2741^{40}$, $4325^{40}$, $6921^{40}$, $10855^{8}$, $20915^{8}$
\item $(h,k)=(3,9)$ : $1^{4}$, $3^{14}$, $5^{10}$, $10^{2}$, $12^{40}$, $20^{2}$, $24^{20}$, $30^{2}$, $41^{20}$, $45^{4}$, $73^{20}$, $99^{20}$, $102^{20}$, $105^{20}$, $116^{20}$, $120^{20}$, $123^{20}$, $209^{20}$, $243^{20}$, $256^{20}$, $258^{20}$, $266^{20}$, $368^{20}$, $544^{20}$, $553^{20}$, $606^{20}$, $623^{20}$, $629^{20}$, $633^{20}$, $724^{20}$, $764^{20}$, $785^{20}$, $829^{20}$, $846^{20}$, $851^{20}$, $879^{20}$, $910^{20}$, $940^{20}$, $970^{20}$, $1023^{20}$, $1194^{20}$, $1246^{20}$, $1264^{20}$, $1309^{20}$, $1337^{20}$, $1347^{20}$, $1431^{20}$, $1639^{20}$, $1691^{20}$, $1886^{20}$, $1917^{20}$, $1983^{20}$, $2021^{20}$, $2086^{20}$, $2175^{20}$, $2434^{20}$, $2705^{20}$, $2719^{20}$, $4489^{20}$
\item $(h,k)=(3,11)$ : $1^{16}$, $5^{16}$, $7^{50}$, $8^{20}$, $10^{12}$, $12^{10}$, $14^{10}$, $15^{18}$, $18^{20}$, $19^{20}$, $21^{40}$, $24^{20}$, $25^{20}$, $31^{20}$, $37^{20}$, $45^{8}$, $58^{40}$, $59^{20}$, $60^{2}$, $61^{20}$, $62^{20}$, $65^{4}$, $69^{20}$, $70^{40}$, $83^{20}$, $89^{20}$, $90^{40}$, $92^{20}$, $97^{20}$, $101^{20}$, $102^{40}$, $109^{20}$, $121^{20}$, $133^{20}$, $142^{20}$, $152^{20}$, $155^{20}$, $162^{20}$, $173^{20}$, $187^{20}$, $191^{20}$, $198^{40}$, $204^{20}$, $220^{8}$, $223^{20}$, $230^{20}$, $237^{20}$, $256^{20}$, $287^{20}$, $300^{20}$, $307^{20}$, $308^{20}$, $310^{20}$, $328^{20}$, $375^{20}$, $408^{20}$, $424^{20}$, $428^{20}$, $452^{20}$, $495^{20}$, $525^{20}$, $545^{8}$, $559^{20}$, $629^{20}$, $860^{2}$, $980^{8}$, $2964^{40}$, $4105^{8}$, $5660^{8}$, $6535^{8}$, $17855^{8}$, $19000^{8}$, $34000^{8}$
\item $(h,k)=(3,13)$ : $1^{1024}$, $3^{284}$, $4^{320}$, $5^{296}$, $6^{340}$, $7^{80}$, $8^{90}$, $9^{40}$, $10^{240}$, $11^{120}$, $12^{130}$, $13^{100}$, $14^{100}$, $15^{160}$, $16^{160}$, $17^{40}$, $18^{80}$, $19^{120}$, $20^{100}$, $21^{100}$, $22^{120}$, $24^{140}$, $25^{80}$, $26^{40}$, $27^{40}$, $28^{60}$, $30^{60}$, $31^{40}$, $32^{60}$, $33^{120}$, $34^{20}$, $35^{40}$, $36^{130}$, $37^{40}$, $38^{40}$, $40^{60}$, $41^{40}$, $42^{20}$, $45^{88}$, $46^{80}$, $47^{40}$, $49^{60}$, $50^{4}$, $51^{40}$, $52^{40}$, $55^{40}$, $59^{40}$, $60^{20}$, $64^{60}$, $65^{40}$, $68^{20}$, $69^{20}$, $70^{40}$, $74^{40}$, $76^{60}$, $77^{20}$, $83^{40}$, $86^{40}$, $87^{40}$, $88^{10}$, $89^{40}$, $92^{40}$, $94^{20}$, $96^{40}$, $98^{40}$, $100^{140}$, $102^{40}$, $108^{60}$, $110^{20}$, $114^{20}$, $119^{40}$, $122^{40}$, $123^{20}$, $126^{20}$, $128^{40}$, $132^{20}$, $138^{20}$, $139^{20}$, $140^{20}$, $144^{20}$, $147^{20}$, $150^{20}$, $152^{20}$, $154^{60}$, $159^{40}$, $162^{80}$, $165^{40}$, $167^{60}$, $169^{60}$, $172^{20}$, $181^{20}$, $184^{40}$, $186^{60}$, $190^{40}$, $192^{20}$, $193^{60}$, $198^{20}$, $200^{40}$, $205^{40}$, $210^{40}$, $220^{10}$, $221^{20}$, $222^{20}$, $223^{20}$, $226^{40}$, $227^{40}$, $228^{40}$, $238^{40}$, $240^{20}$, $241^{20}$, $242^{20}$, $243^{40}$, $244^{80}$, $245^{20}$, $248^{40}$, $250^{40}$, $256^{40}$, $258^{60}$, $260^{30}$, $266^{40}$, $270^{20}$, $274^{20}$, $284^{30}$, $285^{20}$, $308^{20}$, $309^{40}$, $310^{20}$, $318^{20}$, $330^{20}$, $346^{20}$, $360^{20}$, $375^{20}$, $377^{20}$, $378^{20}$, $386^{20}$, $388^{20}$, $393^{20}$, $396^{20}$, $398^{20}$, $400^{20}$, $401^{20}$, $429^{20}$, $431^{20}$, $433^{20}$, $439^{20}$, $450^{40}$, $451^{20}$, $452^{20}$, $453^{20}$, $455^{40}$, $460^{30}$, $465^{20}$, $468^{20}$, $469^{20}$, $470^{4}$, $473^{20}$, $475^{20}$, $477^{20}$, $480^{10}$, $483^{20}$, $484^{20}$, $486^{20}$, $489^{40}$, $490^{20}$, $491^{20}$, $493^{20}$, $495^{40}$, $500^{20}$, $503^{20}$, $521^{20}$, $523^{20}$, $525^{20}$, $527^{20}$, $528^{20}$, $531^{20}$, $532^{20}$, $535^{40}$
\item $(h,k)=(3,17)$ : $1^{4}$, $3^{349524}$
\item $(h,k)=(3,19)$ : $1^{16}$, $3^{10}$, $5^{2}$, $7^{20}$, $10^{26}$, $12^{20}$, $13^{40}$, $20^{20}$, $21^{20}$, $24^{40}$, $25^{28}$, $30^{2}$, $40^{2}$, $44^{20}$, $45^{4}$, $48^{20}$, $49^{20}$, $50^{20}$, $52^{40}$, $55^{20}$, $57^{20}$, $60^{2}$, $65^{20}$, $66^{20}$, $75^{20}$, $76^{20}$, $85^{20}$, $87^{20}$, $91^{20}$, $103^{20}$, $116^{20}$, $123^{20}$, $140^{20}$, $148^{20}$, $150^{40}$, $151^{20}$, $169^{20}$, $171^{20}$, $174^{20}$, $178^{20}$, $180^{20}$, $184^{20}$, $185^{20}$, $218^{20}$, $260^{20}$, $273^{20}$, $292^{20}$, $309^{20}$, $326^{20}$, $328^{20}$, $337^{20}$, $390^{8}$, $439^{20}$, $455^{20}$, $487^{20}$, $625^{8}$, $739^{20}$, $808^{20}$, $970^{8}$, $1050^{8}$, $1785^{20}$, $6350^{8}$, $7095^{8}$, $8805^{8}$, $11500^{8}$, $16680^{8}$, $16865^{8}$, $34400^{8}$
\item $(h,k)=(7,9)$ : $1^{4}$, $3^{4}$, $7^{10}$, $10^{4}$, $12^{10}$, $14^{10}$, $15^{2}$, $16^{20}$, $18^{20}$, $19^{20}$, $40^{12}$, $46^{20}$, $59^{20}$, $65^{4}$, $76^{20}$, $91^{20}$, $105^{20}$, $107^{20}$, $120^{2}$, $137^{20}$, $171^{20}$, $199^{20}$, $243^{20}$, $271^{20}$, $293^{20}$, $297^{20}$, $328^{20}$, $340^{4}$, $360^{4}$, $374^{20}$, $376^{20}$, $384^{20}$, $442^{20}$, $452^{20}$, $457^{20}$, $485^{20}$, $497^{20}$, $561^{20}$, $580^{20}$, $625^{20}$, $670^{20}$, $694^{20}$, $723^{20}$, $735^{20}$, $850^{20}$, $930^{20}$, $1039^{20}$, $1112^{20}$, $1168^{20}$, $1189^{20}$, $1191^{20}$, $1488^{20}$, $1503^{20}$, $1514^{20}$, $1534^{20}$, $1802^{20}$, $1883^{20}$, $1976^{20}$, $2097^{20}$, $2111^{20}$, $3200^{20}$, $3256^{20}$, $3921^{20}$, $4492^{20}$, $8580^{8}$
\item $(h,k)=(7,11)$ : $1^{16}$, $3^{10}$, $5^{26}$, $8^{40}$, $9^{20}$, $10^{22}$, $12^{20}$, $14^{20}$, $16^{20}$, $17^{20}$, $18^{20}$, $23^{20}$, $24^{20}$, $25^{40}$, $27^{20}$, $29^{20}$, $30^{2}$, $39^{20}$, $40^{8}$, $41^{40}$, $43^{20}$, $44^{20}$, $45^{4}$, $50^{4}$, $66^{20}$, $70^{4}$, $72^{20}$, $76^{20}$, $99^{20}$, $100^{20}$, $105^{20}$, $116^{20}$, $150^{20}$, $151^{20}$, $154^{20}$, $163^{20}$, $167^{20}$, $169^{20}$, $176^{20}$, $180^{20}$, $182^{20}$, $197^{20}$, $203^{20}$, $212^{20}$, $235^{8}$, $242^{20}$, $275^{20}$, $281^{20}$, $283^{20}$, $291^{20}$, $295^{20}$, $302^{20}$, $306^{20}$, $314^{20}$, $337^{20}$, $347^{20}$, $408^{20}$, $464^{20}$, $484^{20}$, $584^{20}$, $628^{20}$, $704^{20}$, $723^{40}$, $827^{20}$, $1600^{8}$, $1825^{8}$, $1898^{40}$, $5505^{8}$, $10875^{8}$, $14305^{8}$, $23820^{8}$, $33190^{8}$
\item $(h,k)=(7,13)$ : $1^{4}$, $3^{349524}$
\item $(h,k)=(7,17)$ : $1^{1024}$, $3^{284}$, $4^{320}$, $5^{296}$, $6^{340}$, $7^{80}$, $8^{90}$, $9^{40}$, $10^{240}$, $11^{120}$, $12^{130}$, $13^{100}$, $14^{100}$, $15^{160}$, $16^{160}$, $17^{40}$, $18^{80}$, $19^{120}$, $20^{100}$, $21^{100}$, $22^{120}$, $24^{140}$, $25^{80}$, $26^{40}$, $27^{40}$, $28^{60}$, $30^{60}$, $31^{40}$, $32^{60}$, $33^{120}$, $34^{20}$, $35^{40}$, $36^{130}$, $37^{40}$, $38^{40}$, $40^{60}$, $41^{40}$, $42^{20}$, $45^{88}$, $46^{80}$, $47^{40}$, $49^{60}$, $50^{4}$, $51^{40}$, $52^{40}$, $55^{40}$, $59^{40}$, $60^{20}$, $64^{60}$, $65^{40}$, $68^{20}$, $69^{20}$, $70^{40}$, $74^{40}$, $76^{60}$, $77^{20}$, $83^{40}$, $86^{40}$, $87^{40}$, $88^{10}$, $89^{40}$, $92^{40}$, $94^{20}$, $96^{40}$, $98^{40}$, $100^{140}$, $102^{40}$, $108^{60}$, $110^{20}$, $114^{20}$, $119^{40}$, $122^{40}$, $123^{20}$, $126^{20}$, $128^{40}$, $132^{20}$, $138^{20}$, $139^{20}$, $140^{20}$, $144^{20}$, $147^{20}$, $150^{20}$, $152^{20}$, $154^{60}$, $159^{40}$, $162^{80}$, $165^{40}$, $167^{60}$, $169^{60}$, $172^{20}$, $181^{20}$, $184^{40}$, $186^{60}$, $190^{40}$, $192^{20}$, $193^{60}$, $198^{20}$, $200^{40}$, $205^{40}$, $210^{40}$, $220^{10}$, $221^{20}$, $222^{20}$, $223^{20}$, $226^{40}$, $227^{40}$, $228^{40}$, $238^{40}$, $240^{20}$, $241^{20}$, $242^{20}$, $243^{40}$, $244^{80}$, $245^{20}$, $248^{40}$, $250^{40}$, $256^{40}$, $258^{60}$, $260^{30}$, $266^{40}$, $270^{20}$, $274^{20}$, $284^{30}$, $285^{20}$, $308^{20}$, $309^{40}$, $310^{20}$, $318^{20}$, $330^{20}$, $346^{20}$, $360^{20}$, $375^{20}$, $377^{20}$, $378^{20}$, $386^{20}$, $388^{20}$, $393^{20}$, $396^{20}$, $398^{20}$, $400^{20}$, $401^{20}$, $429^{20}$, $431^{20}$, $433^{20}$, $439^{20}$, $450^{40}$, $451^{20}$, $452^{20}$, $453^{20}$, $455^{40}$, $460^{30}$, $465^{20}$, $468^{20}$, $469^{20}$, $470^{4}$, $473^{20}$, $475^{20}$, $477^{20}$, $480^{10}$, $483^{20}$, $484^{20}$, $486^{20}$, $489^{40}$, $490^{20}$, $491^{20}$, $493^{20}$, $495^{40}$, $500^{20}$, $503^{20}$, $521^{20}$, $523^{20}$, $525^{20}$, $527^{20}$, $528^{20}$, $531^{20}$, $532^{20}$, $535^{40}$
\item $(h,k)=(7,19)$ : $1^{16}$, $7^{10}$, $10^{4}$, $12^{10}$, $14^{10}$, $15^{2}$, $16^{10}$, $18^{20}$, $22^{20}$, $23^{20}$, $30^{20}$, $39^{20}$, $40^{8}$, $48^{20}$, $49^{20}$, $50^{20}$, $56^{20}$, $59^{20}$, $62^{20}$, $64^{20}$, $65^{4}$, $66^{20}$, $70^{20}$, $73^{20}$, $80^{20}$, $82^{20}$, $84^{20}$, $85^{20}$, $88^{20}$, $91^{20}$, $95^{8}$, $100^{20}$, $109^{20}$, $113^{20}$, $115^{8}$, $116^{20}$, $117^{20}$, $127^{20}$, $140^{20}$, $143^{20}$, $145^{20}$, $146^{20}$, $174^{20}$, $177^{20}$, $179^{20}$, $188^{20}$, $190^{20}$, $195^{20}$, $205^{20}$, $215^{8}$, $222^{20}$, $225^{20}$, $227^{20}$, $274^{20}$, $278^{20}$, $285^{8}$, $313^{20}$, $335^{20}$, $341^{20}$, $359^{20}$, $366^{20}$, $414^{20}$, $524^{40}$, $647^{20}$, $685^{20}$, $805^{8}$, $1120^{8}$, $1326^{20}$, $1536^{40}$, $1585^{8}$, $4875^{8}$, $6700^{2}$, $9340^{8}$, $14790^{8}$, $16355^{8}$, $17565^{8}$, $27270^{8}$
\item $(h,k)=(9,11)$ : $1^{4}$, $3^{349524}$
\item $(h,k)=(9,13)$ : $1^{16}$, $3^{10}$, $5^{26}$, $8^{40}$, $9^{20}$, $10^{22}$, $12^{20}$, $14^{20}$, $16^{20}$, $17^{20}$, $18^{20}$, $23^{20}$, $24^{20}$, $25^{40}$, $27^{20}$, $29^{20}$, $30^{2}$, $39^{20}$, $40^{8}$, $41^{40}$, $43^{20}$, $44^{20}$, $45^{4}$, $50^{4}$, $66^{20}$, $70^{4}$, $72^{20}$, $76^{20}$, $99^{20}$, $100^{20}$, $105^{20}$, $116^{20}$, $150^{20}$, $151^{20}$, $154^{20}$, $163^{20}$, $167^{20}$, $169^{20}$, $176^{20}$, $180^{20}$, $182^{20}$, $197^{20}$, $203^{20}$, $212^{20}$, $235^{8}$, $242^{20}$, $275^{20}$, $281^{20}$, $283^{20}$, $291^{20}$, $295^{20}$, $302^{20}$, $306^{20}$, $314^{20}$, $337^{20}$, $347^{20}$, $408^{20}$, $464^{20}$, $484^{20}$, $584^{20}$, $628^{20}$, $704^{20}$, $723^{40}$, $827^{20}$, $1600^{8}$, $1825^{8}$, $1898^{40}$, $5505^{8}$, $10875^{8}$, $14305^{8}$, $23820^{8}$, $33190^{8}$
\item $(h,k)=(9,17)$ : $1^{16}$, $5^{16}$, $7^{50}$, $8^{20}$, $10^{12}$, $12^{10}$, $14^{10}$, $15^{18}$, $18^{20}$, $19^{20}$, $21^{40}$, $24^{20}$, $25^{20}$, $31^{20}$, $37^{20}$, $45^{8}$, $58^{40}$, $59^{20}$, $60^{2}$, $61^{20}$, $62^{20}$, $65^{4}$, $69^{20}$, $70^{40}$, $83^{20}$, $89^{20}$, $90^{40}$, $92^{20}$, $97^{20}$, $101^{20}$, $102^{40}$, $109^{20}$, $121^{20}$, $133^{20}$, $142^{20}$, $152^{20}$, $155^{20}$, $162^{20}$, $173^{20}$, $187^{20}$, $191^{20}$, $198^{40}$, $204^{20}$, $220^{8}$, $223^{20}$, $230^{20}$, $237^{20}$, $256^{20}$, $287^{20}$, $300^{20}$, $307^{20}$, $308^{20}$, $310^{20}$, $328^{20}$, $375^{20}$, $408^{20}$, $424^{20}$, $428^{20}$, $452^{20}$, $495^{20}$, $525^{20}$, $545^{8}$, $559^{20}$, $629^{20}$, $860^{2}$, $980^{8}$, $2964^{40}$, $4105^{8}$, $5660^{8}$, $6535^{8}$, $17855^{8}$, $19000^{8}$, $34000^{8}$
\item $(h,k)=(9,19)$ : $1^{1024}$, $3^{444}$, $4^{840}$, $5^{220}$, $6^{380}$, $7^{160}$, $8^{200}$, $9^{160}$, $10^{292}$, $12^{180}$, $13^{80}$, $14^{60}$, $15^{60}$, $16^{80}$, $17^{80}$, $18^{140}$, $20^{130}$, $21^{40}$, $22^{140}$, $23^{80}$, $24^{20}$, $26^{80}$, $27^{80}$, $28^{120}$, $29^{40}$, $30^{36}$, $31^{120}$, $32^{30}$, $33^{40}$, $34^{40}$, $36^{120}$, $37^{40}$, $38^{60}$, $39^{120}$, $40^{6}$, $41^{40}$, $42^{40}$, $44^{40}$, $45^{80}$, $47^{40}$, $48^{80}$, $50^{40}$, $56^{60}$, $57^{80}$, $59^{40}$, $60^{40}$, $62^{80}$, $64^{80}$, $65^{40}$, $66^{80}$, $70^{24}$, $74^{40}$, $77^{40}$, $82^{40}$, $83^{80}$, $85^{40}$, $88^{60}$, $90^{20}$, $92^{20}$, $95^{20}$, $96^{20}$, $98^{20}$, $100^{42}$, $103^{80}$, $104^{40}$, $107^{40}$, $108^{60}$, $109^{40}$, $110^{20}$, $112^{60}$, $113^{40}$, $116^{40}$, $118^{20}$, $121^{60}$, $122^{40}$, $128^{50}$, $129^{40}$, $134^{20}$, $138^{20}$, $140^{20}$, $142^{80}$, $144^{20}$, $145^{20}$, $146^{20}$, $147^{20}$, $150^{40}$, $154^{60}$, $155^{40}$, $156^{80}$, $158^{40}$, $160^{100}$, $162^{80}$, $164^{20}$, $166^{20}$, $167^{40}$, $172^{10}$, $176^{20}$, $177^{20}$, $184^{20}$, $191^{20}$, $194^{60}$, $199^{40}$, $202^{20}$, $206^{20}$, $209^{40}$, $216^{40}$, $228^{40}$, $233^{20}$, $236^{40}$, $240^{8}$, $242^{40}$, $248^{60}$, $252^{40}$, $253^{40}$, $254^{40}$, $258^{20}$, $260^{40}$, $262^{40}$, $264^{30}$, $269^{40}$, $281^{20}$, $282^{20}$, $289^{20}$, $291^{20}$, $293^{40}$, $296^{20}$, $302^{20}$, $308^{20}$, $312^{40}$, $313^{20}$, $316^{20}$, $326^{20}$, $329^{20}$, $330^{20}$, $341^{20}$, $344^{20}$, $348^{20}$, $355^{20}$, $356^{10}$, $358^{20}$, $361^{20}$, $367^{20}$, $377^{20}$, $379^{20}$, $382^{20}$, $388^{20}$, $394^{20}$, $399^{20}$, $407^{20}$, $412^{20}$, $423^{20}$, $433^{20}$, $438^{20}$, $441^{40}$, $443^{20}$, $444^{10}$, $445^{40}$, $446^{20}$, $448^{40}$, $468^{20}$, $471^{40}$, $473^{20}$, $477^{20}$, $484^{20}$, $485^{20}$, $489^{20}$, $497^{20}$, $501^{20}$, $503^{20}$, $504^{20}$, $511^{20}$, $514^{20}$, $515^{20}$, $519^{20}$, $520^{20}$, $521^{20}$, $527^{20}$, $533^{20}$, $535^{20}$
\item $(h,k)=(11,13)$ : $1^{4}$, $3^{4}$, $7^{10}$, $10^{4}$, $12^{10}$, $14^{10}$, $15^{2}$, $16^{20}$, $18^{20}$, $19^{20}$, $40^{12}$, $46^{20}$, $59^{20}$, $65^{4}$, $76^{20}$, $91^{20}$, $105^{20}$, $107^{20}$, $120^{2}$, $137^{20}$, $171^{20}$, $199^{20}$, $243^{20}$, $271^{20}$, $293^{20}$, $297^{20}$, $328^{20}$, $340^{4}$, $360^{4}$, $374^{20}$, $376^{20}$, $384^{20}$, $442^{20}$, $452^{20}$, $457^{20}$, $485^{20}$, $497^{20}$, $561^{20}$, $580^{20}$, $625^{20}$, $670^{20}$, $694^{20}$, $723^{20}$, $735^{20}$, $850^{20}$, $930^{20}$, $1039^{20}$, $1112^{20}$, $1168^{20}$, $1189^{20}$, $1191^{20}$, $1488^{20}$, $1503^{20}$, $1514^{20}$, $1534^{20}$, $1802^{20}$, $1883^{20}$, $1976^{20}$, $2097^{20}$, $2111^{20}$, $3200^{20}$, $3256^{20}$, $3921^{20}$, $4492^{20}$, $8580^{8}$
\item $(h,k)=(11,17)$ : $1^{4}$, $3^{14}$, $5^{10}$, $10^{2}$, $12^{40}$, $20^{2}$, $24^{20}$, $30^{2}$, $41^{20}$, $45^{4}$, $73^{20}$, $99^{20}$, $102^{20}$, $105^{20}$, $116^{20}$, $120^{20}$, $123^{20}$, $209^{20}$, $243^{20}$, $256^{20}$, $258^{20}$, $266^{20}$, $368^{20}$, $544^{20}$, $553^{20}$, $606^{20}$, $623^{20}$, $629^{20}$, $633^{20}$, $724^{20}$, $764^{20}$, $785^{20}$, $829^{20}$, $846^{20}$, $851^{20}$, $879^{20}$, $910^{20}$, $940^{20}$, $970^{20}$, $1023^{20}$, $1194^{20}$, $1246^{20}$, $1264^{20}$, $1309^{20}$, $1337^{20}$, $1347^{20}$, $1431^{20}$, $1639^{20}$, $1691^{20}$, $1886^{20}$, $1917^{20}$, $1983^{20}$, $2021^{20}$, $2086^{20}$, $2175^{20}$, $2434^{20}$, $2705^{20}$, $2719^{20}$, $4489^{20}$
\item $(h,k)=(11,19)$ : $1^{16}$, $3^{340}$, $4^{20}$, $5^{8}$, $9^{20}$, $11^{20}$, $12^{20}$, $14^{40}$, $25^{8}$, $29^{20}$, $32^{20}$, $33^{20}$, $35^{8}$, $41^{20}$, $42^{20}$, $43^{20}$, $52^{20}$, $56^{20}$, $60^{40}$, $74^{20}$, $78^{20}$, $88^{20}$, $99^{20}$, $105^{20}$, $114^{20}$, $120^{24}$, $151^{20}$, $154^{20}$, $161^{20}$, $165^{20}$, $182^{20}$, $198^{20}$, $211^{20}$, $217^{20}$, $234^{20}$, $237^{20}$, $238^{20}$, $242^{20}$, $245^{8}$, $250^{20}$, $257^{20}$, $261^{20}$, $267^{20}$, $287^{20}$, $303^{20}$, $319^{20}$, $352^{20}$, $363^{20}$, $371^{20}$, $400^{20}$, $401^{20}$, $428^{20}$, $460^{8}$, $521^{20}$, $617^{20}$, $643^{20}$, $831^{20}$, $924^{40}$, $7005^{40}$, $9450^{8}$, $20810^{8}$, $34080^{8}$
\item $(h,k)=(13,17)$ : $1^{16}$, $3^{340}$, $4^{20}$, $14^{40}$, $15^{8}$, $21^{20}$, $34^{20}$, $37^{20}$, $45^{20}$, $47^{20}$, $48^{20}$, $51^{20}$, $61^{20}$, $66^{20}$, $72^{20}$, $76^{20}$, $79^{20}$, $80^{2}$, $101^{20}$, $106^{20}$, $113^{20}$, $114^{20}$, $118^{20}$, $119^{20}$, $127^{20}$, $131^{20}$, $134^{40}$, $136^{20}$, $148^{20}$, $151^{20}$, $159^{20}$, $161^{20}$, $164^{20}$, $177^{20}$, $185^{8}$, $187^{20}$, $202^{20}$, $214^{20}$, $217^{20}$, $235^{8}$, $237^{20}$, $242^{20}$, $250^{20}$, $251^{20}$, $279^{20}$, $280^{8}$, $294^{20}$, $311^{20}$, $327^{20}$, $338^{20}$, $340^{8}$, $346^{20}$, $354^{20}$, $377^{40}$, $390^{8}$, $430^{20}$, $440^{4}$, $466^{20}$, $471^{20}$, $514^{20}$, $526^{20}$, $849^{20}$, $2741^{40}$, $4325^{40}$, $6921^{40}$, $10855^{8}$, $20915^{8}$
\item $(h,k)=(13,19)$ : $1^{4}$, $3^{14}$, $5^{2}$, $8^{20}$, $10^{2}$, $12^{20}$, $15^{8}$, $24^{20}$, $30^{2}$, $37^{20}$, $45^{4}$, $47^{20}$, $60^{8}$, $65^{8}$, $77^{20}$, $84^{20}$, $87^{20}$, $103^{20}$, $142^{20}$, $174^{20}$, $229^{20}$, $230^{4}$, $252^{20}$, $257^{20}$, $265^{20}$, $280^{2}$, $291^{20}$, $311^{20}$, $323^{20}$, $332^{20}$, $376^{20}$, $396^{20}$, $406^{20}$, $531^{20}$, $546^{20}$, $556^{20}$, $588^{20}$, $618^{20}$, $640^{4}$, $656^{20}$, $703^{20}$, $780^{20}$, $804^{20}$, $817^{20}$, $978^{20}$, $1058^{20}$, $1081^{20}$, $1111^{20}$, $1163^{20}$, $1258^{20}$, $1452^{20}$, $1508^{20}$, $1509^{20}$, $1665^{20}$, $1769^{20}$, $1841^{20}$, $2138^{20}$, $2170^{20}$, $2250^{20}$, $2299^{20}$, $2335^{20}$, $2696^{20}$, $3195^{20}$, $3488^{20}$, $3561^{20}$, $7980^{2}$
\item $(h,k)=(17,19)$ : $1^{4}$, $3^{4}$, $7^{10}$, $10^{4}$, $12^{10}$, $14^{10}$, $15^{2}$, $18^{20}$, $20^{20}$, $23^{20}$, $39^{40}$, $42^{20}$, $43^{20}$, $45^{20}$, $52^{20}$, $65^{12}$, $94^{20}$, $104^{20}$, $116^{20}$, $124^{20}$, $132^{20}$, $150^{4}$, $191^{20}$, $196^{20}$, $217^{20}$, $220^{2}$, $258^{20}$, $272^{20}$, $291^{20}$, $309^{20}$, $357^{20}$, $387^{20}$, $407^{20}$, $488^{20}$, $602^{20}$, $669^{20}$, $825^{20}$, $854^{20}$, $868^{20}$, $871^{20}$, $883^{20}$, $947^{20}$, $987^{20}$, $990^{20}$, $1010^{20}$, $1011^{20}$, $1020^{20}$, $1102^{20}$, $1128^{20}$, $1180^{20}$, $1288^{20}$, $1315^{8}$, $1383^{20}$, $1407^{20}$, $1453^{20}$, $1514^{20}$, $1565^{20}$, $1999^{20}$, $2216^{20}$, $2333^{20}$, $3543^{20}$, $3947^{20}$, $4277^{20}$, $5655^{20}$
\end{itemize}

\section{Results for $n=21$}
\begin{itemize}
\item $(h,k)=(1,2)$ : $1^{2}$, $3^{2}$, $7^{8}$, $9^{28}$, $11^{84}$, $14^{2}$, $15^{42}$, $21^{6}$, $22^{252}$, $24^{98}$, $28^{24}$, $30^{532}$, $32^{84}$, $34^{42}$, $36^{112}$, $42^{16}$, $44^{42}$, $45^{126}$, $46^{84}$, $48^{84}$, $49^{60}$, $56^{228}$, $57^{42}$, $60^{140}$, $63^{58}$, $69^{168}$, $70^{120}$, $72^{112}$, $75^{28}$, $77^{42}$, $78^{140}$, $84^{168}$, $91^{60}$, $92^{42}$, $96^{42}$, $98^{192}$, $102^{42}$, $105^{20}$, $106^{42}$, $108^{28}$, $110^{84}$, $112^{84}$, $120^{70}$, $126^{222}$, $129^{42}$, $133^{90}$, $138^{28}$, $142^{42}$, $147^{80}$, $154^{90}$, $161^{42}$, $162^{28}$, $168^{132}$, $175^{60}$, $177^{28}$, $182^{84}$, $189^{30}$, $192^{42}$, $195^{28}$, $196^{6}$, $198^{42}$, $204^{42}$, $210^{142}$, $217^{12}$, $224^{42}$, $231^{122}$, $238^{12}$, $240^{14}$, $245^{36}$, $252^{96}$, $259^{18}$, $273^{20}$, $280^{12}$, $287^{12}$, $291^{14}$, $294^{98}$, $308^{12}$, $315^{124}$, $318^{14}$, $330^{14}$, $336^{50}$, $357^{46}$, $360^{14}$, $364^{6}$, $366^{14}$, $372^{28}$, $378^{60}$, $385^{12}$, $399^{58}$, $406^{12}$, $413^{12}$, $420^{54}$, $434^{6}$, $448^{12}$, $462^{14}$, $476^{12}$, $483^{24}$, $497^{6}$, $504^{48}$, $511^{18}$, $525^{20}$, $546^{26}$, $552^{14}$, $560^{12}$, $567^{12}$, $574^{12}$, $588^{32}$, $602^{30}$, $609^{14}$, $623^{6}$, $630^{14}$, $644^{6}$, $651^{24}$, $658^{6}$, $672^{60}$, $679^{12}$, $686^{18}$, $693^{12}$, $707^{12}$, $714^{14}$, $728^{12}$, $735^{12}$, $742^{18}$, $749^{6}$, $756^{12}$, $777^{8}$, $798^{24}$, $805^{6}$, $812^{6}$, $819^{6}$, $833^{6}$, $840^{22}$, $861^{14}$, $868^{6}$, $882^{34}$, $903^{30}$, $924^{14}$, $931^{6}$, $945^{6}$, $966^{12}$, $987^{10}$, $1008^{24}$, $1015^{6}$, $1029^{4}$, $1050^{20}$, $1071^{12}$, $1092^{16}$, $1113^{6}$, $1127^{6}$, $1134^{4}$, $1141^{6}$, $1155^{4}$, $1162^{6}$, $1176^{20}$, $1204^{6}$, $1218^{16}$, $1260^{10}$, $1281^{8}$, $1302^{14}$, $1323^{2}$, $1344^{8}$, $1358^{6}$, $1365^{8}$, $1386^{10}$, $1428^{8}$, $1449^{6}$, $1470^{14}$, $1484^{6}$, $1491^{4}$, $1512^{18}$, $1554^{2}$, $1575^{6}$, $1596^{6}$, $1617^{8}$, $1638^{8}$, $1659^{2}$, $1680^{8}$, $1722^{6}$, $1743^{2}$, $1764^{10}$, $1848^{2}$, $1869^{4}$, $1890^{2}$, $1932^{2}$, $1953^{4}$, $1974^{2}$, $1995^{2}$, $2037^{2}$, $2058^{4}$, $2079^{2}$, $2121^{4}$, $2142^{2}$, $2163^{4}$, $2184^{8}$, $2205^{2}$, $2226^{2}$, $2247^{2}$, $2310^{2}$, $2331^{2}$, $2352^{8}$, $2394^{2}$, $2415^{2}$, $2436^{2}$, $2478^{2}$, $2499^{2}$, $2520^{2}$, $2604^{10}$, $2688^{2}$, $2730^{2}$, $2772^{2}$, $2793^{6}$, $2814^{2}$, $2835^{4}$, $2877^{2}$, $2898^{2}$, $2919^{2}$, $2982^{2}$, $3045^{2}$, $3066^{2}$, $3192^{4}$, $3234^{8}$, $3339^{2}$, $3402^{2}$, $3465^{2}$, $3507^{2}$, $3570^{2}$, $3612^{4}$, $3738^{2}$, $4326^{2}$, $4704^{2}$, $4998^{2}$
\item $(h,k)=(1,4)$ : $1^{8}$, $7^{6}$, $57^{42}$, $63^{2}$, $574^{6}$, $693^{6}$, $1477^{6}$, $1603^{6}$, $1722^{2}$, $1827^{2}$, $5061^{2}$, $11130^{6}$, $11259^{14}$, $18207^{2}$, $21658^{6}$, $34076^{6}$, $40026^{2}$, $43890^{6}$, $49483^{6}$, $135961^{6}$
\item $(h,k)=(1,5)$ : $1^{2}$, $3^{2}$, $21^{16}$, $28^{6}$, $42^{2}$, $63^{2}$, $567^{2}$, $1587^{14}$, $1785^{2}$, $1852^{42}$, $36981^{6}$, $88470^{14}$, $265629^{2}$
\item $(h,k)=(1,8)$ : $1^{128}$, $3^{170}$, $4^{42}$, $6^{70}$, $8^{42}$, $9^{14}$, $10^{84}$, $11^{42}$, $12^{28}$, $15^{28}$, $18^{56}$, $21^{2}$, $24^{14}$, $27^{28}$, $30^{14}$, $36^{14}$, $38^{42}$, $39^{28}$, $42^{14}$, $46^{42}$, $63^{14}$, $70^{6}$, $78^{14}$, $81^{14}$, $84^{14}$, $88^{42}$, $106^{42}$, $111^{14}$, $123^{42}$, $130^{42}$, $138^{14}$, $149^{42}$, $150^{14}$, $162^{42}$, $174^{14}$, $254^{42}$, $288^{14}$, $300^{14}$, $329^{6}$, $437^{42}$, $447^{14}$, $531^{14}$, $540^{42}$, $543^{14}$, $612^{14}$, $645^{14}$, $651^{14}$, $684^{14}$, $719^{42}$, $747^{14}$, $1059^{14}$, $1107^{14}$, $1116^{14}$, $1120^{6}$, $1232^{6}$, $1242^{14}$, $1277^{42}$, $1338^{14}$, $1686^{14}$, $2433^{14}$, $2542^{42}$, $2580^{14}$, $2669^{42}$, $2711^{42}$, $2742^{42}$, $2763^{14}$, $2804^{42}$, $2889^{14}$, $3885^{14}$, $4515^{14}$, $5091^{14}$, $5226^{14}$, $5571^{14}$, $7035^{14}$, $7239^{14}$, $7668^{14}$, $8148^{14}$, $8223^{14}$, $8259^{14}$
\item $(h,k)=(1,10)$ : $1^{8}$, $7^{6}$, $14^{6}$, $35^{6}$, $63^{2}$, $343^{6}$, $3038^{6}$, $3528^{6}$, $5691^{6}$, $12660^{14}$, $15862^{6}$, $44394^{2}$, $130172^{6}$, $146475^{6}$
\item $(h,k)=(1,11)$ : $1^{2}$, $3^{2}$, $21^{2}$, $28^{12}$, $63^{2}$, $126^{48}$, $168^{96}$, $504^{4116}$
\item $(h,k)=(1,13)$ : $1^{8}$, $3^{42}$, $14^{6}$, $21^{8}$, $84^{2}$, $455^{6}$, $812^{6}$, $5117^{6}$, $6489^{6}$, $45976^{6}$, $95781^{6}$, $144823^{6}$, $149940^{2}$
\item $(h,k)=(1,16)$ : $1^{8}$, $7^{2}$, $14^{2}$, $21^{10}$, $30^{14}$, $42^{2}$, $84^{6}$, $119^{6}$, $273^{6}$, $841^{42}$, $1120^{6}$, $1820^{6}$, $3227^{6}$, $3560^{42}$, $5822^{42}$, $6909^{2}$, $8953^{6}$, $10220^{6}$, $14203^{6}$, $14651^{6}$, $45647^{6}$, $99960^{2}$, $141897^{6}$
\item $(h,k)=(1,17)$ : $1^{2}$, $3^{2}$, $7^{6}$, $14^{6}$, $63^{2}$, $499611^{2}$, $548835^{2}$
\item $(h,k)=(1,19)$ : $1^{8}$, $7^{18}$, $8^{42}$, $9^{84}$, $12^{168}$, $13^{756}$, $14^{6}$, $16^{84}$, $18^{126}$, $19^{42}$, $20^{336}$, $21^{26}$, $22^{294}$, $24^{210}$, $26^{378}$, $28^{96}$, $29^{42}$, $30^{84}$, $31^{42}$, $35^{78}$, $38^{168}$, $42^{90}$, $44^{84}$, $46^{126}$, $47^{84}$, $48^{84}$, $49^{210}$, $50^{84}$, $55^{84}$, $56^{336}$, $58^{42}$, $61^{42}$, $63^{296}$, $70^{228}$, $71^{84}$, $73^{84}$, $74^{42}$, $75^{42}$, $76^{42}$, $77^{150}$, $79^{42}$, $81^{42}$, $84^{762}$, $91^{234}$, $92^{42}$, $93^{42}$, $97^{42}$, $98^{294}$, $103^{42}$, $105^{108}$, $112^{216}$, $114^{42}$, $115^{42}$, $116^{42}$, $117^{126}$, $119^{102}$, $120^{42}$, $126^{288}$, $128^{42}$, $133^{126}$, $134^{42}$, $137^{42}$, $140^{48}$, $141^{42}$, $147^{30}$, $154^{150}$, $161^{66}$, $168^{90}$, $171^{42}$, $175^{60}$, $182^{120}$, $184^{42}$, $189^{6}$, $196^{84}$, $203^{78}$, $210^{84}$, $217^{84}$, $224^{72}$, $231^{42}$, $238^{54}$, $245^{90}$, $252^{96}$, $259^{54}$, $273^{66}$, $280^{54}$, $287^{24}$, $294^{60}$, $301^{24}$, $308^{72}$, $315^{18}$, $322^{90}$, $329^{12}$, $336^{48}$, $343^{18}$, $350^{78}$, $357^{36}$, $364^{18}$, $371^{12}$, $385^{48}$, $392^{6}$, $399^{30}$, $406^{30}$, $413^{42}$, $420^{66}$, $427^{78}$, $434^{24}$, $441^{24}$, $448^{36}$, $455^{6}$, $462^{42}$, $476^{24}$, $483^{12}$, $490^{18}$, $497^{18}$, $504^{42}$, $511^{12}$, $518^{12}$, $525^{36}$, $532^{12}$, $539^{30}$, $553^{6}$, $560^{18}$, $567^{12}$, $574^{6}$, $581^{6}$, $588^{6}$, $595^{6}$, $602^{18}$, $609^{18}$, $616^{12}$, $623^{12}$, $630^{60}$, $637^{6}$, $644^{30}$, $651^{6}$, $658^{24}$, $672^{12}$, $686^{36}$, $693^{18}$, $700^{6}$, $735^{12}$, $742^{18}$, $756^{24}$, $763^{24}$, $770^{6}$, $777^{12}$, $791^{6}$, $798^{18}$, $805^{6}$, $812^{6}$, $819^{24}$, $826^{12}$, $840^{6}$, $854^{18}$, $875^{6}$, $882^{24}$, $889^{6}$, $910^{18}$, $917^{18}$, $924^{6}$, $931^{6}$, $945^{6}$, $952^{12}$, $966^{24}$, $980^{6}$, $987^{6}$, $1001^{12}$, $1008^{24}$, $1015^{6}$, $1036^{6}$, $1050^{6}$, $1057^{6}$, $1071^{12}$, $1085^{12}$, $1099^{6}$, $1106^{6}$, $1113^{6}$, $1127^{6}$, $1141^{6}$, $1148^{12}$, $1190^{6}$, $1211^{6}$, $1218^{12}$, $1232^{12}$, $1239^{6}$, $1267^{6}$, $1288^{6}$, $1295^{6}$, $1302^{6}$, $1323^{6}$, $1337^{6}$, $1400^{6}$, $1421^{6}$, $1435^{6}$, $1484^{6}$, $1491^{6}$, $1505^{18}$, $1540^{6}$, $1568^{12}$, $1596^{6}$, $1617^{6}$, $1673^{6}$
\item $(h,k)=(1,20)$ : $1^{2}$, $3^{699050}$
\item $(h,k)=(2,4)$ : $1^{2}$, $3^{2}$, $14^{2}$, $21^{2}$, $49^{2}$, $168^{2}$, $2688^{2}$, $15330^{2}$, $198135^{2}$, $277389^{6}$
\item $(h,k)=(2,5)$ : $1^{8}$, $3^{42}$, $7^{6}$, $21^{2}$, $42^{6}$, $245^{6}$, $1099^{6}$, $1344^{6}$, $1806^{6}$, $1869^{2}$, $3387^{42}$, $12460^{6}$, $33747^{2}$, $35707^{6}$, $36253^{6}$, $75285^{6}$, $149667^{6}$
\item $(h,k)=(2,8)$ : $1^{8}$, $2^{84}$, $7^{8}$, $10^{42}$, $14^{8}$, $21^{4}$, $259^{6}$, $336^{6}$, $420^{6}$, $1645^{6}$, $1680^{6}$, $2100^{6}$, $3955^{6}$, $12348^{2}$, $62062^{6}$, $87556^{6}$, $132804^{2}$, $140987^{6}$
\item $(h,k)=(2,10)$ : $1^{2}$, $3^{2}$, $63^{2}$, $462^{2}$, $32256^{2}$, $53340^{6}$, $855771^{2}$
\item $(h,k)=(2,11)$ : $1^{8}$, $7^{6}$, $13^{42}$, $14^{2}$, $35^{6}$, $49^{2}$, $63^{6}$, $91^{6}$, $347^{42}$, $588^{6}$, $1323^{6}$, $3752^{6}$, $3768^{42}$, $25977^{6}$, $31920^{2}$, $32949^{2}$, $35826^{6}$, $112203^{6}$, $119119^{6}$
\item $(h,k)=(2,13)$ : $1^{2}$, $3^{58}$, $63^{2}$, $189^{2}$, $301^{6}$, $336^{6}$, $756^{2}$, $8449^{6}$, $21588^{6}$, $74613^{2}$, $293615^{6}$
\item $(h,k)=(2,16)$ : $1^{128}$, $3^{170}$, $6^{42}$, $9^{56}$, $10^{42}$, $12^{14}$, $13^{42}$, $15^{42}$, $16^{42}$, $18^{14}$, $21^{14}$, $22^{42}$, $24^{14}$, $32^{42}$, $33^{14}$, $39^{14}$, $42^{14}$, $43^{42}$, $45^{14}$, $48^{14}$, $51^{14}$, $54^{14}$, $68^{42}$, $99^{14}$, $140^{42}$, $141^{14}$, $153^{14}$, $159^{14}$, $201^{14}$, $255^{14}$, $309^{14}$, $320^{42}$, $360^{14}$, $462^{2}$, $552^{42}$, $733^{42}$, $898^{42}$, $1151^{42}$, $1176^{2}$, $1885^{42}$, $2029^{42}$, $2139^{14}$, $2154^{14}$, $2217^{14}$, $2517^{14}$, $2664^{42}$, $2683^{42}$, $2767^{42}$, $3045^{14}$, $3090^{14}$, $3273^{14}$, $4527^{14}$, $4542^{14}$, $5025^{14}$, $5187^{14}$, $5742^{14}$, $6549^{14}$, $6636^{2}$, $7899^{14}$, $7905^{14}$, $7950^{14}$, $8097^{14}$, $8193^{14}$, $8277^{14}$
\item $(h,k)=(2,17)$ : $1^{8}$, $7^{12}$, $21^{6}$, $63^{14}$, $168^{2}$, $217^{6}$, $308^{6}$, $665^{6}$, $882^{2}$, $2049^{14}$, $2114^{6}$, $3297^{6}$, $3738^{2}$, $6755^{6}$, $15526^{6}$, $21826^{6}$, $24752^{6}$, $46529^{6}$, $52507^{6}$, $124614^{2}$, $126931^{6}$
\item $(h,k)=(2,19)$ : $1^{2}$, $3^{699050}$
\item $(h,k)=(2,20)$ : $1^{8}$, $7^{18}$, $8^{42}$, $9^{84}$, $12^{168}$, $13^{756}$, $14^{6}$, $16^{84}$, $18^{126}$, $19^{42}$, $20^{336}$, $21^{26}$, $22^{294}$, $24^{210}$, $26^{378}$, $28^{96}$, $29^{42}$, $30^{84}$, $31^{42}$, $35^{78}$, $38^{168}$, $42^{90}$, $44^{84}$, $46^{126}$, $47^{84}$, $48^{84}$, $49^{210}$, $50^{84}$, $55^{84}$, $56^{336}$, $58^{42}$, $61^{42}$, $63^{296}$, $70^{228}$, $71^{84}$, $73^{84}$, $74^{42}$, $75^{42}$, $76^{42}$, $77^{150}$, $79^{42}$, $81^{42}$, $84^{762}$, $91^{234}$, $92^{42}$, $93^{42}$, $97^{42}$, $98^{294}$, $103^{42}$, $105^{108}$, $112^{216}$, $114^{42}$, $115^{42}$, $116^{42}$, $117^{126}$, $119^{102}$, $120^{42}$, $126^{288}$, $128^{42}$, $133^{126}$, $134^{42}$, $137^{42}$, $140^{48}$, $141^{42}$, $147^{30}$, $154^{150}$, $161^{66}$, $168^{90}$, $171^{42}$, $175^{60}$, $182^{120}$, $184^{42}$, $189^{6}$, $196^{84}$, $203^{78}$, $210^{84}$, $217^{84}$, $224^{72}$, $231^{42}$, $238^{54}$, $245^{90}$, $252^{96}$, $259^{54}$, $273^{66}$, $280^{54}$, $287^{24}$, $294^{60}$, $301^{24}$, $308^{72}$, $315^{18}$, $322^{90}$, $329^{12}$, $336^{48}$, $343^{18}$, $350^{78}$, $357^{36}$, $364^{18}$, $371^{12}$, $385^{48}$, $392^{6}$, $399^{30}$, $406^{30}$, $413^{42}$, $420^{66}$, $427^{78}$, $434^{24}$, $441^{24}$, $448^{36}$, $455^{6}$, $462^{42}$, $476^{24}$, $483^{12}$, $490^{18}$, $497^{18}$, $504^{42}$, $511^{12}$, $518^{12}$, $525^{36}$, $532^{12}$, $539^{30}$, $553^{6}$, $560^{18}$, $567^{12}$, $574^{6}$, $581^{6}$, $588^{6}$, $595^{6}$, $602^{18}$, $609^{18}$, $616^{12}$, $623^{12}$, $630^{60}$, $637^{6}$, $644^{30}$, $651^{6}$, $658^{24}$, $672^{12}$, $686^{36}$, $693^{18}$, $700^{6}$, $735^{12}$, $742^{18}$, $756^{24}$, $763^{24}$, $770^{6}$, $777^{12}$, $791^{6}$, $798^{18}$, $805^{6}$, $812^{6}$, $819^{24}$, $826^{12}$, $840^{6}$, $854^{18}$, $875^{6}$, $882^{24}$, $889^{6}$, $910^{18}$, $917^{18}$, $924^{6}$, $931^{6}$, $945^{6}$, $952^{12}$, $966^{24}$, $980^{6}$, $987^{6}$, $1001^{12}$, $1008^{24}$, $1015^{6}$, $1036^{6}$, $1050^{6}$, $1057^{6}$, $1071^{12}$, $1085^{12}$, $1099^{6}$, $1106^{6}$, $1113^{6}$, $1127^{6}$, $1141^{6}$, $1148^{12}$, $1190^{6}$, $1211^{6}$, $1218^{12}$, $1232^{12}$, $1239^{6}$, $1267^{6}$, $1288^{6}$, $1295^{6}$, $1302^{6}$, $1323^{6}$, $1337^{6}$, $1400^{6}$, $1421^{6}$, $1435^{6}$, $1484^{6}$, $1491^{6}$, $1505^{18}$, $1540^{6}$, $1568^{12}$, $1596^{6}$, $1617^{6}$, $1673^{6}$
\item $(h,k)=(4,5)$ : $1^{2}$, $3^{2}$, $9^{14}$, $21^{14}$, $63^{2}$, $9555^{2}$, $20790^{2}$, $37583^{6}$, $100268^{6}$, $107037^{6}$, $283290^{2}$
\item $(h,k)=(4,8)$ : $1^{2}$, $3^{2}$, $14^{6}$, $28^{6}$, $63^{2}$, $2954^{6}$, $3528^{2}$, $4494^{2}$, $13139^{6}$, $15456^{14}$, $61866^{2}$, $62286^{2}$, $253246^{6}$
\item $(h,k)=(4,10)$ : $1^{8}$, $3^{42}$, $21^{12}$, $63^{2}$, $84^{6}$, $120^{14}$, $248^{42}$, $451^{42}$, $1463^{42}$, $69027^{2}$, $80157^{2}$, $136479^{6}$, $147735^{6}$
\item $(h,k)=(4,11)$ : $1^{128}$, $3^{254}$, $4^{42}$, $5^{42}$, $6^{84}$, $7^{6}$, $8^{42}$, $9^{140}$, $12^{42}$, $15^{42}$, $18^{14}$, $19^{42}$, $21^{56}$, $27^{42}$, $33^{28}$, $36^{14}$, $51^{14}$, $57^{14}$, $60^{84}$, $84^{14}$, $87^{14}$, $111^{14}$, $112^{42}$, $126^{14}$, $127^{42}$, $144^{14}$, $167^{42}$, $219^{14}$, $297^{14}$, $315^{14}$, $330^{28}$, $387^{14}$, $399^{14}$, $450^{42}$, $582^{14}$, $714^{6}$, $741^{14}$, $873^{14}$, $1134^{42}$, $1302^{42}$, $1434^{14}$, $1484^{42}$, $1702^{42}$, $1980^{14}$, $2196^{14}$, $2274^{42}$, $2520^{42}$, $2643^{42}$, $2775^{14}$, $3288^{14}$, $3453^{14}$, $3684^{14}$, $4005^{14}$, $4128^{14}$, $4218^{14}$, $7293^{14}$, $7467^{14}$, $7476^{14}$, $7758^{14}$, $7842^{14}$, $7878^{14}$, $8202^{14}$, $8214^{14}$, $8295^{14}$
\item $(h,k)=(4,13)$ : $1^{8}$, $14^{6}$, $21^{2}$, $28^{12}$, $38^{42}$, $63^{2}$, $105^{6}$, $154^{6}$, $210^{6}$, $231^{6}$, $243^{14}$, $273^{2}$, $315^{6}$, $322^{6}$, $826^{6}$, $1652^{6}$, $1746^{14}$, $3822^{6}$, $4456^{42}$, $10724^{6}$, $12124^{6}$, $14966^{6}$, $16879^{42}$, $58170^{2}$, $59479^{6}$, $70763^{6}$
\item $(h,k)=(4,16)$ : $1^{8}$, $7^{6}$, $14^{2}$, $21^{8}$, $28^{6}$, $42^{6}$, $49^{2}$, $77^{6}$, $112^{6}$, $147^{2}$, $259^{6}$, $329^{6}$, $1491^{6}$, $1589^{6}$, $1743^{6}$, $2541^{6}$, $2723^{6}$, $4284^{6}$, $4753^{6}$, $30597^{2}$, $119259^{2}$, $139223^{6}$, $140273^{6}$
\item $(h,k)=(4,17)$ : $1^{2}$, $3^{699050}$
\item $(h,k)=(4,19)$ : $1^{8}$, $7^{12}$, $21^{6}$, $63^{14}$, $168^{2}$, $217^{6}$, $308^{6}$, $665^{6}$, $882^{2}$, $2049^{14}$, $2114^{6}$, $3297^{6}$, $3738^{2}$, $6755^{6}$, $15526^{6}$, $21826^{6}$, $24752^{6}$, $46529^{6}$, $52507^{6}$, $124614^{2}$, $126931^{6}$
\item $(h,k)=(4,20)$ : $1^{2}$, $3^{2}$, $7^{6}$, $14^{6}$, $63^{2}$, $499611^{2}$, $548835^{2}$
\item $(h,k)=(5,8)$ : $1^{8}$, $63^{2}$, $147^{6}$, $168^{6}$, $189^{6}$, $210^{2}$, $231^{6}$, $245^{12}$, $259^{6}$, $336^{2}$, $686^{6}$, $2898^{2}$, $11620^{6}$, $31059^{2}$, $111657^{2}$, $137984^{6}$, $149009^{6}$
\item $(h,k)=(5,10)$ : $1^{2}$, $3^{2}$, $14^{2}$, $28^{6}$, $49^{2}$, $378^{2}$, $385^{6}$, $861^{2}$, $1260^{2}$, $2580^{14}$, $6468^{2}$, $11494^{6}$, $60900^{2}$, $100695^{2}$, $105000^{6}$, $509166^{2}$
\item $(h,k)=(5,11)$ : $1^{8}$, $21^{2}$, $28^{6}$, $45^{42}$, $49^{18}$, $63^{2}$, $77^{6}$, $496^{42}$, $1022^{6}$, $2688^{2}$, $5461^{42}$, $12684^{2}$, $19502^{6}$, $22267^{6}$, $22792^{6}$, $47229^{6}$, $60025^{6}$, $66234^{2}$, $107191^{6}$
\item $(h,k)=(5,13)$ : $1^{2}$, $3^{2}$, $7^{2}$, $14^{8}$, $21^{4}$, $35^{12}$, $398^{42}$, $427^{6}$, $567^{2}$, $1260^{2}$, $1522^{42}$, $2961^{2}$, $4431^{6}$, $8589^{2}$, $48510^{2}$, $68892^{14}$, $70658^{6}$, $95088^{2}$, $142170^{2}$
\item $(h,k)=(5,16)$ : $1^{2}$, $3^{699050}$
\item $(h,k)=(5,17)$ : $1^{8}$, $7^{6}$, $14^{2}$, $21^{8}$, $28^{6}$, $42^{6}$, $49^{2}$, $77^{6}$, $112^{6}$, $147^{2}$, $259^{6}$, $329^{6}$, $1491^{6}$, $1589^{6}$, $1743^{6}$, $2541^{6}$, $2723^{6}$, $4284^{6}$, $4753^{6}$, $30597^{2}$, $119259^{2}$, $139223^{6}$, $140273^{6}$
\item $(h,k)=(5,19)$ : $1^{128}$, $3^{170}$, $6^{42}$, $9^{56}$, $10^{42}$, $12^{14}$, $13^{42}$, $15^{42}$, $16^{42}$, $18^{14}$, $21^{14}$, $22^{42}$, $24^{14}$, $32^{42}$, $33^{14}$, $39^{14}$, $42^{14}$, $43^{42}$, $45^{14}$, $48^{14}$, $51^{14}$, $54^{14}$, $68^{42}$, $99^{14}$, $140^{42}$, $141^{14}$, $153^{14}$, $159^{14}$, $201^{14}$, $255^{14}$, $309^{14}$, $320^{42}$, $360^{14}$, $462^{2}$, $552^{42}$, $733^{42}$, $898^{42}$, $1151^{42}$, $1176^{2}$, $1885^{42}$, $2029^{42}$, $2139^{14}$, $2154^{14}$, $2217^{14}$, $2517^{14}$, $2664^{42}$, $2683^{42}$, $2767^{42}$, $3045^{14}$, $3090^{14}$, $3273^{14}$, $4527^{14}$, $4542^{14}$, $5025^{14}$, $5187^{14}$, $5742^{14}$, $6549^{14}$, $6636^{2}$, $7899^{14}$, $7905^{14}$, $7950^{14}$, $8097^{14}$, $8193^{14}$, $8277^{14}$
\item $(h,k)=(5,20)$ : $1^{8}$, $7^{2}$, $14^{2}$, $21^{10}$, $30^{14}$, $42^{2}$, $84^{6}$, $119^{6}$, $273^{6}$, $841^{42}$, $1120^{6}$, $1820^{6}$, $3227^{6}$, $3560^{42}$, $5822^{42}$, $6909^{2}$, $8953^{6}$, $10220^{6}$, $14203^{6}$, $14651^{6}$, $45647^{6}$, $99960^{2}$, $141897^{6}$
\item $(h,k)=(8,10)$ : $1^{2}$, $3^{58}$, $42^{2}$, $63^{2}$, $840^{2}$, $7665^{2}$, $45969^{6}$, $65730^{2}$, $836241^{2}$
\item $(h,k)=(8,11)$ : $1^{8}$, $2^{84}$, $7^{18}$, $21^{6}$, $63^{2}$, $77^{6}$, $85^{42}$, $1285^{42}$, $1414^{6}$, $1876^{6}$, $2094^{14}$, $4550^{6}$, $8673^{6}$, $27804^{2}$, $30576^{6}$, $45157^{6}$, $51506^{6}$, $58562^{6}$, $87423^{6}$, $107625^{2}$
\item $(h,k)=(8,13)$ : $1^{2}$, $3^{699050}$
\item $(h,k)=(8,16)$ : $1^{2}$, $3^{2}$, $7^{2}$, $14^{8}$, $21^{4}$, $35^{12}$, $398^{42}$, $427^{6}$, $567^{2}$, $1260^{2}$, $1522^{42}$, $2961^{2}$, $4431^{6}$, $8589^{2}$, $48510^{2}$, $68892^{14}$, $70658^{6}$, $95088^{2}$, $142170^{2}$
\item $(h,k)=(8,17)$ : $1^{8}$, $14^{6}$, $21^{2}$, $28^{12}$, $38^{42}$, $63^{2}$, $105^{6}$, $154^{6}$, $210^{6}$, $231^{6}$, $243^{14}$, $273^{2}$, $315^{6}$, $322^{6}$, $826^{6}$, $1652^{6}$, $1746^{14}$, $3822^{6}$, $4456^{42}$, $10724^{6}$, $12124^{6}$, $14966^{6}$, $16879^{42}$, $58170^{2}$, $59479^{6}$, $70763^{6}$
\item $(h,k)=(8,19)$ : $1^{2}$, $3^{58}$, $63^{2}$, $189^{2}$, $301^{6}$, $336^{6}$, $756^{2}$, $8449^{6}$, $21588^{6}$, $74613^{2}$, $293615^{6}$
\item $(h,k)=(8,20)$ : $1^{8}$, $3^{42}$, $14^{6}$, $21^{8}$, $84^{2}$, $455^{6}$, $812^{6}$, $5117^{6}$, $6489^{6}$, $45976^{6}$, $95781^{6}$, $144823^{6}$, $149940^{2}$
\item $(h,k)=(10,11)$ : $1^{2}$, $3^{699050}$
\item $(h,k)=(10,13)$ : $1^{8}$, $2^{84}$, $7^{18}$, $21^{6}$, $63^{2}$, $77^{6}$, $85^{42}$, $1285^{42}$, $1414^{6}$, $1876^{6}$, $2094^{14}$, $4550^{6}$, $8673^{6}$, $27804^{2}$, $30576^{6}$, $45157^{6}$, $51506^{6}$, $58562^{6}$, $87423^{6}$, $107625^{2}$
\item $(h,k)=(10,16)$ : $1^{8}$, $21^{2}$, $28^{6}$, $45^{42}$, $49^{18}$, $63^{2}$, $77^{6}$, $496^{42}$, $1022^{6}$, $2688^{2}$, $5461^{42}$, $12684^{2}$, $19502^{6}$, $22267^{6}$, $22792^{6}$, $47229^{6}$, $60025^{6}$, $66234^{2}$, $107191^{6}$
\item $(h,k)=(10,17)$ : $1^{128}$, $3^{254}$, $4^{42}$, $5^{42}$, $6^{84}$, $7^{6}$, $8^{42}$, $9^{140}$, $12^{42}$, $15^{42}$, $18^{14}$, $19^{42}$, $21^{56}$, $27^{42}$, $33^{28}$, $36^{14}$, $51^{14}$, $57^{14}$, $60^{84}$, $84^{14}$, $87^{14}$, $111^{14}$, $112^{42}$, $126^{14}$, $127^{42}$, $144^{14}$, $167^{42}$, $219^{14}$, $297^{14}$, $315^{14}$, $330^{28}$, $387^{14}$, $399^{14}$, $450^{42}$, $582^{14}$, $714^{6}$, $741^{14}$, $873^{14}$, $1134^{42}$, $1302^{42}$, $1434^{14}$, $1484^{42}$, $1702^{42}$, $1980^{14}$, $2196^{14}$, $2274^{42}$, $2520^{42}$, $2643^{42}$, $2775^{14}$, $3288^{14}$, $3453^{14}$, $3684^{14}$, $4005^{14}$, $4128^{14}$, $4218^{14}$, $7293^{14}$, $7467^{14}$, $7476^{14}$, $7758^{14}$, $7842^{14}$, $7878^{14}$, $8202^{14}$, $8214^{14}$, $8295^{14}$
\item $(h,k)=(10,19)$ : $1^{8}$, $7^{6}$, $13^{42}$, $14^{2}$, $35^{6}$, $49^{2}$, $63^{6}$, $91^{6}$, $347^{42}$, $588^{6}$, $1323^{6}$, $3752^{6}$, $3768^{42}$, $25977^{6}$, $31920^{2}$, $32949^{2}$, $35826^{6}$, $112203^{6}$, $119119^{6}$
\item $(h,k)=(10,20)$ : $1^{2}$, $3^{2}$, $21^{2}$, $28^{12}$, $63^{2}$, $126^{48}$, $168^{96}$, $504^{4116}$
\item $(h,k)=(11,13)$ : $1^{2}$, $3^{58}$, $42^{2}$, $63^{2}$, $840^{2}$, $7665^{2}$, $45969^{6}$, $65730^{2}$, $836241^{2}$
\item $(h,k)=(11,16)$ : $1^{2}$, $3^{2}$, $14^{2}$, $28^{6}$, $49^{2}$, $378^{2}$, $385^{6}$, $861^{2}$, $1260^{2}$, $2580^{14}$, $6468^{2}$, $11494^{6}$, $60900^{2}$, $100695^{2}$, $105000^{6}$, $509166^{2}$
\item $(h,k)=(11,17)$ : $1^{8}$, $3^{42}$, $21^{12}$, $63^{2}$, $84^{6}$, $120^{14}$, $248^{42}$, $451^{42}$, $1463^{42}$, $69027^{2}$, $80157^{2}$, $136479^{6}$, $147735^{6}$
\item $(h,k)=(11,19)$ : $1^{2}$, $3^{2}$, $63^{2}$, $462^{2}$, $32256^{2}$, $53340^{6}$, $855771^{2}$
\item $(h,k)=(11,20)$ : $1^{8}$, $7^{6}$, $14^{6}$, $35^{6}$, $63^{2}$, $343^{6}$, $3038^{6}$, $3528^{6}$, $5691^{6}$, $12660^{14}$, $15862^{6}$, $44394^{2}$, $130172^{6}$, $146475^{6}$
\item $(h,k)=(13,16)$ : $1^{8}$, $63^{2}$, $147^{6}$, $168^{6}$, $189^{6}$, $210^{2}$, $231^{6}$, $245^{12}$, $259^{6}$, $336^{2}$, $686^{6}$, $2898^{2}$, $11620^{6}$, $31059^{2}$, $111657^{2}$, $137984^{6}$, $149009^{6}$
\item $(h,k)=(13,17)$ : $1^{2}$, $3^{2}$, $14^{6}$, $28^{6}$, $63^{2}$, $2954^{6}$, $3528^{2}$, $4494^{2}$, $13139^{6}$, $15456^{14}$, $61866^{2}$, $62286^{2}$, $253246^{6}$
\item $(h,k)=(13,19)$ : $1^{8}$, $2^{84}$, $7^{8}$, $10^{42}$, $14^{8}$, $21^{4}$, $259^{6}$, $336^{6}$, $420^{6}$, $1645^{6}$, $1680^{6}$, $2100^{6}$, $3955^{6}$, $12348^{2}$, $62062^{6}$, $87556^{6}$, $132804^{2}$, $140987^{6}$
\item $(h,k)=(13,20)$ : $1^{128}$, $3^{170}$, $4^{42}$, $6^{70}$, $8^{42}$, $9^{14}$, $10^{84}$, $11^{42}$, $12^{28}$, $15^{28}$, $18^{56}$, $21^{2}$, $24^{14}$, $27^{28}$, $30^{14}$, $36^{14}$, $38^{42}$, $39^{28}$, $42^{14}$, $46^{42}$, $63^{14}$, $70^{6}$, $78^{14}$, $81^{14}$, $84^{14}$, $88^{42}$, $106^{42}$, $111^{14}$, $123^{42}$, $130^{42}$, $138^{14}$, $149^{42}$, $150^{14}$, $162^{42}$, $174^{14}$, $254^{42}$, $288^{14}$, $300^{14}$, $329^{6}$, $437^{42}$, $447^{14}$, $531^{14}$, $540^{42}$, $543^{14}$, $612^{14}$, $645^{14}$, $651^{14}$, $684^{14}$, $719^{42}$, $747^{14}$, $1059^{14}$, $1107^{14}$, $1116^{14}$, $1120^{6}$, $1232^{6}$, $1242^{14}$, $1277^{42}$, $1338^{14}$, $1686^{14}$, $2433^{14}$, $2542^{42}$, $2580^{14}$, $2669^{42}$, $2711^{42}$, $2742^{42}$, $2763^{14}$, $2804^{42}$, $2889^{14}$, $3885^{14}$, $4515^{14}$, $5091^{14}$, $5226^{14}$, $5571^{14}$, $7035^{14}$, $7239^{14}$, $7668^{14}$, $8148^{14}$, $8223^{14}$, $8259^{14}$
\item $(h,k)=(16,17)$ : $1^{2}$, $3^{2}$, $9^{14}$, $21^{14}$, $63^{2}$, $9555^{2}$, $20790^{2}$, $37583^{6}$, $100268^{6}$, $107037^{6}$, $283290^{2}$
\item $(h,k)=(16,19)$ : $1^{8}$, $3^{42}$, $7^{6}$, $21^{2}$, $42^{6}$, $245^{6}$, $1099^{6}$, $1344^{6}$, $1806^{6}$, $1869^{2}$, $3387^{42}$, $12460^{6}$, $33747^{2}$, $35707^{6}$, $36253^{6}$, $75285^{6}$, $149667^{6}$
\item $(h,k)=(16,20)$ : $1^{2}$, $3^{2}$, $21^{16}$, $28^{6}$, $42^{2}$, $63^{2}$, $567^{2}$, $1587^{14}$, $1785^{2}$, $1852^{42}$, $36981^{6}$, $88470^{14}$, $265629^{2}$
\item $(h,k)=(17,19)$ : $1^{2}$, $3^{2}$, $14^{2}$, $21^{2}$, $49^{2}$, $168^{2}$, $2688^{2}$, $15330^{2}$, $198135^{2}$, $277389^{6}$
\item $(h,k)=(17,20)$ : $1^{8}$, $7^{6}$, $57^{42}$, $63^{2}$, $574^{6}$, $693^{6}$, $1477^{6}$, $1603^{6}$, $1722^{2}$, $1827^{2}$, $5061^{2}$, $11130^{6}$, $11259^{14}$, $18207^{2}$, $21658^{6}$, $34076^{6}$, $40026^{2}$, $43890^{6}$, $49483^{6}$, $135961^{6}$
\item $(h,k)=(19,20)$ : $1^{2}$, $3^{2}$, $7^{8}$, $9^{28}$, $11^{84}$, $14^{2}$, $15^{42}$, $21^{6}$, $22^{252}$, $24^{98}$, $28^{24}$, $30^{532}$, $32^{84}$, $34^{42}$, $36^{112}$, $42^{16}$, $44^{42}$, $45^{126}$, $46^{84}$, $48^{84}$, $49^{60}$, $56^{228}$, $57^{42}$, $60^{140}$, $63^{58}$, $69^{168}$, $70^{120}$, $72^{112}$, $75^{28}$, $77^{42}$, $78^{140}$, $84^{168}$, $91^{60}$, $92^{42}$, $96^{42}$, $98^{192}$, $102^{42}$, $105^{20}$, $106^{42}$, $108^{28}$, $110^{84}$, $112^{84}$, $120^{70}$, $126^{222}$, $129^{42}$, $133^{90}$, $138^{28}$, $142^{42}$, $147^{80}$, $154^{90}$, $161^{42}$, $162^{28}$, $168^{132}$, $175^{60}$, $177^{28}$, $182^{84}$, $189^{30}$, $192^{42}$, $195^{28}$, $196^{6}$, $198^{42}$, $204^{42}$, $210^{142}$, $217^{12}$, $224^{42}$, $231^{122}$, $238^{12}$, $240^{14}$, $245^{36}$, $252^{96}$, $259^{18}$, $273^{20}$, $280^{12}$, $287^{12}$, $291^{14}$, $294^{98}$, $308^{12}$, $315^{124}$, $318^{14}$, $330^{14}$, $336^{50}$, $357^{46}$, $360^{14}$, $364^{6}$, $366^{14}$, $372^{28}$, $378^{60}$, $385^{12}$, $399^{58}$, $406^{12}$, $413^{12}$, $420^{54}$, $434^{6}$, $448^{12}$, $462^{14}$, $476^{12}$, $483^{24}$, $497^{6}$, $504^{48}$, $511^{18}$, $525^{20}$, $546^{26}$, $552^{14}$, $560^{12}$, $567^{12}$, $574^{12}$, $588^{32}$, $602^{30}$, $609^{14}$, $623^{6}$, $630^{14}$, $644^{6}$, $651^{24}$, $658^{6}$, $672^{60}$, $679^{12}$, $686^{18}$, $693^{12}$, $707^{12}$, $714^{14}$, $728^{12}$, $735^{12}$, $742^{18}$, $749^{6}$, $756^{12}$, $777^{8}$, $798^{24}$, $805^{6}$, $812^{6}$, $819^{6}$, $833^{6}$, $840^{22}$, $861^{14}$, $868^{6}$, $882^{34}$, $903^{30}$, $924^{14}$, $931^{6}$, $945^{6}$, $966^{12}$, $987^{10}$, $1008^{24}$, $1015^{6}$, $1029^{4}$, $1050^{20}$, $1071^{12}$, $1092^{16}$, $1113^{6}$, $1127^{6}$, $1134^{4}$, $1141^{6}$, $1155^{4}$, $1162^{6}$, $1176^{20}$, $1204^{6}$, $1218^{16}$, $1260^{10}$, $1281^{8}$, $1302^{14}$, $1323^{2}$, $1344^{8}$, $1358^{6}$, $1365^{8}$, $1386^{10}$, $1428^{8}$, $1449^{6}$, $1470^{14}$, $1484^{6}$, $1491^{4}$, $1512^{18}$, $1554^{2}$, $1575^{6}$, $1596^{6}$, $1617^{8}$, $1638^{8}$, $1659^{2}$, $1680^{8}$, $1722^{6}$, $1743^{2}$, $1764^{10}$, $1848^{2}$, $1869^{4}$, $1890^{2}$, $1932^{2}$, $1953^{4}$, $1974^{2}$, $1995^{2}$, $2037^{2}$, $2058^{4}$, $2079^{2}$, $2121^{4}$, $2142^{2}$, $2163^{4}$, $2184^{8}$, $2205^{2}$, $2226^{2}$, $2247^{2}$, $2310^{2}$, $2331^{2}$, $2352^{8}$, $2394^{2}$, $2415^{2}$, $2436^{2}$, $2478^{2}$, $2499^{2}$, $2520^{2}$, $2604^{10}$, $2688^{2}$, $2730^{2}$, $2772^{2}$, $2793^{6}$, $2814^{2}$, $2835^{4}$, $2877^{2}$, $2898^{2}$, $2919^{2}$, $2982^{2}$, $3045^{2}$, $3066^{2}$, $3192^{4}$, $3234^{8}$, $3339^{2}$, $3402^{2}$, $3465^{2}$, $3507^{2}$, $3570^{2}$, $3612^{4}$, $3738^{2}$, $4326^{2}$, $4704^{2}$, $4998^{2}$
\end{itemize}

\section{Results for $n=22$}
\begin{itemize}
\item $(h,k)=(1,3)$ : $1^{4}$, $11^{4}$, $14^{22}$, $15^{22}$, $17^{22}$, $22^{2}$, $31^{22}$, $51^{22}$, $57^{22}$, $59^{22}$, $66^{46}$, $71^{22}$, $76^{22}$, $77^{22}$, $95^{22}$, $102^{22}$, $110^{2}$, $115^{22}$, $116^{22}$, $119^{22}$, $120^{22}$, $125^{22}$, $126^{22}$, $128^{22}$, $132^{22}$, $152^{22}$, $166^{44}$, $185^{22}$, $193^{22}$, $209^{22}$, $218^{22}$, $222^{22}$, $228^{22}$, $260^{22}$, $287^{22}$, $290^{22}$, $318^{22}$, $332^{22}$, $368^{22}$, $385^{22}$, $419^{22}$, $475^{22}$, $515^{22}$, $518^{22}$, $531^{22}$, $538^{22}$, $553^{22}$, $557^{22}$, $558^{22}$, $559^{22}$, $575^{22}$, $588^{22}$, $605^{22}$, $615^{22}$, $638^{2}$, $644^{22}$, $666^{22}$, $667^{22}$, $691^{22}$, $714^{22}$, $722^{22}$, $728^{22}$, $744^{44}$, $776^{22}$, $796^{22}$, $805^{22}$, $869^{2}$, $893^{22}$, $918^{22}$, $947^{22}$, $954^{22}$, $970^{22}$, $976^{22}$, $988^{22}$, $1033^{22}$, $1050^{22}$, $1084^{22}$, $1120^{22}$, $1143^{22}$, $1301^{22}$, $1327^{22}$, $1343^{22}$, $1379^{22}$, $1390^{22}$, $1461^{22}$, $1523^{22}$, $1582^{22}$, $1601^{22}$, $1682^{22}$, $1751^{22}$, $1788^{22}$, $1860^{22}$, $2001^{22}$, $2206^{22}$, $2411^{22}$, $2713^{22}$, $4521^{4}$, $86141^{4}$, $608047^{4}$
\item $(h,k)=(1,5)$ : $1^{4}$, $7^{22}$, $8^{44}$, $10^{22}$, $12^{22}$, $19^{22}$, $20^{22}$, $31^{22}$, $55^{2}$, $66^{4}$, $86^{22}$, $93^{22}$, $96^{22}$, $114^{22}$, $115^{44}$, $133^{22}$, $161^{22}$, $165^{22}$, $169^{22}$, $177^{22}$, $181^{22}$, $182^{22}$, $183^{22}$, $194^{22}$, $198^{4}$, $199^{22}$, $210^{22}$, $218^{22}$, $233^{22}$, $239^{22}$, $253^{22}$, $264^{22}$, $266^{22}$, $273^{22}$, $301^{44}$, $308^{2}$, $335^{22}$, $339^{22}$, $340^{44}$, $359^{22}$, $361^{22}$, $372^{22}$, $388^{22}$, $413^{22}$, $417^{22}$, $422^{22}$, $423^{22}$, $459^{22}$, $482^{22}$, $492^{22}$, $497^{22}$, $498^{22}$, $502^{22}$, $521^{22}$, $536^{22}$, $556^{22}$, $569^{22}$, $606^{22}$, $625^{22}$, $627^{22}$, $660^{2}$, $672^{22}$, $680^{22}$, $686^{22}$, $693^{4}$, $712^{22}$, $725^{44}$, $727^{22}$, $730^{22}$, $744^{22}$, $752^{22}$, $763^{22}$, $794^{22}$, $801^{22}$, $814^{22}$, $866^{22}$, $874^{22}$, $877^{22}$, $921^{22}$, $979^{22}$, $1013^{22}$, $1054^{22}$, $1238^{22}$, $1250^{22}$, $1315^{22}$, $1371^{22}$, $1378^{22}$, $1397^{22}$, $1468^{22}$, $1799^{22}$, $2088^{22}$, $2093^{22}$, $2251^{22}$, $2543^{22}$, $2554^{22}$, $2720^{22}$, $4610^{22}$, $4719^{4}$, $5654^{4}$, $8778^{4}$, $24695^{4}$, $158653^{4}$, $495363^{4}$
\item $(h,k)=(1,7)$ : $1^{4}$, $11^{30}$, $15^{22}$, $18^{22}$, $19^{22}$, $20^{44}$, $22^{4}$, $28^{22}$, $31^{22}$, $42^{22}$, $44^{26}$, $45^{22}$, $56^{44}$, $66^{22}$, $69^{22}$, $87^{22}$, $88^{2}$, $90^{22}$, $92^{22}$, $93^{22}$, $103^{22}$, $109^{22}$, $110^{4}$, $116^{22}$, $135^{22}$, $147^{22}$, $156^{22}$, $161^{22}$, $179^{22}$, $186^{22}$, $189^{22}$, $196^{22}$, $234^{22}$, $239^{22}$, $294^{22}$, $306^{22}$, $334^{22}$, $335^{44}$, $363^{22}$, $376^{22}$, $379^{22}$, $394^{22}$, $423^{22}$, $429^{22}$, $432^{22}$, $454^{22}$, $471^{22}$, $476^{22}$, $498^{22}$, $522^{22}$, $531^{22}$, $553^{22}$, $558^{22}$, $589^{22}$, $595^{22}$, $605^{22}$, $622^{22}$, $628^{22}$, $634^{44}$, $636^{22}$, $667^{22}$, $668^{22}$, $675^{22}$, $678^{22}$, $709^{22}$, $732^{22}$, $773^{22}$, $794^{22}$, $805^{22}$, $807^{22}$, $811^{22}$, $841^{22}$, $889^{22}$, $976^{22}$, $1023^{2}$, $1024^{22}$, $1107^{22}$, $1177^{22}$, $1190^{22}$, $1212^{22}$, $1224^{22}$, $1238^{22}$, $1250^{22}$, $1282^{22}$, $1284^{22}$, $1338^{22}$, $1434^{22}$, $1461^{22}$, $1472^{22}$, $1677^{22}$, $1738^{22}$, $2024^{4}$, $2119^{22}$, $2123^{22}$, $2240^{22}$, $3661^{22}$, $4985^{22}$, $179267^{4}$, $517176^{4}$
\item $(h,k)=(1,9)$ : $1^{4}$, $9^{22}$, $16^{22}$, $17^{22}$, $22^{30}$, $53^{22}$, $63^{22}$, $67^{22}$, $72^{22}$, $73^{22}$, $74^{44}$, $75^{22}$, $77^{4}$, $90^{22}$, $104^{22}$, $115^{22}$, $127^{22}$, $132^{26}$, $135^{22}$, $137^{44}$, $138^{22}$, $139^{22}$, $146^{22}$, $148^{22}$, $152^{22}$, $158^{22}$, $168^{22}$, $175^{22}$, $176^{22}$, $181^{22}$, $183^{22}$, $197^{22}$, $198^{4}$, $201^{22}$, $229^{22}$, $230^{22}$, $233^{22}$, $240^{22}$, $252^{22}$, $253^{22}$, $287^{22}$, $293^{22}$, $342^{22}$, $347^{22}$, $363^{2}$, $417^{22}$, $430^{22}$, $443^{22}$, $448^{22}$, $453^{22}$, $489^{22}$, $493^{22}$, $494^{22}$, $509^{22}$, $520^{22}$, $537^{22}$, $597^{22}$, $602^{22}$, $603^{22}$, $622^{22}$, $649^{4}$, $703^{22}$, $742^{22}$, $746^{22}$, $747^{22}$, $752^{22}$, $771^{22}$, $776^{22}$, $791^{22}$, $820^{22}$, $830^{22}$, $861^{22}$, $868^{22}$, $910^{22}$, $931^{22}$, $937^{22}$, $968^{4}$, $1014^{22}$, $1058^{22}$, $1078^{22}$, $1114^{22}$, $1191^{22}$, $1340^{22}$, $1372^{22}$, $1392^{22}$, $1429^{22}$, $1468^{22}$, $1525^{22}$, $1584^{22}$, $1624^{22}$, $1687^{22}$, $1807^{22}$, $1893^{22}$, $1923^{22}$, $1972^{22}$, $2100^{22}$, $2233^{4}$, $2287^{22}$, $2771^{22}$, $3730^{22}$, $55099^{4}$, $134728^{4}$, $504130^{4}$
\item $(h,k)=(1,13)$ : $1^{4}$, $7^{22}$, $10^{22}$, $11^{10}$, $12^{22}$, $13^{22}$, $16^{22}$, $22^{2}$, $27^{22}$, $31^{22}$, $33^{4}$, $39^{22}$, $44^{8}$, $46^{22}$, $53^{22}$, $55^{10}$, $66^{6}$, $70^{22}$, $75^{22}$, $77^{2}$, $80^{22}$, $88^{2}$, $99^{24}$, $110^{4}$, $121^{2}$, $128^{22}$, $133^{22}$, $144^{22}$, $147^{22}$, $175^{22}$, $188^{22}$, $194^{22}$, $201^{44}$, $208^{22}$, $218^{22}$, $245^{22}$, $246^{22}$, $252^{22}$, $258^{44}$, $263^{22}$, $275^{22}$, $279^{22}$, $281^{44}$, $288^{22}$, $299^{22}$, $304^{22}$, $316^{22}$, $347^{22}$, $397^{22}$, $401^{22}$, $421^{22}$, $424^{22}$, $430^{22}$, $436^{22}$, $440^{22}$, $449^{22}$, $471^{22}$, $490^{22}$, $521^{22}$, $542^{22}$, $588^{22}$, $602^{22}$, $609^{22}$, $626^{44}$, $650^{22}$, $686^{22}$, $694^{22}$, $695^{22}$, $736^{22}$, $742^{22}$, $766^{22}$, $771^{22}$, $781^{22}$, $789^{22}$, $841^{22}$, $871^{22}$, $899^{22}$, $904^{22}$, $917^{22}$, $961^{22}$, $969^{22}$, $1027^{22}$, $1096^{22}$, $1113^{22}$, $1120^{22}$, $1135^{22}$, $1211^{22}$, $1217^{22}$, $1254^{2}$, $1291^{22}$, $1366^{22}$, $1393^{22}$, $1699^{22}$, $1768^{22}$, $1882^{22}$, $2036^{22}$, $2226^{22}$, $2345^{22}$, $2377^{22}$, $2665^{22}$, $2782^{22}$, $2804^{22}$, $89518^{4}$, $609004^{4}$
\item $(h,k)=(1,15)$ : $1^{4}$, $12^{22}$, $15^{22}$, $19^{22}$, $20^{22}$, $31^{22}$, $40^{22}$, $53^{22}$, $55^{22}$, $61^{22}$, $62^{22}$, $84^{22}$, $95^{22}$, $99^{22}$, $105^{22}$, $118^{22}$, $121^{2}$, $134^{44}$, $140^{22}$, $143^{22}$, $167^{22}$, $168^{22}$, $175^{22}$, $189^{22}$, $194^{22}$, $199^{22}$, $226^{22}$, $234^{44}$, $240^{22}$, $247^{22}$, $249^{22}$, $250^{22}$, $253^{22}$, $279^{22}$, $283^{22}$, $298^{22}$, $313^{22}$, $318^{22}$, $336^{22}$, $381^{22}$, $385^{4}$, $405^{22}$, $416^{22}$, $430^{22}$, $438^{22}$, $480^{22}$, $515^{22}$, $517^{22}$, $538^{22}$, $553^{22}$, $589^{22}$, $596^{22}$, $601^{22}$, $644^{22}$, $653^{22}$, $657^{22}$, $673^{22}$, $677^{22}$, $732^{22}$, $754^{22}$, $762^{22}$, $769^{22}$, $833^{22}$, $859^{22}$, $878^{22}$, $902^{2}$, $928^{22}$, $979^{22}$, $1037^{22}$, $1072^{22}$, $1099^{22}$, $1118^{22}$, $1148^{22}$, $1155^{22}$, $1166^{22}$, $1168^{22}$, $1170^{22}$, $1190^{22}$, $1202^{22}$, $1232^{22}$, $1241^{22}$, $1250^{22}$, $1371^{22}$, $1442^{22}$, $1456^{22}$, $1534^{22}$, $1536^{22}$, $1634^{22}$, $1680^{22}$, $1717^{22}$, $1722^{22}$, $1961^{22}$, $2131^{22}$, $2148^{22}$, $2212^{22}$, $3003^{4}$, $4147^{4}$, $7172^{2}$, $8239^{4}$, $59675^{4}$, $90013^{4}$, $246411^{4}$, $289839^{4}$
\item $(h,k)=(1,17)$ : $1^{4}$, $7^{22}$, $11^{6}$, $14^{22}$, $19^{22}$, $30^{22}$, $41^{22}$, $44^{2}$, $45^{22}$, $47^{22}$, $50^{44}$, $57^{44}$, $88^{22}$, $89^{22}$, $90^{44}$, $105^{22}$, $140^{22}$, $143^{4}$, $154^{22}$, $160^{22}$, $166^{22}$, $174^{22}$, $194^{22}$, $200^{22}$, $203^{22}$, $213^{22}$, $223^{22}$, $225^{22}$, $232^{22}$, $237^{22}$, $245^{22}$, $253^{4}$, $270^{44}$, $279^{22}$, $300^{22}$, $317^{22}$, $347^{22}$, $363^{22}$, $381^{22}$, $384^{22}$, $396^{22}$, $420^{22}$, $433^{22}$, $440^{22}$, $454^{22}$, $462^{24}$, $463^{22}$, $479^{22}$, $483^{22}$, $485^{22}$, $493^{22}$, $510^{22}$, $514^{22}$, $532^{22}$, $537^{22}$, $547^{22}$, $562^{22}$, $577^{22}$, $580^{22}$, $592^{22}$, $595^{22}$, $638^{4}$, $657^{22}$, $658^{22}$, $662^{22}$, $673^{22}$, $676^{22}$, $679^{22}$, $767^{22}$, $843^{22}$, $848^{22}$, $857^{22}$, $858^{22}$, $860^{22}$, $864^{22}$, $875^{22}$, $891^{4}$, $905^{22}$, $913^{4}$, $937^{22}$, $945^{22}$, $947^{22}$, $951^{22}$, $1020^{22}$, $1100^{4}$, $1106^{22}$, $1189^{22}$, $1199^{22}$, $1259^{22}$, $1274^{22}$, $1295^{22}$, $1454^{22}$, $1492^{22}$, $1495^{22}$, $1585^{22}$, $1821^{22}$, $1975^{22}$, $2322^{22}$, $2603^{22}$, $3501^{44}$, $3830^{22}$, $4128^{22}$, $11209^{4}$, $28182^{4}$, $102674^{4}$, $191433^{4}$, $322454^{4}$
\item $(h,k)=(1,19)$ : $1^{4}$, $8^{22}$, $11^{4}$, $20^{22}$, $21^{22}$, $40^{22}$, $42^{22}$, $49^{22}$, $51^{22}$, $52^{22}$, $55^{2}$, $59^{22}$, $61^{22}$, $77^{4}$, $122^{22}$, $128^{22}$, $129^{22}$, $134^{22}$, $143^{4}$, $145^{22}$, $151^{22}$, $155^{22}$, $163^{22}$, $164^{22}$, $165^{2}$, $168^{22}$, $171^{22}$, $178^{22}$, $184^{22}$, $186^{22}$, $195^{22}$, $211^{22}$, $213^{22}$, $214^{22}$, $226^{22}$, $234^{22}$, $239^{22}$, $241^{22}$, $253^{22}$, $262^{22}$, $264^{44}$, $268^{22}$, $278^{22}$, $287^{44}$, $298^{22}$, $332^{22}$, $335^{22}$, $341^{22}$, $361^{22}$, $381^{22}$, $438^{22}$, $443^{22}$, $476^{22}$, $509^{22}$, $528^{22}$, $532^{22}$, $537^{22}$, $578^{22}$, $581^{22}$, $582^{22}$, $594^{22}$, $622^{22}$, $635^{22}$, $642^{22}$, $646^{22}$, $649^{2}$, $650^{22}$, $685^{22}$, $686^{22}$, $700^{22}$, $712^{22}$, $715^{4}$, $780^{22}$, $801^{22}$, $809^{22}$, $862^{22}$, $863^{22}$, $924^{22}$, $946^{22}$, $979^{4}$, $1011^{22}$, $1060^{22}$, $1062^{22}$, $1100^{22}$, $1124^{22}$, $1269^{22}$, $1358^{22}$, $1405^{22}$, $1415^{22}$, $1701^{22}$, $1777^{22}$, $1846^{22}$, $1910^{22}$, $1923^{22}$, $1968^{22}$, $1999^{22}$, $2477^{22}$, $2700^{22}$, $2997^{22}$, $4887^{22}$, $31372^{4}$, $56177^{4}$, $610434^{4}$
\item $(h,k)=(1,21)$ : $1^{4}$, $3^{1398100}$
\item $(h,k)=(3,5)$ : $1^{4}$, $11^{4}$, $12^{22}$, $13^{22}$, $22^{4}$, $25^{22}$, $44^{6}$, $47^{22}$, $55^{22}$, $56^{22}$, $66^{8}$, $74^{22}$, $79^{44}$, $93^{22}$, $135^{22}$, $163^{22}$, $169^{22}$, $178^{22}$, $184^{22}$, $194^{22}$, $198^{22}$, $231^{2}$, $236^{22}$, $265^{44}$, $276^{22}$, $313^{22}$, $319^{26}$, $326^{22}$, $331^{22}$, $333^{22}$, $350^{22}$, $354^{22}$, $358^{22}$, $377^{22}$, $384^{22}$, $385^{22}$, $388^{22}$, $404^{22}$, $412^{22}$, $417^{22}$, $423^{22}$, $433^{22}$, $437^{22}$, $442^{22}$, $460^{22}$, $464^{44}$, $476^{22}$, $497^{22}$, $499^{22}$, $502^{22}$, $513^{22}$, $527^{22}$, $553^{22}$, $556^{22}$, $565^{22}$, $572^{2}$, $597^{22}$, $606^{22}$, $620^{22}$, $667^{22}$, $697^{22}$, $726^{22}$, $748^{2}$, $754^{22}$, $804^{22}$, $810^{22}$, $821^{22}$, $828^{22}$, $883^{44}$, $901^{22}$, $938^{22}$, $965^{22}$, $966^{22}$, $974^{22}$, $998^{22}$, $1000^{22}$, $1030^{22}$, $1046^{22}$, $1110^{22}$, $1120^{22}$, $1132^{22}$, $1156^{22}$, $1191^{22}$, $1194^{22}$, $1203^{22}$, $1219^{22}$, $1272^{22}$, $1293^{22}$, $1355^{22}$, $1405^{22}$, $1516^{22}$, $1562^{4}$, $1680^{22}$, $1705^{4}$, $1706^{22}$, $1895^{22}$, $1919^{22}$, $2604^{22}$, $2629^{22}$, $5445^{4}$, $9647^{4}$, $26587^{4}$, $171292^{4}$, $483571^{4}$
\item $(h,k)=(3,7)$ : $1^{4}$, $11^{2}$, $14^{22}$, $22^{4}$, $33^{2}$, $34^{22}$, $37^{22}$, $43^{22}$, $44^{4}$, $55^{2}$, $57^{22}$, $62^{22}$, $66^{4}$, $67^{22}$, $76^{22}$, $77^{2}$, $103^{22}$, $107^{22}$, $111^{22}$, $112^{22}$, $131^{22}$, $132^{24}$, $135^{22}$, $139^{22}$, $145^{22}$, $149^{22}$, $156^{22}$, $160^{22}$, $162^{22}$, $188^{22}$, $194^{22}$, $198^{2}$, $201^{22}$, $204^{22}$, $212^{22}$, $230^{22}$, $246^{22}$, $248^{22}$, $259^{22}$, $263^{22}$, $290^{22}$, $298^{22}$, $299^{22}$, $311^{22}$, $312^{22}$, $330^{22}$, $352^{26}$, $361^{22}$, $367^{22}$, $413^{22}$, $416^{22}$, $433^{22}$, $451^{2}$, $488^{22}$, $491^{22}$, $494^{22}$, $503^{22}$, $538^{22}$, $549^{22}$, $570^{22}$, $643^{22}$, $653^{22}$, $654^{22}$, $656^{22}$, $657^{22}$, $664^{22}$, $704^{22}$, $723^{22}$, $734^{22}$, $735^{22}$, $741^{22}$, $755^{22}$, $757^{22}$, $787^{22}$, $850^{22}$, $851^{22}$, $875^{22}$, $876^{22}$, $910^{22}$, $926^{22}$, $964^{22}$, $995^{22}$, $1005^{22}$, $1011^{22}$, $1012^{2}$, $1050^{22}$, $1063^{22}$, $1077^{22}$, $1083^{22}$, $1086^{22}$, $1111^{22}$, $1133^{22}$, $1195^{22}$, $1393^{22}$, $1550^{22}$, $1566^{22}$, $1865^{22}$, $1940^{22}$, $1972^{22}$, $2155^{22}$, $2175^{22}$, $2305^{22}$, $3087^{22}$, $3146^{4}$, $3297^{22}$, $19349^{4}$, $23584^{4}$, $652212^{4}$
\item $(h,k)=(3,9)$ : $1^{4}$, $13^{22}$, $22^{4}$, $23^{22}$, $25^{22}$, $31^{22}$, $33^{4}$, $34^{22}$, $42^{22}$, $67^{22}$, $83^{22}$, $85^{22}$, $96^{22}$, $99^{4}$, $100^{22}$, $115^{22}$, $119^{22}$, $125^{22}$, $132^{4}$, $151^{44}$, $152^{22}$, $156^{22}$, $157^{22}$, $158^{22}$, $160^{22}$, $165^{44}$, $168^{22}$, $186^{22}$, $192^{22}$, $196^{22}$, $205^{22}$, $214^{22}$, $230^{22}$, $235^{22}$, $241^{22}$, $244^{22}$, $256^{22}$, $271^{22}$, $290^{22}$, $308^{4}$, $324^{22}$, $332^{22}$, $335^{22}$, $341^{22}$, $343^{22}$, $358^{22}$, $359^{22}$, $360^{22}$, $394^{22}$, $405^{22}$, $425^{22}$, $433^{22}$, $434^{22}$, $454^{22}$, $468^{22}$, $478^{22}$, $488^{22}$, $494^{22}$, $495^{22}$, $511^{22}$, $515^{22}$, $546^{22}$, $551^{22}$, $554^{44}$, $597^{22}$, $644^{22}$, $670^{22}$, $673^{22}$, $711^{22}$, $761^{22}$, $798^{22}$, $819^{22}$, $823^{22}$, $824^{22}$, $831^{22}$, $858^{22}$, $898^{22}$, $937^{22}$, $968^{4}$, $984^{22}$, $1023^{2}$, $1034^{22}$, $1055^{22}$, $1058^{22}$, $1086^{22}$, $1090^{22}$, $1250^{22}$, $1415^{22}$, $1491^{22}$, $1517^{22}$, $1647^{22}$, $1668^{22}$, $1701^{22}$, $1711^{22}$, $1762^{22}$, $1896^{22}$, $2089^{22}$, $2181^{22}$, $2508^{4}$, $2635^{22}$, $2819^{22}$, $3014^{22}$, $6127^{4}$, $7629^{44}$, $42240^{4}$, $56859^{4}$, $65098^{4}$, $208395^{4}$, $239261^{4}$
\item $(h,k)=(3,13)$ : $1^{4}$, $8^{22}$, $11^{4}$, $19^{22}$, $20^{22}$, $26^{22}$, $29^{22}$, $32^{22}$, $33^{4}$, $36^{22}$, $48^{22}$, $59^{22}$, $61^{22}$, $68^{22}$, $80^{44}$, $82^{22}$, $86^{22}$, $88^{24}$, $92^{22}$, $96^{22}$, $111^{22}$, $114^{22}$, $151^{22}$, $161^{44}$, $164^{22}$, $166^{22}$, $170^{22}$, $172^{22}$, $176^{22}$, $180^{22}$, $184^{22}$, $215^{22}$, $228^{22}$, $235^{22}$, $248^{22}$, $264^{22}$, $269^{22}$, $302^{22}$, $323^{22}$, $325^{22}$, $335^{22}$, $348^{22}$, $350^{22}$, $416^{22}$, $420^{22}$, $425^{22}$, $446^{22}$, $456^{22}$, $460^{22}$, $465^{22}$, $475^{22}$, $501^{22}$, $568^{22}$, $570^{22}$, $578^{22}$, $596^{22}$, $602^{22}$, $607^{22}$, $625^{22}$, $648^{22}$, $672^{22}$, $678^{22}$, $719^{22}$, $750^{22}$, $751^{22}$, $752^{22}$, $761^{22}$, $795^{22}$, $820^{22}$, $823^{22}$, $831^{22}$, $841^{22}$, $879^{22}$, $910^{22}$, $944^{22}$, $996^{22}$, $1017^{22}$, $1023^{6}$, $1080^{22}$, $1185^{22}$, $1230^{22}$, $1236^{22}$, $1264^{22}$, $1276^{4}$, $1389^{22}$, $1409^{22}$, $1430^{4}$, $1439^{22}$, $1543^{22}$, $1680^{22}$, $1706^{22}$, $1725^{22}$, $1748^{22}$, $1858^{22}$, $2082^{22}$, $2228^{22}$, $2830^{22}$, $2873^{22}$, $3587^{22}$, $6919^{4}$, $8811^{4}$, $39864^{4}$, $163064^{4}$, $477708^{4}$
\item $(h,k)=(3,15)$ : $1^{4}$, $9^{22}$, $13^{22}$, $15^{22}$, $21^{22}$, $22^{10}$, $30^{22}$, $37^{44}$, $38^{22}$, $49^{22}$, $51^{22}$, $52^{22}$, $55^{22}$, $57^{22}$, $66^{6}$, $67^{22}$, $68^{22}$, $70^{22}$, $78^{22}$, $81^{22}$, $82^{22}$, $83^{22}$, $91^{22}$, $99^{4}$, $104^{22}$, $125^{22}$, $140^{22}$, $154^{22}$, $155^{22}$, $156^{44}$, $158^{22}$, $165^{22}$, $200^{22}$, $220^{22}$, $232^{22}$, $233^{22}$, $238^{22}$, $267^{22}$, $272^{22}$, $276^{22}$, $288^{22}$, $289^{22}$, $307^{22}$, $327^{22}$, $341^{22}$, $348^{22}$, $355^{22}$, $367^{22}$, $388^{22}$, $410^{22}$, $446^{22}$, $462^{22}$, $467^{22}$, $473^{22}$, $479^{22}$, $483^{22}$, $493^{22}$, $503^{22}$, $515^{22}$, $527^{22}$, $550^{22}$, $583^{22}$, $603^{22}$, $638^{4}$, $657^{22}$, $663^{22}$, $704^{22}$, $784^{22}$, $794^{22}$, $796^{22}$, $807^{22}$, $823^{22}$, $826^{22}$, $834^{22}$, $844^{22}$, $863^{22}$, $891^{22}$, $935^{2}$, $949^{22}$, $967^{22}$, $1004^{44}$, $1035^{22}$, $1096^{22}$, $1241^{22}$, $1244^{22}$, $1287^{4}$, $1300^{22}$, $1309^{4}$, $1358^{22}$, $1363^{22}$, $1395^{22}$, $1477^{22}$, $1546^{22}$, $1560^{22}$, $1589^{22}$, $1738^{4}$, $1776^{22}$, $1781^{22}$, $2128^{22}$, $2253^{22}$, $2271^{22}$, $2497^{22}$, $2701^{22}$, $3157^{22}$, $14832^{44}$, $15818^{4}$, $29799^{4}$, $33000^{4}$, $102179^{4}$, $350691^{4}$
\item $(h,k)=(3,17)$ : $1^{4}$, $7^{22}$, $11^{2}$, $23^{22}$, $39^{22}$, $54^{44}$, $66^{4}$, $68^{22}$, $69^{22}$, $70^{22}$, $74^{22}$, $81^{22}$, $84^{22}$, $85^{44}$, $115^{22}$, $132^{66}$, $137^{22}$, $143^{4}$, $146^{22}$, $154^{2}$, $161^{22}$, $168^{22}$, $169^{22}$, $172^{22}$, $194^{22}$, $198^{22}$, $201^{22}$, $230^{22}$, $233^{22}$, $235^{44}$, $240^{22}$, $246^{22}$, $256^{22}$, $258^{22}$, $264^{2}$, $276^{22}$, $279^{22}$, $286^{2}$, $308^{24}$, $311^{22}$, $318^{22}$, $330^{22}$, $332^{22}$, $348^{22}$, $353^{22}$, $354^{22}$, $359^{22}$, $380^{22}$, $381^{22}$, $388^{22}$, $395^{22}$, $400^{22}$, $406^{22}$, $414^{22}$, $426^{22}$, $437^{22}$, $472^{22}$, $474^{22}$, $518^{22}$, $580^{22}$, $590^{22}$, $595^{22}$, $606^{22}$, $608^{22}$, $618^{22}$, $643^{22}$, $669^{22}$, $700^{22}$, $723^{22}$, $724^{22}$, $748^{22}$, $814^{24}$, $816^{22}$, $833^{22}$, $900^{22}$, $901^{22}$, $935^{22}$, $945^{22}$, $991^{22}$, $1030^{22}$, $1041^{22}$, $1058^{22}$, $1168^{22}$, $1207^{22}$, $1265^{22}$, $1339^{22}$, $1378^{22}$, $1490^{22}$, $1507^{4}$, $1512^{22}$, $1545^{22}$, $1551^{4}$, $1575^{22}$, $1633^{22}$, $1649^{22}$, $2035^{22}$, $2055^{22}$, $2315^{22}$, $2937^{22}$, $3138^{22}$, $3664^{22}$, $29271^{4}$, $162151^{4}$, $184052^{4}$, $320133^{4}$
\item $(h,k)=(3,19)$ : $1^{4}$, $3^{1398100}$
\item $(h,k)=(3,21)$ : $1^{4}$, $8^{22}$, $11^{4}$, $20^{22}$, $21^{22}$, $40^{22}$, $42^{22}$, $49^{22}$, $51^{22}$, $52^{22}$, $55^{2}$, $59^{22}$, $61^{22}$, $77^{4}$, $122^{22}$, $128^{22}$, $129^{22}$, $134^{22}$, $143^{4}$, $145^{22}$, $151^{22}$, $155^{22}$, $163^{22}$, $164^{22}$, $165^{2}$, $168^{22}$, $171^{22}$, $178^{22}$, $184^{22}$, $186^{22}$, $195^{22}$, $211^{22}$, $213^{22}$, $214^{22}$, $226^{22}$, $234^{22}$, $239^{22}$, $241^{22}$, $253^{22}$, $262^{22}$, $264^{44}$, $268^{22}$, $278^{22}$, $287^{44}$, $298^{22}$, $332^{22}$, $335^{22}$, $341^{22}$, $361^{22}$, $381^{22}$, $438^{22}$, $443^{22}$, $476^{22}$, $509^{22}$, $528^{22}$, $532^{22}$, $537^{22}$, $578^{22}$, $581^{22}$, $582^{22}$, $594^{22}$, $622^{22}$, $635^{22}$, $642^{22}$, $646^{22}$, $649^{2}$, $650^{22}$, $685^{22}$, $686^{22}$, $700^{22}$, $712^{22}$, $715^{4}$, $780^{22}$, $801^{22}$, $809^{22}$, $862^{22}$, $863^{22}$, $924^{22}$, $946^{22}$, $979^{4}$, $1011^{22}$, $1060^{22}$, $1062^{22}$, $1100^{22}$, $1124^{22}$, $1269^{22}$, $1358^{22}$, $1405^{22}$, $1415^{22}$, $1701^{22}$, $1777^{22}$, $1846^{22}$, $1910^{22}$, $1923^{22}$, $1968^{22}$, $1999^{22}$, $2477^{22}$, $2700^{22}$, $2997^{22}$, $4887^{22}$, $31372^{4}$, $56177^{4}$, $610434^{4}$
\item $(h,k)=(5,7)$ : $1^{4}$, $7^{22}$, $11^{26}$, $22^{2}$, $25^{22}$, $30^{22}$, $31^{22}$, $44^{2}$, $46^{22}$, $53^{22}$, $74^{22}$, $80^{22}$, $81^{22}$, $84^{22}$, $86^{22}$, $89^{22}$, $97^{22}$, $110^{4}$, $119^{22}$, $125^{22}$, $134^{22}$, $152^{22}$, $163^{22}$, $165^{26}$, $176^{2}$, $177^{22}$, $182^{22}$, $195^{22}$, $198^{22}$, $207^{22}$, $232^{22}$, $233^{22}$, $237^{22}$, $240^{22}$, $256^{22}$, $260^{22}$, $264^{2}$, $282^{22}$, $289^{22}$, $300^{22}$, $381^{22}$, $383^{22}$, $384^{22}$, $402^{22}$, $405^{22}$, $407^{22}$, $417^{22}$, $442^{22}$, $460^{22}$, $462^{22}$, $470^{22}$, $504^{44}$, $509^{22}$, $519^{22}$, $568^{22}$, $589^{22}$, $608^{22}$, $642^{22}$, $646^{22}$, $653^{22}$, $704^{22}$, $709^{22}$, $725^{22}$, $727^{22}$, $735^{22}$, $748^{22}$, $769^{22}$, $780^{22}$, $781^{2}$, $803^{22}$, $830^{22}$, $845^{66}$, $883^{22}$, $889^{22}$, $894^{22}$, $949^{22}$, $1141^{22}$, $1145^{22}$, $1200^{22}$, $1204^{22}$, $1279^{22}$, $1308^{22}$, $1377^{22}$, $1436^{22}$, $1488^{22}$, $1588^{22}$, $1596^{22}$, $1625^{22}$, $1635^{22}$, $1713^{22}$, $1730^{22}$, $1929^{22}$, $1963^{22}$, $2002^{4}$, $2046^{22}$, $2082^{22}$, $2115^{22}$, $2574^{4}$, $2920^{22}$, $3685^{4}$, $170973^{4}$, $519189^{4}$
\item $(h,k)=(5,9)$ : $1^{4}$, $5^{22}$, $17^{22}$, $26^{22}$, $28^{22}$, $40^{22}$, $44^{4}$, $45^{22}$, $51^{22}$, $57^{22}$, $58^{22}$, $70^{22}$, $77^{22}$, $78^{22}$, $90^{44}$, $94^{22}$, $95^{22}$, $105^{22}$, $114^{22}$, $125^{22}$, $154^{22}$, $160^{22}$, $175^{22}$, $189^{44}$, $220^{22}$, $224^{22}$, $226^{22}$, $286^{24}$, $288^{22}$, $313^{22}$, $347^{22}$, $350^{22}$, $402^{22}$, $412^{22}$, $424^{22}$, $429^{22}$, $437^{22}$, $489^{22}$, $493^{22}$, $506^{2}$, $518^{22}$, $525^{22}$, $568^{22}$, $585^{22}$, $590^{44}$, $593^{22}$, $595^{22}$, $608^{44}$, $613^{22}$, $647^{22}$, $673^{22}$, $674^{22}$, $680^{22}$, $685^{22}$, $687^{44}$, $712^{22}$, $719^{22}$, $732^{22}$, $760^{22}$, $789^{22}$, $809^{22}$, $819^{22}$, $823^{22}$, $845^{22}$, $881^{22}$, $929^{22}$, $942^{22}$, $988^{22}$, $1007^{22}$, $1016^{22}$, $1018^{22}$, $1023^{2}$, $1037^{22}$, $1041^{22}$, $1050^{22}$, $1111^{22}$, $1140^{22}$, $1330^{22}$, $1338^{22}$, $1386^{22}$, $1397^{44}$, $1469^{22}$, $1472^{22}$, $1495^{22}$, $1543^{22}$, $1545^{22}$, $1942^{22}$, $2136^{22}$, $2160^{22}$, $2458^{22}$, $2605^{22}$, $3960^{4}$, $11649^{4}$, $22341^{4}$, $660759^{4}$
\item $(h,k)=(5,13)$ : $1^{4}$, $4^{22}$, $9^{22}$, $12^{22}$, $15^{22}$, $16^{22}$, $17^{22}$, $22^{26}$, $24^{22}$, $35^{22}$, $44^{4}$, $50^{22}$, $65^{22}$, $66^{2}$, $73^{22}$, $75^{22}$, $78^{22}$, $83^{66}$, $88^{22}$, $96^{22}$, $98^{22}$, $100^{22}$, $132^{4}$, $146^{22}$, $162^{22}$, $166^{22}$, $181^{22}$, $189^{22}$, $205^{22}$, $208^{22}$, $213^{22}$, $225^{22}$, $242^{2}$, $249^{22}$, $257^{22}$, $309^{22}$, $312^{22}$, $317^{22}$, $319^{4}$, $324^{22}$, $352^{22}$, $370^{22}$, $374^{22}$, $384^{22}$, $392^{22}$, $396^{22}$, $406^{22}$, $424^{22}$, $432^{22}$, $436^{22}$, $460^{22}$, $478^{22}$, $487^{22}$, $488^{22}$, $497^{22}$, $543^{22}$, $554^{22}$, $578^{22}$, $598^{22}$, $613^{22}$, $615^{22}$, $630^{22}$, $637^{22}$, $703^{22}$, $759^{22}$, $760^{22}$, $767^{22}$, $774^{22}$, $797^{22}$, $806^{22}$, $840^{22}$, $851^{22}$, $855^{22}$, $876^{22}$, $894^{22}$, $904^{22}$, $989^{22}$, $1035^{22}$, $1051^{22}$, $1157^{22}$, $1169^{22}$, $1184^{22}$, $1190^{22}$, $1221^{22}$, $1289^{22}$, $1338^{22}$, $1416^{22}$, $1465^{22}$, $1718^{22}$, $1775^{22}$, $1943^{22}$, $1984^{22}$, $2085^{22}$, $2100^{22}$, $2175^{22}$, $2514^{22}$, $3038^{22}$, $3429^{22}$, $145013^{4}$, $553179^{4}$
\item $(h,k)=(5,15)$ : $1^{4}$, $5^{22}$, $8^{44}$, $11^{22}$, $15^{22}$, $40^{22}$, $58^{44}$, $59^{22}$, $62^{22}$, $81^{44}$, $102^{22}$, $109^{22}$, $111^{22}$, $118^{22}$, $162^{22}$, $166^{22}$, $176^{22}$, $180^{22}$, $194^{44}$, $200^{22}$, $202^{22}$, $212^{22}$, $220^{2}$, $224^{44}$, $231^{22}$, $242^{22}$, $244^{22}$, $266^{22}$, $276^{22}$, $297^{22}$, $308^{22}$, $323^{22}$, $330^{22}$, $331^{22}$, $369^{22}$, $374^{22}$, $384^{22}$, $390^{22}$, $415^{22}$, $418^{22}$, $444^{22}$, $464^{22}$, $486^{22}$, $514^{22}$, $520^{22}$, $523^{22}$, $525^{22}$, $536^{22}$, $546^{22}$, $551^{22}$, $564^{22}$, $566^{22}$, $570^{22}$, $584^{22}$, $593^{22}$, $596^{22}$, $616^{22}$, $696^{22}$, $722^{22}$, $779^{22}$, $780^{22}$, $783^{22}$, $800^{22}$, $803^{2}$, $819^{22}$, $835^{22}$, $839^{22}$, $841^{22}$, $843^{22}$, $859^{22}$, $862^{22}$, $933^{22}$, $942^{22}$, $946^{4}$, $980^{22}$, $1023^{26}$, $1129^{22}$, $1184^{22}$, $1195^{22}$, $1306^{22}$, $1321^{22}$, $1332^{22}$, $1343^{22}$, $1468^{22}$, $1561^{22}$, $1764^{22}$, $1883^{22}$, $1946^{22}$, $1970^{22}$, $2079^{22}$, $2228^{22}$, $2294^{22}$, $2522^{22}$, $3178^{22}$, $45397^{4}$, $51128^{4}$, $219021^{4}$, $381326^{4}$
\item $(h,k)=(5,17)$ : $1^{4}$, $3^{1398100}$
\item $(h,k)=(5,19)$ : $1^{4}$, $7^{22}$, $11^{2}$, $23^{22}$, $39^{22}$, $54^{44}$, $66^{4}$, $68^{22}$, $69^{22}$, $70^{22}$, $74^{22}$, $81^{22}$, $84^{22}$, $85^{44}$, $115^{22}$, $132^{66}$, $137^{22}$, $143^{4}$, $146^{22}$, $154^{2}$, $161^{22}$, $168^{22}$, $169^{22}$, $172^{22}$, $194^{22}$, $198^{22}$, $201^{22}$, $230^{22}$, $233^{22}$, $235^{44}$, $240^{22}$, $246^{22}$, $256^{22}$, $258^{22}$, $264^{2}$, $276^{22}$, $279^{22}$, $286^{2}$, $308^{24}$, $311^{22}$, $318^{22}$, $330^{22}$, $332^{22}$, $348^{22}$, $353^{22}$, $354^{22}$, $359^{22}$, $380^{22}$, $381^{22}$, $388^{22}$, $395^{22}$, $400^{22}$, $406^{22}$, $414^{22}$, $426^{22}$, $437^{22}$, $472^{22}$, $474^{22}$, $518^{22}$, $580^{22}$, $590^{22}$, $595^{22}$, $606^{22}$, $608^{22}$, $618^{22}$, $643^{22}$, $669^{22}$, $700^{22}$, $723^{22}$, $724^{22}$, $748^{22}$, $814^{24}$, $816^{22}$, $833^{22}$, $900^{22}$, $901^{22}$, $935^{22}$, $945^{22}$, $991^{22}$, $1030^{22}$, $1041^{22}$, $1058^{22}$, $1168^{22}$, $1207^{22}$, $1265^{22}$, $1339^{22}$, $1378^{22}$, $1490^{22}$, $1507^{4}$, $1512^{22}$, $1545^{22}$, $1551^{4}$, $1575^{22}$, $1633^{22}$, $1649^{22}$, $2035^{22}$, $2055^{22}$, $2315^{22}$, $2937^{22}$, $3138^{22}$, $3664^{22}$, $29271^{4}$, $162151^{4}$, $184052^{4}$, $320133^{4}$
\item $(h,k)=(5,21)$ : $1^{4}$, $7^{22}$, $11^{6}$, $14^{22}$, $19^{22}$, $30^{22}$, $41^{22}$, $44^{2}$, $45^{22}$, $47^{22}$, $50^{44}$, $57^{44}$, $88^{22}$, $89^{22}$, $90^{44}$, $105^{22}$, $140^{22}$, $143^{4}$, $154^{22}$, $160^{22}$, $166^{22}$, $174^{22}$, $194^{22}$, $200^{22}$, $203^{22}$, $213^{22}$, $223^{22}$, $225^{22}$, $232^{22}$, $237^{22}$, $245^{22}$, $253^{4}$, $270^{44}$, $279^{22}$, $300^{22}$, $317^{22}$, $347^{22}$, $363^{22}$, $381^{22}$, $384^{22}$, $396^{22}$, $420^{22}$, $433^{22}$, $440^{22}$, $454^{22}$, $462^{24}$, $463^{22}$, $479^{22}$, $483^{22}$, $485^{22}$, $493^{22}$, $510^{22}$, $514^{22}$, $532^{22}$, $537^{22}$, $547^{22}$, $562^{22}$, $577^{22}$, $580^{22}$, $592^{22}$, $595^{22}$, $638^{4}$, $657^{22}$, $658^{22}$, $662^{22}$, $673^{22}$, $676^{22}$, $679^{22}$, $767^{22}$, $843^{22}$, $848^{22}$, $857^{22}$, $858^{22}$, $860^{22}$, $864^{22}$, $875^{22}$, $891^{4}$, $905^{22}$, $913^{4}$, $937^{22}$, $945^{22}$, $947^{22}$, $951^{22}$, $1020^{22}$, $1100^{4}$, $1106^{22}$, $1189^{22}$, $1199^{22}$, $1259^{22}$, $1274^{22}$, $1295^{22}$, $1454^{22}$, $1492^{22}$, $1495^{22}$, $1585^{22}$, $1821^{22}$, $1975^{22}$, $2322^{22}$, $2603^{22}$, $3501^{44}$, $3830^{22}$, $4128^{22}$, $11209^{4}$, $28182^{4}$, $102674^{4}$, $191433^{4}$, $322454^{4}$
\item $(h,k)=(7,9)$ : $1^{4}$, $3^{22}$, $5^{22}$, $11^{6}$, $18^{22}$, $23^{22}$, $37^{22}$, $41^{22}$, $44^{6}$, $56^{22}$, $66^{6}$, $70^{44}$, $75^{22}$, $78^{22}$, $88^{22}$, $92^{22}$, $108^{22}$, $125^{22}$, $132^{22}$, $139^{22}$, $145^{22}$, $162^{22}$, $178^{22}$, $195^{22}$, $198^{6}$, $199^{22}$, $216^{22}$, $219^{22}$, $226^{22}$, $229^{22}$, $231^{2}$, $249^{22}$, $256^{22}$, $261^{22}$, $278^{22}$, $315^{22}$, $337^{22}$, $343^{22}$, $352^{22}$, $358^{22}$, $373^{22}$, $380^{22}$, $407^{88}$, $413^{22}$, $427^{22}$, $444^{22}$, $453^{22}$, $464^{22}$, $468^{22}$, $469^{22}$, $470^{22}$, $473^{2}$, $483^{22}$, $505^{22}$, $531^{22}$, $556^{22}$, $558^{22}$, $577^{22}$, $583^{22}$, $613^{22}$, $628^{22}$, $646^{22}$, $666^{22}$, $702^{22}$, $705^{22}$, $708^{22}$, $745^{22}$, $791^{22}$, $807^{22}$, $817^{22}$, $820^{44}$, $886^{22}$, $904^{22}$, $942^{22}$, $1003^{22}$, $1006^{22}$, $1077^{22}$, $1109^{22}$, $1122^{22}$, $1163^{22}$, $1260^{22}$, $1268^{22}$, $1363^{22}$, $1426^{22}$, $1495^{22}$, $1504^{22}$, $1768^{22}$, $1855^{22}$, $1865^{22}$, $2025^{22}$, $2048^{22}$, $2097^{22}$, $2173^{22}$, $2928^{22}$, $3862^{22}$, $4752^{2}$, $7007^{4}$, $11319^{4}$, $146905^{4}$, $261360^{4}$, $271854^{4}$
\item $(h,k)=(7,13)$ : $1^{4}$, $6^{44}$, $7^{22}$, $11^{2}$, $14^{22}$, $19^{66}$, $22^{4}$, $31^{22}$, $38^{22}$, $44^{2}$, $45^{22}$, $51^{22}$, $75^{22}$, $76^{22}$, $86^{22}$, $88^{2}$, $99^{24}$, $106^{22}$, $111^{22}$, $114^{22}$, $121^{2}$, $123^{22}$, $132^{2}$, $137^{22}$, $148^{22}$, $149^{22}$, $164^{44}$, $165^{2}$, $167^{22}$, $199^{22}$, $212^{22}$, $228^{22}$, $231^{22}$, $276^{22}$, $277^{44}$, $306^{22}$, $309^{22}$, $318^{22}$, $336^{22}$, $341^{22}$, $342^{22}$, $346^{22}$, $351^{22}$, $352^{22}$, $392^{22}$, $395^{22}$, $405^{44}$, $425^{22}$, $448^{22}$, $454^{22}$, $477^{22}$, $493^{22}$, $511^{22}$, $517^{22}$, $519^{22}$, $526^{22}$, $542^{22}$, $621^{22}$, $627^{4}$, $640^{22}$, $659^{22}$, $689^{22}$, $699^{22}$, $715^{22}$, $724^{22}$, $750^{22}$, $787^{22}$, $788^{22}$, $831^{22}$, $843^{22}$, $879^{22}$, $887^{22}$, $898^{22}$, $902^{22}$, $910^{22}$, $945^{22}$, $970^{22}$, $985^{22}$, $1030^{22}$, $1062^{22}$, $1094^{22}$, $1178^{22}$, $1188^{2}$, $1195^{22}$, $1324^{22}$, $1394^{22}$, $1467^{22}$, $1486^{22}$, $1572^{22}$, $1723^{22}$, $1769^{22}$, $1833^{22}$, $2229^{22}$, $2619^{22}$, $2624^{22}$, $2722^{22}$, $2847^{22}$, $2915^{4}$, $2966^{22}$, $5110^{44}$, $8195^{4}$, $12650^{4}$, $102212^{4}$, $515900^{4}$
\item $(h,k)=(7,15)$ : $1^{4}$, $3^{1398100}$
\item $(h,k)=(7,17)$ : $1^{4}$, $5^{22}$, $8^{44}$, $11^{22}$, $15^{22}$, $40^{22}$, $58^{44}$, $59^{22}$, $62^{22}$, $81^{44}$, $102^{22}$, $109^{22}$, $111^{22}$, $118^{22}$, $162^{22}$, $166^{22}$, $176^{22}$, $180^{22}$, $194^{44}$, $200^{22}$, $202^{22}$, $212^{22}$, $220^{2}$, $224^{44}$, $231^{22}$, $242^{22}$, $244^{22}$, $266^{22}$, $276^{22}$, $297^{22}$, $308^{22}$, $323^{22}$, $330^{22}$, $331^{22}$, $369^{22}$, $374^{22}$, $384^{22}$, $390^{22}$, $415^{22}$, $418^{22}$, $444^{22}$, $464^{22}$, $486^{22}$, $514^{22}$, $520^{22}$, $523^{22}$, $525^{22}$, $536^{22}$, $546^{22}$, $551^{22}$, $564^{22}$, $566^{22}$, $570^{22}$, $584^{22}$, $593^{22}$, $596^{22}$, $616^{22}$, $696^{22}$, $722^{22}$, $779^{22}$, $780^{22}$, $783^{22}$, $800^{22}$, $803^{2}$, $819^{22}$, $835^{22}$, $839^{22}$, $841^{22}$, $843^{22}$, $859^{22}$, $862^{22}$, $933^{22}$, $942^{22}$, $946^{4}$, $980^{22}$, $1023^{26}$, $1129^{22}$, $1184^{22}$, $1195^{22}$, $1306^{22}$, $1321^{22}$, $1332^{22}$, $1343^{22}$, $1468^{22}$, $1561^{22}$, $1764^{22}$, $1883^{22}$, $1946^{22}$, $1970^{22}$, $2079^{22}$, $2228^{22}$, $2294^{22}$, $2522^{22}$, $3178^{22}$, $45397^{4}$, $51128^{4}$, $219021^{4}$, $381326^{4}$
\item $(h,k)=(7,19)$ : $1^{4}$, $9^{22}$, $13^{22}$, $15^{22}$, $21^{22}$, $22^{10}$, $30^{22}$, $37^{44}$, $38^{22}$, $49^{22}$, $51^{22}$, $52^{22}$, $55^{22}$, $57^{22}$, $66^{6}$, $67^{22}$, $68^{22}$, $70^{22}$, $78^{22}$, $81^{22}$, $82^{22}$, $83^{22}$, $91^{22}$, $99^{4}$, $104^{22}$, $125^{22}$, $140^{22}$, $154^{22}$, $155^{22}$, $156^{44}$, $158^{22}$, $165^{22}$, $200^{22}$, $220^{22}$, $232^{22}$, $233^{22}$, $238^{22}$, $267^{22}$, $272^{22}$, $276^{22}$, $288^{22}$, $289^{22}$, $307^{22}$, $327^{22}$, $341^{22}$, $348^{22}$, $355^{22}$, $367^{22}$, $388^{22}$, $410^{22}$, $446^{22}$, $462^{22}$, $467^{22}$, $473^{22}$, $479^{22}$, $483^{22}$, $493^{22}$, $503^{22}$, $515^{22}$, $527^{22}$, $550^{22}$, $583^{22}$, $603^{22}$, $638^{4}$, $657^{22}$, $663^{22}$, $704^{22}$, $784^{22}$, $794^{22}$, $796^{22}$, $807^{22}$, $823^{22}$, $826^{22}$, $834^{22}$, $844^{22}$, $863^{22}$, $891^{22}$, $935^{2}$, $949^{22}$, $967^{22}$, $1004^{44}$, $1035^{22}$, $1096^{22}$, $1241^{22}$, $1244^{22}$, $1287^{4}$, $1300^{22}$, $1309^{4}$, $1358^{22}$, $1363^{22}$, $1395^{22}$, $1477^{22}$, $1546^{22}$, $1560^{22}$, $1589^{22}$, $1738^{4}$, $1776^{22}$, $1781^{22}$, $2128^{22}$, $2253^{22}$, $2271^{22}$, $2497^{22}$, $2701^{22}$, $3157^{22}$, $14832^{44}$, $15818^{4}$, $29799^{4}$, $33000^{4}$, $102179^{4}$, $350691^{4}$
\item $(h,k)=(7,21)$ : $1^{4}$, $12^{22}$, $15^{22}$, $19^{22}$, $20^{22}$, $31^{22}$, $40^{22}$, $53^{22}$, $55^{22}$, $61^{22}$, $62^{22}$, $84^{22}$, $95^{22}$, $99^{22}$, $105^{22}$, $118^{22}$, $121^{2}$, $134^{44}$, $140^{22}$, $143^{22}$, $167^{22}$, $168^{22}$, $175^{22}$, $189^{22}$, $194^{22}$, $199^{22}$, $226^{22}$, $234^{44}$, $240^{22}$, $247^{22}$, $249^{22}$, $250^{22}$, $253^{22}$, $279^{22}$, $283^{22}$, $298^{22}$, $313^{22}$, $318^{22}$, $336^{22}$, $381^{22}$, $385^{4}$, $405^{22}$, $416^{22}$, $430^{22}$, $438^{22}$, $480^{22}$, $515^{22}$, $517^{22}$, $538^{22}$, $553^{22}$, $589^{22}$, $596^{22}$, $601^{22}$, $644^{22}$, $653^{22}$, $657^{22}$, $673^{22}$, $677^{22}$, $732^{22}$, $754^{22}$, $762^{22}$, $769^{22}$, $833^{22}$, $859^{22}$, $878^{22}$, $902^{2}$, $928^{22}$, $979^{22}$, $1037^{22}$, $1072^{22}$, $1099^{22}$, $1118^{22}$, $1148^{22}$, $1155^{22}$, $1166^{22}$, $1168^{22}$, $1170^{22}$, $1190^{22}$, $1202^{22}$, $1232^{22}$, $1241^{22}$, $1250^{22}$, $1371^{22}$, $1442^{22}$, $1456^{22}$, $1534^{22}$, $1536^{22}$, $1634^{22}$, $1680^{22}$, $1717^{22}$, $1722^{22}$, $1961^{22}$, $2131^{22}$, $2148^{22}$, $2212^{22}$, $3003^{4}$, $4147^{4}$, $7172^{2}$, $8239^{4}$, $59675^{4}$, $90013^{4}$, $246411^{4}$, $289839^{4}$
\item $(h,k)=(9,13)$ : $1^{4}$, $3^{1398100}$
\item $(h,k)=(9,15)$ : $1^{4}$, $6^{44}$, $7^{22}$, $11^{2}$, $14^{22}$, $19^{66}$, $22^{4}$, $31^{22}$, $38^{22}$, $44^{2}$, $45^{22}$, $51^{22}$, $75^{22}$, $76^{22}$, $86^{22}$, $88^{2}$, $99^{24}$, $106^{22}$, $111^{22}$, $114^{22}$, $121^{2}$, $123^{22}$, $132^{2}$, $137^{22}$, $148^{22}$, $149^{22}$, $164^{44}$, $165^{2}$, $167^{22}$, $199^{22}$, $212^{22}$, $228^{22}$, $231^{22}$, $276^{22}$, $277^{44}$, $306^{22}$, $309^{22}$, $318^{22}$, $336^{22}$, $341^{22}$, $342^{22}$, $346^{22}$, $351^{22}$, $352^{22}$, $392^{22}$, $395^{22}$, $405^{44}$, $425^{22}$, $448^{22}$, $454^{22}$, $477^{22}$, $493^{22}$, $511^{22}$, $517^{22}$, $519^{22}$, $526^{22}$, $542^{22}$, $621^{22}$, $627^{4}$, $640^{22}$, $659^{22}$, $689^{22}$, $699^{22}$, $715^{22}$, $724^{22}$, $750^{22}$, $787^{22}$, $788^{22}$, $831^{22}$, $843^{22}$, $879^{22}$, $887^{22}$, $898^{22}$, $902^{22}$, $910^{22}$, $945^{22}$, $970^{22}$, $985^{22}$, $1030^{22}$, $1062^{22}$, $1094^{22}$, $1178^{22}$, $1188^{2}$, $1195^{22}$, $1324^{22}$, $1394^{22}$, $1467^{22}$, $1486^{22}$, $1572^{22}$, $1723^{22}$, $1769^{22}$, $1833^{22}$, $2229^{22}$, $2619^{22}$, $2624^{22}$, $2722^{22}$, $2847^{22}$, $2915^{4}$, $2966^{22}$, $5110^{44}$, $8195^{4}$, $12650^{4}$, $102212^{4}$, $515900^{4}$
\item $(h,k)=(9,17)$ : $1^{4}$, $4^{22}$, $9^{22}$, $12^{22}$, $15^{22}$, $16^{22}$, $17^{22}$, $22^{26}$, $24^{22}$, $35^{22}$, $44^{4}$, $50^{22}$, $65^{22}$, $66^{2}$, $73^{22}$, $75^{22}$, $78^{22}$, $83^{66}$, $88^{22}$, $96^{22}$, $98^{22}$, $100^{22}$, $132^{4}$, $146^{22}$, $162^{22}$, $166^{22}$, $181^{22}$, $189^{22}$, $205^{22}$, $208^{22}$, $213^{22}$, $225^{22}$, $242^{2}$, $249^{22}$, $257^{22}$, $309^{22}$, $312^{22}$, $317^{22}$, $319^{4}$, $324^{22}$, $352^{22}$, $370^{22}$, $374^{22}$, $384^{22}$, $392^{22}$, $396^{22}$, $406^{22}$, $424^{22}$, $432^{22}$, $436^{22}$, $460^{22}$, $478^{22}$, $487^{22}$, $488^{22}$, $497^{22}$, $543^{22}$, $554^{22}$, $578^{22}$, $598^{22}$, $613^{22}$, $615^{22}$, $630^{22}$, $637^{22}$, $703^{22}$, $759^{22}$, $760^{22}$, $767^{22}$, $774^{22}$, $797^{22}$, $806^{22}$, $840^{22}$, $851^{22}$, $855^{22}$, $876^{22}$, $894^{22}$, $904^{22}$, $989^{22}$, $1035^{22}$, $1051^{22}$, $1157^{22}$, $1169^{22}$, $1184^{22}$, $1190^{22}$, $1221^{22}$, $1289^{22}$, $1338^{22}$, $1416^{22}$, $1465^{22}$, $1718^{22}$, $1775^{22}$, $1943^{22}$, $1984^{22}$, $2085^{22}$, $2100^{22}$, $2175^{22}$, $2514^{22}$, $3038^{22}$, $3429^{22}$, $145013^{4}$, $553179^{4}$
\item $(h,k)=(9,19)$ : $1^{4}$, $8^{22}$, $11^{4}$, $19^{22}$, $20^{22}$, $26^{22}$, $29^{22}$, $32^{22}$, $33^{4}$, $36^{22}$, $48^{22}$, $59^{22}$, $61^{22}$, $68^{22}$, $80^{44}$, $82^{22}$, $86^{22}$, $88^{24}$, $92^{22}$, $96^{22}$, $111^{22}$, $114^{22}$, $151^{22}$, $161^{44}$, $164^{22}$, $166^{22}$, $170^{22}$, $172^{22}$, $176^{22}$, $180^{22}$, $184^{22}$, $215^{22}$, $228^{22}$, $235^{22}$, $248^{22}$, $264^{22}$, $269^{22}$, $302^{22}$, $323^{22}$, $325^{22}$, $335^{22}$, $348^{22}$, $350^{22}$, $416^{22}$, $420^{22}$, $425^{22}$, $446^{22}$, $456^{22}$, $460^{22}$, $465^{22}$, $475^{22}$, $501^{22}$, $568^{22}$, $570^{22}$, $578^{22}$, $596^{22}$, $602^{22}$, $607^{22}$, $625^{22}$, $648^{22}$, $672^{22}$, $678^{22}$, $719^{22}$, $750^{22}$, $751^{22}$, $752^{22}$, $761^{22}$, $795^{22}$, $820^{22}$, $823^{22}$, $831^{22}$, $841^{22}$, $879^{22}$, $910^{22}$, $944^{22}$, $996^{22}$, $1017^{22}$, $1023^{6}$, $1080^{22}$, $1185^{22}$, $1230^{22}$, $1236^{22}$, $1264^{22}$, $1276^{4}$, $1389^{22}$, $1409^{22}$, $1430^{4}$, $1439^{22}$, $1543^{22}$, $1680^{22}$, $1706^{22}$, $1725^{22}$, $1748^{22}$, $1858^{22}$, $2082^{22}$, $2228^{22}$, $2830^{22}$, $2873^{22}$, $3587^{22}$, $6919^{4}$, $8811^{4}$, $39864^{4}$, $163064^{4}$, $477708^{4}$
\item $(h,k)=(9,21)$ : $1^{4}$, $7^{22}$, $10^{22}$, $11^{10}$, $12^{22}$, $13^{22}$, $16^{22}$, $22^{2}$, $27^{22}$, $31^{22}$, $33^{4}$, $39^{22}$, $44^{8}$, $46^{22}$, $53^{22}$, $55^{10}$, $66^{6}$, $70^{22}$, $75^{22}$, $77^{2}$, $80^{22}$, $88^{2}$, $99^{24}$, $110^{4}$, $121^{2}$, $128^{22}$, $133^{22}$, $144^{22}$, $147^{22}$, $175^{22}$, $188^{22}$, $194^{22}$, $201^{44}$, $208^{22}$, $218^{22}$, $245^{22}$, $246^{22}$, $252^{22}$, $258^{44}$, $263^{22}$, $275^{22}$, $279^{22}$, $281^{44}$, $288^{22}$, $299^{22}$, $304^{22}$, $316^{22}$, $347^{22}$, $397^{22}$, $401^{22}$, $421^{22}$, $424^{22}$, $430^{22}$, $436^{22}$, $440^{22}$, $449^{22}$, $471^{22}$, $490^{22}$, $521^{22}$, $542^{22}$, $588^{22}$, $602^{22}$, $609^{22}$, $626^{44}$, $650^{22}$, $686^{22}$, $694^{22}$, $695^{22}$, $736^{22}$, $742^{22}$, $766^{22}$, $771^{22}$, $781^{22}$, $789^{22}$, $841^{22}$, $871^{22}$, $899^{22}$, $904^{22}$, $917^{22}$, $961^{22}$, $969^{22}$, $1027^{22}$, $1096^{22}$, $1113^{22}$, $1120^{22}$, $1135^{22}$, $1211^{22}$, $1217^{22}$, $1254^{2}$, $1291^{22}$, $1366^{22}$, $1393^{22}$, $1699^{22}$, $1768^{22}$, $1882^{22}$, $2036^{22}$, $2226^{22}$, $2345^{22}$, $2377^{22}$, $2665^{22}$, $2782^{22}$, $2804^{22}$, $89518^{4}$, $609004^{4}$
\item $(h,k)=(13,15)$ : $1^{4}$, $3^{22}$, $5^{22}$, $11^{6}$, $18^{22}$, $23^{22}$, $37^{22}$, $41^{22}$, $44^{6}$, $56^{22}$, $66^{6}$, $70^{44}$, $75^{22}$, $78^{22}$, $88^{22}$, $92^{22}$, $108^{22}$, $125^{22}$, $132^{22}$, $139^{22}$, $145^{22}$, $162^{22}$, $178^{22}$, $195^{22}$, $198^{6}$, $199^{22}$, $216^{22}$, $219^{22}$, $226^{22}$, $229^{22}$, $231^{2}$, $249^{22}$, $256^{22}$, $261^{22}$, $278^{22}$, $315^{22}$, $337^{22}$, $343^{22}$, $352^{22}$, $358^{22}$, $373^{22}$, $380^{22}$, $407^{88}$, $413^{22}$, $427^{22}$, $444^{22}$, $453^{22}$, $464^{22}$, $468^{22}$, $469^{22}$, $470^{22}$, $473^{2}$, $483^{22}$, $505^{22}$, $531^{22}$, $556^{22}$, $558^{22}$, $577^{22}$, $583^{22}$, $613^{22}$, $628^{22}$, $646^{22}$, $666^{22}$, $702^{22}$, $705^{22}$, $708^{22}$, $745^{22}$, $791^{22}$, $807^{22}$, $817^{22}$, $820^{44}$, $886^{22}$, $904^{22}$, $942^{22}$, $1003^{22}$, $1006^{22}$, $1077^{22}$, $1109^{22}$, $1122^{22}$, $1163^{22}$, $1260^{22}$, $1268^{22}$, $1363^{22}$, $1426^{22}$, $1495^{22}$, $1504^{22}$, $1768^{22}$, $1855^{22}$, $1865^{22}$, $2025^{22}$, $2048^{22}$, $2097^{22}$, $2173^{22}$, $2928^{22}$, $3862^{22}$, $4752^{2}$, $7007^{4}$, $11319^{4}$, $146905^{4}$, $261360^{4}$, $271854^{4}$
\item $(h,k)=(13,17)$ : $1^{4}$, $5^{22}$, $17^{22}$, $26^{22}$, $28^{22}$, $40^{22}$, $44^{4}$, $45^{22}$, $51^{22}$, $57^{22}$, $58^{22}$, $70^{22}$, $77^{22}$, $78^{22}$, $90^{44}$, $94^{22}$, $95^{22}$, $105^{22}$, $114^{22}$, $125^{22}$, $154^{22}$, $160^{22}$, $175^{22}$, $189^{44}$, $220^{22}$, $224^{22}$, $226^{22}$, $286^{24}$, $288^{22}$, $313^{22}$, $347^{22}$, $350^{22}$, $402^{22}$, $412^{22}$, $424^{22}$, $429^{22}$, $437^{22}$, $489^{22}$, $493^{22}$, $506^{2}$, $518^{22}$, $525^{22}$, $568^{22}$, $585^{22}$, $590^{44}$, $593^{22}$, $595^{22}$, $608^{44}$, $613^{22}$, $647^{22}$, $673^{22}$, $674^{22}$, $680^{22}$, $685^{22}$, $687^{44}$, $712^{22}$, $719^{22}$, $732^{22}$, $760^{22}$, $789^{22}$, $809^{22}$, $819^{22}$, $823^{22}$, $845^{22}$, $881^{22}$, $929^{22}$, $942^{22}$, $988^{22}$, $1007^{22}$, $1016^{22}$, $1018^{22}$, $1023^{2}$, $1037^{22}$, $1041^{22}$, $1050^{22}$, $1111^{22}$, $1140^{22}$, $1330^{22}$, $1338^{22}$, $1386^{22}$, $1397^{44}$, $1469^{22}$, $1472^{22}$, $1495^{22}$, $1543^{22}$, $1545^{22}$, $1942^{22}$, $2136^{22}$, $2160^{22}$, $2458^{22}$, $2605^{22}$, $3960^{4}$, $11649^{4}$, $22341^{4}$, $660759^{4}$
\item $(h,k)=(13,19)$ : $1^{4}$, $13^{22}$, $22^{4}$, $23^{22}$, $25^{22}$, $31^{22}$, $33^{4}$, $34^{22}$, $42^{22}$, $67^{22}$, $83^{22}$, $85^{22}$, $96^{22}$, $99^{4}$, $100^{22}$, $115^{22}$, $119^{22}$, $125^{22}$, $132^{4}$, $151^{44}$, $152^{22}$, $156^{22}$, $157^{22}$, $158^{22}$, $160^{22}$, $165^{44}$, $168^{22}$, $186^{22}$, $192^{22}$, $196^{22}$, $205^{22}$, $214^{22}$, $230^{22}$, $235^{22}$, $241^{22}$, $244^{22}$, $256^{22}$, $271^{22}$, $290^{22}$, $308^{4}$, $324^{22}$, $332^{22}$, $335^{22}$, $341^{22}$, $343^{22}$, $358^{22}$, $359^{22}$, $360^{22}$, $394^{22}$, $405^{22}$, $425^{22}$, $433^{22}$, $434^{22}$, $454^{22}$, $468^{22}$, $478^{22}$, $488^{22}$, $494^{22}$, $495^{22}$, $511^{22}$, $515^{22}$, $546^{22}$, $551^{22}$, $554^{44}$, $597^{22}$, $644^{22}$, $670^{22}$, $673^{22}$, $711^{22}$, $761^{22}$, $798^{22}$, $819^{22}$, $823^{22}$, $824^{22}$, $831^{22}$, $858^{22}$, $898^{22}$, $937^{22}$, $968^{4}$, $984^{22}$, $1023^{2}$, $1034^{22}$, $1055^{22}$, $1058^{22}$, $1086^{22}$, $1090^{22}$, $1250^{22}$, $1415^{22}$, $1491^{22}$, $1517^{22}$, $1647^{22}$, $1668^{22}$, $1701^{22}$, $1711^{22}$, $1762^{22}$, $1896^{22}$, $2089^{22}$, $2181^{22}$, $2508^{4}$, $2635^{22}$, $2819^{22}$, $3014^{22}$, $6127^{4}$, $7629^{44}$, $42240^{4}$, $56859^{4}$, $65098^{4}$, $208395^{4}$, $239261^{4}$
\item $(h,k)=(13,21)$ : $1^{4}$, $9^{22}$, $16^{22}$, $17^{22}$, $22^{30}$, $53^{22}$, $63^{22}$, $67^{22}$, $72^{22}$, $73^{22}$, $74^{44}$, $75^{22}$, $77^{4}$, $90^{22}$, $104^{22}$, $115^{22}$, $127^{22}$, $132^{26}$, $135^{22}$, $137^{44}$, $138^{22}$, $139^{22}$, $146^{22}$, $148^{22}$, $152^{22}$, $158^{22}$, $168^{22}$, $175^{22}$, $176^{22}$, $181^{22}$, $183^{22}$, $197^{22}$, $198^{4}$, $201^{22}$, $229^{22}$, $230^{22}$, $233^{22}$, $240^{22}$, $252^{22}$, $253^{22}$, $287^{22}$, $293^{22}$, $342^{22}$, $347^{22}$, $363^{2}$, $417^{22}$, $430^{22}$, $443^{22}$, $448^{22}$, $453^{22}$, $489^{22}$, $493^{22}$, $494^{22}$, $509^{22}$, $520^{22}$, $537^{22}$, $597^{22}$, $602^{22}$, $603^{22}$, $622^{22}$, $649^{4}$, $703^{22}$, $742^{22}$, $746^{22}$, $747^{22}$, $752^{22}$, $771^{22}$, $776^{22}$, $791^{22}$, $820^{22}$, $830^{22}$, $861^{22}$, $868^{22}$, $910^{22}$, $931^{22}$, $937^{22}$, $968^{4}$, $1014^{22}$, $1058^{22}$, $1078^{22}$, $1114^{22}$, $1191^{22}$, $1340^{22}$, $1372^{22}$, $1392^{22}$, $1429^{22}$, $1468^{22}$, $1525^{22}$, $1584^{22}$, $1624^{22}$, $1687^{22}$, $1807^{22}$, $1893^{22}$, $1923^{22}$, $1972^{22}$, $2100^{22}$, $2233^{4}$, $2287^{22}$, $2771^{22}$, $3730^{22}$, $55099^{4}$, $134728^{4}$, $504130^{4}$
\item $(h,k)=(15,17)$ : $1^{4}$, $7^{22}$, $11^{26}$, $22^{2}$, $25^{22}$, $30^{22}$, $31^{22}$, $44^{2}$, $46^{22}$, $53^{22}$, $74^{22}$, $80^{22}$, $81^{22}$, $84^{22}$, $86^{22}$, $89^{22}$, $97^{22}$, $110^{4}$, $119^{22}$, $125^{22}$, $134^{22}$, $152^{22}$, $163^{22}$, $165^{26}$, $176^{2}$, $177^{22}$, $182^{22}$, $195^{22}$, $198^{22}$, $207^{22}$, $232^{22}$, $233^{22}$, $237^{22}$, $240^{22}$, $256^{22}$, $260^{22}$, $264^{2}$, $282^{22}$, $289^{22}$, $300^{22}$, $381^{22}$, $383^{22}$, $384^{22}$, $402^{22}$, $405^{22}$, $407^{22}$, $417^{22}$, $442^{22}$, $460^{22}$, $462^{22}$, $470^{22}$, $504^{44}$, $509^{22}$, $519^{22}$, $568^{22}$, $589^{22}$, $608^{22}$, $642^{22}$, $646^{22}$, $653^{22}$, $704^{22}$, $709^{22}$, $725^{22}$, $727^{22}$, $735^{22}$, $748^{22}$, $769^{22}$, $780^{22}$, $781^{2}$, $803^{22}$, $830^{22}$, $845^{66}$, $883^{22}$, $889^{22}$, $894^{22}$, $949^{22}$, $1141^{22}$, $1145^{22}$, $1200^{22}$, $1204^{22}$, $1279^{22}$, $1308^{22}$, $1377^{22}$, $1436^{22}$, $1488^{22}$, $1588^{22}$, $1596^{22}$, $1625^{22}$, $1635^{22}$, $1713^{22}$, $1730^{22}$, $1929^{22}$, $1963^{22}$, $2002^{4}$, $2046^{22}$, $2082^{22}$, $2115^{22}$, $2574^{4}$, $2920^{22}$, $3685^{4}$, $170973^{4}$, $519189^{4}$
\item $(h,k)=(15,19)$ : $1^{4}$, $11^{2}$, $14^{22}$, $22^{4}$, $33^{2}$, $34^{22}$, $37^{22}$, $43^{22}$, $44^{4}$, $55^{2}$, $57^{22}$, $62^{22}$, $66^{4}$, $67^{22}$, $76^{22}$, $77^{2}$, $103^{22}$, $107^{22}$, $111^{22}$, $112^{22}$, $131^{22}$, $132^{24}$, $135^{22}$, $139^{22}$, $145^{22}$, $149^{22}$, $156^{22}$, $160^{22}$, $162^{22}$, $188^{22}$, $194^{22}$, $198^{2}$, $201^{22}$, $204^{22}$, $212^{22}$, $230^{22}$, $246^{22}$, $248^{22}$, $259^{22}$, $263^{22}$, $290^{22}$, $298^{22}$, $299^{22}$, $311^{22}$, $312^{22}$, $330^{22}$, $352^{26}$, $361^{22}$, $367^{22}$, $413^{22}$, $416^{22}$, $433^{22}$, $451^{2}$, $488^{22}$, $491^{22}$, $494^{22}$, $503^{22}$, $538^{22}$, $549^{22}$, $570^{22}$, $643^{22}$, $653^{22}$, $654^{22}$, $656^{22}$, $657^{22}$, $664^{22}$, $704^{22}$, $723^{22}$, $734^{22}$, $735^{22}$, $741^{22}$, $755^{22}$, $757^{22}$, $787^{22}$, $850^{22}$, $851^{22}$, $875^{22}$, $876^{22}$, $910^{22}$, $926^{22}$, $964^{22}$, $995^{22}$, $1005^{22}$, $1011^{22}$, $1012^{2}$, $1050^{22}$, $1063^{22}$, $1077^{22}$, $1083^{22}$, $1086^{22}$, $1111^{22}$, $1133^{22}$, $1195^{22}$, $1393^{22}$, $1550^{22}$, $1566^{22}$, $1865^{22}$, $1940^{22}$, $1972^{22}$, $2155^{22}$, $2175^{22}$, $2305^{22}$, $3087^{22}$, $3146^{4}$, $3297^{22}$, $19349^{4}$, $23584^{4}$, $652212^{4}$
\item $(h,k)=(15,21)$ : $1^{4}$, $11^{30}$, $15^{22}$, $18^{22}$, $19^{22}$, $20^{44}$, $22^{4}$, $28^{22}$, $31^{22}$, $42^{22}$, $44^{26}$, $45^{22}$, $56^{44}$, $66^{22}$, $69^{22}$, $87^{22}$, $88^{2}$, $90^{22}$, $92^{22}$, $93^{22}$, $103^{22}$, $109^{22}$, $110^{4}$, $116^{22}$, $135^{22}$, $147^{22}$, $156^{22}$, $161^{22}$, $179^{22}$, $186^{22}$, $189^{22}$, $196^{22}$, $234^{22}$, $239^{22}$, $294^{22}$, $306^{22}$, $334^{22}$, $335^{44}$, $363^{22}$, $376^{22}$, $379^{22}$, $394^{22}$, $423^{22}$, $429^{22}$, $432^{22}$, $454^{22}$, $471^{22}$, $476^{22}$, $498^{22}$, $522^{22}$, $531^{22}$, $553^{22}$, $558^{22}$, $589^{22}$, $595^{22}$, $605^{22}$, $622^{22}$, $628^{22}$, $634^{44}$, $636^{22}$, $667^{22}$, $668^{22}$, $675^{22}$, $678^{22}$, $709^{22}$, $732^{22}$, $773^{22}$, $794^{22}$, $805^{22}$, $807^{22}$, $811^{22}$, $841^{22}$, $889^{22}$, $976^{22}$, $1023^{2}$, $1024^{22}$, $1107^{22}$, $1177^{22}$, $1190^{22}$, $1212^{22}$, $1224^{22}$, $1238^{22}$, $1250^{22}$, $1282^{22}$, $1284^{22}$, $1338^{22}$, $1434^{22}$, $1461^{22}$, $1472^{22}$, $1677^{22}$, $1738^{22}$, $2024^{4}$, $2119^{22}$, $2123^{22}$, $2240^{22}$, $3661^{22}$, $4985^{22}$, $179267^{4}$, $517176^{4}$
\item $(h,k)=(17,19)$ : $1^{4}$, $11^{4}$, $12^{22}$, $13^{22}$, $22^{4}$, $25^{22}$, $44^{6}$, $47^{22}$, $55^{22}$, $56^{22}$, $66^{8}$, $74^{22}$, $79^{44}$, $93^{22}$, $135^{22}$, $163^{22}$, $169^{22}$, $178^{22}$, $184^{22}$, $194^{22}$, $198^{22}$, $231^{2}$, $236^{22}$, $265^{44}$, $276^{22}$, $313^{22}$, $319^{26}$, $326^{22}$, $331^{22}$, $333^{22}$, $350^{22}$, $354^{22}$, $358^{22}$, $377^{22}$, $384^{22}$, $385^{22}$, $388^{22}$, $404^{22}$, $412^{22}$, $417^{22}$, $423^{22}$, $433^{22}$, $437^{22}$, $442^{22}$, $460^{22}$, $464^{44}$, $476^{22}$, $497^{22}$, $499^{22}$, $502^{22}$, $513^{22}$, $527^{22}$, $553^{22}$, $556^{22}$, $565^{22}$, $572^{2}$, $597^{22}$, $606^{22}$, $620^{22}$, $667^{22}$, $697^{22}$, $726^{22}$, $748^{2}$, $754^{22}$, $804^{22}$, $810^{22}$, $821^{22}$, $828^{22}$, $883^{44}$, $901^{22}$, $938^{22}$, $965^{22}$, $966^{22}$, $974^{22}$, $998^{22}$, $1000^{22}$, $1030^{22}$, $1046^{22}$, $1110^{22}$, $1120^{22}$, $1132^{22}$, $1156^{22}$, $1191^{22}$, $1194^{22}$, $1203^{22}$, $1219^{22}$, $1272^{22}$, $1293^{22}$, $1355^{22}$, $1405^{22}$, $1516^{22}$, $1562^{4}$, $1680^{22}$, $1705^{4}$, $1706^{22}$, $1895^{22}$, $1919^{22}$, $2604^{22}$, $2629^{22}$, $5445^{4}$, $9647^{4}$, $26587^{4}$, $171292^{4}$, $483571^{4}$
\item $(h,k)=(17,21)$ : $1^{4}$, $7^{22}$, $8^{44}$, $10^{22}$, $12^{22}$, $19^{22}$, $20^{22}$, $31^{22}$, $55^{2}$, $66^{4}$, $86^{22}$, $93^{22}$, $96^{22}$, $114^{22}$, $115^{44}$, $133^{22}$, $161^{22}$, $165^{22}$, $169^{22}$, $177^{22}$, $181^{22}$, $182^{22}$, $183^{22}$, $194^{22}$, $198^{4}$, $199^{22}$, $210^{22}$, $218^{22}$, $233^{22}$, $239^{22}$, $253^{22}$, $264^{22}$, $266^{22}$, $273^{22}$, $301^{44}$, $308^{2}$, $335^{22}$, $339^{22}$, $340^{44}$, $359^{22}$, $361^{22}$, $372^{22}$, $388^{22}$, $413^{22}$, $417^{22}$, $422^{22}$, $423^{22}$, $459^{22}$, $482^{22}$, $492^{22}$, $497^{22}$, $498^{22}$, $502^{22}$, $521^{22}$, $536^{22}$, $556^{22}$, $569^{22}$, $606^{22}$, $625^{22}$, $627^{22}$, $660^{2}$, $672^{22}$, $680^{22}$, $686^{22}$, $693^{4}$, $712^{22}$, $725^{44}$, $727^{22}$, $730^{22}$, $744^{22}$, $752^{22}$, $763^{22}$, $794^{22}$, $801^{22}$, $814^{22}$, $866^{22}$, $874^{22}$, $877^{22}$, $921^{22}$, $979^{22}$, $1013^{22}$, $1054^{22}$, $1238^{22}$, $1250^{22}$, $1315^{22}$, $1371^{22}$, $1378^{22}$, $1397^{22}$, $1468^{22}$, $1799^{22}$, $2088^{22}$, $2093^{22}$, $2251^{22}$, $2543^{22}$, $2554^{22}$, $2720^{22}$, $4610^{22}$, $4719^{4}$, $5654^{4}$, $8778^{4}$, $24695^{4}$, $158653^{4}$, $495363^{4}$
\item $(h,k)=(19,21)$ : $1^{4}$, $11^{4}$, $14^{22}$, $15^{22}$, $17^{22}$, $22^{2}$, $31^{22}$, $51^{22}$, $57^{22}$, $59^{22}$, $66^{46}$, $71^{22}$, $76^{22}$, $77^{22}$, $95^{22}$, $102^{22}$, $110^{2}$, $115^{22}$, $116^{22}$, $119^{22}$, $120^{22}$, $125^{22}$, $126^{22}$, $128^{22}$, $132^{22}$, $152^{22}$, $166^{44}$, $185^{22}$, $193^{22}$, $209^{22}$, $218^{22}$, $222^{22}$, $228^{22}$, $260^{22}$, $287^{22}$, $290^{22}$, $318^{22}$, $332^{22}$, $368^{22}$, $385^{22}$, $419^{22}$, $475^{22}$, $515^{22}$, $518^{22}$, $531^{22}$, $538^{22}$, $553^{22}$, $557^{22}$, $558^{22}$, $559^{22}$, $575^{22}$, $588^{22}$, $605^{22}$, $615^{22}$, $638^{2}$, $644^{22}$, $666^{22}$, $667^{22}$, $691^{22}$, $714^{22}$, $722^{22}$, $728^{22}$, $744^{44}$, $776^{22}$, $796^{22}$, $805^{22}$, $869^{2}$, $893^{22}$, $918^{22}$, $947^{22}$, $954^{22}$, $970^{22}$, $976^{22}$, $988^{22}$, $1033^{22}$, $1050^{22}$, $1084^{22}$, $1120^{22}$, $1143^{22}$, $1301^{22}$, $1327^{22}$, $1343^{22}$, $1379^{22}$, $1390^{22}$, $1461^{22}$, $1523^{22}$, $1582^{22}$, $1601^{22}$, $1682^{22}$, $1751^{22}$, $1788^{22}$, $1860^{22}$, $2001^{22}$, $2206^{22}$, $2411^{22}$, $2713^{22}$, $4521^{4}$, $86141^{4}$, $608047^{4}$
\end{itemize}

\section{Results for $n=23$}
\begin{itemize}
\item $(h,k)=(1,2)$ : $1^{2}$, $23^{4}$, $31^{414}$, $32^{322}$, $34^{276}$, $36^{276}$, $40^{184}$, $44^{138}$, $46^{194}$, $47^{92}$, $52^{276}$, $53^{46}$, $61^{92}$, $62^{92}$, $64^{92}$, $68^{276}$, $76^{92}$, $80^{92}$, $84^{46}$, $92^{114}$, $101^{184}$, $102^{92}$, $108^{92}$, $110^{46}$, $115^{14}$, $119^{92}$, $136^{46}$, $138^{38}$, $139^{46}$, $161^{70}$, $162^{46}$, $176^{46}$, $183^{46}$, $184^{132}$, $197^{46}$, $198^{46}$, $207^{124}$, $214^{46}$, $230^{84}$, $253^{118}$, $261^{46}$, $267^{46}$, $276^{346}$, $278^{46}$, $298^{46}$, $299^{124}$, $322^{432}$, $345^{32}$, $350^{46}$, $368^{174}$, $391^{88}$, $414^{116}$, $437^{20}$, $460^{148}$, $483^{56}$, $506^{188}$, $529^{38}$, $552^{74}$, $575^{38}$, $598^{172}$, $621^{64}$, $644^{122}$, $667^{42}$, $690^{104}$, $713^{32}$, $736^{84}$, $759^{14}$, $782^{58}$, $805^{74}$, $828^{102}$, $851^{44}$, $874^{60}$, $897^{12}$, $920^{76}$, $943^{38}$, $966^{44}$, $989^{32}$, $1012^{76}$, $1035^{42}$, $1058^{66}$, $1081^{24}$, $1104^{48}$, $1127^{8}$, $1150^{42}$, $1173^{6}$, $1196^{30}$, $1219^{12}$, $1242^{74}$, $1265^{28}$, $1288^{72}$, $1311^{42}$, $1334^{30}$, $1357^{6}$, $1380^{40}$, $1403^{10}$, $1426^{16}$, $1449^{8}$, $1472^{46}$, $1495^{36}$, $1518^{26}$, $1541^{8}$, $1564^{34}$, $1587^{18}$, $1610^{28}$, $1656^{40}$, $1679^{8}$, $1702^{38}$, $1725^{8}$, $1748^{12}$, $1771^{8}$, $1794^{44}$, $1817^{4}$, $1840^{24}$, $1863^{6}$, $1886^{38}$, $1909^{4}$, $1932^{32}$, $1978^{28}$, $2001^{4}$, $2024^{6}$, $2070^{32}$, $2093^{22}$, $2116^{10}$, $2139^{8}$, $2162^{28}$, $2185^{4}$, $2208^{34}$, $2231^{2}$, $2254^{32}$, $2277^{20}$, $2300^{10}$, $2323^{2}$, $2346^{16}$, $2392^{10}$, $2415^{10}$, $2438^{16}$, $2461^{6}$, $2484^{22}$, $2507^{6}$, $2530^{24}$, $2553^{8}$, $2576^{20}$, $2599^{6}$, $2622^{14}$, $2668^{14}$, $2714^{22}$, $2737^{6}$, $2760^{20}$, $2783^{4}$, $2806^{10}$, $2829^{8}$, $2852^{14}$, $2875^{18}$, $2898^{18}$, $2921^{6}$, $2944^{10}$, $2967^{6}$, $3013^{4}$, $3036^{6}$, $3059^{4}$, $3082^{6}$, $3105^{16}$, $3128^{18}$, $3151^{10}$, $3174^{2}$, $3197^{6}$, $3220^{4}$, $3243^{4}$, $3266^{6}$, $3312^{8}$, $3335^{16}$, $3358^{16}$, $3381^{6}$, $3404^{8}$, $3427^{10}$, $3450^{16}$, $3473^{4}$, $3496^{10}$, $3519^{4}$, $3542^{12}$, $3588^{4}$, $3634^{8}$, $3657^{6}$, $3680^{10}$, $3703^{6}$, $3726^{12}$, $3749^{6}$, $3772^{4}$, $3818^{8}$, $3841^{6}$, $3864^{8}$, $3887^{10}$, $3910^{10}$, $3933^{2}$, $3956^{2}$, $3979^{4}$, $4002^{2}$, $4025^{4}$, $4048^{8}$, $4071^{6}$, $4094^{14}$, $4117^{2}$, $4140^{4}$, $4163^{4}$, $4186^{8}$, $4209^{6}$, $4232^{8}$, $4255^{2}$, $4278^{2}$, $4301^{2}$, $4347^{6}$, $4370^{4}$, $4393^{2}$, $4416^{4}$, $4439^{8}$, $4485^{8}$, $4508^{12}$, $4531^{2}$, $4554^{2}$, $4577^{2}$, $4600^{2}$, $4623^{4}$, $4646^{6}$, $4669^{8}$, $4692^{6}$, $4715^{4}$, $4738^{8}$, $4761^{6}$, $4784^{6}$, $4807^{2}$, $4830^{2}$, $4853^{6}$, $4876^{2}$, $4899^{4}$, $4922^{12}$, $4945^{6}$, $4968^{2}$, $4991^{2}$, $5014^{2}$, $5037^{4}$, $5060^{4}$, $5106^{2}$, $5129^{6}$, $5152^{6}$, $5175^{6}$, $5198^{6}$, $5221^{6}$, $5244^{4}$, $5336^{6}$, $5382^{2}$, $5405^{2}$, $5428^{4}$, $5474^{6}$, $5497^{4}$, $5566^{4}$, $5589^{4}$, $5612^{2}$, $5635^{2}$, $5658^{2}$, $5681^{2}$, $5704^{2}$, $5727^{4}$, $5750^{8}$, $5796^{2}$, $5842^{2}$, $5865^{2}$, $5911^{4}$, $5934^{8}$, $6003^{2}$, $6026^{2}$, $6072^{2}$, $6118^{4}$, $6187^{4}$, $6256^{6}$, $6279^{2}$, $6440^{4}$, $6463^{2}$, $6486^{2}$, $6532^{4}$, $6555^{2}$, $6693^{4}$, $6716^{4}$, $6808^{2}$, $6877^{2}$, $6900^{4}$, $6969^{2}$, $7107^{2}$, $7130^{2}$, $7153^{4}$, $7176^{2}$, $7383^{2}$, $7406^{2}$, $7452^{2}$, $7475^{2}$, $7567^{6}$, $7613^{2}$, $7751^{2}$, $7912^{4}$, $8004^{4}$, $8280^{2}$, $8326^{2}$, $8395^{2}$, $8901^{2}$, $8970^{2}$, $9269^{2}$, $10051^{2}$, $10948^{2}$
\item $(h,k)=(1,3)$ : $1^{2}$, $23^{2}$, $92^{2}$, $2875^{2}$, $12765^{2}$, $48668^{2}$, $62928^{2}$, $386308^{2}$, $1773300^{2}$, $1907344^{2}$
\item $(h,k)=(1,4)$ : $1^{2}$, $20^{46}$, $1932^{2}$, $3634^{2}$, $26082^{2}$, $87653^{2}$, $117185^{2}$, $198674^{2}$, $3758683^{2}$
\item $(h,k)=(1,5)$ : $1^{2}$, $46^{2}$, $138^{2}$, $253^{2}$, $12489^{2}$, $12604^{2}$, $60559^{2}$, $793063^{2}$, $3315151^{2}$
\item $(h,k)=(1,6)$ : $1^{2}$, $345^{2}$, $3289^{2}$, $47380^{2}$, $69667^{2}$, $142462^{2}$, $3931160^{2}$
\item $(h,k)=(1,7)$ : $1^{2}$, $69^{2}$, $368^{2}$, $2047^{2}$, $6693^{2}$, $12052^{2}$, $36478^{2}$, $48208^{2}$, $186829^{2}$, $339388^{2}$, $1558687^{2}$, $2003484^{2}$
\item $(h,k)=(1,8)$ : $1^{2}$, $23^{2}$, $1219^{2}$, $1449^{2}$, $3128^{2}$, $18472^{46}$, $140208^{2}$, $3623420^{2}$
\item $(h,k)=(1,9)$ : $1^{2}$, $23^{2}$, $230^{2}$, $322^{2}$, $747569^{2}$, $3446159^{2}$
\item $(h,k)=(1,10)$ : $1^{2}$, $23^{2}$, $1081^{2}$, $18538^{2}$, $2065400^{2}$, $2109261^{2}$
\item $(h,k)=(1,11)$ : $1^{2}$, $115^{2}$, $1817^{2}$, $5543^{2}$, $299368^{2}$, $1423746^{2}$, $2463714^{2}$
\item $(h,k)=(1,12)$ : $1^{2}$, $23^{2}$, $4462^{40}$, $8924^{920}$
\item $(h,k)=(1,13)$ : $1^{2}$, $69^{2}$, $138^{2}$, $1091396^{2}$, $3102700^{2}$
\item $(h,k)=(1,14)$ : $1^{2}$, $184^{2}$, $345^{2}$, $667^{2}$, $1472^{2}$, $1541^{2}$, $5152^{2}$, $377407^{2}$, $592135^{2}$, $3215400^{2}$
\item $(h,k)=(1,15)$ : $1^{2}$, $69^{2}$, $299^{2}$, $782^{2}$, $9706^{2}$, $24196^{2}$, $29969^{2}$, $60030^{2}$, $1702483^{2}$, $2366769^{2}$
\item $(h,k)=(1,16)$ : $1^{2}$, $46^{2}$, $552^{2}$, $1656^{2}$, $1702^{2}$, $74980^{2}$, $1693099^{2}$, $2422268^{2}$
\item $(h,k)=(1,17)$ : $1^{2}$, $460^{2}$, $598^{2}$, $5152^{2}$, $40848^{2}$, $271124^{2}$, $727559^{2}$, $844215^{2}$, $867560^{2}$, $1436787^{2}$
\item $(h,k)=(1,18)$ : $1^{2}$, $92^{2}$, $161^{2}$, $3680^{2}$, $41584^{2}$, $127903^{2}$, $186024^{2}$, $235704^{2}$, $3599155^{2}$
\item $(h,k)=(1,19)$ : $1^{2}$, $5^{46}$, $138^{2}$, $391^{2}$, $805^{2}$, $2484^{2}$, $26933^{2}$, $83421^{2}$, $110331^{2}$, $146671^{2}$, $932995^{2}$, $2890019^{2}$
\item $(h,k)=(1,20)$ : $1^{2}$, $23^{2}$, $92^{2}$, $506^{2}$, $1288^{2}$, $4192394^{2}$
\item $(h,k)=(1,21)$ : $1^{2}$, $96^{184}$, $108^{276}$, $138^{188}$, $141^{92}$, $159^{46}$, $240^{92}$, $276^{8}$, $330^{46}$, $345^{10}$, $357^{46}$, $408^{46}$, $417^{46}$, $483^{6}$, $552^{46}$, $621^{62}$, $690^{42}$, $759^{40}$, $828^{116}$, $894^{46}$, $897^{42}$, $966^{150}$, $1035^{12}$, $1104^{44}$, $1173^{2}$, $1242^{60}$, $1311^{6}$, $1380^{52}$, $1518^{100}$, $1656^{4}$, $1794^{84}$, $1863^{12}$, $1932^{38}$, $2001^{16}$, $2070^{24}$, $2208^{24}$, $2346^{18}$, $2415^{14}$, $2484^{32}$, $2553^{6}$, $2622^{10}$, $2760^{10}$, $2829^{16}$, $2898^{12}$, $2967^{22}$, $3036^{30}$, $3105^{18}$, $3174^{38}$, $3243^{4}$, $3312^{8}$, $3381^{4}$, $3450^{8}$, $3588^{12}$, $3726^{20}$, $3795^{8}$, $3864^{8}$, $3933^{16}$, $4002^{12}$, $4140^{12}$, $4278^{4}$, $4347^{4}$, $4416^{22}$, $4485^{10}$, $4554^{14}$, $4623^{6}$, $4692^{16}$, $4761^{2}$, $4830^{4}$, $4968^{16}$, $5106^{14}$, $5175^{4}$, $5244^{4}$, $5313^{4}$, $5382^{10}$, $5451^{2}$, $5520^{20}$, $5589^{4}$, $5658^{12}$, $5796^{4}$, $5934^{8}$, $6072^{4}$, $6210^{12}$, $6348^{10}$, $6486^{6}$, $6624^{16}$, $6762^{14}$, $6831^{6}$, $7176^{2}$, $7245^{4}$, $7314^{8}$, $7383^{2}$, $7452^{6}$, $7521^{6}$, $7590^{8}$, $7659^{6}$, $7728^{2}$, $7797^{2}$, $7866^{8}$, $8004^{8}$, $8142^{2}$, $8211^{4}$, $8280^{12}$, $8349^{2}$, $8418^{4}$, $8487^{4}$, $8556^{4}$, $8625^{16}$, $8694^{4}$, $8763^{4}$, $8832^{4}$, $9039^{2}$, $9108^{2}$, $9315^{10}$, $9384^{10}$, $9453^{4}$, $9591^{2}$, $9660^{2}$, $9729^{4}$, $9798^{6}$, $9936^{4}$, $10005^{4}$, $10074^{8}$, $10212^{8}$, $10281^{8}$, $10419^{2}$, $10488^{2}$, $10557^{4}$, $10626^{4}$, $10764^{2}$, $10902^{4}$, $11040^{4}$, $11109^{2}$, $11178^{4}$, $11316^{4}$, $11730^{2}$, $11799^{2}$, $12144^{4}$, $12282^{4}$, $12420^{2}$, $12558^{4}$, $12627^{4}$, $12696^{2}$, $12903^{2}$, $13041^{2}$, $13110^{4}$, $13179^{2}$, $13317^{2}$, $13455^{4}$, $13524^{4}$, $13731^{2}$, $13800^{2}$, $13869^{2}$, $13938^{2}$, $14007^{2}$, $14076^{2}$, $14145^{2}$, $14214^{2}$, $14283^{2}$, $14628^{2}$, $14697^{2}$, $14766^{4}$, $14835^{4}$, $14973^{2}$, $15042^{2}$, $15180^{2}$, $15387^{2}$, $15456^{2}$, $15594^{2}$, $15663^{2}$, $16008^{2}$, $16215^{2}$, $16422^{2}$, $16767^{2}$, $16836^{2}$, $16974^{2}$, $17043^{2}$, $17112^{2}$, $17181^{4}$, $17802^{4}$, $18009^{2}$, $18078^{2}$, $18354^{2}$, $18561^{4}$, $18768^{2}$, $19320^{2}$, $19458^{2}$, $19596^{4}$, $20079^{2}$, $20148^{2}$, $20424^{2}$, $20700^{2}$, $20907^{2}$, $21321^{2}$, $21390^{2}$, $21459^{2}$, $21528^{2}$, $24012^{2}$, $26910^{2}$, $27807^{2}$
\item $(h,k)=(1,22)$ : $1^{2}$, $3^{2796202}$
\item $(h,k)=(2,3)$ : $1^{2}$, $46^{2}$, $299^{2}$, $23276^{2}$, $66815^{2}$, $80960^{2}$, $4022907^{2}$
\item $(h,k)=(2,4)$ : $1^{2}$, $69^{4}$, $92^{2}$, $552^{2}$, $644^{2}$, $6440^{2}$, $8510^{2}$, $8878^{2}$, $43010^{2}$, $45701^{2}$, $138230^{2}$, $171396^{46}$
\item $(h,k)=(2,5)$ : $1^{2}$, $69^{2}$, $322^{2}$, $6900^{2}$, $8050^{2}$, $21091^{2}$, $21620^{2}$, $50416^{2}$, $4085835^{2}$
\item $(h,k)=(2,6)$ : $1^{2}$, $92^{2}$, $161^{2}$, $391^{2}$, $1771^{2}$, $410872^{2}$, $433481^{2}$, $3347535^{2}$
\item $(h,k)=(2,7)$ : $1^{2}$, $23^{4}$, $276^{2}$, $575^{2}$, $2737^{2}$, $3199^{46}$, $10350^{2}$, $52235^{46}$, $1193102^{2}$, $1712235^{2}$
\item $(h,k)=(2,8)$ : $1^{2}$, $23^{2}$, $69^{2}$, $230^{2}$, $368^{2}$, $3680^{2}$, $4991^{2}$, $5842^{2}$, $55775^{2}$, $4123325^{2}$
\item $(h,k)=(2,9)$ : $1^{2}$, $184^{2}$, $529^{2}$, $5175^{2}$, $5359^{2}$, $20033^{2}$, $42136^{2}$, $168797^{2}$, $208587^{2}$, $3743503^{2}$
\item $(h,k)=(2,10)$ : $1^{2}$, $23^{4}$, $207^{2}$, $621^{2}$, $805^{2}$, $6072^{2}$, $125028^{2}$, $225239^{2}$, $627233^{2}$, $649083^{2}$, $2559969^{2}$
\item $(h,k)=(2,11)$ : $1^{2}$, $92^{2}$, $805^{2}$, $2001^{2}$, $2231^{2}$, $121083^{46}$, $546940^{2}$, $857325^{2}$
\item $(h,k)=(2,12)$ : $1^{2}$, $23^{2}$, $69^{2}$, $184^{2}$, $2438^{2}$, $2760^{2}$, $71047^{2}$, $4117782^{2}$
\item $(h,k)=(2,13)$ : $1^{2}$, $851^{2}$, $1380^{2}$, $41745^{2}$, $152674^{2}$, $3997653^{2}$
\item $(h,k)=(2,14)$ : $1^{2}$, $92^{2}$, $391^{2}$, $9223^{2}$, $28566^{2}$, $159068^{2}$, $322184^{2}$, $1130404^{2}$, $2544375^{2}$
\item $(h,k)=(2,15)$ : $1^{2}$, $23^{2}$, $161^{2}$, $44597^{2}$, $47886^{2}$, $1268013^{2}$, $2833623^{2}$
\item $(h,k)=(2,16)$ : $1^{2}$, $23^{2}$, $184^{2}$, $230^{2}$, $782^{2}$, $16537^{2}$, $103454^{2}$, $4073093^{2}$
\item $(h,k)=(2,17)$ : $1^{2}$, $69^{2}$, $92^{2}$, $9039^{2}$, $749869^{2}$, $3435234^{2}$
\item $(h,k)=(2,18)$ : $1^{2}$, $23^{2}$, $69^{2}$, $322^{2}$, $5152^{2}$, $95910^{2}$, $105179^{2}$, $1782914^{2}$, $2204734^{2}$
\item $(h,k)=(2,19)$ : $1^{2}$, $46^{2}$, $92^{2}$, $552^{2}$, $254794^{2}$, $1581227^{2}$, $2357592^{2}$
\item $(h,k)=(2,20)$ : $1^{2}$, $184^{2}$, $652^{46}$, $4692^{2}$, $50830^{2}$, $199594^{2}$, $338721^{2}$, $1328572^{2}$, $2256714^{2}$
\item $(h,k)=(2,21)$ : $1^{2}$, $3^{2796202}$
\item $(h,k)=(2,22)$ : $1^{2}$, $96^{184}$, $108^{276}$, $138^{188}$, $141^{92}$, $159^{46}$, $240^{92}$, $276^{8}$, $330^{46}$, $345^{10}$, $357^{46}$, $408^{46}$, $417^{46}$, $483^{6}$, $552^{46}$, $621^{62}$, $690^{42}$, $759^{40}$, $828^{116}$, $894^{46}$, $897^{42}$, $966^{150}$, $1035^{12}$, $1104^{44}$, $1173^{2}$, $1242^{60}$, $1311^{6}$, $1380^{52}$, $1518^{100}$, $1656^{4}$, $1794^{84}$, $1863^{12}$, $1932^{38}$, $2001^{16}$, $2070^{24}$, $2208^{24}$, $2346^{18}$, $2415^{14}$, $2484^{32}$, $2553^{6}$, $2622^{10}$, $2760^{10}$, $2829^{16}$, $2898^{12}$, $2967^{22}$, $3036^{30}$, $3105^{18}$, $3174^{38}$, $3243^{4}$, $3312^{8}$, $3381^{4}$, $3450^{8}$, $3588^{12}$, $3726^{20}$, $3795^{8}$, $3864^{8}$, $3933^{16}$, $4002^{12}$, $4140^{12}$, $4278^{4}$, $4347^{4}$, $4416^{22}$, $4485^{10}$, $4554^{14}$, $4623^{6}$, $4692^{16}$, $4761^{2}$, $4830^{4}$, $4968^{16}$, $5106^{14}$, $5175^{4}$, $5244^{4}$, $5313^{4}$, $5382^{10}$, $5451^{2}$, $5520^{20}$, $5589^{4}$, $5658^{12}$, $5796^{4}$, $5934^{8}$, $6072^{4}$, $6210^{12}$, $6348^{10}$, $6486^{6}$, $6624^{16}$, $6762^{14}$, $6831^{6}$, $7176^{2}$, $7245^{4}$, $7314^{8}$, $7383^{2}$, $7452^{6}$, $7521^{6}$, $7590^{8}$, $7659^{6}$, $7728^{2}$, $7797^{2}$, $7866^{8}$, $8004^{8}$, $8142^{2}$, $8211^{4}$, $8280^{12}$, $8349^{2}$, $8418^{4}$, $8487^{4}$, $8556^{4}$, $8625^{16}$, $8694^{4}$, $8763^{4}$, $8832^{4}$, $9039^{2}$, $9108^{2}$, $9315^{10}$, $9384^{10}$, $9453^{4}$, $9591^{2}$, $9660^{2}$, $9729^{4}$, $9798^{6}$, $9936^{4}$, $10005^{4}$, $10074^{8}$, $10212^{8}$, $10281^{8}$, $10419^{2}$, $10488^{2}$, $10557^{4}$, $10626^{4}$, $10764^{2}$, $10902^{4}$, $11040^{4}$, $11109^{2}$, $11178^{4}$, $11316^{4}$, $11730^{2}$, $11799^{2}$, $12144^{4}$, $12282^{4}$, $12420^{2}$, $12558^{4}$, $12627^{4}$, $12696^{2}$, $12903^{2}$, $13041^{2}$, $13110^{4}$, $13179^{2}$, $13317^{2}$, $13455^{4}$, $13524^{4}$, $13731^{2}$, $13800^{2}$, $13869^{2}$, $13938^{2}$, $14007^{2}$, $14076^{2}$, $14145^{2}$, $14214^{2}$, $14283^{2}$, $14628^{2}$, $14697^{2}$, $14766^{4}$, $14835^{4}$, $14973^{2}$, $15042^{2}$, $15180^{2}$, $15387^{2}$, $15456^{2}$, $15594^{2}$, $15663^{2}$, $16008^{2}$, $16215^{2}$, $16422^{2}$, $16767^{2}$, $16836^{2}$, $16974^{2}$, $17043^{2}$, $17112^{2}$, $17181^{4}$, $17802^{4}$, $18009^{2}$, $18078^{2}$, $18354^{2}$, $18561^{4}$, $18768^{2}$, $19320^{2}$, $19458^{2}$, $19596^{4}$, $20079^{2}$, $20148^{2}$, $20424^{2}$, $20700^{2}$, $20907^{2}$, $21321^{2}$, $21390^{2}$, $21459^{2}$, $21528^{2}$, $24012^{2}$, $26910^{2}$, $27807^{2}$
\item $(h,k)=(3,4)$ : $1^{2}$, $115^{2}$, $437^{2}$, $2438^{2}$, $3358^{2}$, $5727^{2}$, $99429^{2}$, $1112648^{2}$, $2970151^{2}$
\item $(h,k)=(3,5)$ : $1^{2}$, $7^{46}$, $667^{2}$, $874^{2}$, $1242^{2}$, $59271^{2}$, $163645^{2}$, $468602^{2}$, $718451^{2}$, $2781390^{2}$
\item $(h,k)=(3,6)$ : $1^{2}$, $23^{2}$, $46^{4}$, $552^{2}$, $1748^{2}$, $3036^{2}$, $3197^{2}$, $5451^{2}$, $40043^{2}$, $113137^{2}$, $121187^{2}$, $175030^{2}$, $335662^{2}$, $1588426^{2}$, $1806719^{2}$
\item $(h,k)=(3,7)$ : $1^{2}$, $23^{2}$, $115^{2}$, $1242^{2}$, $2277^{2}$, $21597^{2}$, $92000^{2}$, $670795^{2}$, $1000477^{2}$, $2405777^{2}$
\item $(h,k)=(3,8)$ : $1^{2}$, $46^{2}$, $598^{2}$, $5704^{2}$, $28267^{2}$, $838120^{2}$, $1541184^{2}$, $1780384^{2}$
\item $(h,k)=(3,9)$ : $1^{2}$, $23^{2}$, $46^{2}$, $69^{2}$, $6348^{2}$, $8970^{2}$, $9131^{2}$, $30383^{2}$, $174363^{2}$, $3964970^{2}$
\item $(h,k)=(3,10)$ : $1^{2}$, $69^{2}$, $207^{2}$, $7222^{2}$, $43792^{2}$, $423085^{2}$, $1323236^{2}$, $2396692^{2}$
\item $(h,k)=(3,11)$ : $1^{2}$, $3151^{2}$, $10971^{2}$, $86802^{2}$, $158194^{2}$, $1409808^{2}$, $2525377^{2}$
\item $(h,k)=(3,12)$ : $1^{2}$, $23^{2}$, $46^{4}$, $115^{2}$, $690^{2}$, $51037^{2}$, $86388^{2}$, $1045005^{2}$, $3010953^{2}$
\item $(h,k)=(3,13)$ : $1^{2}$, $23^{2}$, $69^{2}$, $253^{2}$, $414^{2}$, $483^{2}$, $575^{2}$, $14835^{2}$, $28267^{2}$, $38638^{46}$, $60168^{2}$, $65366^{2}$, $413540^{2}$, $606395^{2}$, $2115241^{2}$
\item $(h,k)=(3,14)$ : $1^{2}$, $299^{2}$, $4194004^{2}$
\item $(h,k)=(3,15)$ : $1^{2}$, $23^{2}$, $92^{2}$, $8303^{2}$, $4185885^{2}$
\item $(h,k)=(3,16)$ : $1^{2}$, $46^{2}$, $184^{2}$, $920^{2}$, $1150^{2}$, $8418^{2}$, $8487^{2}$, $9200^{2}$, $28326^{46}$, $204884^{2}$, $750007^{2}$, $2559509^{2}$
\item $(h,k)=(3,17)$ : $1^{2}$, $23^{4}$, $690^{2}$, $116012^{2}$, $162748^{2}$, $209346^{2}$, $776710^{2}$, $2928751^{2}$
\item $(h,k)=(3,18)$ : $1^{2}$, $46^{2}$, $11799^{2}$, $4182458^{2}$
\item $(h,k)=(3,19)$ : $1^{2}$, $161^{2}$, $460^{2}$, $181608^{2}$, $872321^{2}$, $3139753^{2}$
\item $(h,k)=(3,20)$ : $1^{2}$, $3^{2796202}$
\item $(h,k)=(3,21)$ : $1^{2}$, $184^{2}$, $652^{46}$, $4692^{2}$, $50830^{2}$, $199594^{2}$, $338721^{2}$, $1328572^{2}$, $2256714^{2}$
\item $(h,k)=(3,22)$ : $1^{2}$, $23^{2}$, $92^{2}$, $506^{2}$, $1288^{2}$, $4192394^{2}$
\item $(h,k)=(4,5)$ : $1^{2}$, $66332^{2}$, $275126^{2}$, $3852845^{2}$
\item $(h,k)=(4,6)$ : $1^{2}$, $69^{2}$, $33718^{2}$, $1333977^{2}$, $2826539^{2}$
\item $(h,k)=(4,7)$ : $1^{2}$, $23^{2}$, $2070^{2}$, $98716^{2}$, $371496^{2}$, $1387314^{2}$, $2334684^{2}$
\item $(h,k)=(4,8)$ : $1^{2}$, $23^{2}$, $115^{2}$, $138^{2}$, $230^{2}$, $414^{4}$, $1242^{2}$, $5796^{2}$, $9407^{2}$, $13547^{2}$, $53958^{2}$, $59386^{2}$, $59409^{2}$, $60858^{2}$, $136413^{2}$, $1239700^{2}$, $2553253^{2}$
\item $(h,k)=(4,9)$ : $1^{2}$, $92^{2}$, $161^{2}$, $1265^{2}$, $2047^{2}$, $2691^{2}$, $15065^{2}$, $226734^{2}$, $3946248^{2}$
\item $(h,k)=(4,10)$ : $1^{2}$, $115^{2}$, $1794^{2}$, $4830^{2}$, $39974^{2}$, $50439^{2}$, $942885^{2}$, $1276937^{2}$, $1877329^{2}$
\item $(h,k)=(4,11)$ : $1^{2}$, $46^{2}$, $69^{2}$, $115^{2}$, $276^{2}$, $83306^{2}$, $92920^{2}$, $572976^{2}$, $3444595^{2}$
\item $(h,k)=(4,12)$ : $1^{2}$, $23^{2}$, $46^{2}$, $69^{2}$, $506^{2}$, $713^{2}$, $22471^{2}$, $38226^{2}$, $4132249^{2}$
\item $(h,k)=(4,13)$ : $1^{2}$, $1081^{2}$, $3473^{2}$, $9499^{2}$, $22637^{46}$, $37904^{2}$, $112102^{2}$, $3509593^{2}$
\item $(h,k)=(4,14)$ : $1^{2}$, $782^{2}$, $989^{2}$, $5152^{2}$, $42021^{2}$, $81098^{2}$, $4064261^{2}$
\item $(h,k)=(4,15)$ : $1^{2}$, $23^{4}$, $46^{2}$, $253^{2}$, $598^{2}$, $76130^{2}$, $84433^{2}$, $1441571^{2}$, $2591226^{2}$
\item $(h,k)=(4,16)$ : $1^{2}$, $23^{2}$, $92^{2}$, $5773^{2}$, $20792^{2}$, $4167623^{2}$
\item $(h,k)=(4,17)$ : $1^{2}$, $23^{4}$, $368^{2}$, $1587^{2}$, $2093^{2}$, $2323^{2}$, $11017^{2}$, $41975^{2}$, $165083^{46}$, $337985^{2}$
\item $(h,k)=(4,18)$ : $1^{2}$, $23^{2}$, $391^{2}$, $72312^{2}$, $711804^{2}$, $3409773^{2}$
\item $(h,k)=(4,19)$ : $1^{2}$, $3^{2796202}$
\item $(h,k)=(4,20)$ : $1^{2}$, $161^{2}$, $460^{2}$, $181608^{2}$, $872321^{2}$, $3139753^{2}$
\item $(h,k)=(4,21)$ : $1^{2}$, $46^{2}$, $92^{2}$, $552^{2}$, $254794^{2}$, $1581227^{2}$, $2357592^{2}$
\item $(h,k)=(4,22)$ : $1^{2}$, $5^{46}$, $138^{2}$, $391^{2}$, $805^{2}$, $2484^{2}$, $26933^{2}$, $83421^{2}$, $110331^{2}$, $146671^{2}$, $932995^{2}$, $2890019^{2}$
\item $(h,k)=(5,6)$ : $1^{2}$, $184^{2}$, $1288^{2}$, $8924^{2}$, $26485^{46}$, $639285^{2}$, $2935467^{2}$
\item $(h,k)=(5,7)$ : $1^{2}$, $81880^{2}$, $586730^{2}$, $1689994^{2}$, $1835699^{2}$
\item $(h,k)=(5,8)$ : $1^{2}$, $23^{4}$, $483^{2}$, $40480^{2}$, $64791^{2}$, $4088503^{2}$
\item $(h,k)=(5,9)$ : $1^{2}$, $23^{2}$, $46^{2}$, $1817^{2}$, $5221^{2}$, $7958^{2}$, $52716^{2}$, $4126522^{2}$
\item $(h,k)=(5,10)$ : $1^{2}$, $46^{2}$, $92^{2}$, $230^{2}$, $299^{2}$, $989^{2}$, $1357^{2}$, $4577^{2}$, $10971^{2}$, $21344^{2}$, $150397^{2}$, $339457^{2}$, $1068442^{2}$, $1216585^{2}$, $1379517^{2}$
\item $(h,k)=(5,11)$ : $1^{2}$, $23^{2}$, $483^{2}$, $4140^{2}$, $182159^{46}$
\item $(h,k)=(5,12)$ : $1^{2}$, $92^{2}$, $16836^{2}$, $18676^{2}$, $33603^{2}$, $85698^{2}$, $772662^{2}$, $951579^{2}$, $2315157^{2}$
\item $(h,k)=(5,13)$ : $1^{2}$, $23^{2}$, $276^{2}$, $529^{2}$, $20838^{2}$, $185932^{2}$, $641608^{2}$, $1606458^{2}$, $1738639^{2}$
\item $(h,k)=(5,14)$ : $1^{2}$, $46^{2}$, $69^{2}$, $92^{2}$, $713^{2}$, $235014^{2}$, $403696^{2}$, $721602^{2}$, $2833071^{2}$
\item $(h,k)=(5,15)$ : $1^{2}$, $23^{2}$, $161^{2}$, $230^{2}$, $552^{2}$, $1518^{2}$, $11109^{2}$, $72841^{2}$, $339457^{2}$, $1295337^{2}$, $2473075^{2}$
\item $(h,k)=(5,16)$ : $1^{2}$, $23^{6}$, $69^{2}$, $115^{2}$, $230^{2}$, $5474^{2}$, $471477^{2}$, $3716869^{2}$
\item $(h,k)=(5,17)$ : $1^{2}$, $2622^{2}$, $4531^{2}$, $4187150^{2}$
\item $(h,k)=(5,18)$ : $1^{2}$, $3^{2796202}$
\item $(h,k)=(5,19)$ : $1^{2}$, $23^{2}$, $391^{2}$, $72312^{2}$, $711804^{2}$, $3409773^{2}$
\item $(h,k)=(5,20)$ : $1^{2}$, $46^{2}$, $11799^{2}$, $4182458^{2}$
\item $(h,k)=(5,21)$ : $1^{2}$, $23^{2}$, $69^{2}$, $322^{2}$, $5152^{2}$, $95910^{2}$, $105179^{2}$, $1782914^{2}$, $2204734^{2}$
\item $(h,k)=(5,22)$ : $1^{2}$, $92^{2}$, $161^{2}$, $3680^{2}$, $41584^{2}$, $127903^{2}$, $186024^{2}$, $235704^{2}$, $3599155^{2}$
\item $(h,k)=(6,7)$ : $1^{2}$, $345^{2}$, $1472^{2}$, $14467^{2}$, $944633^{2}$, $3233386^{2}$
\item $(h,k)=(6,8)$ : $1^{2}$, $23^{2}$, $92^{4}$, $161^{2}$, $368^{2}$, $529^{46}$, $9683^{2}$, $10143^{2}$, $88849^{2}$, $210795^{2}$, $3861930^{2}$
\item $(h,k)=(6,9)$ : $1^{2}$, $23^{2}$, $69^{2}$, $2093^{2}$, $3956^{2}$, $123441^{2}$, $134412^{2}$, $962481^{2}$, $2967828^{2}$
\item $(h,k)=(6,10)$ : $1^{2}$, $69^{2}$, $115^{2}$, $1503^{46}$, $2553^{2}$, $4692^{2}$, $5819^{2}$, $6348^{2}$, $69377^{46}$, $570492^{2}$, $1973975^{2}$
\item $(h,k)=(6,11)$ : $1^{2}$, $23^{2}$, $69^{2}$, $161^{2}$, $9890^{2}$, $4184160^{2}$
\item $(h,k)=(6,12)$ : $1^{2}$, $23^{2}$, $46^{4}$, $34937^{2}$, $53935^{2}$, $104075^{2}$, $317998^{2}$, $815764^{2}$, $2867479^{2}$
\item $(h,k)=(6,13)$ : $1^{2}$, $46^{2}$, $92^{2}$, $112976^{2}$, $1824705^{2}$, $2256484^{2}$
\item $(h,k)=(6,14)$ : $1^{2}$, $23^{2}$, $161^{2}$, $184^{2}$, $1909^{2}$, $2760^{2}$, $5612^{2}$, $15962^{2}$, $18331^{2}$, $32591^{2}$, $73991^{2}$, $4042779^{2}$
\item $(h,k)=(6,15)$ : $1^{2}$, $6302^{2}$, $84456^{2}$, $92621^{2}$, $906683^{2}$, $3104241^{2}$
\item $(h,k)=(6,16)$ : $1^{2}$, $1311^{2}$, $1564^{2}$, $1702^{2}$, $26910^{2}$, $731998^{2}$, $1078631^{2}$, $1087463^{2}$, $1264724^{2}$
\item $(h,k)=(6,17)$ : $1^{2}$, $3^{2796202}$
\item $(h,k)=(6,18)$ : $1^{2}$, $2622^{2}$, $4531^{2}$, $4187150^{2}$
\item $(h,k)=(6,19)$ : $1^{2}$, $23^{4}$, $368^{2}$, $1587^{2}$, $2093^{2}$, $2323^{2}$, $11017^{2}$, $41975^{2}$, $165083^{46}$, $337985^{2}$
\item $(h,k)=(6,20)$ : $1^{2}$, $23^{4}$, $690^{2}$, $116012^{2}$, $162748^{2}$, $209346^{2}$, $776710^{2}$, $2928751^{2}$
\item $(h,k)=(6,21)$ : $1^{2}$, $69^{2}$, $92^{2}$, $9039^{2}$, $749869^{2}$, $3435234^{2}$
\item $(h,k)=(6,22)$ : $1^{2}$, $460^{2}$, $598^{2}$, $5152^{2}$, $40848^{2}$, $271124^{2}$, $727559^{2}$, $844215^{2}$, $867560^{2}$, $1436787^{2}$
\item $(h,k)=(7,8)$ : $1^{2}$, $23^{2}$, $46^{2}$, $713^{2}$, $1081^{2}$, $1587^{2}$, $1633^{2}$, $1725^{2}$, $7245^{2}$, $96048^{2}$, $4084202^{2}$
\item $(h,k)=(7,9)$ : $1^{2}$, $368^{2}$, $1840^{2}$, $3933^{2}$, $4188162^{2}$
\item $(h,k)=(7,10)$ : $1^{2}$, $69^{2}$, $529^{2}$, $690^{2}$, $492223^{2}$, $3700792^{2}$
\item $(h,k)=(7,11)$ : $1^{2}$, $23^{2}$, $69^{2}$, $598^{2}$, $1058^{2}$, $87998^{2}$, $238717^{2}$, $1111498^{2}$, $2754342^{2}$
\item $(h,k)=(7,12)$ : $1^{2}$, $621^{2}$, $4193682^{2}$
\item $(h,k)=(7,13)$ : $1^{2}$, $23^{2}$, $92^{2}$, $1978^{2}$, $3657^{2}$, $6946^{2}$, $49657^{2}$, $4131950^{2}$
\item $(h,k)=(7,14)$ : $1^{2}$, $23^{2}$, $69^{2}$, $207^{2}$, $345^{2}$, $1817^{2}$, $2340^{46}$, $8510^{2}$, $22724^{2}$, $248538^{2}$, $319033^{2}$, $520444^{2}$, $924899^{2}$, $2093874^{2}$
\item $(h,k)=(7,15)$ : $1^{2}$, $23^{2}$, $46^{4}$, $69^{2}$, $322^{2}$, $1104^{2}$, $1173^{2}$, $1196^{2}$, $2737^{2}$, $3128^{2}$, $3519^{2}$, $19711^{2}$, $20447^{2}$, $30659^{2}$, $34178^{2}$, $39537^{2}$, $66562^{2}$, $72226^{46}$, $94024^{2}$, $189382^{2}$, $291571^{2}$, $1733671^{2}$
\item $(h,k)=(7,16)$ : $1^{2}$, $3^{2796202}$
\item $(h,k)=(7,17)$ : $1^{2}$, $1311^{2}$, $1564^{2}$, $1702^{2}$, $26910^{2}$, $731998^{2}$, $1078631^{2}$, $1087463^{2}$, $1264724^{2}$
\item $(h,k)=(7,18)$ : $1^{2}$, $23^{6}$, $69^{2}$, $115^{2}$, $230^{2}$, $5474^{2}$, $471477^{2}$, $3716869^{2}$
\item $(h,k)=(7,19)$ : $1^{2}$, $23^{2}$, $92^{2}$, $5773^{2}$, $20792^{2}$, $4167623^{2}$
\item $(h,k)=(7,20)$ : $1^{2}$, $46^{2}$, $184^{2}$, $920^{2}$, $1150^{2}$, $8418^{2}$, $8487^{2}$, $9200^{2}$, $28326^{46}$, $204884^{2}$, $750007^{2}$, $2559509^{2}$
\item $(h,k)=(7,21)$ : $1^{2}$, $23^{2}$, $184^{2}$, $230^{2}$, $782^{2}$, $16537^{2}$, $103454^{2}$, $4073093^{2}$
\item $(h,k)=(7,22)$ : $1^{2}$, $46^{2}$, $552^{2}$, $1656^{2}$, $1702^{2}$, $74980^{2}$, $1693099^{2}$, $2422268^{2}$
\item $(h,k)=(8,9)$ : $1^{2}$, $46^{2}$, $69^{2}$, $1219^{2}$, $9338^{2}$, $16307^{2}$, $83444^{2}$, $156377^{2}$, $271929^{2}$, $3655574^{2}$
\item $(h,k)=(8,10)$ : $1^{2}$, $23^{2}$, $46^{2}$, $276^{2}$, $20746^{2}$, $97819^{2}$, $4075393^{2}$
\item $(h,k)=(8,11)$ : $1^{2}$, $46^{2}$, $115^{2}$, $437^{2}$, $966^{2}$, $3565^{2}$, $41932^{46}$, $52785^{2}$, $3171953^{2}$
\item $(h,k)=(8,12)$ : $1^{2}$, $253^{2}$, $1886^{2}$, $10856^{2}$, $15180^{2}$, $336605^{2}$, $3829523^{2}$
\item $(h,k)=(8,13)$ : $1^{2}$, $23^{2}$, $46^{2}$, $92^{2}$, $575^{2}$, $4002^{2}$, $14904^{2}$, $16928^{2}$, $35765^{2}$, $58696^{2}$, $186185^{2}$, $3877087^{2}$
\item $(h,k)=(8,14)$ : $1^{2}$, $8464^{2}$, $28635^{2}$, $58351^{2}$, $221122^{2}$, $297781^{2}$, $3579950^{2}$
\item $(h,k)=(8,15)$ : $1^{2}$, $3^{2796202}$
\item $(h,k)=(8,16)$ : $1^{2}$, $23^{2}$, $46^{4}$, $69^{2}$, $322^{2}$, $1104^{2}$, $1173^{2}$, $1196^{2}$, $2737^{2}$, $3128^{2}$, $3519^{2}$, $19711^{2}$, $20447^{2}$, $30659^{2}$, $34178^{2}$, $39537^{2}$, $66562^{2}$, $72226^{46}$, $94024^{2}$, $189382^{2}$, $291571^{2}$, $1733671^{2}$
\item $(h,k)=(8,17)$ : $1^{2}$, $6302^{2}$, $84456^{2}$, $92621^{2}$, $906683^{2}$, $3104241^{2}$
\item $(h,k)=(8,18)$ : $1^{2}$, $23^{2}$, $161^{2}$, $230^{2}$, $552^{2}$, $1518^{2}$, $11109^{2}$, $72841^{2}$, $339457^{2}$, $1295337^{2}$, $2473075^{2}$
\item $(h,k)=(8,19)$ : $1^{2}$, $23^{4}$, $46^{2}$, $253^{2}$, $598^{2}$, $76130^{2}$, $84433^{2}$, $1441571^{2}$, $2591226^{2}$
\item $(h,k)=(8,20)$ : $1^{2}$, $23^{2}$, $92^{2}$, $8303^{2}$, $4185885^{2}$
\item $(h,k)=(8,21)$ : $1^{2}$, $23^{2}$, $161^{2}$, $44597^{2}$, $47886^{2}$, $1268013^{2}$, $2833623^{2}$
\item $(h,k)=(8,22)$ : $1^{2}$, $69^{2}$, $299^{2}$, $782^{2}$, $9706^{2}$, $24196^{2}$, $29969^{2}$, $60030^{2}$, $1702483^{2}$, $2366769^{2}$
\item $(h,k)=(9,10)$ : $1^{2}$, $322^{2}$, $621^{2}$, $263557^{2}$, $282785^{2}$, $950107^{2}$, $2696911^{2}$
\item $(h,k)=(9,11)$ : $1^{2}$, $23^{4}$, $253^{2}$, $322^{2}$, $7820^{2}$, $9200^{2}$, $20447^{2}$, $425730^{2}$, $3730485^{2}$
\item $(h,k)=(9,12)$ : $1^{2}$, $13616^{2}$, $20447^{2}$, $38663^{2}$, $43999^{2}$, $54924^{2}$, $58259^{2}$, $3964395^{2}$
\item $(h,k)=(9,13)$ : $1^{2}$, $69^{2}$, $207^{2}$, $851^{2}$, $1035^{2}$, $1104^{2}$, $34721^{46}$, $59156^{2}$, $1437730^{2}$, $1895568^{2}$
\item $(h,k)=(9,14)$ : $1^{2}$, $3^{2796202}$
\item $(h,k)=(9,15)$ : $1^{2}$, $8464^{2}$, $28635^{2}$, $58351^{2}$, $221122^{2}$, $297781^{2}$, $3579950^{2}$
\item $(h,k)=(9,16)$ : $1^{2}$, $23^{2}$, $69^{2}$, $207^{2}$, $345^{2}$, $1817^{2}$, $2340^{46}$, $8510^{2}$, $22724^{2}$, $248538^{2}$, $319033^{2}$, $520444^{2}$, $924899^{2}$, $2093874^{2}$
\item $(h,k)=(9,17)$ : $1^{2}$, $23^{2}$, $161^{2}$, $184^{2}$, $1909^{2}$, $2760^{2}$, $5612^{2}$, $15962^{2}$, $18331^{2}$, $32591^{2}$, $73991^{2}$, $4042779^{2}$
\item $(h,k)=(9,18)$ : $1^{2}$, $46^{2}$, $69^{2}$, $92^{2}$, $713^{2}$, $235014^{2}$, $403696^{2}$, $721602^{2}$, $2833071^{2}$
\item $(h,k)=(9,19)$ : $1^{2}$, $782^{2}$, $989^{2}$, $5152^{2}$, $42021^{2}$, $81098^{2}$, $4064261^{2}$
\item $(h,k)=(9,20)$ : $1^{2}$, $299^{2}$, $4194004^{2}$
\item $(h,k)=(9,21)$ : $1^{2}$, $92^{2}$, $391^{2}$, $9223^{2}$, $28566^{2}$, $159068^{2}$, $322184^{2}$, $1130404^{2}$, $2544375^{2}$
\item $(h,k)=(9,22)$ : $1^{2}$, $184^{2}$, $345^{2}$, $667^{2}$, $1472^{2}$, $1541^{2}$, $5152^{2}$, $377407^{2}$, $592135^{2}$, $3215400^{2}$
\item $(h,k)=(10,11)$ : $1^{2}$, $46^{2}$, $414^{2}$, $626^{46}$, $4179445^{2}$
\item $(h,k)=(10,12)$ : $1^{2}$, $395577^{2}$, $3798726^{2}$
\item $(h,k)=(10,13)$ : $1^{2}$, $3^{2796202}$
\item $(h,k)=(10,14)$ : $1^{2}$, $69^{2}$, $207^{2}$, $851^{2}$, $1035^{2}$, $1104^{2}$, $34721^{46}$, $59156^{2}$, $1437730^{2}$, $1895568^{2}$
\item $(h,k)=(10,15)$ : $1^{2}$, $23^{2}$, $46^{2}$, $92^{2}$, $575^{2}$, $4002^{2}$, $14904^{2}$, $16928^{2}$, $35765^{2}$, $58696^{2}$, $186185^{2}$, $3877087^{2}$
\item $(h,k)=(10,16)$ : $1^{2}$, $23^{2}$, $92^{2}$, $1978^{2}$, $3657^{2}$, $6946^{2}$, $49657^{2}$, $4131950^{2}$
\item $(h,k)=(10,17)$ : $1^{2}$, $46^{2}$, $92^{2}$, $112976^{2}$, $1824705^{2}$, $2256484^{2}$
\item $(h,k)=(10,18)$ : $1^{2}$, $23^{2}$, $276^{2}$, $529^{2}$, $20838^{2}$, $185932^{2}$, $641608^{2}$, $1606458^{2}$, $1738639^{2}$
\item $(h,k)=(10,19)$ : $1^{2}$, $1081^{2}$, $3473^{2}$, $9499^{2}$, $22637^{46}$, $37904^{2}$, $112102^{2}$, $3509593^{2}$
\item $(h,k)=(10,20)$ : $1^{2}$, $23^{2}$, $69^{2}$, $253^{2}$, $414^{2}$, $483^{2}$, $575^{2}$, $14835^{2}$, $28267^{2}$, $38638^{46}$, $60168^{2}$, $65366^{2}$, $413540^{2}$, $606395^{2}$, $2115241^{2}$
\item $(h,k)=(10,21)$ : $1^{2}$, $851^{2}$, $1380^{2}$, $41745^{2}$, $152674^{2}$, $3997653^{2}$
\item $(h,k)=(10,22)$ : $1^{2}$, $69^{2}$, $138^{2}$, $1091396^{2}$, $3102700^{2}$
\item $(h,k)=(11,12)$ : $1^{2}$, $3^{2796202}$
\item $(h,k)=(11,13)$ : $1^{2}$, $395577^{2}$, $3798726^{2}$
\item $(h,k)=(11,14)$ : $1^{2}$, $13616^{2}$, $20447^{2}$, $38663^{2}$, $43999^{2}$, $54924^{2}$, $58259^{2}$, $3964395^{2}$
\item $(h,k)=(11,15)$ : $1^{2}$, $253^{2}$, $1886^{2}$, $10856^{2}$, $15180^{2}$, $336605^{2}$, $3829523^{2}$
\item $(h,k)=(11,16)$ : $1^{2}$, $621^{2}$, $4193682^{2}$
\item $(h,k)=(11,17)$ : $1^{2}$, $23^{2}$, $46^{4}$, $34937^{2}$, $53935^{2}$, $104075^{2}$, $317998^{2}$, $815764^{2}$, $2867479^{2}$
\item $(h,k)=(11,18)$ : $1^{2}$, $92^{2}$, $16836^{2}$, $18676^{2}$, $33603^{2}$, $85698^{2}$, $772662^{2}$, $951579^{2}$, $2315157^{2}$
\item $(h,k)=(11,19)$ : $1^{2}$, $23^{2}$, $46^{2}$, $69^{2}$, $506^{2}$, $713^{2}$, $22471^{2}$, $38226^{2}$, $4132249^{2}$
\item $(h,k)=(11,20)$ : $1^{2}$, $23^{2}$, $46^{4}$, $115^{2}$, $690^{2}$, $51037^{2}$, $86388^{2}$, $1045005^{2}$, $3010953^{2}$
\item $(h,k)=(11,21)$ : $1^{2}$, $23^{2}$, $69^{2}$, $184^{2}$, $2438^{2}$, $2760^{2}$, $71047^{2}$, $4117782^{2}$
\item $(h,k)=(11,22)$ : $1^{2}$, $23^{2}$, $4462^{40}$, $8924^{920}$
\item $(h,k)=(12,13)$ : $1^{2}$, $46^{2}$, $414^{2}$, $626^{46}$, $4179445^{2}$
\item $(h,k)=(12,14)$ : $1^{2}$, $23^{4}$, $253^{2}$, $322^{2}$, $7820^{2}$, $9200^{2}$, $20447^{2}$, $425730^{2}$, $3730485^{2}$
\item $(h,k)=(12,15)$ : $1^{2}$, $46^{2}$, $115^{2}$, $437^{2}$, $966^{2}$, $3565^{2}$, $41932^{46}$, $52785^{2}$, $3171953^{2}$
\item $(h,k)=(12,16)$ : $1^{2}$, $23^{2}$, $69^{2}$, $598^{2}$, $1058^{2}$, $87998^{2}$, $238717^{2}$, $1111498^{2}$, $2754342^{2}$
\item $(h,k)=(12,17)$ : $1^{2}$, $23^{2}$, $69^{2}$, $161^{2}$, $9890^{2}$, $4184160^{2}$
\item $(h,k)=(12,18)$ : $1^{2}$, $23^{2}$, $483^{2}$, $4140^{2}$, $182159^{46}$
\item $(h,k)=(12,19)$ : $1^{2}$, $46^{2}$, $69^{2}$, $115^{2}$, $276^{2}$, $83306^{2}$, $92920^{2}$, $572976^{2}$, $3444595^{2}$
\item $(h,k)=(12,20)$ : $1^{2}$, $3151^{2}$, $10971^{2}$, $86802^{2}$, $158194^{2}$, $1409808^{2}$, $2525377^{2}$
\item $(h,k)=(12,21)$ : $1^{2}$, $92^{2}$, $805^{2}$, $2001^{2}$, $2231^{2}$, $121083^{46}$, $546940^{2}$, $857325^{2}$
\item $(h,k)=(12,22)$ : $1^{2}$, $115^{2}$, $1817^{2}$, $5543^{2}$, $299368^{2}$, $1423746^{2}$, $2463714^{2}$
\item $(h,k)=(13,14)$ : $1^{2}$, $322^{2}$, $621^{2}$, $263557^{2}$, $282785^{2}$, $950107^{2}$, $2696911^{2}$
\item $(h,k)=(13,15)$ : $1^{2}$, $23^{2}$, $46^{2}$, $276^{2}$, $20746^{2}$, $97819^{2}$, $4075393^{2}$
\item $(h,k)=(13,16)$ : $1^{2}$, $69^{2}$, $529^{2}$, $690^{2}$, $492223^{2}$, $3700792^{2}$
\item $(h,k)=(13,17)$ : $1^{2}$, $69^{2}$, $115^{2}$, $1503^{46}$, $2553^{2}$, $4692^{2}$, $5819^{2}$, $6348^{2}$, $69377^{46}$, $570492^{2}$, $1973975^{2}$
\item $(h,k)=(13,18)$ : $1^{2}$, $46^{2}$, $92^{2}$, $230^{2}$, $299^{2}$, $989^{2}$, $1357^{2}$, $4577^{2}$, $10971^{2}$, $21344^{2}$, $150397^{2}$, $339457^{2}$, $1068442^{2}$, $1216585^{2}$, $1379517^{2}$
\item $(h,k)=(13,19)$ : $1^{2}$, $115^{2}$, $1794^{2}$, $4830^{2}$, $39974^{2}$, $50439^{2}$, $942885^{2}$, $1276937^{2}$, $1877329^{2}$
\item $(h,k)=(13,20)$ : $1^{2}$, $69^{2}$, $207^{2}$, $7222^{2}$, $43792^{2}$, $423085^{2}$, $1323236^{2}$, $2396692^{2}$
\item $(h,k)=(13,21)$ : $1^{2}$, $23^{4}$, $207^{2}$, $621^{2}$, $805^{2}$, $6072^{2}$, $125028^{2}$, $225239^{2}$, $627233^{2}$, $649083^{2}$, $2559969^{2}$
\item $(h,k)=(13,22)$ : $1^{2}$, $23^{2}$, $1081^{2}$, $18538^{2}$, $2065400^{2}$, $2109261^{2}$
\item $(h,k)=(14,15)$ : $1^{2}$, $46^{2}$, $69^{2}$, $1219^{2}$, $9338^{2}$, $16307^{2}$, $83444^{2}$, $156377^{2}$, $271929^{2}$, $3655574^{2}$
\item $(h,k)=(14,16)$ : $1^{2}$, $368^{2}$, $1840^{2}$, $3933^{2}$, $4188162^{2}$
\item $(h,k)=(14,17)$ : $1^{2}$, $23^{2}$, $69^{2}$, $2093^{2}$, $3956^{2}$, $123441^{2}$, $134412^{2}$, $962481^{2}$, $2967828^{2}$
\item $(h,k)=(14,18)$ : $1^{2}$, $23^{2}$, $46^{2}$, $1817^{2}$, $5221^{2}$, $7958^{2}$, $52716^{2}$, $4126522^{2}$
\item $(h,k)=(14,19)$ : $1^{2}$, $92^{2}$, $161^{2}$, $1265^{2}$, $2047^{2}$, $2691^{2}$, $15065^{2}$, $226734^{2}$, $3946248^{2}$
\item $(h,k)=(14,20)$ : $1^{2}$, $23^{2}$, $46^{2}$, $69^{2}$, $6348^{2}$, $8970^{2}$, $9131^{2}$, $30383^{2}$, $174363^{2}$, $3964970^{2}$
\item $(h,k)=(14,21)$ : $1^{2}$, $184^{2}$, $529^{2}$, $5175^{2}$, $5359^{2}$, $20033^{2}$, $42136^{2}$, $168797^{2}$, $208587^{2}$, $3743503^{2}$
\item $(h,k)=(14,22)$ : $1^{2}$, $23^{2}$, $230^{2}$, $322^{2}$, $747569^{2}$, $3446159^{2}$
\item $(h,k)=(15,16)$ : $1^{2}$, $23^{2}$, $46^{2}$, $713^{2}$, $1081^{2}$, $1587^{2}$, $1633^{2}$, $1725^{2}$, $7245^{2}$, $96048^{2}$, $4084202^{2}$
\item $(h,k)=(15,17)$ : $1^{2}$, $23^{2}$, $92^{4}$, $161^{2}$, $368^{2}$, $529^{46}$, $9683^{2}$, $10143^{2}$, $88849^{2}$, $210795^{2}$, $3861930^{2}$
\item $(h,k)=(15,18)$ : $1^{2}$, $23^{4}$, $483^{2}$, $40480^{2}$, $64791^{2}$, $4088503^{2}$
\item $(h,k)=(15,19)$ : $1^{2}$, $23^{2}$, $115^{2}$, $138^{2}$, $230^{2}$, $414^{4}$, $1242^{2}$, $5796^{2}$, $9407^{2}$, $13547^{2}$, $53958^{2}$, $59386^{2}$, $59409^{2}$, $60858^{2}$, $136413^{2}$, $1239700^{2}$, $2553253^{2}$
\item $(h,k)=(15,20)$ : $1^{2}$, $46^{2}$, $598^{2}$, $5704^{2}$, $28267^{2}$, $838120^{2}$, $1541184^{2}$, $1780384^{2}$
\item $(h,k)=(15,21)$ : $1^{2}$, $23^{2}$, $69^{2}$, $230^{2}$, $368^{2}$, $3680^{2}$, $4991^{2}$, $5842^{2}$, $55775^{2}$, $4123325^{2}$
\item $(h,k)=(15,22)$ : $1^{2}$, $23^{2}$, $1219^{2}$, $1449^{2}$, $3128^{2}$, $18472^{46}$, $140208^{2}$, $3623420^{2}$
\item $(h,k)=(16,17)$ : $1^{2}$, $345^{2}$, $1472^{2}$, $14467^{2}$, $944633^{2}$, $3233386^{2}$
\item $(h,k)=(16,18)$ : $1^{2}$, $81880^{2}$, $586730^{2}$, $1689994^{2}$, $1835699^{2}$
\item $(h,k)=(16,19)$ : $1^{2}$, $23^{2}$, $2070^{2}$, $98716^{2}$, $371496^{2}$, $1387314^{2}$, $2334684^{2}$
\item $(h,k)=(16,20)$ : $1^{2}$, $23^{2}$, $115^{2}$, $1242^{2}$, $2277^{2}$, $21597^{2}$, $92000^{2}$, $670795^{2}$, $1000477^{2}$, $2405777^{2}$
\item $(h,k)=(16,21)$ : $1^{2}$, $23^{4}$, $276^{2}$, $575^{2}$, $2737^{2}$, $3199^{46}$, $10350^{2}$, $52235^{46}$, $1193102^{2}$, $1712235^{2}$
\item $(h,k)=(16,22)$ : $1^{2}$, $69^{2}$, $368^{2}$, $2047^{2}$, $6693^{2}$, $12052^{2}$, $36478^{2}$, $48208^{2}$, $186829^{2}$, $339388^{2}$, $1558687^{2}$, $2003484^{2}$
\item $(h,k)=(17,18)$ : $1^{2}$, $184^{2}$, $1288^{2}$, $8924^{2}$, $26485^{46}$, $639285^{2}$, $2935467^{2}$
\item $(h,k)=(17,19)$ : $1^{2}$, $69^{2}$, $33718^{2}$, $1333977^{2}$, $2826539^{2}$
\item $(h,k)=(17,20)$ : $1^{2}$, $23^{2}$, $46^{4}$, $552^{2}$, $1748^{2}$, $3036^{2}$, $3197^{2}$, $5451^{2}$, $40043^{2}$, $113137^{2}$, $121187^{2}$, $175030^{2}$, $335662^{2}$, $1588426^{2}$, $1806719^{2}$
\item $(h,k)=(17,21)$ : $1^{2}$, $92^{2}$, $161^{2}$, $391^{2}$, $1771^{2}$, $410872^{2}$, $433481^{2}$, $3347535^{2}$
\item $(h,k)=(17,22)$ : $1^{2}$, $345^{2}$, $3289^{2}$, $47380^{2}$, $69667^{2}$, $142462^{2}$, $3931160^{2}$
\item $(h,k)=(18,19)$ : $1^{2}$, $66332^{2}$, $275126^{2}$, $3852845^{2}$
\item $(h,k)=(18,20)$ : $1^{2}$, $7^{46}$, $667^{2}$, $874^{2}$, $1242^{2}$, $59271^{2}$, $163645^{2}$, $468602^{2}$, $718451^{2}$, $2781390^{2}$
\item $(h,k)=(18,21)$ : $1^{2}$, $69^{2}$, $322^{2}$, $6900^{2}$, $8050^{2}$, $21091^{2}$, $21620^{2}$, $50416^{2}$, $4085835^{2}$
\item $(h,k)=(18,22)$ : $1^{2}$, $46^{2}$, $138^{2}$, $253^{2}$, $12489^{2}$, $12604^{2}$, $60559^{2}$, $793063^{2}$, $3315151^{2}$
\item $(h,k)=(19,20)$ : $1^{2}$, $115^{2}$, $437^{2}$, $2438^{2}$, $3358^{2}$, $5727^{2}$, $99429^{2}$, $1112648^{2}$, $2970151^{2}$
\item $(h,k)=(19,21)$ : $1^{2}$, $69^{4}$, $92^{2}$, $552^{2}$, $644^{2}$, $6440^{2}$, $8510^{2}$, $8878^{2}$, $43010^{2}$, $45701^{2}$, $138230^{2}$, $171396^{46}$
\item $(h,k)=(19,22)$ : $1^{2}$, $20^{46}$, $1932^{2}$, $3634^{2}$, $26082^{2}$, $87653^{2}$, $117185^{2}$, $198674^{2}$, $3758683^{2}$
\item $(h,k)=(20,21)$ : $1^{2}$, $46^{2}$, $299^{2}$, $23276^{2}$, $66815^{2}$, $80960^{2}$, $4022907^{2}$
\item $(h,k)=(20,22)$ : $1^{2}$, $23^{2}$, $92^{2}$, $2875^{2}$, $12765^{2}$, $48668^{2}$, $62928^{2}$, $386308^{2}$, $1773300^{2}$, $1907344^{2}$
\item $(h,k)=(21,22)$ : $1^{2}$, $23^{4}$, $31^{414}$, $32^{322}$, $34^{276}$, $36^{276}$, $40^{184}$, $44^{138}$, $46^{194}$, $47^{92}$, $52^{276}$, $53^{46}$, $61^{92}$, $62^{92}$, $64^{92}$, $68^{276}$, $76^{92}$, $80^{92}$, $84^{46}$, $92^{114}$, $101^{184}$, $102^{92}$, $108^{92}$, $110^{46}$, $115^{14}$, $119^{92}$, $136^{46}$, $138^{38}$, $139^{46}$, $161^{70}$, $162^{46}$, $176^{46}$, $183^{46}$, $184^{132}$, $197^{46}$, $198^{46}$, $207^{124}$, $214^{46}$, $230^{84}$, $253^{118}$, $261^{46}$, $267^{46}$, $276^{346}$, $278^{46}$, $298^{46}$, $299^{124}$, $322^{432}$, $345^{32}$, $350^{46}$, $368^{174}$, $391^{88}$, $414^{116}$, $437^{20}$, $460^{148}$, $483^{56}$, $506^{188}$, $529^{38}$, $552^{74}$, $575^{38}$, $598^{172}$, $621^{64}$, $644^{122}$, $667^{42}$, $690^{104}$, $713^{32}$, $736^{84}$, $759^{14}$, $782^{58}$, $805^{74}$, $828^{102}$, $851^{44}$, $874^{60}$, $897^{12}$, $920^{76}$, $943^{38}$, $966^{44}$, $989^{32}$, $1012^{76}$, $1035^{42}$, $1058^{66}$, $1081^{24}$, $1104^{48}$, $1127^{8}$, $1150^{42}$, $1173^{6}$, $1196^{30}$, $1219^{12}$, $1242^{74}$, $1265^{28}$, $1288^{72}$, $1311^{42}$, $1334^{30}$, $1357^{6}$, $1380^{40}$, $1403^{10}$, $1426^{16}$, $1449^{8}$, $1472^{46}$, $1495^{36}$, $1518^{26}$, $1541^{8}$, $1564^{34}$, $1587^{18}$, $1610^{28}$, $1656^{40}$, $1679^{8}$, $1702^{38}$, $1725^{8}$, $1748^{12}$, $1771^{8}$, $1794^{44}$, $1817^{4}$, $1840^{24}$, $1863^{6}$, $1886^{38}$, $1909^{4}$, $1932^{32}$, $1978^{28}$, $2001^{4}$, $2024^{6}$, $2070^{32}$, $2093^{22}$, $2116^{10}$, $2139^{8}$, $2162^{28}$, $2185^{4}$, $2208^{34}$, $2231^{2}$, $2254^{32}$, $2277^{20}$, $2300^{10}$, $2323^{2}$, $2346^{16}$, $2392^{10}$, $2415^{10}$, $2438^{16}$, $2461^{6}$, $2484^{22}$, $2507^{6}$, $2530^{24}$, $2553^{8}$, $2576^{20}$, $2599^{6}$, $2622^{14}$, $2668^{14}$, $2714^{22}$, $2737^{6}$, $2760^{20}$, $2783^{4}$, $2806^{10}$, $2829^{8}$, $2852^{14}$, $2875^{18}$, $2898^{18}$, $2921^{6}$, $2944^{10}$, $2967^{6}$, $3013^{4}$, $3036^{6}$, $3059^{4}$, $3082^{6}$, $3105^{16}$, $3128^{18}$, $3151^{10}$, $3174^{2}$, $3197^{6}$, $3220^{4}$, $3243^{4}$, $3266^{6}$, $3312^{8}$, $3335^{16}$, $3358^{16}$, $3381^{6}$, $3404^{8}$, $3427^{10}$, $3450^{16}$, $3473^{4}$, $3496^{10}$, $3519^{4}$, $3542^{12}$, $3588^{4}$, $3634^{8}$, $3657^{6}$, $3680^{10}$, $3703^{6}$, $3726^{12}$, $3749^{6}$, $3772^{4}$, $3818^{8}$, $3841^{6}$, $3864^{8}$, $3887^{10}$, $3910^{10}$, $3933^{2}$, $3956^{2}$, $3979^{4}$, $4002^{2}$, $4025^{4}$, $4048^{8}$, $4071^{6}$, $4094^{14}$, $4117^{2}$, $4140^{4}$, $4163^{4}$, $4186^{8}$, $4209^{6}$, $4232^{8}$, $4255^{2}$, $4278^{2}$, $4301^{2}$, $4347^{6}$, $4370^{4}$, $4393^{2}$, $4416^{4}$, $4439^{8}$, $4485^{8}$, $4508^{12}$, $4531^{2}$, $4554^{2}$, $4577^{2}$, $4600^{2}$, $4623^{4}$, $4646^{6}$, $4669^{8}$, $4692^{6}$, $4715^{4}$, $4738^{8}$, $4761^{6}$, $4784^{6}$, $4807^{2}$, $4830^{2}$, $4853^{6}$, $4876^{2}$, $4899^{4}$, $4922^{12}$, $4945^{6}$, $4968^{2}$, $4991^{2}$, $5014^{2}$, $5037^{4}$, $5060^{4}$, $5106^{2}$, $5129^{6}$, $5152^{6}$, $5175^{6}$, $5198^{6}$, $5221^{6}$, $5244^{4}$, $5336^{6}$, $5382^{2}$, $5405^{2}$, $5428^{4}$, $5474^{6}$, $5497^{4}$, $5566^{4}$, $5589^{4}$, $5612^{2}$, $5635^{2}$, $5658^{2}$, $5681^{2}$, $5704^{2}$, $5727^{4}$, $5750^{8}$, $5796^{2}$, $5842^{2}$, $5865^{2}$, $5911^{4}$, $5934^{8}$, $6003^{2}$, $6026^{2}$, $6072^{2}$, $6118^{4}$, $6187^{4}$, $6256^{6}$, $6279^{2}$, $6440^{4}$, $6463^{2}$, $6486^{2}$, $6532^{4}$, $6555^{2}$, $6693^{4}$, $6716^{4}$, $6808^{2}$, $6877^{2}$, $6900^{4}$, $6969^{2}$, $7107^{2}$, $7130^{2}$, $7153^{4}$, $7176^{2}$, $7383^{2}$, $7406^{2}$, $7452^{2}$, $7475^{2}$, $7567^{6}$, $7613^{2}$, $7751^{2}$, $7912^{4}$, $8004^{4}$, $8280^{2}$, $8326^{2}$, $8395^{2}$, $8901^{2}$, $8970^{2}$, $9269^{2}$, $10051^{2}$, $10948^{2}$
\end{itemize}

\section{Results for $n=24$}
\begin{itemize}
\item $(h,k)=(1,5)$ : $1^{16}$, $3^{40}$, $4^{20}$, $7^{32}$, $8^{14}$, $11^{20}$, $12^{8}$, $16^{12}$, $19^{12}$, $21^{24}$, $24^{16}$, $27^{8}$, $30^{24}$, $35^{24}$, $43^{24}$, $57^{8}$, $69^{24}$, $71^{24}$, $95^{24}$, $96^{24}$, $112^{24}$, $174^{32}$, $179^{24}$, $230^{24}$, $235^{48}$, $240^{24}$, $243^{16}$, $295^{24}$, $301^{24}$, $313^{24}$, $349^{24}$, $367^{24}$, $402^{24}$, $406^{24}$, $414^{8}$, $421^{24}$, $426^{24}$, $466^{24}$, $486^{24}$, $538^{24}$, $559^{24}$, $565^{24}$, $590^{24}$, $646^{24}$, $649^{24}$, $674^{24}$, $675^{24}$, $678^{24}$, $699^{24}$, $709^{24}$, $715^{24}$, $751^{48}$, $754^{24}$, $781^{24}$, $795^{24}$, $853^{24}$, $873^{24}$, $881^{24}$, $924^{24}$, $966^{24}$, $1066^{24}$, $1080^{16}$, $1105^{48}$, $1106^{24}$, $1197^{24}$, $1215^{24}$, $1229^{24}$, $1260^{24}$, $1316^{24}$, $1346^{24}$, $1355^{24}$, $1373^{24}$, $1419^{24}$, $1471^{24}$, $1472^{24}$, $1524^{12}$, $1588^{24}$, $1590^{24}$, $1632^{24}$, $1637^{24}$, $1656^{24}$, $1694^{24}$, $1696^{24}$, $1754^{24}$, $1776^{24}$, $1782^{24}$, $1863^{24}$, $1865^{24}$, $1884^{24}$, $1923^{24}$, $1970^{24}$, $1981^{24}$, $2023^{24}$, $2045^{24}$, $2082^{24}$, $2177^{24}$, $2213^{24}$, $2229^{24}$, $2282^{24}$, $2291^{24}$, $2409^{24}$, $2450^{24}$, $2463^{24}$, $2480^{48}$, $2519^{24}$, $2521^{24}$, $2539^{24}$, $2544^{24}$, $2549^{24}$, $2625^{24}$, $2642^{24}$, $2663^{24}$, $2712^{24}$, $2730^{24}$, $2784^{2}$, $2789^{24}$, $2866^{24}$, $2875^{24}$, $2901^{24}$, $3027^{24}$, $3091^{24}$, $3168^{24}$, $3375^{24}$, $3444^{24}$, $3557^{24}$, $3704^{24}$, $3748^{24}$, $3831^{24}$, $3842^{24}$, $3928^{24}$, $4102^{24}$, $4111^{24}$, $4138^{24}$, $4156^{24}$, $4165^{24}$, $4338^{16}$, $4465^{24}$, $4474^{24}$, $4621^{24}$, $4636^{24}$, $4691^{24}$, $4735^{24}$, $4785^{24}$, $4824^{4}$, $4834^{24}$, $4901^{24}$, $4948^{24}$, $5157^{24}$, $5199^{24}$, $5234^{24}$, $5313^{24}$, $5410^{24}$, $5425^{24}$, $5511^{24}$, $5599^{24}$, $5677^{24}$, $5873^{24}$, $5922^{24}$, $6424^{24}$, $6505^{24}$, $6550^{24}$, $6745^{24}$, $6757^{24}$, $7176^{24}$, $7201^{24}$, $7518^{24}$, $7659^{24}$, $8119^{24}$, $8540^{24}$, $8700^{24}$, $9096^{24}$, $9141^{24}$, $9197^{24}$, $9234^{24}$, $9356^{24}$, $9607^{24}$, $9623^{24}$, $9844^{24}$, $10192^{24}$, $10330^{24}$, $10504^{24}$, $10805^{24}$, $10916^{24}$, $11273^{24}$, $11730^{24}$, $12336^{24}$, $12486^{24}$, $13397^{24}$, $13797^{24}$, $17599^{24}$, $19687^{24}$, $20170^{24}$, $32358^{24}$, $37072^{6}$
\item $(h,k)=(1,7)$ : $1^{64}$, $3^{108}$, $4^{36}$, $5^{48}$, $6^{8}$, $8^{78}$, $9^{8}$, $10^{48}$, $12^{8}$, $13^{12}$, $14^{24}$, $16^{36}$, $24^{18}$, $28^{24}$, $29^{12}$, $31^{12}$, $32^{48}$, $35^{12}$, $39^{12}$, $40^{6}$, $52^{12}$, $60^{24}$, $76^{48}$, $82^{24}$, $88^{18}$, $139^{48}$, $158^{24}$, $164^{12}$, $170^{24}$, $200^{48}$, $205^{24}$, $207^{24}$, $220^{24}$, $232^{12}$, $252^{24}$, $258^{24}$, $264^{24}$, $270^{24}$, $283^{24}$, $325^{24}$, $361^{24}$, $382^{24}$, $414^{24}$, $427^{24}$, $429^{24}$, $440^{6}$, $444^{24}$, $455^{24}$, $456^{24}$, $480^{26}$, $482^{24}$, $548^{12}$, $563^{24}$, $564^{24}$, $666^{24}$, $669^{24}$, $695^{24}$, $742^{24}$, $789^{24}$, $806^{24}$, $899^{24}$, $902^{24}$, $967^{24}$, $990^{24}$, $992^{24}$, $1010^{24}$, $1014^{24}$, $1070^{24}$, $1095^{24}$, $1193^{24}$, $1211^{24}$, $1216^{24}$, $1298^{24}$, $1315^{24}$, $1395^{24}$, $1434^{24}$, $1442^{24}$, $1545^{24}$, $1591^{24}$, $1593^{24}$, $1598^{24}$, $1609^{24}$, $1631^{24}$, $1649^{24}$, $1733^{24}$, $1754^{24}$, $1797^{24}$, $1806^{24}$, $1863^{24}$, $1888^{24}$, $1891^{24}$, $1925^{24}$, $1952^{24}$, $2003^{24}$, $2004^{24}$, $2049^{24}$, $2059^{24}$, $2082^{24}$, $2084^{12}$, $2139^{24}$, $2180^{24}$, $2192^{24}$, $2195^{24}$, $2285^{24}$, $2331^{24}$, $2384^{24}$, $2392^{12}$, $2444^{24}$, $2449^{24}$, $2524^{24}$, $2537^{24}$, $2620^{24}$, $2642^{24}$, $2671^{24}$, $2913^{24}$, $2928^{24}$, $2940^{24}$, $2957^{24}$, $3073^{24}$, $3119^{24}$, $3137^{24}$, $3138^{24}$, $3176^{24}$, $3202^{24}$, $3262^{24}$, $3555^{24}$, $3592^{24}$, $3594^{24}$, $3642^{24}$, $3876^{24}$, $3886^{24}$, $3949^{24}$, $4028^{24}$, $4032^{24}$, $4071^{24}$, $4120^{24}$, $4170^{24}$, $4196^{24}$, $4236^{24}$, $4321^{24}$, $4442^{24}$, $4557^{24}$, $4718^{24}$, $4734^{48}$, $4786^{24}$, $4793^{24}$, $4983^{24}$, $5011^{24}$, $5100^{12}$, $5148^{24}$, $5166^{24}$, $5204^{24}$, $5276^{24}$, $5292^{24}$, $5328^{24}$, $5448^{24}$, $5525^{24}$, $5669^{24}$, $5717^{24}$, $5728^{24}$, $5812^{24}$, $5840^{12}$, $6081^{24}$, $6112^{24}$, $6196^{24}$, $6281^{24}$, $6344^{24}$, $6496^{12}$, $6606^{24}$, $6675^{24}$, $6921^{24}$, $7121^{24}$, $7156^{24}$, $7190^{24}$, $7250^{24}$, $7298^{24}$, $7607^{24}$, $7700^{24}$, $7898^{24}$, $7984^{24}$, $7986^{24}$, $8033^{24}$, $8194^{24}$, $8327^{24}$, $8432^{24}$, $8941^{24}$, $8996^{24}$, $9058^{24}$, $9178^{24}$, $9243^{24}$, $9264^{24}$, $9524^{24}$, $9732^{24}$, $9740^{24}$, $9769^{24}$, $10003^{24}$, $10134^{24}$, $10195^{24}$, $10610^{24}$, $10729^{24}$, $11117^{24}$, $11617^{24}$, $12798^{24}$, $13927^{24}$, $19945^{24}$, $30000^{6}$
\item $(h,k)=(1,11)$ : $1^{4}$, $3^{1364}$, $9^{16}$, $24^{4}$, $29^{24}$, $69^{16}$, $77^{24}$, $89^{24}$, $121^{24}$, $180^{24}$, $181^{24}$, $188^{24}$, $226^{24}$, $255^{24}$, $294^{24}$, $295^{24}$, $312^{24}$, $330^{8}$, $339^{24}$, $346^{24}$, $390^{24}$, $393^{24}$, $394^{24}$, $422^{24}$, $461^{24}$, $480^{2}$, $537^{24}$, $552^{24}$, $567^{24}$, $577^{24}$, $654^{24}$, $663^{24}$, $674^{24}$, $709^{24}$, $739^{24}$, $762^{24}$, $765^{24}$, $771^{24}$, $844^{24}$, $884^{24}$, $893^{24}$, $912^{24}$, $957^{24}$, $960^{2}$, $983^{24}$, $1046^{24}$, $1111^{24}$, $1151^{24}$, $1158^{24}$, $1175^{24}$, $1185^{24}$, $1250^{24}$, $1275^{24}$, $1292^{24}$, $1294^{24}$, $1302^{24}$, $1311^{24}$, $1343^{24}$, $1477^{24}$, $1563^{24}$, $1659^{24}$, $1680^{24}$, $1688^{24}$, $1690^{24}$, $1694^{24}$, $1732^{24}$, $1736^{24}$, $1744^{24}$, $1778^{48}$, $1781^{24}$, $1844^{24}$, $1893^{24}$, $1939^{24}$, $1967^{24}$, $1996^{24}$, $2057^{24}$, $2058^{24}$, $2088^{24}$, $2093^{24}$, $2134^{24}$, $2145^{24}$, $2248^{6}$, $2322^{24}$, $2337^{24}$, $2430^{24}$, $2441^{24}$, $2478^{24}$, $2559^{24}$, $2620^{24}$, $2671^{24}$, $2698^{24}$, $2782^{24}$, $2884^{24}$, $2983^{24}$, $2987^{24}$, $3000^{24}$, $3031^{24}$, $3123^{24}$, $3264^{24}$, $3282^{24}$, $3302^{24}$, $3344^{24}$, $3380^{24}$, $3639^{24}$, $3732^{24}$, $3733^{24}$, $3813^{24}$, $3821^{24}$, $3868^{24}$, $3898^{24}$, $3909^{24}$, $3919^{24}$, $3944^{24}$, $3959^{24}$, $3993^{24}$, $4088^{24}$, $4140^{24}$, $4216^{24}$, $4280^{24}$, $4291^{24}$, $4344^{24}$, $4347^{24}$, $4381^{24}$, $4440^{24}$, $4491^{24}$, $4624^{24}$, $4807^{24}$, $4840^{24}$, $4905^{24}$, $5096^{24}$, $5114^{24}$, $5325^{24}$, $5388^{24}$, $5450^{24}$, $5504^{24}$, $5548^{24}$, $5568^{24}$, $5580^{24}$, $5609^{24}$, $5765^{24}$, $5772^{24}$, $6149^{24}$, $6191^{24}$, $6211^{24}$, $6285^{24}$, $6356^{24}$, $6406^{24}$, $6432^{6}$, $6559^{24}$, $6614^{24}$, $6704^{24}$, $6929^{24}$, $7143^{24}$, $7576^{24}$, $7643^{24}$, $7703^{24}$, $8203^{24}$, $8211^{24}$, $8283^{24}$, $8304^{2}$, $8626^{24}$, $8943^{24}$, $9127^{24}$, $9312^{24}$, $9637^{24}$, $10256^{24}$, $10375^{24}$, $11428^{24}$, $11542^{24}$, $12263^{24}$, $12448^{24}$, $12903^{24}$, $14058^{24}$, $14080^{24}$, $14301^{24}$, $15010^{24}$, $15581^{24}$, $16671^{24}$, $19645^{24}$, $21036^{4}$, $24253^{24}$
\item $(h,k)=(1,13)$ : $1^{4096}$, $3^{1184}$, $4^{3068}$, $5^{672}$, $6^{1392}$, $7^{488}$, $8^{1046}$, $9^{728}$, $10^{768}$, $11^{248}$, $12^{820}$, $13^{240}$, $14^{720}$, $15^{336}$, $16^{696}$, $17^{288}$, $18^{640}$, $19^{528}$, $20^{336}$, $21^{336}$, $22^{288}$, $23^{288}$, $24^{476}$, $25^{48}$, $26^{216}$, $27^{96}$, $28^{240}$, $29^{144}$, $30^{112}$, $32^{216}$, $33^{144}$, $34^{192}$, $35^{144}$, $36^{296}$, $37^{144}$, $38^{240}$, $39^{144}$, $40^{192}$, $41^{144}$, $42^{168}$, $43^{144}$, $44^{228}$, $45^{192}$, $46^{288}$, $48^{144}$, $49^{72}$, $50^{96}$, $51^{168}$, $52^{96}$, $53^{48}$, $54^{264}$, $55^{144}$, $56^{168}$, $57^{72}$, $58^{120}$, $60^{152}$, $62^{96}$, $63^{48}$, $64^{72}$, $65^{48}$, $66^{72}$, $68^{192}$, $69^{96}$, $70^{216}$, $72^{96}$, $73^{48}$, $74^{24}$, $75^{48}$, $76^{96}$, $77^{48}$, $78^{216}$, $79^{72}$, $80^{96}$, $81^{48}$, $82^{120}$, $85^{48}$, $86^{96}$, $87^{48}$, $88^{48}$, $90^{96}$, $91^{48}$, $92^{72}$, $93^{48}$, $94^{168}$, $95^{48}$, $96^{146}$, $98^{48}$, $99^{96}$, $100^{216}$, $101^{96}$, $102^{24}$, $104^{48}$, $106^{96}$, $108^{148}$, $109^{48}$, $112^{108}$, $113^{48}$, $114^{240}$, $116^{24}$, $117^{64}$, $118^{216}$, $119^{48}$, $120^{72}$, $122^{48}$, $124^{192}$, $126^{96}$, $128^{72}$, $130^{192}$, $132^{48}$, $133^{48}$, $136^{48}$, $138^{120}$, $139^{72}$, $140^{72}$, $141^{24}$, $144^{48}$, $145^{48}$, $146^{72}$, $147^{48}$, $148^{24}$, $149^{192}$, $150^{48}$, $151^{48}$, $153^{48}$, $154^{48}$, $156^{48}$, $158^{72}$, $160^{48}$, $161^{48}$, $162^{120}$, $164^{72}$, $167^{48}$, $168^{48}$, $170^{48}$, $171^{48}$, $173^{48}$, $174^{32}$, $175^{48}$, $178^{24}$, $179^{48}$, $180^{24}$, $182^{120}$, $183^{24}$, $184^{48}$, $185^{24}$, $186^{96}$, $188^{96}$, $190^{24}$, $191^{48}$, $192^{48}$, $193^{24}$, $196^{96}$, $197^{48}$, $198^{72}$, $200^{72}$, $202^{120}$, $204^{144}$, $205^{48}$, $206^{72}$, $207^{48}$, $208^{24}$, $209^{72}$, $210^{48}$, $212^{72}$, $213^{48}$, $214^{24}$, $216^{24}$, $218^{48}$, $221^{48}$, $222^{120}$, $224^{48}$, $225^{96}$, $226^{48}$, $228^{48}$, $229^{48}$, $230^{48}$, $231^{96}$, $232^{72}$, $233^{24}$, $236^{120}$, $240^{26}$, $246^{24}$, $247^{48}$, $250^{72}$, $251^{48}$, $252^{24}$, $253^{96}$, $254^{144}$, $257^{96}$, $258^{24}$, $260^{48}$, $264^{48}$, $266^{96}$, $268^{72}$, $270^{96}$, $272^{120}$, $274^{48}$, $276^{48}$, $278^{24}$, $282^{72}$, $284^{24}$, $285^{96}$, $290^{48}$, $292^{96}$, $300^{36}$, $301^{72}$, $302^{24}$, $305^{96}$, $307^{72}$, $310^{48}$, $312^{24}$, $313^{72}$, $314^{24}$, $317^{48}$, $318^{40}$, $321^{48}$, $322^{24}$, $324^{48}$, $325^{48}$, $326^{48}$, $327^{96}$, $331^{48}$, $332^{24}$, $333^{24}$, $335^{24}$, $340^{96}$, $342^{24}$, $346^{24}$, $348^{48}$, $362^{24}$, $363^{96}$, $364^{48}$, $366^{24}$, $368^{48}$, $369^{96}$, $372^{24}$, $374^{24}$, $376^{24}$, $378^{48}$, $380^{24}$, $386^{72}$, $395^{48}$, $396^{36}$, $398^{72}$, $401^{48}$, $402^{96}$, $403^{96}$, $404^{72}$, $408^{24}$, $409^{72}$, $411^{48}$, $412^{48}$, $414^{48}$, $416^{48}$, $417^{48}$, $418^{24}$, $420^{24}$, $421^{72}$, $422^{24}$, $425^{48}$, $426^{48}$, $430^{72}$, $432^{24}$, $439^{24}$, $440^{24}$, $441^{48}$, $444^{48}$, $448^{24}$, $451^{48}$, $453^{24}$, $455^{48}$, $456^{48}$, $458^{24}$, $462^{72}$, $466^{24}$, $470^{96}$, $476^{48}$, $478^{48}$, $482^{48}$, $486^{48}$, $487^{48}$, $496^{96}$, $497^{48}$, $498^{24}$, $501^{24}$, $502^{24}$, $504^{54}$, $505^{48}$, $508^{72}$, $510^{72}$, $512^{24}$, $521^{120}$, $522^{72}$, $523^{48}$, $527^{48}$, $528^{48}$, $531^{24}$, $534^{48}$, $536^{24}$, $537^{48}$, $540^{48}$, $542^{24}$, $543^{48}$, $544^{48}$, $545^{48}$, $550^{24}$, $552^{72}$, $554^{24}$, $555^{48}$, $562^{96}$, $564^{48}$, $568^{72}$, $569^{48}$, $575^{48}$, $577^{48}$, $578^{48}$, $579^{24}$, $583^{48}$, $584^{24}$, $585^{72}$, $586^{48}$, $587^{48}$, $591^{24}$, $592^{24}$, $593^{24}$, $595^{24}$, $596^{24}$, $600^{24}$, $601^{24}$, $602^{24}$, $610^{72}$, $618^{24}$, $621^{24}$, $622^{48}$, $625^{48}$, $626^{24}$, $630^{24}$, $645^{96}$, $653^{48}$, $655^{24}$, $658^{24}$, $659^{72}$, $661^{48}$, $670^{72}$, $675^{48}$, $678^{48}$, $680^{24}$, $682^{24}$, $683^{24}$, $690^{24}$, $691^{48}$, $692^{24}$, $698^{24}$, $699^{96}$, $704^{24}$, $707^{24}$, $712^{24}$, $714^{24}$, $716^{24}$, $718^{48}$, $720^{84}$, $724^{24}$, $726^{24}$, $728^{24}$, $731^{72}$, $734^{48}$, $744^{24}$, $749^{48}$, $750^{24}$, $752^{48}$, $755^{48}$, $756^{4}$, $757^{48}$, $760^{24}$, $762^{24}$, $765^{24}$, $771^{24}$, $773^{24}$, $776^{48}$, $778^{48}$, $780^{24}$, $783^{48}$, $786^{24}$, $789^{24}$, $791^{48}$, $799^{48}$, $816^{72}$, $821^{24}$, $822^{24}$, $823^{48}$, $824^{24}$, $829^{48}$, $834^{48}$, $835^{24}$, $840^{72}$, $842^{24}$, $843^{24}$, $845^{24}$, $847^{48}$, $854^{24}$, $856^{24}$, $858^{48}$, $859^{48}$, $861^{24}$, $862^{72}$, $866^{96}$, $877^{48}$, $880^{24}$, $884^{24}$, $886^{24}$, $887^{48}$, $888^{24}$, $892^{48}$, $893^{48}$, $896^{72}$, $901^{24}$, $903^{48}$, $904^{48}$, $914^{24}$, $922^{24}$, $924^{72}$, $925^{48}$, $926^{48}$, $929^{96}$, $930^{16}$, $933^{24}$, $934^{24}$, $937^{48}$, $938^{48}$, $941^{48}$, $951^{48}$, $954^{48}$, $955^{96}$, $957^{24}$, $966^{96}$, $970^{24}$, $974^{24}$, $977^{24}$, $982^{48}$, $983^{24}$, $987^{64}$, $993^{96}$, $996^{48}$, $998^{48}$, $999^{24}$, $1002^{48}$, $1003^{72}$, $1005^{24}$, $1006^{24}$, $1008^{24}$, $1010^{48}$, $1011^{48}$, $1013^{48}$, $1016^{48}$, $1017^{48}$, $1018^{96}$, $1020^{120}$, $1022^{48}$, $1023^{24}$, $1026^{48}$, $1032^{24}$, $1034^{48}$, $1037^{24}$, $1038^{96}$, $1041^{24}$, $1042^{24}$, $1048^{24}$, $1059^{24}$, $1074^{24}$, $1076^{24}$, $1089^{24}$, $1126^{24}$, $1138^{24}$, $1144^{48}$, $1164^{24}$, $1169^{24}$, $1172^{24}$, $1187^{24}$, $1190^{24}$, $1193^{24}$, $1199^{24}$, $1201^{24}$, $1204^{24}$, $1206^{24}$, $1213^{24}$, $1217^{24}$, $1219^{24}$, $1222^{24}$, $1227^{24}$, $1231^{24}$, $1247^{24}$, $1255^{24}$, $1266^{24}$, $1267^{24}$, $1270^{24}$, $1271^{24}$, $1272^{24}$, $1285^{24}$, $1290^{24}$, $1292^{24}$, $1302^{24}$, $1305^{24}$, $1325^{24}$, $1326^{24}$, $1335^{24}$, $1342^{24}$, $1343^{24}$, $1348^{24}$, $1354^{24}$, $1364^{24}$, $1365^{24}$, $1367^{24}$, $1378^{24}$, $1384^{24}$, $1429^{24}$, $1432^{24}$, $1443^{24}$, $1446^{24}$, $1452^{24}$, $1459^{24}$, $1463^{24}$, $1466^{24}$, $1470^{48}$, $1472^{48}$, $1473^{24}$, $1475^{24}$, $1476^{24}$, $1480^{24}$, $1481^{24}$, $1491^{24}$, $1493^{48}$, $1502^{24}$, $1504^{24}$, $1516^{24}$, $1527^{24}$, $1536^{24}$, $1561^{24}$, $1564^{24}$, $1567^{24}$, $1583^{24}$, $1584^{6}$, $1587^{24}$, $1590^{24}$, $1596^{48}$, $1599^{24}$, $1611^{24}$, $1614^{24}$, $1646^{24}$, $1662^{72}$, $1667^{24}$, $1672^{12}$, $1673^{24}$, $1678^{48}$, $1683^{24}$, $1685^{24}$, $1692^{24}$, $1697^{24}$, $1699^{24}$, $1705^{24}$, $1708^{24}$, $1721^{24}$, $1722^{24}$, $1727^{24}$, $1735^{24}$, $1741^{24}$, $1749^{24}$, $1751^{24}$, $1756^{24}$, $1759^{24}$, $1775^{24}$, $1781^{24}$, $1783^{24}$, $1791^{24}$, $1802^{24}$, $1816^{24}$, $1817^{24}$, $1818^{24}$, $1824^{8}$, $1825^{24}$, $1830^{24}$, $1834^{24}$, $1851^{24}$, $1857^{48}$, $1863^{24}$, $1865^{24}$, $1866^{24}$, $1872^{24}$, $1875^{48}$, $1877^{24}$, $1879^{24}$, $1887^{24}$, $1889^{24}$, $1894^{24}$, $1917^{24}$, $1929^{48}$, $1931^{24}$, $1932^{12}$, $1937^{24}$, $1938^{48}$, $1939^{24}$, $1942^{48}$, $1962^{24}$, $1963^{24}$, $1965^{24}$, $1969^{24}$, $1975^{24}$, $1980^{24}$, $1981^{24}$, $1991^{48}$, $1997^{24}$, $2001^{48}$, $2015^{24}$, $2017^{24}$, $2023^{24}$, $2024^{24}$, $2027^{24}$, $2029^{24}$, $2031^{24}$, $2037^{24}$, $2039^{24}$, $2040^{24}$, $2045^{24}$, $2061^{48}$, $2063^{24}$, $2064^{24}$, $2065^{24}$, $2067^{24}$, $2068^{12}$, $2071^{24}$, $2077^{24}$, $2079^{24}$, $2080^{24}$, $2083^{48}$, $2085^{24}$, $2103^{24}$
\item $(h,k)=(1,17)$ : $1^{256}$, $3^{248}$, $4^{96}$, $5^{168}$, $6^{56}$, $7^{72}$, $8^{60}$, $9^{104}$, $10^{24}$, $11^{60}$, $12^{168}$, $13^{72}$, $14^{48}$, $15^{48}$, $16^{12}$, $17^{48}$, $18^{64}$, $19^{156}$, $20^{96}$, $21^{56}$, $24^{56}$, $25^{72}$, $27^{56}$, $28^{48}$, $30^{32}$, $32^{24}$, $33^{16}$, $34^{24}$, $35^{24}$, $37^{24}$, $38^{24}$, $39^{88}$, $40^{24}$, $42^{24}$, $43^{24}$, $47^{48}$, $48^{40}$, $51^{88}$, $52^{24}$, $53^{48}$, $54^{48}$, $55^{48}$, $57^{8}$, $59^{72}$, $60^{24}$, $62^{24}$, $63^{24}$, $65^{24}$, $73^{24}$, $75^{16}$, $77^{24}$, $79^{24}$, $81^{16}$, $82^{24}$, $85^{24}$, $86^{24}$, $87^{16}$, $88^{48}$, $89^{24}$, $92^{48}$, $93^{16}$, $94^{48}$, $95^{24}$, $96^{8}$, $97^{24}$, $98^{24}$, $102^{88}$, $104^{24}$, $108^{40}$, $110^{24}$, $111^{16}$, $113^{48}$, $114^{32}$, $115^{24}$, $117^{96}$, $119^{48}$, $120^{24}$, $123^{16}$, $124^{48}$, $127^{96}$, $128^{24}$, $131^{24}$, $134^{48}$, $135^{16}$, $137^{24}$, $139^{24}$, $140^{48}$, $141^{72}$, $145^{24}$, $146^{24}$, $147^{48}$, $149^{24}$, $150^{16}$, $152^{24}$, $159^{24}$, $164^{24}$, $167^{24}$, $170^{24}$, $171^{16}$, $173^{24}$, $177^{16}$, $180^{16}$, $183^{16}$, $185^{24}$, $186^{32}$, $190^{72}$, $195^{16}$, $196^{24}$, $200^{24}$, $201^{24}$, $202^{48}$, $204^{48}$, $206^{24}$, $207^{24}$, $208^{72}$, $210^{24}$, $215^{24}$, $218^{24}$, $222^{40}$, $227^{24}$, $229^{24}$, $231^{24}$, $234^{32}$, $249^{40}$, $251^{48}$, $255^{16}$, $257^{24}$, $258^{24}$, $259^{24}$, $260^{48}$, $263^{48}$, $264^{24}$, $267^{40}$, $270^{24}$, $271^{24}$, $272^{24}$, $288^{8}$, $289^{24}$, $290^{24}$, $297^{56}$, $300^{16}$, $302^{96}$, $306^{32}$, $310^{48}$, $312^{24}$, $313^{24}$, $315^{16}$, $334^{24}$, $336^{24}$, $339^{24}$, $349^{24}$, $353^{24}$, $358^{24}$, $368^{72}$, $373^{24}$, $381^{24}$, $386^{24}$, $395^{24}$, $396^{16}$, $398^{48}$, $399^{16}$, $426^{24}$, $429^{16}$, $437^{24}$, $438^{24}$, $469^{24}$, $490^{24}$, $496^{24}$, $515^{24}$, $522^{48}$, $525^{24}$, $543^{24}$, $555^{24}$, $573^{32}$, $576^{16}$, $588^{16}$, $607^{24}$, $616^{24}$, $633^{16}$, $638^{24}$, $654^{32}$, $684^{24}$, $703^{24}$, $705^{24}$, $715^{24}$, $762^{24}$, $786^{16}$, $793^{24}$, $810^{16}$, $815^{48}$, $837^{16}$, $845^{24}$, $878^{48}$, $888^{16}$, $939^{24}$, $951^{24}$, $987^{24}$, $999^{16}$, $1077^{48}$, $1089^{16}$, $1091^{24}$, $1146^{16}$, $1317^{48}$, $1320^{16}$, $1407^{16}$, $1496^{24}$, $1515^{64}$, $1584^{16}$, $2001^{16}$, $2157^{48}$, $2250^{16}$, $2439^{16}$, $2871^{48}$, $3033^{16}$, $3075^{16}$, $3327^{16}$, $3473^{48}$, $3604^{48}$, $3696^{16}$, $4848^{48}$, $4893^{48}$, $5415^{16}$, $6261^{48}$, $6435^{16}$, $6689^{48}$, $6894^{48}$, $7384^{48}$, $8169^{16}$, $8958^{16}$, $10313^{48}$, $10419^{16}$, $10508^{48}$, $10548^{48}$, $10765^{48}$, $10803^{16}$, $11373^{16}$, $12378^{16}$, $12585^{16}$, $12996^{16}$, $15549^{16}$, $16575^{16}$, $16674^{16}$, $16998^{16}$, $17658^{16}$, $19401^{16}$, $20010^{16}$, $20094^{16}$, $20913^{16}$, $21174^{16}$, $24951^{16}$, $25905^{16}$, $28659^{16}$, $28902^{16}$, $28947^{16}$, $29775^{16}$, $31107^{16}$, $31881^{16}$, $32592^{16}$, $32637^{16}$, $32679^{16}$, $33048^{16}$
\item $(h,k)=(1,19)$ : $1^{64}$, $3^{28}$, $4^{36}$, $9^{24}$, $12^{32}$, $13^{60}$, $14^{24}$, $16^{78}$, $22^{24}$, $24^{46}$, $28^{12}$, $29^{12}$, $31^{12}$, $32^{12}$, $35^{12}$, $37^{48}$, $39^{12}$, $44^{12}$, $52^{12}$, $64^{24}$, $83^{48}$, $90^{24}$, $119^{24}$, $120^{12}$, $125^{24}$, $189^{24}$, $201^{24}$, $204^{12}$, $205^{24}$, $232^{24}$, $234^{24}$, $239^{24}$, $240^{24}$, $244^{24}$, $281^{24}$, $302^{24}$, $305^{24}$, $345^{24}$, $354^{24}$, $380^{24}$, $392^{12}$, $440^{12}$, $478^{24}$, $503^{24}$, $525^{24}$, $530^{48}$, $538^{24}$, $564^{24}$, $567^{24}$, $597^{24}$, $613^{24}$, $638^{48}$, $728^{24}$, $729^{24}$, $743^{24}$, $786^{24}$, $844^{12}$, $873^{24}$, $883^{24}$, $915^{24}$, $964^{36}$, $1002^{24}$, $1012^{24}$, $1023^{24}$, $1069^{24}$, $1092^{12}$, $1108^{24}$, $1133^{24}$, $1160^{24}$, $1190^{24}$, $1284^{24}$, $1448^{24}$, $1467^{24}$, $1480^{24}$, $1494^{8}$, $1499^{24}$, $1595^{24}$, $1603^{24}$, $1639^{24}$, $1680^{24}$, $1776^{24}$, $1829^{24}$, $1837^{24}$, $1880^{24}$, $1908^{24}$, $1945^{24}$, $1964^{24}$, $1996^{24}$, $2074^{24}$, $2081^{24}$, $2092^{24}$, $2149^{24}$, $2156^{24}$, $2187^{24}$, $2232^{12}$, $2270^{24}$, $2281^{24}$, $2336^{24}$, $2373^{24}$, $2382^{24}$, $2417^{24}$, $2435^{24}$, $2508^{8}$, $2693^{48}$, $2719^{24}$, $2725^{24}$, $2736^{24}$, $2752^{12}$, $2785^{24}$, $2837^{24}$, $2959^{24}$, $2970^{24}$, $2982^{24}$, $2992^{24}$, $3037^{24}$, $3067^{24}$, $3093^{24}$, $3121^{24}$, $3210^{24}$, $3236^{24}$, $3265^{24}$, $3268^{24}$, $3421^{24}$, $3469^{24}$, $3483^{24}$, $3493^{24}$, $3504^{24}$, $3527^{24}$, $3575^{24}$, $3615^{24}$, $3665^{24}$, $3741^{24}$, $3849^{24}$, $3851^{24}$, $3853^{24}$, $3955^{24}$, $3973^{24}$, $4088^{24}$, $4197^{24}$, $4255^{24}$, $4330^{24}$, $4397^{24}$, $4413^{24}$, $4414^{24}$, $4481^{24}$, $4527^{24}$, $4552^{24}$, $4630^{24}$, $4642^{24}$, $4644^{24}$, $4698^{24}$, $4730^{24}$, $4795^{24}$, $4904^{24}$, $4921^{24}$, $4976^{24}$, $5083^{24}$, $5140^{24}$, $5316^{24}$, $5336^{24}$, $5378^{24}$, $5393^{24}$, $5432^{24}$, $5484^{24}$, $5522^{24}$, $5673^{24}$, $5763^{24}$, $5809^{24}$, $5870^{24}$, $6132^{24}$, $6341^{24}$, $6363^{24}$, $6406^{24}$, $6478^{24}$, $6603^{24}$, $6620^{24}$, $6690^{24}$, $6708^{24}$, $7048^{24}$, $7091^{24}$, $7395^{24}$, $7481^{24}$, $7549^{24}$, $7759^{24}$, $7870^{24}$, $8169^{24}$, $8443^{24}$, $8466^{24}$, $8747^{24}$, $8842^{24}$, $8884^{24}$, $8932^{24}$, $9168^{24}$, $9328^{12}$, $9519^{24}$, $10158^{24}$, $10188^{24}$, $10834^{24}$, $11488^{6}$, $11494^{24}$, $11896^{24}$, $12668^{24}$, $13188^{24}$, $15493^{24}$, $19594^{24}$, $26449^{24}$
\item $(h,k)=(1,23)$ : $1^{4}$, $3^{5592404}$
\item $(h,k)=(5,7)$ : $1^{4}$, $3^{1364}$, $9^{16}$, $24^{4}$, $29^{24}$, $31^{24}$, $53^{24}$, $60^{48}$, $68^{24}$, $72^{6}$, $102^{24}$, $148^{24}$, $192^{24}$, $224^{12}$, $233^{24}$, $248^{24}$, $253^{24}$, $258^{24}$, $283^{24}$, $288^{24}$, $309^{24}$, $319^{24}$, $382^{24}$, $386^{24}$, $406^{24}$, $415^{24}$, $431^{24}$, $464^{24}$, $469^{24}$, $475^{24}$, $498^{24}$, $542^{24}$, $546^{24}$, $553^{24}$, $580^{24}$, $584^{24}$, $607^{24}$, $618^{24}$, $619^{24}$, $658^{24}$, $737^{24}$, $820^{24}$, $837^{24}$, $847^{24}$, $862^{24}$, $868^{24}$, $876^{24}$, $881^{24}$, $961^{24}$, $1013^{24}$, $1071^{24}$, $1091^{48}$, $1107^{24}$, $1139^{24}$, $1195^{24}$, $1199^{24}$, $1258^{24}$, $1289^{24}$, $1304^{24}$, $1353^{24}$, $1375^{24}$, $1392^{24}$, $1413^{24}$, $1441^{24}$, $1508^{24}$, $1583^{24}$, $1589^{24}$, $1786^{24}$, $1791^{24}$, $1848^{24}$, $1857^{24}$, $1867^{24}$, $1915^{24}$, $1966^{24}$, $1976^{24}$, $2001^{24}$, $2025^{24}$, $2071^{24}$, $2129^{24}$, $2148^{4}$, $2475^{24}$, $2539^{24}$, $2555^{24}$, $2683^{24}$, $2752^{24}$, $2761^{24}$, $2806^{24}$, $2854^{24}$, $2925^{24}$, $2953^{24}$, $3020^{24}$, $3118^{24}$, $3140^{24}$, $3266^{24}$, $3385^{24}$, $3397^{24}$, $3408^{24}$, $3446^{24}$, $3513^{24}$, $3575^{24}$, $3629^{24}$, $3635^{24}$, $3664^{24}$, $3727^{24}$, $3745^{24}$, $3758^{24}$, $3803^{24}$, $3875^{24}$, $3882^{24}$, $3905^{24}$, $3922^{24}$, $3958^{24}$, $4259^{24}$, $4287^{24}$, $4297^{24}$, $4355^{24}$, $4432^{24}$, $4719^{24}$, $4765^{24}$, $4805^{24}$, $4870^{24}$, $4941^{24}$, $5149^{24}$, $5158^{24}$, $5299^{24}$, $5317^{24}$, $5371^{24}$, $5399^{24}$, $5587^{24}$, $5594^{24}$, $5717^{24}$, $5806^{24}$, $6026^{24}$, $6069^{24}$, $6177^{24}$, $6226^{24}$, $6236^{24}$, $6246^{48}$, $6447^{24}$, $6455^{24}$, $6554^{24}$, $6556^{24}$, $6568^{24}$, $6594^{24}$, $6610^{24}$, $6746^{24}$, $6756^{24}$, $7073^{24}$, $7075^{24}$, $7186^{24}$, $7370^{24}$, $7445^{24}$, $7776^{24}$, $7784^{24}$, $8039^{24}$, $8098^{24}$, $8390^{24}$, $8691^{24}$, $8922^{24}$, $9029^{24}$, $9074^{24}$, $9161^{24}$, $9220^{24}$, $10157^{24}$, $10200^{24}$, $10603^{24}$, $10961^{24}$, $11187^{24}$, $11395^{24}$, $11547^{24}$, $13733^{24}$, $15858^{24}$, $16158^{24}$, $16397^{24}$, $20002^{24}$, $33639^{24}$
\item $(h,k)=(5,11)$ : $1^{64}$, $3^{348}$, $4^{60}$, $7^{24}$, $8^{18}$, $9^{8}$, $12^{72}$, $13^{12}$, $14^{24}$, $16^{54}$, $17^{48}$, $18^{48}$, $24^{42}$, $28^{24}$, $29^{12}$, $30^{24}$, $31^{36}$, $32^{12}$, $35^{12}$, $39^{36}$, $40^{12}$, $48^{24}$, $52^{12}$, $64^{36}$, $68^{12}$, $69^{24}$, $82^{72}$, $124^{12}$, $150^{24}$, $160^{12}$, $172^{12}$, $192^{4}$, $204^{24}$, $216^{2}$, $252^{12}$, $274^{24}$, $276^{24}$, $288^{2}$, $378^{16}$, $381^{24}$, $422^{48}$, $426^{24}$, $428^{12}$, $437^{48}$, $440^{24}$, $486^{24}$, $493^{24}$, $502^{24}$, $559^{24}$, $562^{24}$, $578^{24}$, $608^{24}$, $616^{24}$, $631^{24}$, $636^{12}$, $644^{12}$, $658^{48}$, $668^{24}$, $678^{24}$, $720^{2}$, $745^{24}$, $788^{12}$, $793^{24}$, $808^{6}$, $914^{24}$, $1040^{6}$, $1081^{24}$, $1087^{24}$, $1094^{24}$, $1184^{24}$, $1186^{24}$, $1193^{24}$, $1196^{12}$, $1233^{24}$, $1290^{24}$, $1344^{24}$, $1348^{24}$, $1370^{24}$, $1371^{24}$, $1375^{24}$, $1384^{24}$, $1420^{24}$, $1424^{24}$, $1434^{48}$, $1455^{24}$, $1464^{24}$, $1510^{24}$, $1626^{24}$, $1648^{24}$, $1655^{24}$, $1664^{12}$, $1714^{24}$, $1731^{24}$, $1748^{24}$, $1773^{24}$, $1784^{12}$, $1785^{24}$, $1813^{24}$, $1831^{24}$, $1848^{24}$, $1877^{24}$, $1912^{48}$, $2036^{24}$, $2074^{24}$, $2083^{24}$, $2175^{24}$, $2196^{24}$, $2200^{24}$, $2269^{24}$, $2348^{24}$, $2405^{24}$, $2422^{24}$, $2441^{24}$, $2521^{24}$, $2645^{24}$, $2703^{24}$, $2752^{24}$, $2783^{24}$, $2795^{24}$, $2858^{24}$, $2866^{24}$, $2879^{24}$, $2927^{24}$, $2941^{24}$, $2953^{24}$, $2959^{24}$, $2963^{24}$, $2969^{24}$, $3002^{24}$, $3003^{24}$, $3048^{24}$, $3084^{12}$, $3129^{24}$, $3174^{24}$, $3204^{12}$, $3269^{24}$, $3298^{24}$, $3331^{24}$, $3362^{24}$, $3372^{24}$, $3444^{24}$, $3481^{24}$, $3494^{24}$, $3536^{24}$, $3542^{24}$, $3586^{24}$, $3600^{24}$, $3651^{24}$, $3661^{24}$, $3730^{24}$, $3818^{24}$, $3845^{24}$, $3943^{24}$, $3996^{24}$, $4106^{24}$, $4232^{24}$, $4340^{24}$, $4518^{8}$, $4651^{24}$, $4677^{24}$, $4687^{24}$, $4752^{24}$, $4794^{24}$, $4846^{24}$, $4859^{24}$, $4865^{24}$, $4928^{24}$, $4989^{24}$, $5055^{24}$, $5186^{24}$, $5278^{24}$, $5281^{24}$, $5430^{24}$, $5432^{24}$, $5436^{24}$, $5508^{24}$, $5582^{24}$, $5590^{24}$, $5663^{24}$, $5799^{48}$, $5835^{24}$, $5866^{24}$, $5880^{24}$, $6089^{24}$, $6100^{24}$, $6107^{24}$, $6188^{24}$, $6208^{24}$, $6398^{24}$, $6629^{24}$, $6812^{24}$, $6838^{24}$, $7021^{24}$, $7204^{24}$, $7255^{24}$, $7302^{48}$, $7639^{24}$, $8000^{24}$, $8182^{24}$, $8568^{24}$, $8787^{24}$, $8926^{24}$, $8986^{24}$, $9061^{24}$, $9257^{24}$, $9335^{24}$, $9395^{24}$, $9609^{24}$, $10037^{24}$, $10101^{24}$, $10620^{24}$, $11165^{24}$, $13080^{6}$, $14769^{24}$, $15162^{24}$, $16976^{24}$, $19524^{12}$, $21146^{24}$
\item $(h,k)=(5,13)$ : $1^{256}$, $2^{360}$, $3^{200}$, $5^{120}$, $6^{72}$, $7^{24}$, $8^{12}$, $9^{112}$, $11^{180}$, $12^{64}$, $13^{48}$, $15^{168}$, $16^{84}$, $17^{24}$, $18^{168}$, $19^{12}$, $21^{80}$, $22^{24}$, $23^{24}$, $24^{40}$, $25^{48}$, $26^{24}$, $27^{40}$, $29^{24}$, $33^{56}$, $35^{96}$, $38^{48}$, $39^{24}$, $40^{24}$, $41^{120}$, $42^{32}$, $43^{48}$, $48^{16}$, $50^{24}$, $51^{16}$, $52^{96}$, $54^{24}$, $55^{48}$, $57^{40}$, $61^{24}$, $63^{80}$, $65^{24}$, $71^{24}$, $72^{40}$, $76^{24}$, $79^{48}$, $80^{24}$, $81^{24}$, $82^{48}$, $85^{48}$, $87^{24}$, $89^{72}$, $92^{24}$, $94^{24}$, $96^{24}$, $102^{48}$, $103^{24}$, $107^{24}$, $108^{24}$, $110^{24}$, $111^{72}$, $113^{48}$, $114^{56}$, $115^{24}$, $116^{24}$, $118^{24}$, $120^{40}$, $122^{48}$, $123^{40}$, $126^{16}$, $130^{24}$, $132^{48}$, $133^{24}$, $141^{40}$, $147^{16}$, $148^{48}$, $149^{24}$, $152^{24}$, $153^{16}$, $159^{48}$, $160^{24}$, $164^{48}$, $167^{24}$, $168^{24}$, $171^{24}$, $172^{24}$, $175^{48}$, $179^{24}$, $180^{40}$, $181^{24}$, $182^{48}$, $183^{72}$, $186^{40}$, $192^{16}$, $197^{24}$, $199^{24}$, $200^{24}$, $203^{24}$, $204^{32}$, $205^{24}$, $206^{24}$, $209^{24}$, $210^{24}$, $211^{48}$, $215^{24}$, $217^{48}$, $222^{16}$, $225^{40}$, $226^{72}$, $233^{24}$, $240^{32}$, $241^{24}$, $243^{48}$, $245^{24}$, $246^{24}$, $247^{24}$, $248^{24}$, $251^{24}$, $259^{48}$, $261^{16}$, $262^{48}$, $277^{24}$, $281^{48}$, $297^{24}$, $315^{16}$, $317^{24}$, $318^{24}$, $324^{24}$, $325^{24}$, $329^{24}$, $330^{24}$, $333^{16}$, $334^{24}$, $336^{40}$, $337^{24}$, $343^{48}$, $353^{24}$, $355^{24}$, $360^{24}$, $374^{24}$, $375^{24}$, $380^{24}$, $381^{24}$, $390^{24}$, $397^{48}$, $398^{72}$, $408^{24}$, $410^{48}$, $414^{16}$, $417^{24}$, $423^{16}$, $425^{24}$, $438^{24}$, $444^{24}$, $446^{24}$, $447^{24}$, $456^{24}$, $471^{24}$, $474^{16}$, $500^{24}$, $511^{24}$, $517^{48}$, $519^{24}$, $534^{24}$, $554^{24}$, $569^{24}$, $576^{16}$, $577^{24}$, $588^{24}$, $672^{48}$, $678^{16}$, $707^{24}$, $726^{32}$, $754^{24}$, $780^{32}$, $786^{16}$, $788^{24}$, $819^{16}$, $849^{16}$, $889^{24}$, $900^{24}$, $906^{48}$, $914^{24}$, $918^{16}$, $927^{48}$, $1056^{16}$, $1118^{48}$, $1143^{16}$, $1188^{16}$, $1260^{16}$, $1293^{24}$, $1369^{24}$, $1471^{48}$, $1477^{48}$, $1482^{16}$, $1731^{48}$, $1735^{24}$, $1740^{16}$, $2115^{16}$, $2121^{48}$, $2283^{48}$, $2286^{16}$, $2575^{48}$, $2649^{16}$, $3038^{48}$, $3222^{16}$, $3318^{16}$, $3483^{16}$, $3684^{48}$, $3921^{16}$, $4185^{16}$, $4404^{16}$, $4686^{16}$, $4743^{16}$, $4914^{48}$, $4947^{16}$, $5122^{48}$, $5367^{16}$, $5904^{16}$, $6618^{16}$, $6793^{48}$, $6909^{16}$, $7626^{16}$, $7873^{48}$, $7963^{48}$, $8137^{48}$, $8178^{16}$, $9618^{16}$, $10546^{48}$, $10733^{48}$, $10734^{48}$, $11577^{16}$, $11580^{16}$, $12939^{16}$, $15234^{16}$, $18975^{16}$, $20118^{16}$, $20589^{16}$, $21816^{16}$, $23736^{16}$, $27375^{16}$, $27414^{16}$, $27708^{16}$, $28011^{16}$, $28287^{16}$, $28557^{16}$, $29961^{16}$, $31041^{16}$, $31752^{16}$, $32085^{16}$, $32139^{16}$, $32157^{16}$, $32343^{16}$, $32949^{16}$
\item $(h,k)=(5,17)$ : $1^{4096}$, $3^{1680}$, $4^{1772}$, $5^{480}$, $6^{1152}$, $7^{584}$, $8^{1214}$, $9^{336}$, $10^{984}$, $11^{248}$, $12^{648}$, $13^{288}$, $14^{696}$, $15^{448}$, $16^{540}$, $17^{672}$, $18^{392}$, $19^{192}$, $20^{492}$, $21^{240}$, $22^{360}$, $23^{192}$, $24^{400}$, $25^{96}$, $26^{408}$, $27^{96}$, $28^{156}$, $29^{192}$, $30^{72}$, $31^{192}$, $32^{276}$, $33^{168}$, $34^{216}$, $35^{144}$, $36^{312}$, $37^{96}$, $38^{192}$, $39^{144}$, $40^{372}$, $41^{192}$, $42^{192}$, $43^{192}$, $44^{264}$, $45^{144}$, $46^{360}$, $47^{96}$, $48^{360}$, $49^{48}$, $50^{192}$, $52^{48}$, $53^{144}$, $54^{120}$, $55^{48}$, $56^{48}$, $57^{48}$, $58^{168}$, $60^{180}$, $61^{96}$, $62^{24}$, $63^{24}$, $64^{72}$, $65^{48}$, $66^{200}$, $67^{48}$, $68^{120}$, $69^{144}$, $70^{144}$, $71^{192}$, $72^{240}$, $73^{48}$, $74^{120}$, $75^{48}$, $76^{120}$, $77^{96}$, $78^{48}$, $80^{72}$, $82^{192}$, $83^{24}$, $84^{136}$, $86^{96}$, $87^{48}$, $88^{192}$, $89^{48}$, $90^{24}$, $92^{144}$, $93^{48}$, $94^{48}$, $95^{48}$, $96^{100}$, $97^{48}$, $98^{24}$, $102^{48}$, $103^{96}$, $104^{72}$, $106^{96}$, $107^{48}$, $108^{96}$, $110^{120}$, $112^{72}$, $113^{96}$, $114^{48}$, $116^{168}$, $117^{48}$, $118^{192}$, $122^{48}$, $123^{16}$, $126^{24}$, $127^{72}$, $128^{96}$, $130^{96}$, $136^{72}$, $137^{48}$, $140^{36}$, $141^{48}$, $142^{24}$, $144^{48}$, $146^{24}$, $147^{48}$, $148^{24}$, $149^{48}$, $150^{24}$, $152^{48}$, $154^{48}$, $157^{48}$, $158^{72}$, $159^{48}$, $160^{96}$, $162^{144}$, $164^{96}$, $165^{96}$, $166^{48}$, $167^{48}$, $168^{96}$, $169^{48}$, $170^{48}$, $171^{96}$, $172^{96}$, $173^{24}$, $176^{144}$, $178^{24}$, $179^{72}$, $181^{48}$, $182^{48}$, $186^{96}$, $189^{48}$, $190^{96}$, $191^{48}$, $192^{72}$, $193^{48}$, $194^{96}$, $196^{24}$, $197^{72}$, $200^{48}$, $202^{24}$, $203^{48}$, $205^{48}$, $206^{48}$, $208^{48}$, $210^{72}$, $212^{24}$, $214^{48}$, $215^{96}$, $216^{30}$, $218^{24}$, $219^{48}$, $221^{48}$, $224^{48}$, $225^{48}$, $226^{48}$, $230^{24}$, $232^{48}$, $235^{24}$, $236^{48}$, $238^{48}$, $239^{48}$, $240^{96}$, $242^{96}$, $243^{96}$, $244^{48}$, $248^{24}$, $249^{72}$, $250^{48}$, $251^{48}$, $252^{24}$, $258^{96}$, $262^{48}$, $264^{48}$, $266^{48}$, $268^{24}$, $270^{24}$, $271^{96}$, $279^{48}$, $280^{96}$, $282^{24}$, $283^{24}$, $284^{144}$, $288^{32}$, $292^{48}$, $293^{48}$, $294^{24}$, $297^{48}$, $300^{96}$, $301^{48}$, $302^{24}$, $303^{24}$, $305^{48}$, $306^{72}$, $308^{48}$, $310^{72}$, $311^{24}$, $313^{24}$, $314^{72}$, $315^{48}$, $316^{48}$, $318^{24}$, $323^{48}$, $324^{72}$, $330^{72}$, $333^{48}$, $334^{24}$, $339^{48}$, $340^{24}$, $342^{24}$, $344^{24}$, $347^{24}$, $351^{64}$, $352^{36}$, $354^{48}$, $355^{48}$, $356^{24}$, $357^{24}$, $358^{48}$, $359^{72}$, $360^{86}$, $363^{48}$, $365^{144}$, $366^{24}$, $367^{48}$, $369^{24}$, $373^{48}$, $375^{24}$, $377^{72}$, $380^{96}$, $382^{24}$, $385^{24}$, $386^{72}$, $388^{24}$, $392^{72}$, $393^{48}$, $394^{24}$, $396^{8}$, $398^{24}$, $406^{24}$, $411^{48}$, $413^{24}$, $414^{24}$, $418^{96}$, $421^{96}$, $424^{72}$, $428^{24}$, $432^{24}$, $433^{48}$, $435^{48}$, $439^{48}$, $440^{24}$, $442^{72}$, $445^{24}$, $453^{72}$, $455^{48}$, $456^{24}$, $461^{24}$, $466^{48}$, $468^{48}$, $469^{48}$, $472^{24}$, $473^{48}$, $475^{24}$, $476^{48}$, $477^{24}$, $479^{48}$, $481^{48}$, $482^{72}$, $484^{48}$, $485^{72}$, $494^{48}$, $497^{48}$, $498^{24}$, $500^{24}$, $502^{48}$, $504^{48}$, $507^{48}$, $508^{24}$, $509^{24}$, $510^{24}$, $516^{24}$, $520^{120}$, $521^{48}$, $522^{24}$, $523^{24}$, $530^{24}$, $531^{48}$, $534^{24}$, $535^{24}$, $538^{48}$, $540^{24}$, $542^{48}$, $548^{48}$, $549^{48}$, $550^{120}$, $556^{48}$, $560^{24}$, $562^{24}$, $563^{48}$, $564^{16}$, $566^{24}$, $568^{48}$, $572^{24}$, $573^{48}$, $576^{60}$, $588^{96}$, $592^{48}$, $594^{24}$, $600^{50}$, $602^{24}$, $603^{96}$, $605^{48}$, $606^{24}$, $613^{48}$, $622^{72}$, $624^{24}$, $625^{96}$, $631^{72}$, $632^{48}$, $638^{48}$, $643^{24}$, $648^{24}$, $650^{24}$, $653^{48}$, $654^{24}$, $656^{24}$, $657^{64}$, $661^{24}$, $664^{48}$, $666^{24}$, $671^{48}$, $672^{48}$, $674^{72}$, $676^{48}$, $677^{48}$, $679^{24}$, $684^{24}$, $692^{24}$, $697^{72}$, $703^{48}$, $709^{48}$, $711^{48}$, $712^{72}$, $714^{24}$, $724^{48}$, $725^{48}$, $728^{24}$, $729^{48}$, $737^{48}$, $744^{48}$, $746^{48}$, $747^{48}$, $748^{48}$, $764^{72}$, $766^{48}$, $767^{24}$, $770^{24}$, $772^{24}$, $774^{48}$, $775^{48}$, $782^{48}$, $791^{72}$, $794^{48}$, $795^{96}$, $800^{24}$, $809^{24}$, $812^{24}$, $815^{48}$, $819^{24}$, $820^{24}$, $821^{48}$, $826^{72}$, $827^{72}$, $830^{48}$, $831^{24}$, $833^{24}$, $841^{24}$, $842^{24}$, $847^{24}$, $848^{24}$, $852^{48}$, $853^{48}$, $854^{48}$, $861^{48}$, $863^{48}$, $866^{24}$, $867^{24}$, $868^{24}$, $870^{24}$, $871^{24}$, $876^{24}$, $877^{48}$, $882^{48}$, $884^{48}$, $888^{24}$, $892^{48}$, $895^{48}$, $901^{48}$, $903^{24}$, $907^{48}$, $912^{72}$, $916^{24}$, $923^{72}$, $928^{24}$, $930^{48}$, $935^{48}$, $938^{48}$, $947^{48}$, $948^{24}$, $963^{48}$, $965^{48}$, $967^{24}$, $968^{72}$, $971^{48}$, $981^{48}$, $982^{96}$, $985^{48}$, $988^{48}$, $989^{72}$, $992^{72}$, $998^{24}$, $1000^{120}$, $1001^{48}$, $1003^{48}$, $1004^{24}$, $1005^{48}$, $1009^{24}$, $1011^{24}$, $1012^{144}$, $1022^{48}$, $1024^{96}$, $1025^{48}$, $1028^{24}$, $1031^{24}$, $1033^{24}$, $1036^{72}$, $1038^{48}$, $1040^{48}$, $1048^{144}$, $1053^{24}$, $1054^{24}$, $1063^{24}$, $1068^{24}$, $1070^{24}$, $1078^{24}$, $1082^{24}$, $1097^{24}$, $1108^{24}$, $1110^{24}$, $1111^{24}$, $1130^{24}$, $1135^{24}$, $1136^{24}$, $1139^{24}$, $1164^{24}$, $1172^{24}$, $1173^{48}$, $1178^{24}$, $1184^{12}$, $1187^{24}$, $1189^{24}$, $1202^{24}$, $1209^{48}$, $1223^{24}$, $1235^{24}$, $1245^{48}$, $1247^{24}$, $1284^{24}$, $1287^{24}$, $1294^{24}$, $1295^{24}$, $1297^{24}$, $1299^{24}$, $1302^{24}$, $1304^{24}$, $1308^{24}$, $1320^{24}$, $1321^{24}$, $1328^{24}$, $1330^{24}$, $1332^{24}$, $1342^{48}$, $1358^{24}$, $1365^{24}$, $1366^{24}$, $1374^{24}$, $1381^{24}$, $1389^{24}$, $1400^{24}$, $1408^{24}$, $1432^{6}$, $1447^{24}$, $1448^{24}$, $1454^{48}$, $1461^{48}$, $1479^{24}$, $1495^{24}$, $1501^{24}$, $1504^{24}$, $1505^{24}$, $1511^{24}$, $1531^{24}$, $1535^{24}$, $1543^{48}$, $1544^{24}$, $1549^{24}$, $1571^{24}$, $1572^{24}$, $1574^{24}$, $1583^{24}$, $1587^{24}$, $1588^{24}$, $1591^{24}$, $1596^{12}$, $1600^{24}$, $1609^{24}$, $1615^{24}$, $1625^{24}$, $1639^{24}$, $1655^{24}$, $1669^{24}$, $1671^{24}$, $1672^{24}$, $1678^{24}$, $1684^{24}$, $1689^{48}$, $1706^{24}$, $1711^{24}$, $1715^{24}$, $1717^{24}$, $1725^{24}$, $1727^{24}$, $1744^{24}$, $1752^{24}$, $1755^{24}$, $1763^{24}$, $1765^{24}$, $1777^{24}$, $1778^{48}$, $1781^{24}$, $1787^{24}$, $1800^{24}$, $1801^{24}$, $1808^{24}$, $1817^{24}$, $1825^{24}$, $1827^{24}$, $1831^{24}$, $1837^{24}$, $1838^{24}$, $1839^{24}$, $1840^{24}$, $1847^{24}$, $1852^{24}$, $1864^{24}$, $1872^{24}$, $1887^{48}$, $1888^{36}$, $1890^{24}$, $1891^{24}$, $1892^{24}$, $1896^{24}$, $1899^{24}$, $1904^{24}$, $1920^{12}$, $1924^{24}$, $1927^{24}$, $1928^{24}$, $1929^{24}$, $1937^{24}$, $1942^{24}$, $1952^{24}$, $1954^{24}$, $1955^{24}$, $1961^{24}$, $1963^{24}$, $1965^{24}$, $1966^{24}$, $1968^{24}$, $1973^{24}$, $1980^{24}$, $1985^{24}$, $1989^{24}$, $1991^{72}$, $1992^{24}$, $1996^{24}$, $1999^{24}$, $2000^{24}$, $2001^{24}$, $2007^{24}$, $2009^{24}$, $2010^{24}$, $2017^{48}$, $2018^{24}$, $2019^{24}$, $2023^{24}$, $2031^{24}$, $2033^{24}$, $2036^{24}$, $2040^{24}$, $2041^{24}$, $2043^{24}$, $2051^{48}$, $2053^{24}$, $2055^{24}$, $2056^{24}$, $2059^{48}$, $2060^{24}$, $2063^{24}$, $2067^{24}$, $2075^{24}$, $2077^{24}$, $2079^{48}$, $2091^{24}$, $2096^{24}$
\item $(h,k)=(5,19)$ : $1^{4}$, $3^{5592404}$
\item $(h,k)=(5,23)$ : $1^{64}$, $3^{28}$, $4^{36}$, $9^{24}$, $12^{32}$, $13^{60}$, $14^{24}$, $16^{78}$, $22^{24}$, $24^{46}$, $28^{12}$, $29^{12}$, $31^{12}$, $32^{12}$, $35^{12}$, $37^{48}$, $39^{12}$, $44^{12}$, $52^{12}$, $64^{24}$, $83^{48}$, $90^{24}$, $119^{24}$, $120^{12}$, $125^{24}$, $189^{24}$, $201^{24}$, $204^{12}$, $205^{24}$, $232^{24}$, $234^{24}$, $239^{24}$, $240^{24}$, $244^{24}$, $281^{24}$, $302^{24}$, $305^{24}$, $345^{24}$, $354^{24}$, $380^{24}$, $392^{12}$, $440^{12}$, $478^{24}$, $503^{24}$, $525^{24}$, $530^{48}$, $538^{24}$, $564^{24}$, $567^{24}$, $597^{24}$, $613^{24}$, $638^{48}$, $728^{24}$, $729^{24}$, $743^{24}$, $786^{24}$, $844^{12}$, $873^{24}$, $883^{24}$, $915^{24}$, $964^{36}$, $1002^{24}$, $1012^{24}$, $1023^{24}$, $1069^{24}$, $1092^{12}$, $1108^{24}$, $1133^{24}$, $1160^{24}$, $1190^{24}$, $1284^{24}$, $1448^{24}$, $1467^{24}$, $1480^{24}$, $1494^{8}$, $1499^{24}$, $1595^{24}$, $1603^{24}$, $1639^{24}$, $1680^{24}$, $1776^{24}$, $1829^{24}$, $1837^{24}$, $1880^{24}$, $1908^{24}$, $1945^{24}$, $1964^{24}$, $1996^{24}$, $2074^{24}$, $2081^{24}$, $2092^{24}$, $2149^{24}$, $2156^{24}$, $2187^{24}$, $2232^{12}$, $2270^{24}$, $2281^{24}$, $2336^{24}$, $2373^{24}$, $2382^{24}$, $2417^{24}$, $2435^{24}$, $2508^{8}$, $2693^{48}$, $2719^{24}$, $2725^{24}$, $2736^{24}$, $2752^{12}$, $2785^{24}$, $2837^{24}$, $2959^{24}$, $2970^{24}$, $2982^{24}$, $2992^{24}$, $3037^{24}$, $3067^{24}$, $3093^{24}$, $3121^{24}$, $3210^{24}$, $3236^{24}$, $3265^{24}$, $3268^{24}$, $3421^{24}$, $3469^{24}$, $3483^{24}$, $3493^{24}$, $3504^{24}$, $3527^{24}$, $3575^{24}$, $3615^{24}$, $3665^{24}$, $3741^{24}$, $3849^{24}$, $3851^{24}$, $3853^{24}$, $3955^{24}$, $3973^{24}$, $4088^{24}$, $4197^{24}$, $4255^{24}$, $4330^{24}$, $4397^{24}$, $4413^{24}$, $4414^{24}$, $4481^{24}$, $4527^{24}$, $4552^{24}$, $4630^{24}$, $4642^{24}$, $4644^{24}$, $4698^{24}$, $4730^{24}$, $4795^{24}$, $4904^{24}$, $4921^{24}$, $4976^{24}$, $5083^{24}$, $5140^{24}$, $5316^{24}$, $5336^{24}$, $5378^{24}$, $5393^{24}$, $5432^{24}$, $5484^{24}$, $5522^{24}$, $5673^{24}$, $5763^{24}$, $5809^{24}$, $5870^{24}$, $6132^{24}$, $6341^{24}$, $6363^{24}$, $6406^{24}$, $6478^{24}$, $6603^{24}$, $6620^{24}$, $6690^{24}$, $6708^{24}$, $7048^{24}$, $7091^{24}$, $7395^{24}$, $7481^{24}$, $7549^{24}$, $7759^{24}$, $7870^{24}$, $8169^{24}$, $8443^{24}$, $8466^{24}$, $8747^{24}$, $8842^{24}$, $8884^{24}$, $8932^{24}$, $9168^{24}$, $9328^{12}$, $9519^{24}$, $10158^{24}$, $10188^{24}$, $10834^{24}$, $11488^{6}$, $11494^{24}$, $11896^{24}$, $12668^{24}$, $13188^{24}$, $15493^{24}$, $19594^{24}$, $26449^{24}$
\item $(h,k)=(7,11)$ : $1^{16}$, $2^{72}$, $3^{40}$, $4^{20}$, $5^{48}$, $6^{16}$, $7^{32}$, $8^{14}$, $9^{48}$, $11^{20}$, $12^{36}$, $16^{12}$, $19^{12}$, $21^{24}$, $27^{8}$, $29^{48}$, $35^{24}$, $43^{24}$, $45^{16}$, $54^{8}$, $57^{8}$, $62^{24}$, $71^{24}$, $80^{24}$, $84^{24}$, $118^{24}$, $156^{24}$, $182^{48}$, $190^{24}$, $206^{24}$, $227^{24}$, $234^{8}$, $254^{48}$, $260^{24}$, $262^{24}$, $267^{16}$, $338^{24}$, $355^{24}$, $361^{24}$, $378^{8}$, $390^{24}$, $409^{24}$, $419^{24}$, $429^{24}$, $445^{24}$, $454^{24}$, $470^{24}$, $479^{24}$, $482^{24}$, $491^{24}$, $548^{24}$, $631^{24}$, $683^{24}$, $767^{24}$, $789^{24}$, $792^{24}$, $831^{24}$, $845^{24}$, $852^{24}$, $871^{24}$, $900^{12}$, $922^{24}$, $972^{8}$, $975^{24}$, $995^{24}$, $1008^{26}$, $1011^{24}$, $1024^{24}$, $1025^{24}$, $1048^{24}$, $1073^{24}$, $1119^{24}$, $1140^{4}$, $1197^{24}$, $1304^{24}$, $1400^{24}$, $1448^{24}$, $1468^{24}$, $1536^{24}$, $1613^{24}$, $1683^{24}$, $1733^{24}$, $1746^{24}$, $1759^{24}$, $1771^{24}$, $1813^{24}$, $1900^{24}$, $1906^{24}$, $1944^{2}$, $1951^{24}$, $1982^{24}$, $1983^{24}$, $2062^{24}$, $2092^{24}$, $2139^{24}$, $2171^{24}$, $2197^{24}$, $2201^{24}$, $2216^{24}$, $2225^{24}$, $2285^{24}$, $2303^{24}$, $2356^{24}$, $2385^{24}$, $2408^{24}$, $2474^{24}$, $2538^{24}$, $2587^{24}$, $2620^{24}$, $2637^{24}$, $2691^{24}$, $2694^{24}$, $2792^{24}$, $2844^{24}$, $2976^{24}$, $3003^{24}$, $3073^{24}$, $3169^{24}$, $3174^{24}$, $3209^{24}$, $3263^{24}$, $3289^{24}$, $3296^{24}$, $3497^{24}$, $3507^{24}$, $3529^{24}$, $3534^{24}$, $3598^{24}$, $3652^{24}$, $3737^{24}$, $3810^{24}$, $3877^{24}$, $3941^{24}$, $3947^{24}$, $4002^{24}$, $4010^{48}$, $4205^{24}$, $4440^{24}$, $4630^{24}$, $4660^{24}$, $4691^{24}$, $4786^{24}$, $4819^{24}$, $4832^{24}$, $5084^{24}$, $5100^{24}$, $5230^{24}$, $5594^{24}$, $5650^{24}$, $5880^{24}$, $5994^{24}$, $6136^{24}$, $6164^{24}$, $6332^{24}$, $6397^{24}$, $6430^{24}$, $6438^{24}$, $6470^{24}$, $6653^{24}$, $6667^{24}$, $6721^{24}$, $6785^{24}$, $6787^{24}$, $7476^{24}$, $7632^{24}$, $7666^{24}$, $7954^{24}$, $7992^{8}$, $8054^{24}$, $8117^{24}$, $8148^{24}$, $8244^{24}$, $8283^{24}$, $8325^{24}$, $8409^{24}$, $8607^{24}$, $8753^{24}$, $8808^{24}$, $8857^{24}$, $9116^{24}$, $9452^{24}$, $9481^{24}$, $9828^{24}$, $9855^{24}$, $10362^{24}$, $10400^{24}$, $10582^{24}$, $10928^{24}$, $11107^{24}$, $11227^{24}$, $11794^{24}$, $12009^{24}$, $12531^{24}$, $12809^{24}$, $12868^{24}$, $14533^{24}$, $15007^{24}$, $15035^{24}$, $25476^{8}$
\item $(h,k)=(7,13)$ : $1^{64}$, $3^{124}$, $4^{60}$, $6^{48}$, $8^{48}$, $9^{24}$, $10^{48}$, $12^{48}$, $13^{12}$, $14^{24}$, $16^{96}$, $22^{96}$, $24^{28}$, $27^{24}$, $28^{36}$, $29^{12}$, $30^{48}$, $31^{12}$, $32^{12}$, $35^{12}$, $38^{24}$, $39^{12}$, $52^{48}$, $63^{24}$, $64^{6}$, $66^{24}$, $70^{24}$, $72^{2}$, $73^{48}$, $82^{48}$, $86^{24}$, $98^{24}$, $105^{24}$, $112^{6}$, $116^{36}$, $126^{48}$, $136^{36}$, $144^{14}$, $154^{24}$, $160^{24}$, $262^{24}$, $264^{18}$, $289^{24}$, $299^{24}$, $328^{12}$, $339^{48}$, $342^{24}$, $348^{24}$, $364^{12}$, $381^{24}$, $443^{24}$, $444^{24}$, $475^{24}$, $492^{24}$, $545^{24}$, $559^{24}$, $606^{24}$, $632^{24}$, $635^{24}$, $648^{24}$, $659^{48}$, $713^{24}$, $732^{24}$, $742^{24}$, $764^{24}$, $799^{24}$, $801^{24}$, $817^{24}$, $820^{24}$, $823^{24}$, $837^{24}$, $847^{24}$, $848^{24}$, $884^{24}$, $925^{24}$, $939^{24}$, $952^{24}$, $1133^{24}$, $1135^{48}$, $1148^{24}$, $1188^{12}$, $1285^{24}$, $1309^{24}$, $1312^{24}$, $1327^{24}$, $1348^{24}$, $1361^{24}$, $1399^{24}$, $1409^{24}$, $1471^{24}$, $1532^{24}$, $1547^{24}$, $1550^{24}$, $1555^{24}$, $1566^{24}$, $1569^{24}$, $1618^{24}$, $1621^{24}$, $1626^{24}$, $1628^{24}$, $1635^{24}$, $1644^{24}$, $1657^{24}$, $1689^{48}$, $1736^{24}$, $1774^{24}$, $1778^{24}$, $1792^{24}$, $1818^{24}$, $1846^{24}$, $1886^{24}$, $1987^{24}$, $1999^{24}$, $2087^{24}$, $2089^{24}$, $2102^{24}$, $2134^{24}$, $2200^{24}$, $2249^{24}$, $2296^{24}$, $2342^{24}$, $2511^{24}$, $2512^{24}$, $2554^{24}$, $2571^{24}$, $2631^{24}$, $2682^{24}$, $2707^{24}$, $2723^{24}$, $2769^{24}$, $2799^{24}$, $2800^{12}$, $2852^{24}$, $2853^{24}$, $2858^{24}$, $2910^{24}$, $2925^{24}$, $2933^{24}$, $2954^{24}$, $3164^{12}$, $3183^{24}$, $3188^{24}$, $3194^{24}$, $3272^{24}$, $3323^{24}$, $3418^{24}$, $3441^{24}$, $3457^{24}$, $3468^{24}$, $3476^{24}$, $3501^{24}$, $3788^{24}$, $3911^{24}$, $3956^{48}$, $4069^{24}$, $4177^{48}$, $4262^{24}$, $4303^{24}$, $4341^{24}$, $4362^{24}$, $4400^{24}$, $4439^{24}$, $4560^{24}$, $4652^{12}$, $4961^{24}$, $4978^{24}$, $4992^{24}$, $4998^{24}$, $5293^{24}$, $5317^{24}$, $5358^{24}$, $5367^{24}$, $5396^{24}$, $5490^{24}$, $5898^{24}$, $5911^{24}$, $5960^{24}$, $6128^{24}$, $6293^{24}$, $6494^{24}$, $6616^{24}$, $6622^{24}$, $6624^{24}$, $6713^{24}$, $6731^{24}$, $7225^{24}$, $7306^{24}$, $7460^{24}$, $7482^{24}$, $7580^{24}$, $7776^{6}$, $7818^{24}$, $8260^{24}$, $8298^{24}$, $8340^{24}$, $8768^{24}$, $8804^{24}$, $8881^{24}$, $9056^{24}$, $9090^{24}$, $9116^{12}$, $9503^{24}$, $9984^{2}$, $10054^{24}$, $10266^{24}$, $10267^{24}$, $10295^{24}$, $10546^{24}$, $10848^{24}$, $11781^{24}$, $13081^{24}$, $13563^{24}$, $13954^{24}$, $14413^{24}$, $15666^{24}$, $15872^{24}$, $16709^{24}$, $19387^{24}$
\item $(h,k)=(7,17)$ : $1^{4}$, $3^{5592404}$
\item $(h,k)=(7,19)$ : $1^{4096}$, $3^{1680}$, $4^{1772}$, $5^{480}$, $6^{1152}$, $7^{584}$, $8^{1214}$, $9^{336}$, $10^{984}$, $11^{248}$, $12^{648}$, $13^{288}$, $14^{696}$, $15^{448}$, $16^{540}$, $17^{672}$, $18^{392}$, $19^{192}$, $20^{492}$, $21^{240}$, $22^{360}$, $23^{192}$, $24^{400}$, $25^{96}$, $26^{408}$, $27^{96}$, $28^{156}$, $29^{192}$, $30^{72}$, $31^{192}$, $32^{276}$, $33^{168}$, $34^{216}$, $35^{144}$, $36^{312}$, $37^{96}$, $38^{192}$, $39^{144}$, $40^{372}$, $41^{192}$, $42^{192}$, $43^{192}$, $44^{264}$, $45^{144}$, $46^{360}$, $47^{96}$, $48^{360}$, $49^{48}$, $50^{192}$, $52^{48}$, $53^{144}$, $54^{120}$, $55^{48}$, $56^{48}$, $57^{48}$, $58^{168}$, $60^{180}$, $61^{96}$, $62^{24}$, $63^{24}$, $64^{72}$, $65^{48}$, $66^{200}$, $67^{48}$, $68^{120}$, $69^{144}$, $70^{144}$, $71^{192}$, $72^{240}$, $73^{48}$, $74^{120}$, $75^{48}$, $76^{120}$, $77^{96}$, $78^{48}$, $80^{72}$, $82^{192}$, $83^{24}$, $84^{136}$, $86^{96}$, $87^{48}$, $88^{192}$, $89^{48}$, $90^{24}$, $92^{144}$, $93^{48}$, $94^{48}$, $95^{48}$, $96^{100}$, $97^{48}$, $98^{24}$, $102^{48}$, $103^{96}$, $104^{72}$, $106^{96}$, $107^{48}$, $108^{96}$, $110^{120}$, $112^{72}$, $113^{96}$, $114^{48}$, $116^{168}$, $117^{48}$, $118^{192}$, $122^{48}$, $123^{16}$, $126^{24}$, $127^{72}$, $128^{96}$, $130^{96}$, $136^{72}$, $137^{48}$, $140^{36}$, $141^{48}$, $142^{24}$, $144^{48}$, $146^{24}$, $147^{48}$, $148^{24}$, $149^{48}$, $150^{24}$, $152^{48}$, $154^{48}$, $157^{48}$, $158^{72}$, $159^{48}$, $160^{96}$, $162^{144}$, $164^{96}$, $165^{96}$, $166^{48}$, $167^{48}$, $168^{96}$, $169^{48}$, $170^{48}$, $171^{96}$, $172^{96}$, $173^{24}$, $176^{144}$, $178^{24}$, $179^{72}$, $181^{48}$, $182^{48}$, $186^{96}$, $189^{48}$, $190^{96}$, $191^{48}$, $192^{72}$, $193^{48}$, $194^{96}$, $196^{24}$, $197^{72}$, $200^{48}$, $202^{24}$, $203^{48}$, $205^{48}$, $206^{48}$, $208^{48}$, $210^{72}$, $212^{24}$, $214^{48}$, $215^{96}$, $216^{30}$, $218^{24}$, $219^{48}$, $221^{48}$, $224^{48}$, $225^{48}$, $226^{48}$, $230^{24}$, $232^{48}$, $235^{24}$, $236^{48}$, $238^{48}$, $239^{48}$, $240^{96}$, $242^{96}$, $243^{96}$, $244^{48}$, $248^{24}$, $249^{72}$, $250^{48}$, $251^{48}$, $252^{24}$, $258^{96}$, $262^{48}$, $264^{48}$, $266^{48}$, $268^{24}$, $270^{24}$, $271^{96}$, $279^{48}$, $280^{96}$, $282^{24}$, $283^{24}$, $284^{144}$, $288^{32}$, $292^{48}$, $293^{48}$, $294^{24}$, $297^{48}$, $300^{96}$, $301^{48}$, $302^{24}$, $303^{24}$, $305^{48}$, $306^{72}$, $308^{48}$, $310^{72}$, $311^{24}$, $313^{24}$, $314^{72}$, $315^{48}$, $316^{48}$, $318^{24}$, $323^{48}$, $324^{72}$, $330^{72}$, $333^{48}$, $334^{24}$, $339^{48}$, $340^{24}$, $342^{24}$, $344^{24}$, $347^{24}$, $351^{64}$, $352^{36}$, $354^{48}$, $355^{48}$, $356^{24}$, $357^{24}$, $358^{48}$, $359^{72}$, $360^{86}$, $363^{48}$, $365^{144}$, $366^{24}$, $367^{48}$, $369^{24}$, $373^{48}$, $375^{24}$, $377^{72}$, $380^{96}$, $382^{24}$, $385^{24}$, $386^{72}$, $388^{24}$, $392^{72}$, $393^{48}$, $394^{24}$, $396^{8}$, $398^{24}$, $406^{24}$, $411^{48}$, $413^{24}$, $414^{24}$, $418^{96}$, $421^{96}$, $424^{72}$, $428^{24}$, $432^{24}$, $433^{48}$, $435^{48}$, $439^{48}$, $440^{24}$, $442^{72}$, $445^{24}$, $453^{72}$, $455^{48}$, $456^{24}$, $461^{24}$, $466^{48}$, $468^{48}$, $469^{48}$, $472^{24}$, $473^{48}$, $475^{24}$, $476^{48}$, $477^{24}$, $479^{48}$, $481^{48}$, $482^{72}$, $484^{48}$, $485^{72}$, $494^{48}$, $497^{48}$, $498^{24}$, $500^{24}$, $502^{48}$, $504^{48}$, $507^{48}$, $508^{24}$, $509^{24}$, $510^{24}$, $516^{24}$, $520^{120}$, $521^{48}$, $522^{24}$, $523^{24}$, $530^{24}$, $531^{48}$, $534^{24}$, $535^{24}$, $538^{48}$, $540^{24}$, $542^{48}$, $548^{48}$, $549^{48}$, $550^{120}$, $556^{48}$, $560^{24}$, $562^{24}$, $563^{48}$, $564^{16}$, $566^{24}$, $568^{48}$, $572^{24}$, $573^{48}$, $576^{60}$, $588^{96}$, $592^{48}$, $594^{24}$, $600^{50}$, $602^{24}$, $603^{96}$, $605^{48}$, $606^{24}$, $613^{48}$, $622^{72}$, $624^{24}$, $625^{96}$, $631^{72}$, $632^{48}$, $638^{48}$, $643^{24}$, $648^{24}$, $650^{24}$, $653^{48}$, $654^{24}$, $656^{24}$, $657^{64}$, $661^{24}$, $664^{48}$, $666^{24}$, $671^{48}$, $672^{48}$, $674^{72}$, $676^{48}$, $677^{48}$, $679^{24}$, $684^{24}$, $692^{24}$, $697^{72}$, $703^{48}$, $709^{48}$, $711^{48}$, $712^{72}$, $714^{24}$, $724^{48}$, $725^{48}$, $728^{24}$, $729^{48}$, $737^{48}$, $744^{48}$, $746^{48}$, $747^{48}$, $748^{48}$, $764^{72}$, $766^{48}$, $767^{24}$, $770^{24}$, $772^{24}$, $774^{48}$, $775^{48}$, $782^{48}$, $791^{72}$, $794^{48}$, $795^{96}$, $800^{24}$, $809^{24}$, $812^{24}$, $815^{48}$, $819^{24}$, $820^{24}$, $821^{48}$, $826^{72}$, $827^{72}$, $830^{48}$, $831^{24}$, $833^{24}$, $841^{24}$, $842^{24}$, $847^{24}$, $848^{24}$, $852^{48}$, $853^{48}$, $854^{48}$, $861^{48}$, $863^{48}$, $866^{24}$, $867^{24}$, $868^{24}$, $870^{24}$, $871^{24}$, $876^{24}$, $877^{48}$, $882^{48}$, $884^{48}$, $888^{24}$, $892^{48}$, $895^{48}$, $901^{48}$, $903^{24}$, $907^{48}$, $912^{72}$, $916^{24}$, $923^{72}$, $928^{24}$, $930^{48}$, $935^{48}$, $938^{48}$, $947^{48}$, $948^{24}$, $963^{48}$, $965^{48}$, $967^{24}$, $968^{72}$, $971^{48}$, $981^{48}$, $982^{96}$, $985^{48}$, $988^{48}$, $989^{72}$, $992^{72}$, $998^{24}$, $1000^{120}$, $1001^{48}$, $1003^{48}$, $1004^{24}$, $1005^{48}$, $1009^{24}$, $1011^{24}$, $1012^{144}$, $1022^{48}$, $1024^{96}$, $1025^{48}$, $1028^{24}$, $1031^{24}$, $1033^{24}$, $1036^{72}$, $1038^{48}$, $1040^{48}$, $1048^{144}$, $1053^{24}$, $1054^{24}$, $1063^{24}$, $1068^{24}$, $1070^{24}$, $1078^{24}$, $1082^{24}$, $1097^{24}$, $1108^{24}$, $1110^{24}$, $1111^{24}$, $1130^{24}$, $1135^{24}$, $1136^{24}$, $1139^{24}$, $1164^{24}$, $1172^{24}$, $1173^{48}$, $1178^{24}$, $1184^{12}$, $1187^{24}$, $1189^{24}$, $1202^{24}$, $1209^{48}$, $1223^{24}$, $1235^{24}$, $1245^{48}$, $1247^{24}$, $1284^{24}$, $1287^{24}$, $1294^{24}$, $1295^{24}$, $1297^{24}$, $1299^{24}$, $1302^{24}$, $1304^{24}$, $1308^{24}$, $1320^{24}$, $1321^{24}$, $1328^{24}$, $1330^{24}$, $1332^{24}$, $1342^{48}$, $1358^{24}$, $1365^{24}$, $1366^{24}$, $1374^{24}$, $1381^{24}$, $1389^{24}$, $1400^{24}$, $1408^{24}$, $1432^{6}$, $1447^{24}$, $1448^{24}$, $1454^{48}$, $1461^{48}$, $1479^{24}$, $1495^{24}$, $1501^{24}$, $1504^{24}$, $1505^{24}$, $1511^{24}$, $1531^{24}$, $1535^{24}$, $1543^{48}$, $1544^{24}$, $1549^{24}$, $1571^{24}$, $1572^{24}$, $1574^{24}$, $1583^{24}$, $1587^{24}$, $1588^{24}$, $1591^{24}$, $1596^{12}$, $1600^{24}$, $1609^{24}$, $1615^{24}$, $1625^{24}$, $1639^{24}$, $1655^{24}$, $1669^{24}$, $1671^{24}$, $1672^{24}$, $1678^{24}$, $1684^{24}$, $1689^{48}$, $1706^{24}$, $1711^{24}$, $1715^{24}$, $1717^{24}$, $1725^{24}$, $1727^{24}$, $1744^{24}$, $1752^{24}$, $1755^{24}$, $1763^{24}$, $1765^{24}$, $1777^{24}$, $1778^{48}$, $1781^{24}$, $1787^{24}$, $1800^{24}$, $1801^{24}$, $1808^{24}$, $1817^{24}$, $1825^{24}$, $1827^{24}$, $1831^{24}$, $1837^{24}$, $1838^{24}$, $1839^{24}$, $1840^{24}$, $1847^{24}$, $1852^{24}$, $1864^{24}$, $1872^{24}$, $1887^{48}$, $1888^{36}$, $1890^{24}$, $1891^{24}$, $1892^{24}$, $1896^{24}$, $1899^{24}$, $1904^{24}$, $1920^{12}$, $1924^{24}$, $1927^{24}$, $1928^{24}$, $1929^{24}$, $1937^{24}$, $1942^{24}$, $1952^{24}$, $1954^{24}$, $1955^{24}$, $1961^{24}$, $1963^{24}$, $1965^{24}$, $1966^{24}$, $1968^{24}$, $1973^{24}$, $1980^{24}$, $1985^{24}$, $1989^{24}$, $1991^{72}$, $1992^{24}$, $1996^{24}$, $1999^{24}$, $2000^{24}$, $2001^{24}$, $2007^{24}$, $2009^{24}$, $2010^{24}$, $2017^{48}$, $2018^{24}$, $2019^{24}$, $2023^{24}$, $2031^{24}$, $2033^{24}$, $2036^{24}$, $2040^{24}$, $2041^{24}$, $2043^{24}$, $2051^{48}$, $2053^{24}$, $2055^{24}$, $2056^{24}$, $2059^{48}$, $2060^{24}$, $2063^{24}$, $2067^{24}$, $2075^{24}$, $2077^{24}$, $2079^{48}$, $2091^{24}$, $2096^{24}$
\item $(h,k)=(7,23)$ : $1^{256}$, $3^{248}$, $4^{96}$, $5^{168}$, $6^{56}$, $7^{72}$, $8^{60}$, $9^{104}$, $10^{24}$, $11^{60}$, $12^{168}$, $13^{72}$, $14^{48}$, $15^{48}$, $16^{12}$, $17^{48}$, $18^{64}$, $19^{156}$, $20^{96}$, $21^{56}$, $24^{56}$, $25^{72}$, $27^{56}$, $28^{48}$, $30^{32}$, $32^{24}$, $33^{16}$, $34^{24}$, $35^{24}$, $37^{24}$, $38^{24}$, $39^{88}$, $40^{24}$, $42^{24}$, $43^{24}$, $47^{48}$, $48^{40}$, $51^{88}$, $52^{24}$, $53^{48}$, $54^{48}$, $55^{48}$, $57^{8}$, $59^{72}$, $60^{24}$, $62^{24}$, $63^{24}$, $65^{24}$, $73^{24}$, $75^{16}$, $77^{24}$, $79^{24}$, $81^{16}$, $82^{24}$, $85^{24}$, $86^{24}$, $87^{16}$, $88^{48}$, $89^{24}$, $92^{48}$, $93^{16}$, $94^{48}$, $95^{24}$, $96^{8}$, $97^{24}$, $98^{24}$, $102^{88}$, $104^{24}$, $108^{40}$, $110^{24}$, $111^{16}$, $113^{48}$, $114^{32}$, $115^{24}$, $117^{96}$, $119^{48}$, $120^{24}$, $123^{16}$, $124^{48}$, $127^{96}$, $128^{24}$, $131^{24}$, $134^{48}$, $135^{16}$, $137^{24}$, $139^{24}$, $140^{48}$, $141^{72}$, $145^{24}$, $146^{24}$, $147^{48}$, $149^{24}$, $150^{16}$, $152^{24}$, $159^{24}$, $164^{24}$, $167^{24}$, $170^{24}$, $171^{16}$, $173^{24}$, $177^{16}$, $180^{16}$, $183^{16}$, $185^{24}$, $186^{32}$, $190^{72}$, $195^{16}$, $196^{24}$, $200^{24}$, $201^{24}$, $202^{48}$, $204^{48}$, $206^{24}$, $207^{24}$, $208^{72}$, $210^{24}$, $215^{24}$, $218^{24}$, $222^{40}$, $227^{24}$, $229^{24}$, $231^{24}$, $234^{32}$, $249^{40}$, $251^{48}$, $255^{16}$, $257^{24}$, $258^{24}$, $259^{24}$, $260^{48}$, $263^{48}$, $264^{24}$, $267^{40}$, $270^{24}$, $271^{24}$, $272^{24}$, $288^{8}$, $289^{24}$, $290^{24}$, $297^{56}$, $300^{16}$, $302^{96}$, $306^{32}$, $310^{48}$, $312^{24}$, $313^{24}$, $315^{16}$, $334^{24}$, $336^{24}$, $339^{24}$, $349^{24}$, $353^{24}$, $358^{24}$, $368^{72}$, $373^{24}$, $381^{24}$, $386^{24}$, $395^{24}$, $396^{16}$, $398^{48}$, $399^{16}$, $426^{24}$, $429^{16}$, $437^{24}$, $438^{24}$, $469^{24}$, $490^{24}$, $496^{24}$, $515^{24}$, $522^{48}$, $525^{24}$, $543^{24}$, $555^{24}$, $573^{32}$, $576^{16}$, $588^{16}$, $607^{24}$, $616^{24}$, $633^{16}$, $638^{24}$, $654^{32}$, $684^{24}$, $703^{24}$, $705^{24}$, $715^{24}$, $762^{24}$, $786^{16}$, $793^{24}$, $810^{16}$, $815^{48}$, $837^{16}$, $845^{24}$, $878^{48}$, $888^{16}$, $939^{24}$, $951^{24}$, $987^{24}$, $999^{16}$, $1077^{48}$, $1089^{16}$, $1091^{24}$, $1146^{16}$, $1317^{48}$, $1320^{16}$, $1407^{16}$, $1496^{24}$, $1515^{64}$, $1584^{16}$, $2001^{16}$, $2157^{48}$, $2250^{16}$, $2439^{16}$, $2871^{48}$, $3033^{16}$, $3075^{16}$, $3327^{16}$, $3473^{48}$, $3604^{48}$, $3696^{16}$, $4848^{48}$, $4893^{48}$, $5415^{16}$, $6261^{48}$, $6435^{16}$, $6689^{48}$, $6894^{48}$, $7384^{48}$, $8169^{16}$, $8958^{16}$, $10313^{48}$, $10419^{16}$, $10508^{48}$, $10548^{48}$, $10765^{48}$, $10803^{16}$, $11373^{16}$, $12378^{16}$, $12585^{16}$, $12996^{16}$, $15549^{16}$, $16575^{16}$, $16674^{16}$, $16998^{16}$, $17658^{16}$, $19401^{16}$, $20010^{16}$, $20094^{16}$, $20913^{16}$, $21174^{16}$, $24951^{16}$, $25905^{16}$, $28659^{16}$, $28902^{16}$, $28947^{16}$, $29775^{16}$, $31107^{16}$, $31881^{16}$, $32592^{16}$, $32637^{16}$, $32679^{16}$, $33048^{16}$
\item $(h,k)=(11,13)$ : $1^{4}$, $3^{5592404}$
\item $(h,k)=(11,17)$ : $1^{64}$, $3^{124}$, $4^{60}$, $6^{48}$, $8^{48}$, $9^{24}$, $10^{48}$, $12^{48}$, $13^{12}$, $14^{24}$, $16^{96}$, $22^{96}$, $24^{28}$, $27^{24}$, $28^{36}$, $29^{12}$, $30^{48}$, $31^{12}$, $32^{12}$, $35^{12}$, $38^{24}$, $39^{12}$, $52^{48}$, $63^{24}$, $64^{6}$, $66^{24}$, $70^{24}$, $72^{2}$, $73^{48}$, $82^{48}$, $86^{24}$, $98^{24}$, $105^{24}$, $112^{6}$, $116^{36}$, $126^{48}$, $136^{36}$, $144^{14}$, $154^{24}$, $160^{24}$, $262^{24}$, $264^{18}$, $289^{24}$, $299^{24}$, $328^{12}$, $339^{48}$, $342^{24}$, $348^{24}$, $364^{12}$, $381^{24}$, $443^{24}$, $444^{24}$, $475^{24}$, $492^{24}$, $545^{24}$, $559^{24}$, $606^{24}$, $632^{24}$, $635^{24}$, $648^{24}$, $659^{48}$, $713^{24}$, $732^{24}$, $742^{24}$, $764^{24}$, $799^{24}$, $801^{24}$, $817^{24}$, $820^{24}$, $823^{24}$, $837^{24}$, $847^{24}$, $848^{24}$, $884^{24}$, $925^{24}$, $939^{24}$, $952^{24}$, $1133^{24}$, $1135^{48}$, $1148^{24}$, $1188^{12}$, $1285^{24}$, $1309^{24}$, $1312^{24}$, $1327^{24}$, $1348^{24}$, $1361^{24}$, $1399^{24}$, $1409^{24}$, $1471^{24}$, $1532^{24}$, $1547^{24}$, $1550^{24}$, $1555^{24}$, $1566^{24}$, $1569^{24}$, $1618^{24}$, $1621^{24}$, $1626^{24}$, $1628^{24}$, $1635^{24}$, $1644^{24}$, $1657^{24}$, $1689^{48}$, $1736^{24}$, $1774^{24}$, $1778^{24}$, $1792^{24}$, $1818^{24}$, $1846^{24}$, $1886^{24}$, $1987^{24}$, $1999^{24}$, $2087^{24}$, $2089^{24}$, $2102^{24}$, $2134^{24}$, $2200^{24}$, $2249^{24}$, $2296^{24}$, $2342^{24}$, $2511^{24}$, $2512^{24}$, $2554^{24}$, $2571^{24}$, $2631^{24}$, $2682^{24}$, $2707^{24}$, $2723^{24}$, $2769^{24}$, $2799^{24}$, $2800^{12}$, $2852^{24}$, $2853^{24}$, $2858^{24}$, $2910^{24}$, $2925^{24}$, $2933^{24}$, $2954^{24}$, $3164^{12}$, $3183^{24}$, $3188^{24}$, $3194^{24}$, $3272^{24}$, $3323^{24}$, $3418^{24}$, $3441^{24}$, $3457^{24}$, $3468^{24}$, $3476^{24}$, $3501^{24}$, $3788^{24}$, $3911^{24}$, $3956^{48}$, $4069^{24}$, $4177^{48}$, $4262^{24}$, $4303^{24}$, $4341^{24}$, $4362^{24}$, $4400^{24}$, $4439^{24}$, $4560^{24}$, $4652^{12}$, $4961^{24}$, $4978^{24}$, $4992^{24}$, $4998^{24}$, $5293^{24}$, $5317^{24}$, $5358^{24}$, $5367^{24}$, $5396^{24}$, $5490^{24}$, $5898^{24}$, $5911^{24}$, $5960^{24}$, $6128^{24}$, $6293^{24}$, $6494^{24}$, $6616^{24}$, $6622^{24}$, $6624^{24}$, $6713^{24}$, $6731^{24}$, $7225^{24}$, $7306^{24}$, $7460^{24}$, $7482^{24}$, $7580^{24}$, $7776^{6}$, $7818^{24}$, $8260^{24}$, $8298^{24}$, $8340^{24}$, $8768^{24}$, $8804^{24}$, $8881^{24}$, $9056^{24}$, $9090^{24}$, $9116^{12}$, $9503^{24}$, $9984^{2}$, $10054^{24}$, $10266^{24}$, $10267^{24}$, $10295^{24}$, $10546^{24}$, $10848^{24}$, $11781^{24}$, $13081^{24}$, $13563^{24}$, $13954^{24}$, $14413^{24}$, $15666^{24}$, $15872^{24}$, $16709^{24}$, $19387^{24}$
\item $(h,k)=(11,19)$ : $1^{256}$, $2^{360}$, $3^{200}$, $5^{120}$, $6^{72}$, $7^{24}$, $8^{12}$, $9^{112}$, $11^{180}$, $12^{64}$, $13^{48}$, $15^{168}$, $16^{84}$, $17^{24}$, $18^{168}$, $19^{12}$, $21^{80}$, $22^{24}$, $23^{24}$, $24^{40}$, $25^{48}$, $26^{24}$, $27^{40}$, $29^{24}$, $33^{56}$, $35^{96}$, $38^{48}$, $39^{24}$, $40^{24}$, $41^{120}$, $42^{32}$, $43^{48}$, $48^{16}$, $50^{24}$, $51^{16}$, $52^{96}$, $54^{24}$, $55^{48}$, $57^{40}$, $61^{24}$, $63^{80}$, $65^{24}$, $71^{24}$, $72^{40}$, $76^{24}$, $79^{48}$, $80^{24}$, $81^{24}$, $82^{48}$, $85^{48}$, $87^{24}$, $89^{72}$, $92^{24}$, $94^{24}$, $96^{24}$, $102^{48}$, $103^{24}$, $107^{24}$, $108^{24}$, $110^{24}$, $111^{72}$, $113^{48}$, $114^{56}$, $115^{24}$, $116^{24}$, $118^{24}$, $120^{40}$, $122^{48}$, $123^{40}$, $126^{16}$, $130^{24}$, $132^{48}$, $133^{24}$, $141^{40}$, $147^{16}$, $148^{48}$, $149^{24}$, $152^{24}$, $153^{16}$, $159^{48}$, $160^{24}$, $164^{48}$, $167^{24}$, $168^{24}$, $171^{24}$, $172^{24}$, $175^{48}$, $179^{24}$, $180^{40}$, $181^{24}$, $182^{48}$, $183^{72}$, $186^{40}$, $192^{16}$, $197^{24}$, $199^{24}$, $200^{24}$, $203^{24}$, $204^{32}$, $205^{24}$, $206^{24}$, $209^{24}$, $210^{24}$, $211^{48}$, $215^{24}$, $217^{48}$, $222^{16}$, $225^{40}$, $226^{72}$, $233^{24}$, $240^{32}$, $241^{24}$, $243^{48}$, $245^{24}$, $246^{24}$, $247^{24}$, $248^{24}$, $251^{24}$, $259^{48}$, $261^{16}$, $262^{48}$, $277^{24}$, $281^{48}$, $297^{24}$, $315^{16}$, $317^{24}$, $318^{24}$, $324^{24}$, $325^{24}$, $329^{24}$, $330^{24}$, $333^{16}$, $334^{24}$, $336^{40}$, $337^{24}$, $343^{48}$, $353^{24}$, $355^{24}$, $360^{24}$, $374^{24}$, $375^{24}$, $380^{24}$, $381^{24}$, $390^{24}$, $397^{48}$, $398^{72}$, $408^{24}$, $410^{48}$, $414^{16}$, $417^{24}$, $423^{16}$, $425^{24}$, $438^{24}$, $444^{24}$, $446^{24}$, $447^{24}$, $456^{24}$, $471^{24}$, $474^{16}$, $500^{24}$, $511^{24}$, $517^{48}$, $519^{24}$, $534^{24}$, $554^{24}$, $569^{24}$, $576^{16}$, $577^{24}$, $588^{24}$, $672^{48}$, $678^{16}$, $707^{24}$, $726^{32}$, $754^{24}$, $780^{32}$, $786^{16}$, $788^{24}$, $819^{16}$, $849^{16}$, $889^{24}$, $900^{24}$, $906^{48}$, $914^{24}$, $918^{16}$, $927^{48}$, $1056^{16}$, $1118^{48}$, $1143^{16}$, $1188^{16}$, $1260^{16}$, $1293^{24}$, $1369^{24}$, $1471^{48}$, $1477^{48}$, $1482^{16}$, $1731^{48}$, $1735^{24}$, $1740^{16}$, $2115^{16}$, $2121^{48}$, $2283^{48}$, $2286^{16}$, $2575^{48}$, $2649^{16}$, $3038^{48}$, $3222^{16}$, $3318^{16}$, $3483^{16}$, $3684^{48}$, $3921^{16}$, $4185^{16}$, $4404^{16}$, $4686^{16}$, $4743^{16}$, $4914^{48}$, $4947^{16}$, $5122^{48}$, $5367^{16}$, $5904^{16}$, $6618^{16}$, $6793^{48}$, $6909^{16}$, $7626^{16}$, $7873^{48}$, $7963^{48}$, $8137^{48}$, $8178^{16}$, $9618^{16}$, $10546^{48}$, $10733^{48}$, $10734^{48}$, $11577^{16}$, $11580^{16}$, $12939^{16}$, $15234^{16}$, $18975^{16}$, $20118^{16}$, $20589^{16}$, $21816^{16}$, $23736^{16}$, $27375^{16}$, $27414^{16}$, $27708^{16}$, $28011^{16}$, $28287^{16}$, $28557^{16}$, $29961^{16}$, $31041^{16}$, $31752^{16}$, $32085^{16}$, $32139^{16}$, $32157^{16}$, $32343^{16}$, $32949^{16}$
\item $(h,k)=(11,23)$ : $1^{4096}$, $3^{1184}$, $4^{3068}$, $5^{672}$, $6^{1392}$, $7^{488}$, $8^{1046}$, $9^{728}$, $10^{768}$, $11^{248}$, $12^{820}$, $13^{240}$, $14^{720}$, $15^{336}$, $16^{696}$, $17^{288}$, $18^{640}$, $19^{528}$, $20^{336}$, $21^{336}$, $22^{288}$, $23^{288}$, $24^{476}$, $25^{48}$, $26^{216}$, $27^{96}$, $28^{240}$, $29^{144}$, $30^{112}$, $32^{216}$, $33^{144}$, $34^{192}$, $35^{144}$, $36^{296}$, $37^{144}$, $38^{240}$, $39^{144}$, $40^{192}$, $41^{144}$, $42^{168}$, $43^{144}$, $44^{228}$, $45^{192}$, $46^{288}$, $48^{144}$, $49^{72}$, $50^{96}$, $51^{168}$, $52^{96}$, $53^{48}$, $54^{264}$, $55^{144}$, $56^{168}$, $57^{72}$, $58^{120}$, $60^{152}$, $62^{96}$, $63^{48}$, $64^{72}$, $65^{48}$, $66^{72}$, $68^{192}$, $69^{96}$, $70^{216}$, $72^{96}$, $73^{48}$, $74^{24}$, $75^{48}$, $76^{96}$, $77^{48}$, $78^{216}$, $79^{72}$, $80^{96}$, $81^{48}$, $82^{120}$, $85^{48}$, $86^{96}$, $87^{48}$, $88^{48}$, $90^{96}$, $91^{48}$, $92^{72}$, $93^{48}$, $94^{168}$, $95^{48}$, $96^{146}$, $98^{48}$, $99^{96}$, $100^{216}$, $101^{96}$, $102^{24}$, $104^{48}$, $106^{96}$, $108^{148}$, $109^{48}$, $112^{108}$, $113^{48}$, $114^{240}$, $116^{24}$, $117^{64}$, $118^{216}$, $119^{48}$, $120^{72}$, $122^{48}$, $124^{192}$, $126^{96}$, $128^{72}$, $130^{192}$, $132^{48}$, $133^{48}$, $136^{48}$, $138^{120}$, $139^{72}$, $140^{72}$, $141^{24}$, $144^{48}$, $145^{48}$, $146^{72}$, $147^{48}$, $148^{24}$, $149^{192}$, $150^{48}$, $151^{48}$, $153^{48}$, $154^{48}$, $156^{48}$, $158^{72}$, $160^{48}$, $161^{48}$, $162^{120}$, $164^{72}$, $167^{48}$, $168^{48}$, $170^{48}$, $171^{48}$, $173^{48}$, $174^{32}$, $175^{48}$, $178^{24}$, $179^{48}$, $180^{24}$, $182^{120}$, $183^{24}$, $184^{48}$, $185^{24}$, $186^{96}$, $188^{96}$, $190^{24}$, $191^{48}$, $192^{48}$, $193^{24}$, $196^{96}$, $197^{48}$, $198^{72}$, $200^{72}$, $202^{120}$, $204^{144}$, $205^{48}$, $206^{72}$, $207^{48}$, $208^{24}$, $209^{72}$, $210^{48}$, $212^{72}$, $213^{48}$, $214^{24}$, $216^{24}$, $218^{48}$, $221^{48}$, $222^{120}$, $224^{48}$, $225^{96}$, $226^{48}$, $228^{48}$, $229^{48}$, $230^{48}$, $231^{96}$, $232^{72}$, $233^{24}$, $236^{120}$, $240^{26}$, $246^{24}$, $247^{48}$, $250^{72}$, $251^{48}$, $252^{24}$, $253^{96}$, $254^{144}$, $257^{96}$, $258^{24}$, $260^{48}$, $264^{48}$, $266^{96}$, $268^{72}$, $270^{96}$, $272^{120}$, $274^{48}$, $276^{48}$, $278^{24}$, $282^{72}$, $284^{24}$, $285^{96}$, $290^{48}$, $292^{96}$, $300^{36}$, $301^{72}$, $302^{24}$, $305^{96}$, $307^{72}$, $310^{48}$, $312^{24}$, $313^{72}$, $314^{24}$, $317^{48}$, $318^{40}$, $321^{48}$, $322^{24}$, $324^{48}$, $325^{48}$, $326^{48}$, $327^{96}$, $331^{48}$, $332^{24}$, $333^{24}$, $335^{24}$, $340^{96}$, $342^{24}$, $346^{24}$, $348^{48}$, $362^{24}$, $363^{96}$, $364^{48}$, $366^{24}$, $368^{48}$, $369^{96}$, $372^{24}$, $374^{24}$, $376^{24}$, $378^{48}$, $380^{24}$, $386^{72}$, $395^{48}$, $396^{36}$, $398^{72}$, $401^{48}$, $402^{96}$, $403^{96}$, $404^{72}$, $408^{24}$, $409^{72}$, $411^{48}$, $412^{48}$, $414^{48}$, $416^{48}$, $417^{48}$, $418^{24}$, $420^{24}$, $421^{72}$, $422^{24}$, $425^{48}$, $426^{48}$, $430^{72}$, $432^{24}$, $439^{24}$, $440^{24}$, $441^{48}$, $444^{48}$, $448^{24}$, $451^{48}$, $453^{24}$, $455^{48}$, $456^{48}$, $458^{24}$, $462^{72}$, $466^{24}$, $470^{96}$, $476^{48}$, $478^{48}$, $482^{48}$, $486^{48}$, $487^{48}$, $496^{96}$, $497^{48}$, $498^{24}$, $501^{24}$, $502^{24}$, $504^{54}$, $505^{48}$, $508^{72}$, $510^{72}$, $512^{24}$, $521^{120}$, $522^{72}$, $523^{48}$, $527^{48}$, $528^{48}$, $531^{24}$, $534^{48}$, $536^{24}$, $537^{48}$, $540^{48}$, $542^{24}$, $543^{48}$, $544^{48}$, $545^{48}$, $550^{24}$, $552^{72}$, $554^{24}$, $555^{48}$, $562^{96}$, $564^{48}$, $568^{72}$, $569^{48}$, $575^{48}$, $577^{48}$, $578^{48}$, $579^{24}$, $583^{48}$, $584^{24}$, $585^{72}$, $586^{48}$, $587^{48}$, $591^{24}$, $592^{24}$, $593^{24}$, $595^{24}$, $596^{24}$, $600^{24}$, $601^{24}$, $602^{24}$, $610^{72}$, $618^{24}$, $621^{24}$, $622^{48}$, $625^{48}$, $626^{24}$, $630^{24}$, $645^{96}$, $653^{48}$, $655^{24}$, $658^{24}$, $659^{72}$, $661^{48}$, $670^{72}$, $675^{48}$, $678^{48}$, $680^{24}$, $682^{24}$, $683^{24}$, $690^{24}$, $691^{48}$, $692^{24}$, $698^{24}$, $699^{96}$, $704^{24}$, $707^{24}$, $712^{24}$, $714^{24}$, $716^{24}$, $718^{48}$, $720^{84}$, $724^{24}$, $726^{24}$, $728^{24}$, $731^{72}$, $734^{48}$, $744^{24}$, $749^{48}$, $750^{24}$, $752^{48}$, $755^{48}$, $756^{4}$, $757^{48}$, $760^{24}$, $762^{24}$, $765^{24}$, $771^{24}$, $773^{24}$, $776^{48}$, $778^{48}$, $780^{24}$, $783^{48}$, $786^{24}$, $789^{24}$, $791^{48}$, $799^{48}$, $816^{72}$, $821^{24}$, $822^{24}$, $823^{48}$, $824^{24}$, $829^{48}$, $834^{48}$, $835^{24}$, $840^{72}$, $842^{24}$, $843^{24}$, $845^{24}$, $847^{48}$, $854^{24}$, $856^{24}$, $858^{48}$, $859^{48}$, $861^{24}$, $862^{72}$, $866^{96}$, $877^{48}$, $880^{24}$, $884^{24}$, $886^{24}$, $887^{48}$, $888^{24}$, $892^{48}$, $893^{48}$, $896^{72}$, $901^{24}$, $903^{48}$, $904^{48}$, $914^{24}$, $922^{24}$, $924^{72}$, $925^{48}$, $926^{48}$, $929^{96}$, $930^{16}$, $933^{24}$, $934^{24}$, $937^{48}$, $938^{48}$, $941^{48}$, $951^{48}$, $954^{48}$, $955^{96}$, $957^{24}$, $966^{96}$, $970^{24}$, $974^{24}$, $977^{24}$, $982^{48}$, $983^{24}$, $987^{64}$, $993^{96}$, $996^{48}$, $998^{48}$, $999^{24}$, $1002^{48}$, $1003^{72}$, $1005^{24}$, $1006^{24}$, $1008^{24}$, $1010^{48}$, $1011^{48}$, $1013^{48}$, $1016^{48}$, $1017^{48}$, $1018^{96}$, $1020^{120}$, $1022^{48}$, $1023^{24}$, $1026^{48}$, $1032^{24}$, $1034^{48}$, $1037^{24}$, $1038^{96}$, $1041^{24}$, $1042^{24}$, $1048^{24}$, $1059^{24}$, $1074^{24}$, $1076^{24}$, $1089^{24}$, $1126^{24}$, $1138^{24}$, $1144^{48}$, $1164^{24}$, $1169^{24}$, $1172^{24}$, $1187^{24}$, $1190^{24}$, $1193^{24}$, $1199^{24}$, $1201^{24}$, $1204^{24}$, $1206^{24}$, $1213^{24}$, $1217^{24}$, $1219^{24}$, $1222^{24}$, $1227^{24}$, $1231^{24}$, $1247^{24}$, $1255^{24}$, $1266^{24}$, $1267^{24}$, $1270^{24}$, $1271^{24}$, $1272^{24}$, $1285^{24}$, $1290^{24}$, $1292^{24}$, $1302^{24}$, $1305^{24}$, $1325^{24}$, $1326^{24}$, $1335^{24}$, $1342^{24}$, $1343^{24}$, $1348^{24}$, $1354^{24}$, $1364^{24}$, $1365^{24}$, $1367^{24}$, $1378^{24}$, $1384^{24}$, $1429^{24}$, $1432^{24}$, $1443^{24}$, $1446^{24}$, $1452^{24}$, $1459^{24}$, $1463^{24}$, $1466^{24}$, $1470^{48}$, $1472^{48}$, $1473^{24}$, $1475^{24}$, $1476^{24}$, $1480^{24}$, $1481^{24}$, $1491^{24}$, $1493^{48}$, $1502^{24}$, $1504^{24}$, $1516^{24}$, $1527^{24}$, $1536^{24}$, $1561^{24}$, $1564^{24}$, $1567^{24}$, $1583^{24}$, $1584^{6}$, $1587^{24}$, $1590^{24}$, $1596^{48}$, $1599^{24}$, $1611^{24}$, $1614^{24}$, $1646^{24}$, $1662^{72}$, $1667^{24}$, $1672^{12}$, $1673^{24}$, $1678^{48}$, $1683^{24}$, $1685^{24}$, $1692^{24}$, $1697^{24}$, $1699^{24}$, $1705^{24}$, $1708^{24}$, $1721^{24}$, $1722^{24}$, $1727^{24}$, $1735^{24}$, $1741^{24}$, $1749^{24}$, $1751^{24}$, $1756^{24}$, $1759^{24}$, $1775^{24}$, $1781^{24}$, $1783^{24}$, $1791^{24}$, $1802^{24}$, $1816^{24}$, $1817^{24}$, $1818^{24}$, $1824^{8}$, $1825^{24}$, $1830^{24}$, $1834^{24}$, $1851^{24}$, $1857^{48}$, $1863^{24}$, $1865^{24}$, $1866^{24}$, $1872^{24}$, $1875^{48}$, $1877^{24}$, $1879^{24}$, $1887^{24}$, $1889^{24}$, $1894^{24}$, $1917^{24}$, $1929^{48}$, $1931^{24}$, $1932^{12}$, $1937^{24}$, $1938^{48}$, $1939^{24}$, $1942^{48}$, $1962^{24}$, $1963^{24}$, $1965^{24}$, $1969^{24}$, $1975^{24}$, $1980^{24}$, $1981^{24}$, $1991^{48}$, $1997^{24}$, $2001^{48}$, $2015^{24}$, $2017^{24}$, $2023^{24}$, $2024^{24}$, $2027^{24}$, $2029^{24}$, $2031^{24}$, $2037^{24}$, $2039^{24}$, $2040^{24}$, $2045^{24}$, $2061^{48}$, $2063^{24}$, $2064^{24}$, $2065^{24}$, $2067^{24}$, $2068^{12}$, $2071^{24}$, $2077^{24}$, $2079^{24}$, $2080^{24}$, $2083^{48}$, $2085^{24}$, $2103^{24}$
\item $(h,k)=(13,17)$ : $1^{16}$, $2^{72}$, $3^{40}$, $4^{20}$, $5^{48}$, $6^{16}$, $7^{32}$, $8^{14}$, $9^{48}$, $11^{20}$, $12^{36}$, $16^{12}$, $19^{12}$, $21^{24}$, $27^{8}$, $29^{48}$, $35^{24}$, $43^{24}$, $45^{16}$, $54^{8}$, $57^{8}$, $62^{24}$, $71^{24}$, $80^{24}$, $84^{24}$, $118^{24}$, $156^{24}$, $182^{48}$, $190^{24}$, $206^{24}$, $227^{24}$, $234^{8}$, $254^{48}$, $260^{24}$, $262^{24}$, $267^{16}$, $338^{24}$, $355^{24}$, $361^{24}$, $378^{8}$, $390^{24}$, $409^{24}$, $419^{24}$, $429^{24}$, $445^{24}$, $454^{24}$, $470^{24}$, $479^{24}$, $482^{24}$, $491^{24}$, $548^{24}$, $631^{24}$, $683^{24}$, $767^{24}$, $789^{24}$, $792^{24}$, $831^{24}$, $845^{24}$, $852^{24}$, $871^{24}$, $900^{12}$, $922^{24}$, $972^{8}$, $975^{24}$, $995^{24}$, $1008^{26}$, $1011^{24}$, $1024^{24}$, $1025^{24}$, $1048^{24}$, $1073^{24}$, $1119^{24}$, $1140^{4}$, $1197^{24}$, $1304^{24}$, $1400^{24}$, $1448^{24}$, $1468^{24}$, $1536^{24}$, $1613^{24}$, $1683^{24}$, $1733^{24}$, $1746^{24}$, $1759^{24}$, $1771^{24}$, $1813^{24}$, $1900^{24}$, $1906^{24}$, $1944^{2}$, $1951^{24}$, $1982^{24}$, $1983^{24}$, $2062^{24}$, $2092^{24}$, $2139^{24}$, $2171^{24}$, $2197^{24}$, $2201^{24}$, $2216^{24}$, $2225^{24}$, $2285^{24}$, $2303^{24}$, $2356^{24}$, $2385^{24}$, $2408^{24}$, $2474^{24}$, $2538^{24}$, $2587^{24}$, $2620^{24}$, $2637^{24}$, $2691^{24}$, $2694^{24}$, $2792^{24}$, $2844^{24}$, $2976^{24}$, $3003^{24}$, $3073^{24}$, $3169^{24}$, $3174^{24}$, $3209^{24}$, $3263^{24}$, $3289^{24}$, $3296^{24}$, $3497^{24}$, $3507^{24}$, $3529^{24}$, $3534^{24}$, $3598^{24}$, $3652^{24}$, $3737^{24}$, $3810^{24}$, $3877^{24}$, $3941^{24}$, $3947^{24}$, $4002^{24}$, $4010^{48}$, $4205^{24}$, $4440^{24}$, $4630^{24}$, $4660^{24}$, $4691^{24}$, $4786^{24}$, $4819^{24}$, $4832^{24}$, $5084^{24}$, $5100^{24}$, $5230^{24}$, $5594^{24}$, $5650^{24}$, $5880^{24}$, $5994^{24}$, $6136^{24}$, $6164^{24}$, $6332^{24}$, $6397^{24}$, $6430^{24}$, $6438^{24}$, $6470^{24}$, $6653^{24}$, $6667^{24}$, $6721^{24}$, $6785^{24}$, $6787^{24}$, $7476^{24}$, $7632^{24}$, $7666^{24}$, $7954^{24}$, $7992^{8}$, $8054^{24}$, $8117^{24}$, $8148^{24}$, $8244^{24}$, $8283^{24}$, $8325^{24}$, $8409^{24}$, $8607^{24}$, $8753^{24}$, $8808^{24}$, $8857^{24}$, $9116^{24}$, $9452^{24}$, $9481^{24}$, $9828^{24}$, $9855^{24}$, $10362^{24}$, $10400^{24}$, $10582^{24}$, $10928^{24}$, $11107^{24}$, $11227^{24}$, $11794^{24}$, $12009^{24}$, $12531^{24}$, $12809^{24}$, $12868^{24}$, $14533^{24}$, $15007^{24}$, $15035^{24}$, $25476^{8}$
\item $(h,k)=(13,19)$ : $1^{64}$, $3^{348}$, $4^{60}$, $7^{24}$, $8^{18}$, $9^{8}$, $12^{72}$, $13^{12}$, $14^{24}$, $16^{54}$, $17^{48}$, $18^{48}$, $24^{42}$, $28^{24}$, $29^{12}$, $30^{24}$, $31^{36}$, $32^{12}$, $35^{12}$, $39^{36}$, $40^{12}$, $48^{24}$, $52^{12}$, $64^{36}$, $68^{12}$, $69^{24}$, $82^{72}$, $124^{12}$, $150^{24}$, $160^{12}$, $172^{12}$, $192^{4}$, $204^{24}$, $216^{2}$, $252^{12}$, $274^{24}$, $276^{24}$, $288^{2}$, $378^{16}$, $381^{24}$, $422^{48}$, $426^{24}$, $428^{12}$, $437^{48}$, $440^{24}$, $486^{24}$, $493^{24}$, $502^{24}$, $559^{24}$, $562^{24}$, $578^{24}$, $608^{24}$, $616^{24}$, $631^{24}$, $636^{12}$, $644^{12}$, $658^{48}$, $668^{24}$, $678^{24}$, $720^{2}$, $745^{24}$, $788^{12}$, $793^{24}$, $808^{6}$, $914^{24}$, $1040^{6}$, $1081^{24}$, $1087^{24}$, $1094^{24}$, $1184^{24}$, $1186^{24}$, $1193^{24}$, $1196^{12}$, $1233^{24}$, $1290^{24}$, $1344^{24}$, $1348^{24}$, $1370^{24}$, $1371^{24}$, $1375^{24}$, $1384^{24}$, $1420^{24}$, $1424^{24}$, $1434^{48}$, $1455^{24}$, $1464^{24}$, $1510^{24}$, $1626^{24}$, $1648^{24}$, $1655^{24}$, $1664^{12}$, $1714^{24}$, $1731^{24}$, $1748^{24}$, $1773^{24}$, $1784^{12}$, $1785^{24}$, $1813^{24}$, $1831^{24}$, $1848^{24}$, $1877^{24}$, $1912^{48}$, $2036^{24}$, $2074^{24}$, $2083^{24}$, $2175^{24}$, $2196^{24}$, $2200^{24}$, $2269^{24}$, $2348^{24}$, $2405^{24}$, $2422^{24}$, $2441^{24}$, $2521^{24}$, $2645^{24}$, $2703^{24}$, $2752^{24}$, $2783^{24}$, $2795^{24}$, $2858^{24}$, $2866^{24}$, $2879^{24}$, $2927^{24}$, $2941^{24}$, $2953^{24}$, $2959^{24}$, $2963^{24}$, $2969^{24}$, $3002^{24}$, $3003^{24}$, $3048^{24}$, $3084^{12}$, $3129^{24}$, $3174^{24}$, $3204^{12}$, $3269^{24}$, $3298^{24}$, $3331^{24}$, $3362^{24}$, $3372^{24}$, $3444^{24}$, $3481^{24}$, $3494^{24}$, $3536^{24}$, $3542^{24}$, $3586^{24}$, $3600^{24}$, $3651^{24}$, $3661^{24}$, $3730^{24}$, $3818^{24}$, $3845^{24}$, $3943^{24}$, $3996^{24}$, $4106^{24}$, $4232^{24}$, $4340^{24}$, $4518^{8}$, $4651^{24}$, $4677^{24}$, $4687^{24}$, $4752^{24}$, $4794^{24}$, $4846^{24}$, $4859^{24}$, $4865^{24}$, $4928^{24}$, $4989^{24}$, $5055^{24}$, $5186^{24}$, $5278^{24}$, $5281^{24}$, $5430^{24}$, $5432^{24}$, $5436^{24}$, $5508^{24}$, $5582^{24}$, $5590^{24}$, $5663^{24}$, $5799^{48}$, $5835^{24}$, $5866^{24}$, $5880^{24}$, $6089^{24}$, $6100^{24}$, $6107^{24}$, $6188^{24}$, $6208^{24}$, $6398^{24}$, $6629^{24}$, $6812^{24}$, $6838^{24}$, $7021^{24}$, $7204^{24}$, $7255^{24}$, $7302^{48}$, $7639^{24}$, $8000^{24}$, $8182^{24}$, $8568^{24}$, $8787^{24}$, $8926^{24}$, $8986^{24}$, $9061^{24}$, $9257^{24}$, $9335^{24}$, $9395^{24}$, $9609^{24}$, $10037^{24}$, $10101^{24}$, $10620^{24}$, $11165^{24}$, $13080^{6}$, $14769^{24}$, $15162^{24}$, $16976^{24}$, $19524^{12}$, $21146^{24}$
\item $(h,k)=(13,23)$ : $1^{4}$, $3^{1364}$, $9^{16}$, $24^{4}$, $29^{24}$, $69^{16}$, $77^{24}$, $89^{24}$, $121^{24}$, $180^{24}$, $181^{24}$, $188^{24}$, $226^{24}$, $255^{24}$, $294^{24}$, $295^{24}$, $312^{24}$, $330^{8}$, $339^{24}$, $346^{24}$, $390^{24}$, $393^{24}$, $394^{24}$, $422^{24}$, $461^{24}$, $480^{2}$, $537^{24}$, $552^{24}$, $567^{24}$, $577^{24}$, $654^{24}$, $663^{24}$, $674^{24}$, $709^{24}$, $739^{24}$, $762^{24}$, $765^{24}$, $771^{24}$, $844^{24}$, $884^{24}$, $893^{24}$, $912^{24}$, $957^{24}$, $960^{2}$, $983^{24}$, $1046^{24}$, $1111^{24}$, $1151^{24}$, $1158^{24}$, $1175^{24}$, $1185^{24}$, $1250^{24}$, $1275^{24}$, $1292^{24}$, $1294^{24}$, $1302^{24}$, $1311^{24}$, $1343^{24}$, $1477^{24}$, $1563^{24}$, $1659^{24}$, $1680^{24}$, $1688^{24}$, $1690^{24}$, $1694^{24}$, $1732^{24}$, $1736^{24}$, $1744^{24}$, $1778^{48}$, $1781^{24}$, $1844^{24}$, $1893^{24}$, $1939^{24}$, $1967^{24}$, $1996^{24}$, $2057^{24}$, $2058^{24}$, $2088^{24}$, $2093^{24}$, $2134^{24}$, $2145^{24}$, $2248^{6}$, $2322^{24}$, $2337^{24}$, $2430^{24}$, $2441^{24}$, $2478^{24}$, $2559^{24}$, $2620^{24}$, $2671^{24}$, $2698^{24}$, $2782^{24}$, $2884^{24}$, $2983^{24}$, $2987^{24}$, $3000^{24}$, $3031^{24}$, $3123^{24}$, $3264^{24}$, $3282^{24}$, $3302^{24}$, $3344^{24}$, $3380^{24}$, $3639^{24}$, $3732^{24}$, $3733^{24}$, $3813^{24}$, $3821^{24}$, $3868^{24}$, $3898^{24}$, $3909^{24}$, $3919^{24}$, $3944^{24}$, $3959^{24}$, $3993^{24}$, $4088^{24}$, $4140^{24}$, $4216^{24}$, $4280^{24}$, $4291^{24}$, $4344^{24}$, $4347^{24}$, $4381^{24}$, $4440^{24}$, $4491^{24}$, $4624^{24}$, $4807^{24}$, $4840^{24}$, $4905^{24}$, $5096^{24}$, $5114^{24}$, $5325^{24}$, $5388^{24}$, $5450^{24}$, $5504^{24}$, $5548^{24}$, $5568^{24}$, $5580^{24}$, $5609^{24}$, $5765^{24}$, $5772^{24}$, $6149^{24}$, $6191^{24}$, $6211^{24}$, $6285^{24}$, $6356^{24}$, $6406^{24}$, $6432^{6}$, $6559^{24}$, $6614^{24}$, $6704^{24}$, $6929^{24}$, $7143^{24}$, $7576^{24}$, $7643^{24}$, $7703^{24}$, $8203^{24}$, $8211^{24}$, $8283^{24}$, $8304^{2}$, $8626^{24}$, $8943^{24}$, $9127^{24}$, $9312^{24}$, $9637^{24}$, $10256^{24}$, $10375^{24}$, $11428^{24}$, $11542^{24}$, $12263^{24}$, $12448^{24}$, $12903^{24}$, $14058^{24}$, $14080^{24}$, $14301^{24}$, $15010^{24}$, $15581^{24}$, $16671^{24}$, $19645^{24}$, $21036^{4}$, $24253^{24}$
\item $(h,k)=(17,19)$ : $1^{4}$, $3^{1364}$, $9^{16}$, $24^{4}$, $29^{24}$, $31^{24}$, $53^{24}$, $60^{48}$, $68^{24}$, $72^{6}$, $102^{24}$, $148^{24}$, $192^{24}$, $224^{12}$, $233^{24}$, $248^{24}$, $253^{24}$, $258^{24}$, $283^{24}$, $288^{24}$, $309^{24}$, $319^{24}$, $382^{24}$, $386^{24}$, $406^{24}$, $415^{24}$, $431^{24}$, $464^{24}$, $469^{24}$, $475^{24}$, $498^{24}$, $542^{24}$, $546^{24}$, $553^{24}$, $580^{24}$, $584^{24}$, $607^{24}$, $618^{24}$, $619^{24}$, $658^{24}$, $737^{24}$, $820^{24}$, $837^{24}$, $847^{24}$, $862^{24}$, $868^{24}$, $876^{24}$, $881^{24}$, $961^{24}$, $1013^{24}$, $1071^{24}$, $1091^{48}$, $1107^{24}$, $1139^{24}$, $1195^{24}$, $1199^{24}$, $1258^{24}$, $1289^{24}$, $1304^{24}$, $1353^{24}$, $1375^{24}$, $1392^{24}$, $1413^{24}$, $1441^{24}$, $1508^{24}$, $1583^{24}$, $1589^{24}$, $1786^{24}$, $1791^{24}$, $1848^{24}$, $1857^{24}$, $1867^{24}$, $1915^{24}$, $1966^{24}$, $1976^{24}$, $2001^{24}$, $2025^{24}$, $2071^{24}$, $2129^{24}$, $2148^{4}$, $2475^{24}$, $2539^{24}$, $2555^{24}$, $2683^{24}$, $2752^{24}$, $2761^{24}$, $2806^{24}$, $2854^{24}$, $2925^{24}$, $2953^{24}$, $3020^{24}$, $3118^{24}$, $3140^{24}$, $3266^{24}$, $3385^{24}$, $3397^{24}$, $3408^{24}$, $3446^{24}$, $3513^{24}$, $3575^{24}$, $3629^{24}$, $3635^{24}$, $3664^{24}$, $3727^{24}$, $3745^{24}$, $3758^{24}$, $3803^{24}$, $3875^{24}$, $3882^{24}$, $3905^{24}$, $3922^{24}$, $3958^{24}$, $4259^{24}$, $4287^{24}$, $4297^{24}$, $4355^{24}$, $4432^{24}$, $4719^{24}$, $4765^{24}$, $4805^{24}$, $4870^{24}$, $4941^{24}$, $5149^{24}$, $5158^{24}$, $5299^{24}$, $5317^{24}$, $5371^{24}$, $5399^{24}$, $5587^{24}$, $5594^{24}$, $5717^{24}$, $5806^{24}$, $6026^{24}$, $6069^{24}$, $6177^{24}$, $6226^{24}$, $6236^{24}$, $6246^{48}$, $6447^{24}$, $6455^{24}$, $6554^{24}$, $6556^{24}$, $6568^{24}$, $6594^{24}$, $6610^{24}$, $6746^{24}$, $6756^{24}$, $7073^{24}$, $7075^{24}$, $7186^{24}$, $7370^{24}$, $7445^{24}$, $7776^{24}$, $7784^{24}$, $8039^{24}$, $8098^{24}$, $8390^{24}$, $8691^{24}$, $8922^{24}$, $9029^{24}$, $9074^{24}$, $9161^{24}$, $9220^{24}$, $10157^{24}$, $10200^{24}$, $10603^{24}$, $10961^{24}$, $11187^{24}$, $11395^{24}$, $11547^{24}$, $13733^{24}$, $15858^{24}$, $16158^{24}$, $16397^{24}$, $20002^{24}$, $33639^{24}$
\item $(h,k)=(17,23)$ : $1^{64}$, $3^{108}$, $4^{36}$, $5^{48}$, $6^{8}$, $8^{78}$, $9^{8}$, $10^{48}$, $12^{8}$, $13^{12}$, $14^{24}$, $16^{36}$, $24^{18}$, $28^{24}$, $29^{12}$, $31^{12}$, $32^{48}$, $35^{12}$, $39^{12}$, $40^{6}$, $52^{12}$, $60^{24}$, $76^{48}$, $82^{24}$, $88^{18}$, $139^{48}$, $158^{24}$, $164^{12}$, $170^{24}$, $200^{48}$, $205^{24}$, $207^{24}$, $220^{24}$, $232^{12}$, $252^{24}$, $258^{24}$, $264^{24}$, $270^{24}$, $283^{24}$, $325^{24}$, $361^{24}$, $382^{24}$, $414^{24}$, $427^{24}$, $429^{24}$, $440^{6}$, $444^{24}$, $455^{24}$, $456^{24}$, $480^{26}$, $482^{24}$, $548^{12}$, $563^{24}$, $564^{24}$, $666^{24}$, $669^{24}$, $695^{24}$, $742^{24}$, $789^{24}$, $806^{24}$, $899^{24}$, $902^{24}$, $967^{24}$, $990^{24}$, $992^{24}$, $1010^{24}$, $1014^{24}$, $1070^{24}$, $1095^{24}$, $1193^{24}$, $1211^{24}$, $1216^{24}$, $1298^{24}$, $1315^{24}$, $1395^{24}$, $1434^{24}$, $1442^{24}$, $1545^{24}$, $1591^{24}$, $1593^{24}$, $1598^{24}$, $1609^{24}$, $1631^{24}$, $1649^{24}$, $1733^{24}$, $1754^{24}$, $1797^{24}$, $1806^{24}$, $1863^{24}$, $1888^{24}$, $1891^{24}$, $1925^{24}$, $1952^{24}$, $2003^{24}$, $2004^{24}$, $2049^{24}$, $2059^{24}$, $2082^{24}$, $2084^{12}$, $2139^{24}$, $2180^{24}$, $2192^{24}$, $2195^{24}$, $2285^{24}$, $2331^{24}$, $2384^{24}$, $2392^{12}$, $2444^{24}$, $2449^{24}$, $2524^{24}$, $2537^{24}$, $2620^{24}$, $2642^{24}$, $2671^{24}$, $2913^{24}$, $2928^{24}$, $2940^{24}$, $2957^{24}$, $3073^{24}$, $3119^{24}$, $3137^{24}$, $3138^{24}$, $3176^{24}$, $3202^{24}$, $3262^{24}$, $3555^{24}$, $3592^{24}$, $3594^{24}$, $3642^{24}$, $3876^{24}$, $3886^{24}$, $3949^{24}$, $4028^{24}$, $4032^{24}$, $4071^{24}$, $4120^{24}$, $4170^{24}$, $4196^{24}$, $4236^{24}$, $4321^{24}$, $4442^{24}$, $4557^{24}$, $4718^{24}$, $4734^{48}$, $4786^{24}$, $4793^{24}$, $4983^{24}$, $5011^{24}$, $5100^{12}$, $5148^{24}$, $5166^{24}$, $5204^{24}$, $5276^{24}$, $5292^{24}$, $5328^{24}$, $5448^{24}$, $5525^{24}$, $5669^{24}$, $5717^{24}$, $5728^{24}$, $5812^{24}$, $5840^{12}$, $6081^{24}$, $6112^{24}$, $6196^{24}$, $6281^{24}$, $6344^{24}$, $6496^{12}$, $6606^{24}$, $6675^{24}$, $6921^{24}$, $7121^{24}$, $7156^{24}$, $7190^{24}$, $7250^{24}$, $7298^{24}$, $7607^{24}$, $7700^{24}$, $7898^{24}$, $7984^{24}$, $7986^{24}$, $8033^{24}$, $8194^{24}$, $8327^{24}$, $8432^{24}$, $8941^{24}$, $8996^{24}$, $9058^{24}$, $9178^{24}$, $9243^{24}$, $9264^{24}$, $9524^{24}$, $9732^{24}$, $9740^{24}$, $9769^{24}$, $10003^{24}$, $10134^{24}$, $10195^{24}$, $10610^{24}$, $10729^{24}$, $11117^{24}$, $11617^{24}$, $12798^{24}$, $13927^{24}$, $19945^{24}$, $30000^{6}$
\item $(h,k)=(19,23)$ : $1^{16}$, $3^{40}$, $4^{20}$, $7^{32}$, $8^{14}$, $11^{20}$, $12^{8}$, $16^{12}$, $19^{12}$, $21^{24}$, $24^{16}$, $27^{8}$, $30^{24}$, $35^{24}$, $43^{24}$, $57^{8}$, $69^{24}$, $71^{24}$, $95^{24}$, $96^{24}$, $112^{24}$, $174^{32}$, $179^{24}$, $230^{24}$, $235^{48}$, $240^{24}$, $243^{16}$, $295^{24}$, $301^{24}$, $313^{24}$, $349^{24}$, $367^{24}$, $402^{24}$, $406^{24}$, $414^{8}$, $421^{24}$, $426^{24}$, $466^{24}$, $486^{24}$, $538^{24}$, $559^{24}$, $565^{24}$, $590^{24}$, $646^{24}$, $649^{24}$, $674^{24}$, $675^{24}$, $678^{24}$, $699^{24}$, $709^{24}$, $715^{24}$, $751^{48}$, $754^{24}$, $781^{24}$, $795^{24}$, $853^{24}$, $873^{24}$, $881^{24}$, $924^{24}$, $966^{24}$, $1066^{24}$, $1080^{16}$, $1105^{48}$, $1106^{24}$, $1197^{24}$, $1215^{24}$, $1229^{24}$, $1260^{24}$, $1316^{24}$, $1346^{24}$, $1355^{24}$, $1373^{24}$, $1419^{24}$, $1471^{24}$, $1472^{24}$, $1524^{12}$, $1588^{24}$, $1590^{24}$, $1632^{24}$, $1637^{24}$, $1656^{24}$, $1694^{24}$, $1696^{24}$, $1754^{24}$, $1776^{24}$, $1782^{24}$, $1863^{24}$, $1865^{24}$, $1884^{24}$, $1923^{24}$, $1970^{24}$, $1981^{24}$, $2023^{24}$, $2045^{24}$, $2082^{24}$, $2177^{24}$, $2213^{24}$, $2229^{24}$, $2282^{24}$, $2291^{24}$, $2409^{24}$, $2450^{24}$, $2463^{24}$, $2480^{48}$, $2519^{24}$, $2521^{24}$, $2539^{24}$, $2544^{24}$, $2549^{24}$, $2625^{24}$, $2642^{24}$, $2663^{24}$, $2712^{24}$, $2730^{24}$, $2784^{2}$, $2789^{24}$, $2866^{24}$, $2875^{24}$, $2901^{24}$, $3027^{24}$, $3091^{24}$, $3168^{24}$, $3375^{24}$, $3444^{24}$, $3557^{24}$, $3704^{24}$, $3748^{24}$, $3831^{24}$, $3842^{24}$, $3928^{24}$, $4102^{24}$, $4111^{24}$, $4138^{24}$, $4156^{24}$, $4165^{24}$, $4338^{16}$, $4465^{24}$, $4474^{24}$, $4621^{24}$, $4636^{24}$, $4691^{24}$, $4735^{24}$, $4785^{24}$, $4824^{4}$, $4834^{24}$, $4901^{24}$, $4948^{24}$, $5157^{24}$, $5199^{24}$, $5234^{24}$, $5313^{24}$, $5410^{24}$, $5425^{24}$, $5511^{24}$, $5599^{24}$, $5677^{24}$, $5873^{24}$, $5922^{24}$, $6424^{24}$, $6505^{24}$, $6550^{24}$, $6745^{24}$, $6757^{24}$, $7176^{24}$, $7201^{24}$, $7518^{24}$, $7659^{24}$, $8119^{24}$, $8540^{24}$, $8700^{24}$, $9096^{24}$, $9141^{24}$, $9197^{24}$, $9234^{24}$, $9356^{24}$, $9607^{24}$, $9623^{24}$, $9844^{24}$, $10192^{24}$, $10330^{24}$, $10504^{24}$, $10805^{24}$, $10916^{24}$, $11273^{24}$, $11730^{24}$, $12336^{24}$, $12486^{24}$, $13397^{24}$, $13797^{24}$, $17599^{24}$, $19687^{24}$, $20170^{24}$, $32358^{24}$, $37072^{6}$
\end{itemize}

\section{Results for $n=25$}
\begin{itemize}
\item $(h,k)=(1,2)$ : $1^{2}$, $5^{2}$, $10^{2}$, $15^{560}$, $17^{950}$, $19^{350}$, $20^{130}$, $25^{6}$, $26^{100}$, $30^{400}$, $35^{30}$, $37^{50}$, $39^{50}$, $41^{50}$, $44^{800}$, $45^{590}$, $46^{650}$, $50^{866}$, $52^{400}$, $54^{100}$, $57^{300}$, $59^{200}$, $60^{590}$, $63^{200}$, $65^{130}$, $66^{350}$, $70^{320}$, $71^{150}$, $72^{50}$, $74^{500}$, $75^{180}$, $76^{200}$, $78^{100}$, $80^{380}$, $81^{200}$, $82^{200}$, $84^{200}$, $90^{700}$, $91^{200}$, $93^{50}$, $97^{300}$, $98^{200}$, $100^{118}$, $105^{40}$, $106^{400}$, $108^{250}$, $110^{290}$, $112^{200}$, $115^{10}$, $117^{500}$, $118^{50}$, $120^{480}$, $124^{150}$, $125^{224}$, $127^{100}$, $130^{510}$, $140^{440}$, $145^{160}$, $149^{150}$, $150^{196}$, $155^{40}$, $160^{100}$, $164^{50}$, $166^{100}$, $170^{20}$, $173^{100}$, $175^{86}$, $176^{100}$, $179^{100}$, $180^{310}$, $182^{100}$, $184^{150}$, $185^{10}$, $188^{150}$, $190^{120}$, $195^{50}$, $200^{340}$, $205^{50}$, $210^{170}$, $215^{100}$, $220^{320}$, $225^{380}$, $228^{50}$, $230^{80}$, $236^{150}$, $238^{50}$, $240^{130}$, $245^{290}$, $246^{50}$, $250^{270}$, $255^{160}$, $256^{50}$, $260^{40}$, $265^{50}$, $270^{150}$, $273^{50}$, $274^{50}$, $275^{308}$, $280^{30}$, $288^{100}$, $290^{60}$, $295^{10}$, $296^{50}$, $299^{100}$, $300^{360}$, $315^{80}$, $319^{50}$, $320^{150}$, $325^{40}$, $330^{160}$, $335^{50}$, $340^{20}$, $345^{110}$, $350^{808}$, $353^{50}$, $354^{50}$, $358^{50}$, $360^{100}$, $362^{50}$, $365^{20}$, $369^{50}$, $370^{170}$, $375^{58}$, $380^{60}$, $385^{60}$, $387^{50}$, $390^{90}$, $399^{50}$, $400^{260}$, $405^{40}$, $410^{20}$, $415^{80}$, $420^{10}$, $440^{20}$, $445^{40}$, $450^{398}$, $460^{20}$, $465^{40}$, $470^{90}$, $475^{100}$, $480^{20}$, $482^{50}$, $490^{70}$, $500^{326}$, $510^{60}$, $515^{10}$, $520^{30}$, $525^{66}$, $535^{40}$, $540^{20}$, $544^{50}$, $545^{70}$, $550^{220}$, $555^{40}$, $559^{50}$, $560^{110}$, $565^{60}$, $570^{80}$, $575^{214}$, $580^{80}$, $598^{50}$, $600^{264}$, $610^{40}$, $615^{10}$, $625^{32}$, $630^{40}$, $645^{20}$, $650^{182}$, $655^{70}$, $670^{30}$, $675^{132}$, $680^{20}$, $690^{40}$, $700^{100}$, $710^{20}$, $720^{20}$, $725^{150}$, $730^{20}$, $750^{228}$, $760^{20}$, $770^{30}$, $775^{112}$, $790^{40}$, $795^{20}$, $800^{140}$, $805^{10}$, $820^{30}$, $825^{50}$, $830^{60}$, $840^{20}$, $850^{106}$, $860^{20}$, $865^{10}$, $870^{50}$, $875^{66}$, $880^{20}$, $885^{40}$, $890^{50}$, $900^{146}$, $920^{50}$, $925^{42}$, $935^{10}$, $945^{30}$, $950^{206}$, $955^{30}$, $960^{40}$, $965^{20}$, $975^{30}$, $980^{10}$, $985^{20}$, $1000^{136}$, $1010^{20}$, $1020^{50}$, $1025^{62}$, $1035^{20}$, $1045^{20}$, $1050^{166}$, $1075^{4}$, $1080^{10}$, $1085^{10}$, $1100^{110}$, $1125^{110}$, $1140^{40}$, $1150^{168}$, $1165^{10}$, $1170^{10}$, $1175^{40}$, $1180^{20}$, $1185^{20}$, $1200^{50}$, $1210^{10}$, $1215^{10}$, $1220^{20}$, $1225^{64}$, $1250^{116}$, $1260^{10}$, $1270^{20}$, $1275^{46}$, $1290^{10}$, $1295^{30}$, $1300^{120}$, $1305^{30}$, $1310^{20}$, $1325^{86}$, $1340^{30}$, $1350^{128}$, $1370^{10}$, $1375^{42}$, $1380^{10}$, $1400^{152}$, $1425^{16}$, $1430^{10}$, $1445^{10}$, $1450^{108}$, $1460^{20}$, $1465^{10}$, $1470^{20}$, $1475^{18}$, $1480^{10}$, $1500^{58}$, $1525^{10}$, $1545^{10}$, $1550^{116}$, $1555^{10}$, $1560^{10}$, $1570^{10}$, $1575^{36}$, $1580^{30}$, $1585^{10}$, $1600^{60}$, $1605^{10}$, $1615^{20}$, $1625^{68}$, $1650^{44}$, $1660^{10}$, $1670^{30}$, $1675^{12}$, $1700^{98}$, $1725^{32}$, $1750^{52}$, $1760^{10}$, $1765^{10}$, $1775^{26}$, $1790^{20}$, $1800^{60}$, $1810^{10}$, $1815^{10}$, $1825^{32}$, $1830^{10}$, $1835^{10}$, $1850^{94}$, $1875^{24}$, $1880^{20}$, $1895^{10}$, $1900^{64}$, $1905^{10}$, $1910^{10}$, $1915^{10}$, $1925^{6}$, $1945^{10}$, $1950^{90}$, $1955^{10}$, $1975^{12}$, $2000^{46}$, $2025^{26}$, $2030^{10}$, $2040^{10}$, $2050^{130}$, $2075^{20}$, $2100^{94}$, $2125^{14}$, $2140^{10}$, $2150^{52}$, $2175^{36}$, $2200^{80}$, $2225^{18}$, $2230^{10}$, $2250^{26}$, $2275^{18}$, $2300^{32}$, $2305^{10}$, $2310^{10}$, $2325^{24}$, $2340^{10}$, $2350^{56}$, $2375^{20}$, $2395^{10}$, $2400^{62}$, $2410^{10}$, $2420^{10}$, $2450^{58}$, $2475^{38}$, $2500^{26}$, $2525^{26}$, $2550^{20}$, $2575^{32}$, $2585^{10}$, $2600^{54}$, $2625^{10}$, $2650^{30}$, $2675^{10}$, $2680^{10}$, $2690^{10}$, $2700^{30}$, $2750^{22}$, $2775^{8}$, $2800^{48}$, $2825^{16}$, $2850^{56}$, $2870^{10}$, $2875^{6}$, $2900^{20}$, $2925^{28}$, $2950^{56}$, $2965^{10}$, $2975^{12}$, $3000^{60}$, $3025^{14}$, $3050^{14}$, $3075^{26}$, $3100^{46}$, $3125^{8}$, $3150^{46}$, $3175^{10}$, $3200^{20}$, $3225^{14}$, $3230^{10}$, $3250^{32}$, $3275^{6}$, $3300^{20}$, $3325^{10}$, $3345^{10}$, $3350^{38}$, $3375^{18}$, $3400^{22}$, $3425^{16}$, $3450^{30}$, $3475^{4}$, $3500^{24}$, $3525^{10}$, $3550^{28}$, $3575^{14}$, $3600^{16}$, $3625^{6}$, $3650^{30}$, $3675^{6}$, $3700^{32}$, $3725^{28}$, $3750^{22}$, $3775^{4}$, $3800^{18}$, $3825^{6}$, $3850^{30}$, $3875^{4}$, $3900^{30}$, $3950^{22}$, $3975^{10}$, $4000^{20}$, $4025^{12}$, $4050^{44}$, $4075^{6}$, $4100^{18}$, $4125^{16}$, $4150^{30}$, $4175^{8}$, $4200^{28}$, $4250^{20}$, $4275^{8}$, $4300^{16}$, $4325^{6}$, $4350^{16}$, $4375^{4}$, $4400^{16}$, $4425^{4}$, $4450^{22}$, $4475^{6}$, $4500^{22}$, $4525^{6}$, $4550^{20}$, $4575^{4}$, $4600^{12}$, $4625^{4}$, $4650^{22}$, $4675^{2}$, $4700^{4}$, $4750^{22}$, $4800^{22}$, $4825^{4}$, $4850^{22}$, $4875^{8}$, $4900^{12}$, $4925^{8}$, $4950^{16}$, $4975^{10}$, $5025^{8}$, $5050^{14}$, $5075^{4}$, $5100^{12}$, $5125^{8}$, $5150^{12}$, $5175^{8}$, $5200^{10}$, $5225^{10}$, $5250^{26}$, $5275^{2}$, $5300^{22}$, $5325^{18}$, $5350^{12}$, $5375^{6}$, $5400^{8}$, $5450^{8}$, $5475^{8}$, $5500^{14}$, $5550^{20}$, $5575^{4}$, $5600^{10}$, $5625^{12}$, $5650^{16}$, $5675^{12}$, $5700^{32}$, $5725^{10}$, $5750^{16}$, $5775^{14}$, $5800^{8}$, $5850^{12}$, $5875^{4}$, $5900^{12}$, $5925^{6}$, $5950^{24}$, $5975^{6}$, $6000^{14}$, $6025^{6}$, $6050^{4}$, $6075^{6}$, $6100^{10}$, $6125^{10}$, $6150^{10}$, $6175^{4}$, $6200^{14}$, $6225^{8}$, $6250^{4}$, $6275^{6}$, $6300^{8}$, $6325^{8}$, $6350^{6}$, $6375^{8}$, $6400^{18}$, $6450^{8}$, $6500^{8}$, $6525^{4}$, $6550^{20}$, $6625^{6}$, $6650^{6}$, $6700^{12}$, $6725^{8}$, $6750^{4}$, $6800^{4}$, $6825^{10}$, $6850^{14}$, $6875^{4}$, $6900^{2}$, $6925^{4}$, $6950^{18}$, $6975^{8}$, $7000^{22}$, $7025^{6}$, $7050^{2}$, $7075^{4}$, $7100^{6}$, $7125^{2}$, $7150^{2}$, $7200^{8}$, $7225^{4}$, $7250^{6}$, $7275^{4}$, $7300^{6}$, $7325^{4}$, $7350^{6}$, $7375^{6}$, $7400^{4}$, $7425^{4}$, $7450^{2}$, $7475^{2}$, $7500^{4}$, $7525^{2}$, $7550^{10}$, $7575^{6}$, $7600^{8}$, $7625^{4}$, $7650^{4}$, $7675^{6}$, $7700^{8}$, $7725^{4}$, $7750^{8}$, $7775^{8}$, $7800^{2}$, $7825^{4}$, $7850^{2}$, $7875^{4}$, $7900^{2}$, $7925^{2}$, $7950^{6}$, $8000^{4}$, $8025^{2}$, $8050^{12}$, $8075^{2}$, $8100^{2}$, $8125^{4}$, $8150^{4}$, $8175^{10}$, $8200^{10}$, $8225^{4}$, $8250^{4}$, $8275^{4}$, $8300^{4}$, $8350^{6}$, $8375^{4}$, $8425^{6}$, $8450^{6}$, $8475^{4}$, $8500^{8}$, $8525^{8}$, $8550^{4}$, $8575^{4}$, $8600^{6}$, $8650^{4}$, $8675^{8}$, $8700^{6}$, $8725^{4}$, $8750^{10}$, $8775^{2}$, $8800^{2}$, $8875^{4}$, $8900^{2}$, $8925^{4}$, $8950^{2}$, $8975^{6}$, $9000^{2}$, $9025^{4}$, $9050^{4}$, $9075^{6}$, $9100^{18}$, $9150^{4}$, $9200^{6}$, $9225^{4}$, $9275^{2}$, $9300^{12}$, $9325^{4}$, $9350^{2}$, $9375^{6}$, $9400^{2}$, $9425^{2}$, $9450^{2}$, $9475^{4}$, $9525^{2}$, $9550^{2}$, $9650^{2}$, $9675^{2}$, $9725^{4}$, $9750^{2}$, $9775^{4}$, $9825^{2}$, $9850^{8}$, $9875^{8}$, $9950^{6}$, $9975^{2}$, $10000^{6}$, $10100^{2}$, $10125^{4}$, $10150^{2}$, $10175^{2}$, $10200^{4}$, $10225^{2}$, $10250^{2}$, $10275^{10}$, $10300^{2}$, $10350^{4}$, $10375^{2}$, $10400^{4}$, $10425^{2}$, $10450^{2}$, $10500^{2}$, $10525^{2}$, $10550^{2}$, $10575^{2}$, $10650^{2}$, $10675^{2}$, $10700^{6}$, $10725^{4}$, $10800^{2}$, $10825^{2}$, $10850^{4}$, $10900^{4}$, $10925^{2}$, $11000^{4}$, $11025^{2}$, $11050^{2}$, $11075^{2}$, $11125^{2}$, $11200^{4}$, $11225^{4}$, $11250^{2}$, $11325^{2}$, $11400^{4}$, $11500^{2}$, $11525^{2}$, $11550^{6}$, $11625^{2}$, $11650^{6}$, $11675^{2}$, $11700^{2}$, $11750^{4}$, $11800^{2}$, $11825^{2}$, $11925^{2}$, $11950^{2}$, $12000^{2}$, $12025^{4}$, $12050^{6}$, $12100^{4}$, $12150^{2}$, $12175^{2}$, $12200^{2}$, $12225^{4}$, $12250^{2}$, $12400^{4}$, $12450^{6}$, $12525^{2}$, $12575^{4}$, $12700^{6}$, $12750^{2}$, $12800^{4}$, $12850^{2}$, $12900^{2}$, $12950^{2}$, $13000^{4}$, $13050^{2}$, $13075^{2}$, $13150^{6}$, $13175^{2}$, $13200^{6}$, $13225^{2}$, $13275^{2}$, $13325^{2}$, $13350^{2}$, $13375^{2}$, $13450^{2}$, $13525^{2}$, $13625^{2}$, $13650^{2}$, $13900^{2}$, $13925^{2}$, $14025^{2}$, $14050^{2}$, $14175^{2}$, $14225^{2}$, $14250^{2}$, $14450^{2}$, $14650^{2}$, $14700^{2}$, $14725^{4}$, $14750^{2}$, $14800^{2}$, $14925^{2}$, $14950^{2}$, $15050^{2}$, $15125^{2}$, $15250^{4}$, $15400^{2}$, $15600^{2}$, $15650^{2}$, $15825^{2}$, $15875^{2}$, $16225^{2}$, $16325^{2}$, $16350^{2}$, $16375^{2}$, $16500^{2}$, $16550^{2}$, $16575^{2}$, $16650^{2}$, $16900^{2}$, $17075^{2}$, $17600^{2}$, $17650^{2}$, $17700^{2}$, $18400^{2}$, $18725^{2}$, $19150^{2}$, $19275^{2}$, $19600^{2}$, $20300^{2}$, $20625^{2}$, $20875^{2}$, $22000^{2}$, $22675^{2}$
\item $(h,k)=(1,3)$ : $1^{2}$, $15^{12}$, $50^{2}$, $5900^{2}$, $8625^{2}$, $26975^{2}$, $36850^{2}$, $2051535^{10}$, $6441050^{2}$
\item $(h,k)=(1,4)$ : $1^{2}$, $3^{10}$, $8915^{10}$, $9550^{2}$, $15675^{2}$, $262925^{2}$, $784845^{10}$, $12520250^{2}$
\item $(h,k)=(1,6)$ : $1^{32}$, $10^{10}$, $15^{20}$, $20^{30}$, $25^{10}$, $35^{60}$, $40^{10}$, $50^{2}$, $80^{50}$, $90^{10}$, $105^{10}$, $160^{10}$, $425^{2}$, $565^{10}$, $655^{10}$, $670^{10}$, $900^{2}$, $2155^{10}$, $2198^{50}$, $2615^{10}$, $4950^{10}$, $6267^{50}$, $6816^{50}$, $14100^{10}$, $15835^{10}$, $25305^{10}$, $28709^{50}$, $64405^{10}$, $78320^{10}$, $81700^{10}$, $86005^{10}$, $110825^{2}$, $158760^{10}$, $160755^{10}$, $165780^{10}$, $201585^{10}$, $213535^{10}$, $249870^{10}$, $295630^{10}$, $376710^{10}$, $427820^{10}$, $484195^{10}$
\item $(h,k)=(1,7)$ : $1^{2}$, $5^{2}$, $10^{2}$, $475^{2}$, $900^{4}$, $1250^{2}$, $10675^{2}$, $121025^{2}$, $592350^{2}$, $2183650^{2}$, $13865975^{2}$
\item $(h,k)=(1,8)$ : $1^{2}$, $5^{10}$, $15^{2}$, $225^{2}$, $275^{2}$, $475^{2}$, $7160^{10}$, $53225^{2}$, $16687175^{2}$
\item $(h,k)=(1,9)$ : $1^{2}$, $3^{10}$, $164^{50}$, $5050^{2}$, $2161525^{2}$, $14606525^{2}$
\item $(h,k)=(1,11)$ : $1^{32}$, $5^{10}$, $10^{50}$, $15^{20}$, $20^{10}$, $45^{50}$, $60^{10}$, $61^{50}$, $70^{10}$, $85^{10}$, $135^{10}$, $150^{2}$, $175^{12}$, $200^{2}$, $320^{10}$, $335^{20}$, $340^{10}$, $372^{50}$, $385^{10}$, $550^{10}$, $895^{10}$, $1530^{10}$, $1750^{2}$, $2390^{10}$, $3945^{10}$, $4060^{50}$, $4795^{10}$, $5295^{10}$, $5320^{10}$, $7720^{10}$, $15270^{10}$, $18050^{10}$, $22895^{10}$, $28211^{50}$, $35355^{10}$, $37135^{10}$, $44706^{50}$, $46315^{10}$, $62720^{10}$, $69070^{10}$, $71617^{50}$, $87339^{50}$, $117665^{10}$, $144945^{10}$, $146530^{10}$, $378370^{10}$, $464895^{10}$, $470940^{10}$, $539950^{2}$
\item $(h,k)=(1,12)$ : $1^{2}$, $5^{2}$, $10^{2}$, $75^{2}$, $200^{2}$, $3425^{2}$, $13435^{50}$, $28780^{10}$, $324425^{2}$, $338400^{2}$, $364850^{2}$, $2206205^{10}$, $4235025^{2}$
\item $(h,k)=(1,13)$ : $1^{2}$, $15^{2}$, $60^{10}$, $120^{150}$, $900^{1202}$, $1800^{18030}$
\item $(h,k)=(1,14)$ : $1^{2}$, $3^{10}$, $32280^{10}$, $659725^{2}$, $5772600^{2}$, $10183475^{2}$
\item $(h,k)=(1,16)$ : $1^{32}$, $10^{10}$, $21^{50}$, $25^{2}$, $45^{20}$, $50^{10}$, $107^{50}$, $130^{10}$, $245^{10}$, $345^{10}$, $542^{50}$, $790^{10}$, $1330^{50}$, $1360^{10}$, $1825^{2}$, $2665^{10}$, $3280^{10}$, $3880^{10}$, $9067^{50}$, $10525^{10}$, $12635^{10}$, $14235^{10}$, $22240^{10}$, $23330^{10}$, $51230^{10}$, $53880^{10}$, $74055^{10}$, $80125^{10}$, $95945^{10}$, $124885^{10}$, $163865^{10}$, $205895^{10}$, $217810^{10}$, $239495^{10}$, $247930^{10}$, $280535^{10}$, $302350^{10}$, $420930^{10}$, $536920^{10}$, $540375^{2}$
\item $(h,k)=(1,17)$ : $1^{2}$, $5^{2}$, $10^{2}$, $235^{10}$, $6025^{2}$, $16770000^{2}$
\item $(h,k)=(1,18)$ : $1^{2}$, $10^{10}$, $15^{2}$, $100^{2}$, $575^{2}$, $1125^{2}$, $2275^{2}$, $716225^{2}$, $1144025^{2}$, $7008625^{2}$, $7904200^{2}$
\item $(h,k)=(1,19)$ : $1^{2}$, $3^{10}$, $30^{10}$, $55^{10}$, $4975^{2}$, $6150^{2}$, $10350^{2}$, $367083^{50}$, $3231775^{2}$, $4346450^{2}$
\item $(h,k)=(1,21)$ : $1^{32}$, $5^{10}$, $7^{50}$, $8^{50}$, $10^{10}$, $15^{10}$, $20^{10}$, $25^{2}$, $28^{50}$, $75^{10}$, $175^{2}$, $200^{2}$, $250^{10}$, $314^{50}$, $530^{10}$, $740^{10}$, $1630^{10}$, $1814^{50}$, $4094^{50}$, $5710^{10}$, $8300^{10}$, $9025^{2}$, $19536^{50}$, $23248^{50}$, $29975^{2}$, $45180^{10}$, $45685^{10}$, $61640^{10}$, $71670^{10}$, $79520^{50}$, $81225^{10}$, $82450^{50}$, $146885^{10}$, $148450^{2}$, $158915^{10}$, $292775^{10}$, $313870^{10}$, $354275^{2}$, $415800^{10}$, $540990^{10}$
\item $(h,k)=(1,22)$ : $1^{2}$, $5^{2}$, $10^{2}$, $125^{2}$, $295^{10}$, $1533775^{2}$, $15241825^{2}$
\item $(h,k)=(1,23)$ : $1^{2}$, $15^{2}$, $60^{30}$, $90^{200}$, $123^{50}$, $150^{706}$, $180^{280}$, $198^{200}$, $210^{70}$, $213^{150}$, $225^{160}$, $240^{150}$, $243^{200}$, $246^{200}$, $300^{4}$, $315^{40}$, $345^{10}$, $351^{200}$, $360^{200}$, $372^{150}$, $375^{20}$, $420^{200}$, $435^{40}$, $450^{8}$, $510^{20}$, $519^{100}$, $525^{30}$, $540^{140}$, $552^{150}$, $570^{40}$, $600^{76}$, $630^{80}$, $645^{100}$, $660^{280}$, $675^{124}$, $690^{80}$, $714^{50}$, $720^{40}$, $735^{60}$, $738^{50}$, $750^{72}$, $765^{80}$, $768^{50}$, $822^{50}$, $825^{144}$, $870^{10}$, $897^{100}$, $900^{112}$, $945^{80}$, $957^{50}$, $960^{100}$, $975^{34}$, $990^{40}$, $1020^{20}$, $1035^{60}$, $1050^{248}$, $1080^{60}$, $1095^{20}$, $1125^{4}$, $1140^{60}$, $1200^{44}$, $1215^{40}$, $1245^{40}$, $1335^{40}$, $1350^{138}$, $1425^{14}$, $1470^{10}$, $1500^{52}$, $1530^{60}$, $1545^{10}$, $1605^{40}$, $1632^{50}$, $1635^{20}$, $1650^{40}$, $1680^{70}$, $1695^{60}$, $1710^{50}$, $1725^{20}$, $1800^{80}$, $1845^{10}$, $1875^{8}$, $2025^{76}$, $2070^{10}$, $2100^{20}$, $2130^{20}$, $2175^{24}$, $2250^{90}$, $2310^{20}$, $2325^{60}$, $2385^{20}$, $2400^{36}$, $2475^{34}$, $2550^{18}$, $2580^{20}$, $2595^{10}$, $2610^{40}$, $2625^{18}$, $2670^{50}$, $2700^{50}$, $2760^{20}$, $2775^{20}$, $2805^{10}$, $2850^{52}$, $3000^{78}$, $3060^{30}$, $3075^{26}$, $3150^{24}$, $3240^{10}$, $3300^{42}$, $3375^{60}$, $3420^{20}$, $3450^{30}$, $3495^{10}$, $3510^{10}$, $3525^{26}$, $3555^{10}$, $3600^{22}$, $3645^{10}$, $3750^{30}$, $3810^{10}$, $3825^{20}$, $3900^{104}$, $3915^{30}$, $3930^{20}$, $3975^{40}$, $4050^{38}$, $4110^{10}$, $4125^{16}$, $4200^{48}$, $4275^{8}$, $4350^{50}$, $4380^{20}$, $4395^{10}$, $4425^{2}$, $4500^{16}$, $4650^{14}$, $4680^{10}$, $4725^{4}$, $4800^{22}$, $4815^{10}$, $4875^{2}$, $4980^{10}$, $5025^{4}$, $5100^{30}$, $5175^{20}$, $5250^{20}$, $5295^{10}$, $5325^{10}$, $5400^{16}$, $5430^{10}$, $5475^{16}$, $5505^{10}$, $5550^{40}$, $5640^{10}$, $5685^{10}$, $5700^{10}$, $5730^{10}$, $5745^{10}$, $5850^{52}$, $5925^{8}$, $6000^{22}$, $6075^{4}$, $6150^{40}$, $6225^{10}$, $6300^{32}$, $6450^{20}$, $6525^{24}$, $6600^{12}$, $6675^{6}$, $6750^{14}$, $6900^{10}$, $6915^{10}$, $6930^{10}$, $7050^{20}$, $7125^{8}$, $7185^{10}$, $7200^{12}$, $7350^{20}$, $7425^{22}$, $7500^{12}$, $7575^{4}$, $7650^{12}$, $7725^{26}$, $7755^{10}$, $7800^{16}$, $7875^{8}$, $7950^{16}$, $8040^{10}$, $8100^{12}$, $8250^{4}$, $8325^{6}$, $8400^{12}$, $8550^{20}$, $8625^{6}$, $8775^{16}$, $8850^{32}$, $8925^{4}$, $9000^{26}$, $9075^{4}$, $9150^{8}$, $9300^{20}$, $9375^{4}$, $9450^{12}$, $9525^{8}$, $9600^{8}$, $9690^{10}$, $9750^{4}$, $9900^{8}$, $10050^{6}$, $10125^{10}$, $10275^{8}$, $10350^{6}$, $10500^{16}$, $10575^{4}$, $10650^{10}$, $10725^{6}$, $10875^{6}$, $10950^{8}$, $11100^{8}$, $11175^{14}$, $11250^{20}$, $11400^{10}$, $11475^{2}$, $11550^{12}$, $11700^{20}$, $11850^{8}$, $12000^{4}$, $12075^{10}$, $12150^{14}$, $12300^{4}$, $12375^{8}$, $12450^{18}$, $12525^{2}$, $12600^{6}$, $12750^{12}$, $12825^{4}$, $12900^{8}$, $12975^{4}$, $13050^{10}$, $13200^{16}$, $13275^{4}$, $13350^{8}$, $13650^{4}$, $13800^{4}$, $13950^{12}$, $14025^{2}$, $14250^{10}$, $14400^{14}$, $14550^{20}$, $14700^{8}$, $14775^{4}$, $14850^{2}$, $15075^{2}$, $15150^{10}$, $15300^{6}$, $15375^{4}$, $15525^{2}$, $15750^{8}$, $15900^{8}$, $15975^{4}$, $16050^{8}$, $16200^{8}$, $16425^{4}$, $16500^{6}$, $16650^{12}$, $16725^{4}$, $16800^{2}$, $16950^{4}$, $17025^{2}$, $17100^{18}$, $17175^{4}$, $17250^{8}$, $17325^{8}$, $17400^{4}$, $17550^{4}$, $17625^{4}$, $17700^{8}$, $17775^{2}$, $17850^{6}$, $17925^{2}$, $18000^{6}$, $18150^{4}$, $18225^{4}$, $18300^{2}$, $18375^{6}$, $18450^{2}$, $18525^{4}$, $18600^{4}$, $18825^{2}$, $18975^{2}$, $19050^{4}$, $19125^{2}$, $19200^{8}$, $19350^{2}$, $19500^{8}$, $19575^{2}$, $19650^{6}$, $19950^{2}$, $20100^{2}$, $20400^{4}$, $20475^{6}$, $20550^{2}$, $20625^{2}$, $20850^{4}$, $20925^{2}$, $21000^{6}$, $21075^{4}$, $21225^{2}$, $21300^{2}$, $21450^{2}$, $21675^{4}$, $21975^{4}$, $22125^{2}$, $22275^{2}$, $22500^{2}$, $22650^{4}$, $22800^{4}$, $23025^{2}$, $23100^{6}$, $23250^{4}$, $23325^{2}$, $23850^{2}$, $24075^{2}$, $24150^{2}$, $24375^{2}$, $24450^{2}$, $24525^{4}$, $24825^{2}$, $24900^{2}$, $25050^{2}$, $25275^{4}$, $25350^{4}$, $25575^{2}$, $25650^{2}$, $25800^{2}$, $25950^{2}$, $26100^{4}$, $26250^{2}$, $26400^{2}$, $26700^{2}$, $26850^{2}$, $26925^{6}$, $27000^{2}$, $27225^{4}$, $27300^{14}$, $27450^{2}$, $27600^{2}$, $27900^{2}$, $27975^{2}$, $28050^{2}$, $28125^{2}$, $28200^{2}$, $28425^{2}$, $28650^{2}$, $29025^{2}$, $29175^{4}$, $29325^{4}$, $30000^{4}$, $30375^{2}$, $30525^{2}$, $30750^{2}$, $30825^{4}$, $31350^{2}$, $31725^{2}$, $32175^{2}$, $32550^{4}$, $33600^{2}$, $33675^{2}$, $34200^{2}$, $34875^{2}$, $35400^{2}$, $35475^{2}$, $36675^{2}$, $37350^{2}$, $37725^{2}$, $38100^{4}$, $38400^{2}$, $38550^{2}$, $38850^{2}$, $39000^{2}$, $39225^{2}$, $39450^{2}$, $39600^{2}$, $39825^{2}$, $40050^{2}$, $40125^{2}$, $40350^{2}$, $41775^{2}$, $42075^{2}$, $42525^{2}$, $42750^{2}$, $43350^{2}$, $44100^{2}$, $44175^{2}$, $44250^{2}$, $44850^{2}$, $45375^{2}$, $46950^{2}$, $47625^{2}$, $49125^{2}$, $50700^{2}$, $52800^{2}$, $52950^{2}$, $57450^{2}$, $60900^{2}$, $62625^{2}$
\item $(h,k)=(1,24)$ : $1^{2}$, $3^{11184810}$
\item $(h,k)=(2,3)$ : $1^{2}$, $3^{10}$, $5^{50}$, $50^{2}$, $1450^{2}$, $4025^{2}$, $9700^{2}$, $37275^{2}$, $79375^{2}$, $1489850^{2}$, $2798975^{2}$, $12356375^{2}$
\item $(h,k)=(2,4)$ : $1^{2}$, $15^{2}$, $50^{2}$, $135^{10}$, $300^{2}$, $525^{2}$, $1190^{10}$, $1325^{2}$, $2725^{2}$, $2840^{10}$, $26350^{2}$, $84625^{2}$, $191475^{2}$, $216575^{2}$, $326475^{2}$, $662000^{2}$, $675975^{2}$, $14567975^{2}$
\item $(h,k)=(2,6)$ : $1^{2}$, $5^{2}$, $10^{2}$, $25^{10}$, $35^{10}$, $100^{2}$, $3000^{2}$, $17133^{50}$, $111925^{2}$, $173250^{2}$, $16060300^{2}$
\item $(h,k)=(2,7)$ : $1^{32}$, $5^{20}$, $8^{50}$, $10^{10}$, $15^{10}$, $30^{10}$, $65^{10}$, $160^{10}$, $234^{50}$, $235^{10}$, $285^{10}$, $455^{10}$, $860^{10}$, $955^{10}$, $1107^{50}$, $1155^{10}$, $1455^{10}$, $1475^{2}$, $5390^{10}$, $6375^{2}$, $8770^{10}$, $12690^{50}$, $18555^{10}$, $19800^{10}$, $22070^{10}$, $23015^{10}$, $25675^{2}$, $32977^{50}$, $59185^{10}$, $67190^{10}$, $85650^{2}$, $89325^{10}$, $90025^{10}$, $170725^{10}$, $262405^{10}$, $341180^{10}$, $364390^{10}$, $423025^{2}$, $462200^{10}$, $468080^{10}$, $533925^{10}$
\item $(h,k)=(2,8)$ : $1^{2}$, $3^{10}$, $750^{2}$, $7625^{10}$, $27650^{2}$, $1867875^{2}$, $14842800^{2}$
\item $(h,k)=(2,9)$ : $1^{2}$, $15^{2}$, $125^{2}$, $275^{2}$, $325^{2}$, $350625^{2}$, $7793525^{2}$, $8632325^{2}$
\item $(h,k)=(2,11)$ : $1^{2}$, $5^{2}$, $10^{2}$, $350^{2}$, $1150^{2}$, $1825^{2}$, $2315^{10}$, $61325^{2}$, $135725^{2}$, $16565250^{2}$
\item $(h,k)=(2,12)$ : $1^{32}$, $5^{10}$, $15^{10}$, $40^{10}$, $60^{10}$, $100^{2}$, $125^{10}$, $210^{10}$, $225^{10}$, $245^{20}$, $295^{10}$, $325^{2}$, $752^{50}$, $1615^{10}$, $2035^{10}$, $2070^{10}$, $2490^{10}$, $2515^{10}$, $2670^{10}$, $2825^{2}$, $4102^{50}$, $5640^{10}$, $6072^{50}$, $8484^{50}$, $8515^{10}$, $10335^{10}$, $10670^{10}$, $10900^{2}$, $12240^{10}$, $16580^{10}$, $21840^{10}$, $33028^{50}$, $46000^{2}$, $51040^{50}$, $51540^{10}$, $53740^{10}$, $56056^{50}$, $73480^{10}$, $87965^{10}$, $118630^{10}$, $165345^{10}$, $189585^{10}$, $191615^{10}$, $265470^{10}$, $307420^{10}$, $404705^{10}$, $439150^{10}$, $482075^{2}$
\item $(h,k)=(2,13)$ : $1^{2}$, $3^{10}$, $17050^{2}$, $20025^{2}$, $30575^{2}$, $82675^{2}$, $153450^{2}$, $232650^{2}$, $338840^{50}$, $7769775^{2}$
\item $(h,k)=(2,14)$ : $1^{2}$, $15^{2}$, $75^{2}$, $300^{2}$, $525^{2}$, $11950^{2}$, $27250^{2}$, $130550^{2}$, $503000^{2}$, $1377300^{2}$, $2945250^{10}$
\item $(h,k)=(2,16)$ : $1^{2}$, $5^{2}$, $10^{2}$, $15^{10}$, $37950^{2}$, $87450^{2}$, $94200^{2}$, $208650^{2}$, $406975^{2}$, $15941900^{2}$
\item $(h,k)=(2,17)$ : $1^{32}$, $5^{40}$, $6^{50}$, $7^{50}$, $15^{10}$, $40^{10}$, $65^{10}$, $150^{2}$, $200^{10}$, $235^{20}$, $250^{2}$, $475^{2}$, $675^{10}$, $889^{50}$, $935^{10}$, $1146^{50}$, $1400^{50}$, $1800^{10}$, $3536^{50}$, $3655^{10}$, $4220^{50}$, $5815^{10}$, $16895^{10}$, $19789^{50}$, $47500^{10}$, $65250^{50}$, $81815^{10}$, $86770^{10}$, $127060^{10}$, $136220^{10}$, $182485^{10}$, $231685^{10}$, $259975^{10}$, $269800^{10}$, $270130^{10}$, $518460^{10}$, $523295^{10}$, $541350^{2}$
\item $(h,k)=(2,18)$ : $1^{2}$, $3^{10}$, $50^{2}$, $3750^{2}$, $23700^{2}$, $45350^{2}$, $122931^{50}$, $13631075^{2}$
\item $(h,k)=(2,19)$ : $1^{2}$, $15^{12}$, $65^{10}$, $400^{2}$, $1400^{2}$, $1925^{2}$, $3200^{2}$, $3925^{2}$, $37925^{50}$, $473425^{2}$, $15344400^{2}$
\item $(h,k)=(2,21)$ : $1^{2}$, $5^{2}$, $10^{2}$, $3800^{10}$, $223825^{2}$, $16534375^{2}$
\item $(h,k)=(2,22)$ : $1^{32}$, $5^{20}$, $10^{20}$, $15^{10}$, $20^{10}$, $25^{10}$, $40^{10}$, $45^{50}$, $90^{10}$, $105^{10}$, $135^{10}$, $177^{50}$, $290^{10}$, $310^{10}$, $315^{10}$, $325^{10}$, $520^{10}$, $660^{10}$, $670^{10}$, $685^{10}$, $770^{10}$, $1085^{10}$, $2201^{50}$, $3645^{10}$, $9265^{10}$, $10405^{10}$, $11510^{10}$, $27388^{50}$, $29150^{2}$, $44700^{2}$, $97125^{2}$, $105798^{50}$, $126220^{10}$, $177255^{10}$, $226670^{10}$, $250180^{10}$, $272900^{2}$, $286490^{10}$, $402515^{10}$, $537275^{10}$, $541100^{10}$
\item $(h,k)=(2,23)$ : $1^{2}$, $3^{11184810}$
\item $(h,k)=(2,24)$ : $1^{2}$, $15^{2}$, $60^{30}$, $90^{200}$, $123^{50}$, $150^{706}$, $180^{280}$, $198^{200}$, $210^{70}$, $213^{150}$, $225^{160}$, $240^{150}$, $243^{200}$, $246^{200}$, $300^{4}$, $315^{40}$, $345^{10}$, $351^{200}$, $360^{200}$, $372^{150}$, $375^{20}$, $420^{200}$, $435^{40}$, $450^{8}$, $510^{20}$, $519^{100}$, $525^{30}$, $540^{140}$, $552^{150}$, $570^{40}$, $600^{76}$, $630^{80}$, $645^{100}$, $660^{280}$, $675^{124}$, $690^{80}$, $714^{50}$, $720^{40}$, $735^{60}$, $738^{50}$, $750^{72}$, $765^{80}$, $768^{50}$, $822^{50}$, $825^{144}$, $870^{10}$, $897^{100}$, $900^{112}$, $945^{80}$, $957^{50}$, $960^{100}$, $975^{34}$, $990^{40}$, $1020^{20}$, $1035^{60}$, $1050^{248}$, $1080^{60}$, $1095^{20}$, $1125^{4}$, $1140^{60}$, $1200^{44}$, $1215^{40}$, $1245^{40}$, $1335^{40}$, $1350^{138}$, $1425^{14}$, $1470^{10}$, $1500^{52}$, $1530^{60}$, $1545^{10}$, $1605^{40}$, $1632^{50}$, $1635^{20}$, $1650^{40}$, $1680^{70}$, $1695^{60}$, $1710^{50}$, $1725^{20}$, $1800^{80}$, $1845^{10}$, $1875^{8}$, $2025^{76}$, $2070^{10}$, $2100^{20}$, $2130^{20}$, $2175^{24}$, $2250^{90}$, $2310^{20}$, $2325^{60}$, $2385^{20}$, $2400^{36}$, $2475^{34}$, $2550^{18}$, $2580^{20}$, $2595^{10}$, $2610^{40}$, $2625^{18}$, $2670^{50}$, $2700^{50}$, $2760^{20}$, $2775^{20}$, $2805^{10}$, $2850^{52}$, $3000^{78}$, $3060^{30}$, $3075^{26}$, $3150^{24}$, $3240^{10}$, $3300^{42}$, $3375^{60}$, $3420^{20}$, $3450^{30}$, $3495^{10}$, $3510^{10}$, $3525^{26}$, $3555^{10}$, $3600^{22}$, $3645^{10}$, $3750^{30}$, $3810^{10}$, $3825^{20}$, $3900^{104}$, $3915^{30}$, $3930^{20}$, $3975^{40}$, $4050^{38}$, $4110^{10}$, $4125^{16}$, $4200^{48}$, $4275^{8}$, $4350^{50}$, $4380^{20}$, $4395^{10}$, $4425^{2}$, $4500^{16}$, $4650^{14}$, $4680^{10}$, $4725^{4}$, $4800^{22}$, $4815^{10}$, $4875^{2}$, $4980^{10}$, $5025^{4}$, $5100^{30}$, $5175^{20}$, $5250^{20}$, $5295^{10}$, $5325^{10}$, $5400^{16}$, $5430^{10}$, $5475^{16}$, $5505^{10}$, $5550^{40}$, $5640^{10}$, $5685^{10}$, $5700^{10}$, $5730^{10}$, $5745^{10}$, $5850^{52}$, $5925^{8}$, $6000^{22}$, $6075^{4}$, $6150^{40}$, $6225^{10}$, $6300^{32}$, $6450^{20}$, $6525^{24}$, $6600^{12}$, $6675^{6}$, $6750^{14}$, $6900^{10}$, $6915^{10}$, $6930^{10}$, $7050^{20}$, $7125^{8}$, $7185^{10}$, $7200^{12}$, $7350^{20}$, $7425^{22}$, $7500^{12}$, $7575^{4}$, $7650^{12}$, $7725^{26}$, $7755^{10}$, $7800^{16}$, $7875^{8}$, $7950^{16}$, $8040^{10}$, $8100^{12}$, $8250^{4}$, $8325^{6}$, $8400^{12}$, $8550^{20}$, $8625^{6}$, $8775^{16}$, $8850^{32}$, $8925^{4}$, $9000^{26}$, $9075^{4}$, $9150^{8}$, $9300^{20}$, $9375^{4}$, $9450^{12}$, $9525^{8}$, $9600^{8}$, $9690^{10}$, $9750^{4}$, $9900^{8}$, $10050^{6}$, $10125^{10}$, $10275^{8}$, $10350^{6}$, $10500^{16}$, $10575^{4}$, $10650^{10}$, $10725^{6}$, $10875^{6}$, $10950^{8}$, $11100^{8}$, $11175^{14}$, $11250^{20}$, $11400^{10}$, $11475^{2}$, $11550^{12}$, $11700^{20}$, $11850^{8}$, $12000^{4}$, $12075^{10}$, $12150^{14}$, $12300^{4}$, $12375^{8}$, $12450^{18}$, $12525^{2}$, $12600^{6}$, $12750^{12}$, $12825^{4}$, $12900^{8}$, $12975^{4}$, $13050^{10}$, $13200^{16}$, $13275^{4}$, $13350^{8}$, $13650^{4}$, $13800^{4}$, $13950^{12}$, $14025^{2}$, $14250^{10}$, $14400^{14}$, $14550^{20}$, $14700^{8}$, $14775^{4}$, $14850^{2}$, $15075^{2}$, $15150^{10}$, $15300^{6}$, $15375^{4}$, $15525^{2}$, $15750^{8}$, $15900^{8}$, $15975^{4}$, $16050^{8}$, $16200^{8}$, $16425^{4}$, $16500^{6}$, $16650^{12}$, $16725^{4}$, $16800^{2}$, $16950^{4}$, $17025^{2}$, $17100^{18}$, $17175^{4}$, $17250^{8}$, $17325^{8}$, $17400^{4}$, $17550^{4}$, $17625^{4}$, $17700^{8}$, $17775^{2}$, $17850^{6}$, $17925^{2}$, $18000^{6}$, $18150^{4}$, $18225^{4}$, $18300^{2}$, $18375^{6}$, $18450^{2}$, $18525^{4}$, $18600^{4}$, $18825^{2}$, $18975^{2}$, $19050^{4}$, $19125^{2}$, $19200^{8}$, $19350^{2}$, $19500^{8}$, $19575^{2}$, $19650^{6}$, $19950^{2}$, $20100^{2}$, $20400^{4}$, $20475^{6}$, $20550^{2}$, $20625^{2}$, $20850^{4}$, $20925^{2}$, $21000^{6}$, $21075^{4}$, $21225^{2}$, $21300^{2}$, $21450^{2}$, $21675^{4}$, $21975^{4}$, $22125^{2}$, $22275^{2}$, $22500^{2}$, $22650^{4}$, $22800^{4}$, $23025^{2}$, $23100^{6}$, $23250^{4}$, $23325^{2}$, $23850^{2}$, $24075^{2}$, $24150^{2}$, $24375^{2}$, $24450^{2}$, $24525^{4}$, $24825^{2}$, $24900^{2}$, $25050^{2}$, $25275^{4}$, $25350^{4}$, $25575^{2}$, $25650^{2}$, $25800^{2}$, $25950^{2}$, $26100^{4}$, $26250^{2}$, $26400^{2}$, $26700^{2}$, $26850^{2}$, $26925^{6}$, $27000^{2}$, $27225^{4}$, $27300^{14}$, $27450^{2}$, $27600^{2}$, $27900^{2}$, $27975^{2}$, $28050^{2}$, $28125^{2}$, $28200^{2}$, $28425^{2}$, $28650^{2}$, $29025^{2}$, $29175^{4}$, $29325^{4}$, $30000^{4}$, $30375^{2}$, $30525^{2}$, $30750^{2}$, $30825^{4}$, $31350^{2}$, $31725^{2}$, $32175^{2}$, $32550^{4}$, $33600^{2}$, $33675^{2}$, $34200^{2}$, $34875^{2}$, $35400^{2}$, $35475^{2}$, $36675^{2}$, $37350^{2}$, $37725^{2}$, $38100^{4}$, $38400^{2}$, $38550^{2}$, $38850^{2}$, $39000^{2}$, $39225^{2}$, $39450^{2}$, $39600^{2}$, $39825^{2}$, $40050^{2}$, $40125^{2}$, $40350^{2}$, $41775^{2}$, $42075^{2}$, $42525^{2}$, $42750^{2}$, $43350^{2}$, $44100^{2}$, $44175^{2}$, $44250^{2}$, $44850^{2}$, $45375^{2}$, $46950^{2}$, $47625^{2}$, $49125^{2}$, $50700^{2}$, $52800^{2}$, $52950^{2}$, $57450^{2}$, $60900^{2}$, $62625^{2}$
\item $(h,k)=(3,4)$ : $1^{2}$, $5^{2}$, $10^{2}$, $100^{2}$, $175^{2}$, $650^{2}$, $1250^{2}$, $1355^{10}$, $1444^{50}$, $8010^{10}$, $311375^{2}$, $1085650^{2}$, $3059015^{10}$
\item $(h,k)=(3,6)$ : $1^{2}$, $15^{2}$, $50^{2}$, $125^{2}$, $850^{2}$, $1150^{2}$, $1475^{2}$, $2400^{2}$, $4450^{2}$, $6050^{2}$, $8175^{2}$, $22670^{50}$, $137225^{2}$, $319425^{2}$, $375375^{2}$, $464075^{2}$, $832250^{2}$, $1263650^{2}$, $3262500^{2}$, $9531225^{2}$
\item $(h,k)=(3,7)$ : $1^{2}$, $3^{10}$, $10^{10}$, $4350^{2}$, $48800^{2}$, $63935^{10}$, $261725^{2}$, $1982200^{2}$, $2829925^{2}$, $11330475^{2}$
\item $(h,k)=(3,8)$ : $1^{32}$, $2^{300}$, $5^{30}$, $8^{50}$, $9^{50}$, $10^{20}$, $20^{10}$, $22^{50}$, $45^{10}$, $50^{12}$, $65^{10}$, $70^{10}$, $96^{50}$, $175^{2}$, $305^{10}$, $320^{10}$, $475^{10}$, $570^{10}$, $605^{10}$, $1000^{10}$, $3015^{10}$, $6565^{10}$, $7160^{10}$, $17918^{50}$, $32475^{2}$, $38335^{10}$, $100540^{50}$, $106668^{50}$, $448305^{10}$, $509425^{2}$, $533240^{10}$, $539735^{10}$, $540735^{10}$
\item $(h,k)=(3,9)$ : $1^{2}$, $5^{2}$, $10^{2}$, $38^{50}$, $75^{10}$, $29975^{2}$, $86330^{10}$, $236975^{2}$, $453773^{50}$, $1366800^{2}$, $3366150^{2}$
\item $(h,k)=(3,11)$ : $1^{2}$, $15^{2}$, $200^{10}$, $850^{2}$, $2720^{10}$, $12175^{2}$, $16749575^{2}$
\item $(h,k)=(3,12)$ : $1^{2}$, $3^{10}$, $5^{10}$, $25^{2}$, $50^{2}$, $150^{2}$, $3165^{10}$, $162850^{2}$, $16598275^{2}$
\item $(h,k)=(3,13)$ : $1^{32}$, $4^{50}$, $5^{60}$, $10^{10}$, $16^{50}$, $20^{10}$, $22^{50}$, $65^{10}$, $70^{10}$, $100^{10}$, $130^{10}$, $235^{10}$, $269^{50}$, $305^{10}$, $400^{2}$, $713^{50}$, $840^{10}$, $865^{10}$, $1310^{10}$, $2030^{10}$, $2050^{10}$, $2357^{50}$, $2805^{10}$, $2820^{10}$, $2915^{10}$, $3015^{10}$, $3930^{10}$, $5110^{10}$, $5472^{50}$, $6515^{10}$, $8390^{10}$, $15150^{10}$, $24745^{50}$, $27525^{10}$, $29730^{10}$, $40055^{10}$, $64025^{10}$, $66215^{10}$, $105805^{10}$, $125085^{10}$, $196150^{10}$, $198025^{10}$, $204200^{2}$, $222170^{10}$, $256310^{10}$, $299745^{10}$, $309790^{10}$, $314370^{10}$, $333980^{10}$, $498840^{10}$
\item $(h,k)=(3,14)$ : $1^{2}$, $5^{2}$, $10^{2}$, $150^{2}$, $195^{10}$, $250^{2}$, $1075^{2}$, $1400^{2}$, $1925^{2}$, $5050^{2}$, $5725^{2}$, $8150^{2}$, $16752500^{2}$
\item $(h,k)=(3,16)$ : $1^{2}$, $15^{2}$, $2600^{2}$, $76575^{2}$, $667921^{50}$
\item $(h,k)=(3,17)$ : $1^{2}$, $3^{210}$, $25^{2}$, $100^{2}$, $110^{50}$, $600^{2}$, $5550^{2}$, $12250^{2}$, $165102^{50}$, $783325^{2}$, $11844750^{2}$
\item $(h,k)=(3,18)$ : $1^{32}$, $5^{60}$, $10^{50}$, $15^{10}$, $20^{10}$, $25^{10}$, $65^{10}$, $105^{10}$, $230^{10}$, $325^{2}$, $360^{10}$, $415^{10}$, $470^{10}$, $654^{50}$, $750^{10}$, $845^{10}$, $2325^{10}$, $3075^{10}$, $3225^{10}$, $3925^{10}$, $4000^{10}$, $4790^{10}$, $5100^{2}$, $5400^{2}$, $5891^{50}$, $6215^{10}$, $7137^{50}$, $8030^{10}$, $9466^{50}$, $10475^{10}$, $15620^{10}$, $21310^{10}$, $25335^{10}$, $34555^{10}$, $37485^{10}$, $49220^{10}$, $49235^{10}$, $57407^{50}$, $66335^{10}$, $98634^{50}$, $196765^{10}$, $359735^{10}$, $444620^{10}$, $468105^{10}$, $491250^{2}$, $541320^{10}$
\item $(h,k)=(3,19)$ : $1^{2}$, $5^{2}$, $10^{2}$, $575^{2}$, $3450^{2}$, $121725^{2}$, $143225^{2}$, $3889125^{2}$, $12619100^{2}$
\item $(h,k)=(3,21)$ : $1^{2}$, $15^{2}$, $200^{2}$, $1125^{2}$, $2975^{2}$, $39805^{10}$, $533625^{2}$, $16040250^{2}$
\item $(h,k)=(3,22)$ : $1^{2}$, $3^{11184810}$
\item $(h,k)=(3,23)$ : $1^{32}$, $5^{20}$, $10^{20}$, $15^{10}$, $20^{10}$, $25^{10}$, $40^{10}$, $45^{50}$, $90^{10}$, $105^{10}$, $135^{10}$, $177^{50}$, $290^{10}$, $310^{10}$, $315^{10}$, $325^{10}$, $520^{10}$, $660^{10}$, $670^{10}$, $685^{10}$, $770^{10}$, $1085^{10}$, $2201^{50}$, $3645^{10}$, $9265^{10}$, $10405^{10}$, $11510^{10}$, $27388^{50}$, $29150^{2}$, $44700^{2}$, $97125^{2}$, $105798^{50}$, $126220^{10}$, $177255^{10}$, $226670^{10}$, $250180^{10}$, $272900^{2}$, $286490^{10}$, $402515^{10}$, $537275^{10}$, $541100^{10}$
\item $(h,k)=(3,24)$ : $1^{2}$, $5^{2}$, $10^{2}$, $125^{2}$, $295^{10}$, $1533775^{2}$, $15241825^{2}$
\item $(h,k)=(4,6)$ : $1^{2}$, $3^{10}$, $75^{2}$, $250^{10}$, $1750^{2}$, $45700^{2}$, $2243375^{10}$, $5511550^{2}$
\item $(h,k)=(4,7)$ : $1^{2}$, $5^{10}$, $15^{2}$, $50^{2}$, $100^{2}$, $7350^{2}$, $9650^{2}$, $25675^{2}$, $29775^{2}$, $362100^{2}$, $501715^{10}$, $2587275^{2}$, $3798325^{2}$, $7448300^{2}$
\item $(h,k)=(4,8)$ : $1^{2}$, $5^{2}$, $10^{2}$, $25^{2}$, $100^{4}$, $125^{2}$, $275^{2}$, $425^{2}$, $428^{50}$, $600^{2}$, $2050^{2}$, $2865^{10}$, $34702^{50}$, $68925^{2}$, $72350^{2}$, $120325^{10}$, $2522425^{2}$, $2866750^{2}$, $9748850^{2}$
\item $(h,k)=(4,9)$ : $1^{32}$, $5^{90}$, $15^{10}$, $30^{10}$, $48^{50}$, $50^{10}$, $70^{10}$, $100^{10}$, $270^{10}$, $305^{20}$, $390^{10}$, $930^{10}$, $995^{10}$, $2190^{10}$, $2195^{10}$, $3125^{2}$, $3179^{50}$, $7105^{10}$, $8760^{10}$, $8930^{10}$, $9695^{10}$, $12500^{2}$, $25220^{10}$, $39870^{10}$, $45521^{50}$, $62840^{10}$, $102890^{10}$, $115685^{10}$, $154225^{2}$, $166310^{10}$, $212520^{10}$, $237180^{10}$, $296270^{10}$, $334280^{10}$, $372250^{2}$, $410560^{10}$, $425780^{10}$, $531495^{10}$
\item $(h,k)=(4,11)$ : $1^{2}$, $3^{10}$, $25^{12}$, $125^{2}$, $1000^{2}$, $3355185^{10}$
\item $(h,k)=(4,12)$ : $1^{2}$, $15^{2}$, $60^{10}$, $6675^{2}$, $51900^{2}$, $118800^{2}$, $16599525^{2}$
\item $(h,k)=(4,13)$ : $1^{2}$, $5^{2}$, $10^{2}$, $50^{2}$, $500^{2}$, $550^{2}$, $5275^{2}$, $137685^{10}$, $143450^{2}$, $15938950^{2}$
\item $(h,k)=(4,14)$ : $1^{32}$, $3^{50}$, $5^{30}$, $10^{10}$, $15^{60}$, $25^{20}$, $55^{10}$, $200^{50}$, $305^{10}$, $345^{10}$, $435^{10}$, $611^{50}$, $1680^{10}$, $2130^{10}$, $3150^{2}$, $3575^{10}$, $3865^{10}$, $4410^{10}$, $7765^{10}$, $14165^{10}$, $19150^{10}$, $68850^{10}$, $87135^{10}$, $108297^{50}$, $141750^{2}$, $155750^{2}$, $192990^{10}$, $241550^{2}$, $260740^{10}$, $454385^{10}$, $513885^{10}$, $527185^{10}$, $538230^{10}$
\item $(h,k)=(4,16)$ : $1^{2}$, $3^{10}$, $75^{2}$, $125^{4}$, $12950^{2}$, $27775^{10}$, $758225^{2}$, $4202225^{2}$, $11664600^{2}$
\item $(h,k)=(4,17)$ : $1^{2}$, $15^{2}$, $85^{10}$, $175^{6}$, $1400^{2}$, $1550^{2}$, $5650^{2}$, $16275^{2}$, $7783525^{2}$, $8967850^{2}$
\item $(h,k)=(4,18)$ : $1^{2}$, $5^{2}$, $10^{2}$, $100^{2}$, $1243900^{2}$, $2040525^{2}$, $3538300^{2}$, $9954375^{2}$
\item $(h,k)=(4,19)$ : $1^{32}$, $5^{10}$, $8^{50}$, $10^{50}$, $15^{10}$, $25^{10}$, $30^{10}$, $40^{10}$, $75^{10}$, $160^{10}$, $220^{10}$, $232^{50}$, $390^{10}$, $625^{2}$, $775^{10}$, $1095^{10}$, $1110^{10}$, $2025^{2}$, $2045^{10}$, $3385^{10}$, $4066^{50}$, $5105^{10}$, $5510^{10}$, $8945^{10}$, $11940^{10}$, $13173^{50}$, $17553^{50}$, $20440^{10}$, $47635^{10}$, $52465^{10}$, $83060^{10}$, $106935^{10}$, $108610^{10}$, $124735^{10}$, $159485^{10}$, $197110^{10}$, $359400^{10}$, $386365^{10}$, $420965^{10}$, $430850^{10}$, $532870^{10}$, $539525^{2}$
\item $(h,k)=(4,21)$ : $1^{2}$, $3^{11184810}$
\item $(h,k)=(4,22)$ : $1^{2}$, $15^{2}$, $200^{2}$, $1125^{2}$, $2975^{2}$, $39805^{10}$, $533625^{2}$, $16040250^{2}$
\item $(h,k)=(4,23)$ : $1^{2}$, $5^{2}$, $10^{2}$, $3800^{10}$, $223825^{2}$, $16534375^{2}$
\item $(h,k)=(4,24)$ : $1^{32}$, $5^{10}$, $7^{50}$, $8^{50}$, $10^{10}$, $15^{10}$, $20^{10}$, $25^{2}$, $28^{50}$, $75^{10}$, $175^{2}$, $200^{2}$, $250^{10}$, $314^{50}$, $530^{10}$, $740^{10}$, $1630^{10}$, $1814^{50}$, $4094^{50}$, $5710^{10}$, $8300^{10}$, $9025^{2}$, $19536^{50}$, $23248^{50}$, $29975^{2}$, $45180^{10}$, $45685^{10}$, $61640^{10}$, $71670^{10}$, $79520^{50}$, $81225^{10}$, $82450^{50}$, $146885^{10}$, $148450^{2}$, $158915^{10}$, $292775^{10}$, $313870^{10}$, $354275^{2}$, $415800^{10}$, $540990^{10}$
\item $(h,k)=(6,7)$ : $1^{2}$, $5^{2}$, $10^{2}$, $14650^{2}$, $14750^{2}$, $39250^{2}$, $117530^{10}$, $169100^{2}$, $249875^{2}$, $697825^{2}$, $1113600^{2}$, $2778100^{10}$
\item $(h,k)=(6,8)$ : $1^{2}$, $15^{2}$, $115^{10}$, $225^{2}$, $950^{2}$, $2425^{2}$, $15100^{2}$, $202525^{2}$, $219800^{2}$, $685675^{2}$, $1970550^{2}$, $13679375^{2}$
\item $(h,k)=(6,9)$ : $1^{2}$, $3^{10}$, $316^{50}$, $12525^{2}$, $19475^{2}$, $1077000^{2}$, $7377250^{2}$, $8283050^{2}$
\item $(h,k)=(6,11)$ : $1^{32}$, $5^{10}$, $7^{50}$, $10^{20}$, $15^{80}$, $25^{10}$, $45^{10}$, $50^{10}$, $85^{10}$, $106^{50}$, $130^{10}$, $140^{10}$, $144^{50}$, $205^{10}$, $217^{50}$, $235^{10}$, $290^{20}$, $320^{20}$, $390^{50}$, $480^{10}$, $960^{10}$, $1405^{10}$, $1470^{10}$, $1575^{2}$, $1970^{10}$, $2460^{10}$, $3129^{50}$, $3950^{2}$, $4320^{10}$, $7150^{2}$, $7455^{10}$, $13425^{2}$, $23360^{50}$, $24364^{50}$, $31850^{10}$, $36025^{10}$, $42950^{10}$, $45630^{10}$, $53190^{10}$, $55745^{10}$, $60940^{20}$, $66435^{10}$, $67852^{50}$, $68000^{2}$, $73688^{50}$, $104265^{10}$, $149990^{10}$, $218110^{10}$, $315005^{10}$, $431325^{2}$, $480925^{10}$, $539270^{10}$
\item $(h,k)=(6,12)$ : $1^{2}$, $5^{2}$, $10^{2}$, $25^{2}$, $45^{10}$, $75^{2}$, $2525^{2}$, $4225^{2}$, $212710^{10}$, $361375^{2}$, $428350^{2}$, $2457200^{2}$, $4399925^{2}$, $8059725^{2}$
\item $(h,k)=(6,13)$ : $1^{2}$, $15^{2}$, $25^{2}$, $50^{2}$, $365^{10}$, $2625^{2}$, $441950^{2}$, $5203925^{2}$, $11126800^{2}$
\item $(h,k)=(6,14)$ : $1^{2}$, $3^{10}$, $20^{10}$, $60^{10}$, $100^{2}$, $200^{2}$, $625^{2}$, $16775875^{2}$
\item $(h,k)=(6,16)$ : $1^{32}$, $5^{60}$, $10^{20}$, $15^{10}$, $25^{2}$, $35^{10}$, $40^{10}$, $50^{10}$, $320^{10}$, $505^{10}$, $645^{10}$, $685^{10}$, $715^{10}$, $880^{10}$, $900^{2}$, $1018^{50}$, $1795^{10}$, $2425^{50}$, $2655^{10}$, $5200^{10}$, $5625^{10}$, $7216^{50}$, $9910^{10}$, $11880^{10}$, $14771^{50}$, $16290^{10}$, $27600^{2}$, $30004^{50}$, $30300^{10}$, $34637^{50}$, $39627^{50}$, $41439^{50}$, $73660^{10}$, $97540^{10}$, $122370^{10}$, $172345^{10}$, $176120^{10}$, $300415^{10}$, $316775^{10}$, $318040^{10}$, $361910^{10}$, $467280^{10}$
\item $(h,k)=(6,17)$ : $1^{2}$, $5^{2}$, $10^{2}$, $150^{2}$, $1450^{2}$, $48050^{2}$, $112975^{2}$, $530100^{2}$, $561475^{2}$, $624040^{10}$, $2480560^{10}$
\item $(h,k)=(6,18)$ : $1^{2}$, $15^{2}$, $118800^{2}$, $1248675^{2}$, $15409725^{2}$
\item $(h,k)=(6,19)$ : $1^{2}$, $3^{11184810}$
\item $(h,k)=(6,21)$ : $1^{32}$, $5^{10}$, $8^{50}$, $10^{50}$, $15^{10}$, $25^{10}$, $30^{10}$, $40^{10}$, $75^{10}$, $160^{10}$, $220^{10}$, $232^{50}$, $390^{10}$, $625^{2}$, $775^{10}$, $1095^{10}$, $1110^{10}$, $2025^{2}$, $2045^{10}$, $3385^{10}$, $4066^{50}$, $5105^{10}$, $5510^{10}$, $8945^{10}$, $11940^{10}$, $13173^{50}$, $17553^{50}$, $20440^{10}$, $47635^{10}$, $52465^{10}$, $83060^{10}$, $106935^{10}$, $108610^{10}$, $124735^{10}$, $159485^{10}$, $197110^{10}$, $359400^{10}$, $386365^{10}$, $420965^{10}$, $430850^{10}$, $532870^{10}$, $539525^{2}$
\item $(h,k)=(6,22)$ : $1^{2}$, $5^{2}$, $10^{2}$, $575^{2}$, $3450^{2}$, $121725^{2}$, $143225^{2}$, $3889125^{2}$, $12619100^{2}$
\item $(h,k)=(6,23)$ : $1^{2}$, $15^{12}$, $65^{10}$, $400^{2}$, $1400^{2}$, $1925^{2}$, $3200^{2}$, $3925^{2}$, $37925^{50}$, $473425^{2}$, $15344400^{2}$
\item $(h,k)=(6,24)$ : $1^{2}$, $3^{10}$, $30^{10}$, $55^{10}$, $4975^{2}$, $6150^{2}$, $10350^{2}$, $367083^{50}$, $3231775^{2}$, $4346450^{2}$
\item $(h,k)=(7,8)$ : $1^{2}$, $3^{10}$, $50^{2}$, $27180^{10}$, $47000^{2}$, $109925^{2}$, $770225^{2}$, $15714100^{2}$
\item $(h,k)=(7,9)$ : $1^{2}$, $15^{2}$, $25^{2}$, $28600^{2}$, $34790^{10}$, $40405^{10}$, $83575^{2}$, $945050^{2}$, $5051750^{2}$, $10292225^{2}$
\item $(h,k)=(7,11)$ : $1^{2}$, $5^{2}$, $10^{2}$, $25^{2}$, $240^{10}$, $450^{2}$, $650^{2}$, $750^{2}$, $3075^{2}$, $253500^{2}$, $721200^{10}$, $12911550^{2}$
\item $(h,k)=(7,12)$ : $1^{32}$, $5^{20}$, $10^{20}$, $15^{10}$, $20^{10}$, $35^{10}$, $40^{10}$, $45^{10}$, $50^{20}$, $55^{10}$, $140^{10}$, $168^{50}$, $190^{10}$, $355^{10}$, $375^{10}$, $420^{10}$, $610^{10}$, $655^{10}$, $830^{10}$, $1025^{2}$, $1210^{10}$, $1265^{10}$, $1678^{50}$, $2552^{50}$, $2884^{50}$, $3335^{10}$, $4780^{10}$, $5050^{2}$, $6160^{10}$, $9810^{10}$, $13945^{10}$, $15135^{10}$, $16975^{10}$, $26500^{10}$, $45400^{10}$, $46405^{10}$, $46750^{10}$, $67665^{10}$, $73950^{10}$, $95570^{10}$, $100265^{10}$, $140210^{10}$, $203040^{10}$, $228150^{2}$, $307900^{2}$, $315150^{10}$, $347355^{10}$, $348965^{10}$, $378415^{10}$, $444735^{10}$, $453700^{10}$
\item $(h,k)=(7,13)$ : $1^{2}$, $3^{10}$, $2175^{2}$, $29650^{2}$, $16745375^{2}$
\item $(h,k)=(7,14)$ : $1^{2}$, $15^{2}$, $25^{4}$, $50^{2}$, $80^{10}$, $300^{2}$, $450^{2}$, $1450^{2}$, $1950^{2}$, $6500^{2}$, $7250^{2}$, $47495^{10}$, $285425^{2}$, $346925^{2}$, $476125^{2}$, $861975^{2}$, $3575225^{2}$, $10975650^{2}$
\item $(h,k)=(7,16)$ : $1^{2}$, $5^{2}$, $10^{2}$, $75^{2}$, $95^{10}$, $200^{2}$, $275^{2}$, $2525^{2}$, $3525^{2}$, $3950^{2}$, $12300^{2}$, $94050^{2}$, $16659825^{2}$
\item $(h,k)=(7,17)$ : $1^{32}$, $3^{50}$, $5^{10}$, $6^{50}$, $15^{10}$, $20^{30}$, $25^{2}$, $29^{50}$, $35^{20}$, $85^{10}$, $90^{10}$, $95^{10}$, $100^{10}$, $120^{50}$, $195^{10}$, $320^{10}$, $490^{50}$, $510^{10}$, $900^{10}$, $1465^{20}$, $1600^{2}$, $2695^{10}$, $2888^{50}$, $5697^{50}$, $13714^{50}$, $14810^{10}$, $20795^{10}$, $45490^{10}$, $49885^{10}$, $55331^{50}$, $82705^{10}$, $101025^{10}$, $120375^{2}$, $134605^{10}$, $135720^{10}$, $167295^{10}$, $177890^{10}$, $247955^{10}$, $264910^{10}$, $292320^{10}$, $327870^{10}$, $347925^{2}$, $360175^{10}$, $438545^{10}$
\item $(h,k)=(7,18)$ : $1^{2}$, $3^{11184810}$
\item $(h,k)=(7,19)$ : $1^{2}$, $15^{2}$, $118800^{2}$, $1248675^{2}$, $15409725^{2}$
\item $(h,k)=(7,21)$ : $1^{2}$, $5^{2}$, $10^{2}$, $100^{2}$, $1243900^{2}$, $2040525^{2}$, $3538300^{2}$, $9954375^{2}$
\item $(h,k)=(7,22)$ : $1^{32}$, $5^{60}$, $10^{50}$, $15^{10}$, $20^{10}$, $25^{10}$, $65^{10}$, $105^{10}$, $230^{10}$, $325^{2}$, $360^{10}$, $415^{10}$, $470^{10}$, $654^{50}$, $750^{10}$, $845^{10}$, $2325^{10}$, $3075^{10}$, $3225^{10}$, $3925^{10}$, $4000^{10}$, $4790^{10}$, $5100^{2}$, $5400^{2}$, $5891^{50}$, $6215^{10}$, $7137^{50}$, $8030^{10}$, $9466^{50}$, $10475^{10}$, $15620^{10}$, $21310^{10}$, $25335^{10}$, $34555^{10}$, $37485^{10}$, $49220^{10}$, $49235^{10}$, $57407^{50}$, $66335^{10}$, $98634^{50}$, $196765^{10}$, $359735^{10}$, $444620^{10}$, $468105^{10}$, $491250^{2}$, $541320^{10}$
\item $(h,k)=(7,23)$ : $1^{2}$, $3^{10}$, $50^{2}$, $3750^{2}$, $23700^{2}$, $45350^{2}$, $122931^{50}$, $13631075^{2}$
\item $(h,k)=(7,24)$ : $1^{2}$, $10^{10}$, $15^{2}$, $100^{2}$, $575^{2}$, $1125^{2}$, $2275^{2}$, $716225^{2}$, $1144025^{2}$, $7008625^{2}$, $7904200^{2}$
\item $(h,k)=(8,9)$ : $1^{2}$, $5^{2}$, $10^{2}$, $55^{10}$, $625^{2}$, $2975^{2}$, $5042^{50}$, $6475^{2}$, $8875^{2}$, $12500^{10}$, $36325^{2}$, $807700^{2}$, $2180850^{2}$, $3036750^{2}$, $10507800^{2}$
\item $(h,k)=(8,11)$ : $1^{2}$, $15^{2}$, $150^{2}$, $2575^{2}$, $36775^{2}$, $75290^{10}$, $337875^{2}$, $719975^{2}$, $15303400^{2}$
\item $(h,k)=(8,12)$ : $1^{2}$, $3^{10}$, $5^{10}$, $25^{4}$, $8275^{2}$, $12025^{2}$, $49375^{2}$, $169600^{2}$, $647275^{2}$, $3178115^{10}$
\item $(h,k)=(8,13)$ : $1^{32}$, $5^{20}$, $15^{10}$, $20^{10}$, $25^{10}$, $30^{10}$, $31^{50}$, $65^{50}$, $85^{10}$, $200^{2}$, $210^{10}$, $215^{10}$, $239^{50}$, $295^{10}$, $350^{10}$, $580^{10}$, $1080^{10}$, $1735^{50}$, $2070^{10}$, $2325^{10}$, $3585^{10}$, $4535^{10}$, $5780^{10}$, $5985^{10}$, $11421^{50}$, $11430^{10}$, $14130^{10}$, $21022^{50}$, $32945^{10}$, $33225^{2}$, $54178^{50}$, $68360^{10}$, $110625^{2}$, $146515^{10}$, $200315^{10}$, $213290^{10}$, $214950^{10}$, $249100^{10}$, $267260^{10}$, $392775^{2}$, $394045^{10}$, $470660^{10}$, $494425^{10}$
\item $(h,k)=(8,14)$ : $1^{2}$, $5^{2}$, $10^{2}$, $50^{2}$, $4000^{2}$, $7450^{2}$, $219525^{2}$, $515650^{2}$, $1931150^{2}$, $14099375^{2}$
\item $(h,k)=(8,16)$ : $1^{2}$, $15^{2}$, $55^{10}$, $100^{2}$, $775^{2}$, $850^{2}$, $2755^{10}$, $9250^{2}$, $14730^{10}$, $35775^{2}$, $142000^{2}$, $197850^{2}$, $565225^{2}$, $690325^{2}$, $1343375^{2}$, $1713550^{2}$, $11990425^{2}$
\item $(h,k)=(8,17)$ : $1^{2}$, $3^{11184810}$
\item $(h,k)=(8,18)$ : $1^{32}$, $3^{50}$, $5^{10}$, $6^{50}$, $15^{10}$, $20^{30}$, $25^{2}$, $29^{50}$, $35^{20}$, $85^{10}$, $90^{10}$, $95^{10}$, $100^{10}$, $120^{50}$, $195^{10}$, $320^{10}$, $490^{50}$, $510^{10}$, $900^{10}$, $1465^{20}$, $1600^{2}$, $2695^{10}$, $2888^{50}$, $5697^{50}$, $13714^{50}$, $14810^{10}$, $20795^{10}$, $45490^{10}$, $49885^{10}$, $55331^{50}$, $82705^{10}$, $101025^{10}$, $120375^{2}$, $134605^{10}$, $135720^{10}$, $167295^{10}$, $177890^{10}$, $247955^{10}$, $264910^{10}$, $292320^{10}$, $327870^{10}$, $347925^{2}$, $360175^{10}$, $438545^{10}$
\item $(h,k)=(8,19)$ : $1^{2}$, $5^{2}$, $10^{2}$, $150^{2}$, $1450^{2}$, $48050^{2}$, $112975^{2}$, $530100^{2}$, $561475^{2}$, $624040^{10}$, $2480560^{10}$
\item $(h,k)=(8,21)$ : $1^{2}$, $15^{2}$, $85^{10}$, $175^{6}$, $1400^{2}$, $1550^{2}$, $5650^{2}$, $16275^{2}$, $7783525^{2}$, $8967850^{2}$
\item $(h,k)=(8,22)$ : $1^{2}$, $3^{210}$, $25^{2}$, $100^{2}$, $110^{50}$, $600^{2}$, $5550^{2}$, $12250^{2}$, $165102^{50}$, $783325^{2}$, $11844750^{2}$
\item $(h,k)=(8,23)$ : $1^{32}$, $5^{40}$, $6^{50}$, $7^{50}$, $15^{10}$, $40^{10}$, $65^{10}$, $150^{2}$, $200^{10}$, $235^{20}$, $250^{2}$, $475^{2}$, $675^{10}$, $889^{50}$, $935^{10}$, $1146^{50}$, $1400^{50}$, $1800^{10}$, $3536^{50}$, $3655^{10}$, $4220^{50}$, $5815^{10}$, $16895^{10}$, $19789^{50}$, $47500^{10}$, $65250^{50}$, $81815^{10}$, $86770^{10}$, $127060^{10}$, $136220^{10}$, $182485^{10}$, $231685^{10}$, $259975^{10}$, $269800^{10}$, $270130^{10}$, $518460^{10}$, $523295^{10}$, $541350^{2}$
\item $(h,k)=(8,24)$ : $1^{2}$, $5^{2}$, $10^{2}$, $235^{10}$, $6025^{2}$, $16770000^{2}$
\item $(h,k)=(9,11)$ : $1^{2}$, $3^{10}$, $25^{2}$, $125^{2}$, $300^{2}$, $15600^{2}$, $377950^{2}$, $1205755^{10}$, $10354425^{2}$
\item $(h,k)=(9,12)$ : $1^{2}$, $15^{2}$, $250^{2}$, $615^{10}$, $1075^{2}$, $25045^{10}$, $140675^{2}$, $145150^{2}$, $245765^{10}$, $325275^{2}$, $14807650^{2}$
\item $(h,k)=(9,13)$ : $1^{2}$, $5^{2}$, $10^{2}$, $25^{10}$, $50^{2}$, $2875^{2}$, $7425^{2}$, $16766725^{2}$
\item $(h,k)=(9,14)$ : $1^{32}$, $5^{30}$, $10^{30}$, $30^{10}$, $75^{10}$, $115^{20}$, $265^{10}$, $285^{10}$, $485^{10}$, $555^{10}$, $1027^{50}$, $1325^{2}$, $2920^{50}$, $2973^{50}$, $3150^{2}$, $5200^{2}$, $7390^{10}$, $7799^{50}$, $10022^{50}$, $15364^{50}$, $20755^{10}$, $32590^{10}$, $34201^{50}$, $40425^{10}$, $45550^{2}$, $60433^{50}$, $198735^{10}$, $235725^{10}$, $253860^{10}$, $260460^{10}$, $265860^{10}$, $286915^{10}$, $339675^{2}$, $456645^{10}$, $541435^{10}$
\item $(h,k)=(9,16)$ : $1^{2}$, $3^{11184810}$
\item $(h,k)=(9,17)$ : $1^{2}$, $15^{2}$, $55^{10}$, $100^{2}$, $775^{2}$, $850^{2}$, $2755^{10}$, $9250^{2}$, $14730^{10}$, $35775^{2}$, $142000^{2}$, $197850^{2}$, $565225^{2}$, $690325^{2}$, $1343375^{2}$, $1713550^{2}$, $11990425^{2}$
\item $(h,k)=(9,18)$ : $1^{2}$, $5^{2}$, $10^{2}$, $75^{2}$, $95^{10}$, $200^{2}$, $275^{2}$, $2525^{2}$, $3525^{2}$, $3950^{2}$, $12300^{2}$, $94050^{2}$, $16659825^{2}$
\item $(h,k)=(9,19)$ : $1^{32}$, $5^{60}$, $10^{20}$, $15^{10}$, $25^{2}$, $35^{10}$, $40^{10}$, $50^{10}$, $320^{10}$, $505^{10}$, $645^{10}$, $685^{10}$, $715^{10}$, $880^{10}$, $900^{2}$, $1018^{50}$, $1795^{10}$, $2425^{50}$, $2655^{10}$, $5200^{10}$, $5625^{10}$, $7216^{50}$, $9910^{10}$, $11880^{10}$, $14771^{50}$, $16290^{10}$, $27600^{2}$, $30004^{50}$, $30300^{10}$, $34637^{50}$, $39627^{50}$, $41439^{50}$, $73660^{10}$, $97540^{10}$, $122370^{10}$, $172345^{10}$, $176120^{10}$, $300415^{10}$, $316775^{10}$, $318040^{10}$, $361910^{10}$, $467280^{10}$
\item $(h,k)=(9,21)$ : $1^{2}$, $3^{10}$, $75^{2}$, $125^{4}$, $12950^{2}$, $27775^{10}$, $758225^{2}$, $4202225^{2}$, $11664600^{2}$
\item $(h,k)=(9,22)$ : $1^{2}$, $15^{2}$, $2600^{2}$, $76575^{2}$, $667921^{50}$
\item $(h,k)=(9,23)$ : $1^{2}$, $5^{2}$, $10^{2}$, $15^{10}$, $37950^{2}$, $87450^{2}$, $94200^{2}$, $208650^{2}$, $406975^{2}$, $15941900^{2}$
\item $(h,k)=(9,24)$ : $1^{32}$, $10^{10}$, $21^{50}$, $25^{2}$, $45^{20}$, $50^{10}$, $107^{50}$, $130^{10}$, $245^{10}$, $345^{10}$, $542^{50}$, $790^{10}$, $1330^{50}$, $1360^{10}$, $1825^{2}$, $2665^{10}$, $3280^{10}$, $3880^{10}$, $9067^{50}$, $10525^{10}$, $12635^{10}$, $14235^{10}$, $22240^{10}$, $23330^{10}$, $51230^{10}$, $53880^{10}$, $74055^{10}$, $80125^{10}$, $95945^{10}$, $124885^{10}$, $163865^{10}$, $205895^{10}$, $217810^{10}$, $239495^{10}$, $247930^{10}$, $280535^{10}$, $302350^{10}$, $420930^{10}$, $536920^{10}$, $540375^{2}$
\item $(h,k)=(11,12)$ : $1^{2}$, $5^{2}$, $10^{2}$, $25^{2}$, $109^{50}$, $275^{2}$, $300^{2}$, $1175^{2}$, $777080^{10}$, $1014375^{2}$, $11872925^{2}$
\item $(h,k)=(11,13)$ : $1^{2}$, $15^{2}$, $20^{10}$, $75^{2}$, $130^{10}$, $280^{10}$, $675^{2}$, $78470^{10}$, $89375^{2}$, $163600^{2}$, $16128975^{2}$
\item $(h,k)=(11,14)$ : $1^{2}$, $3^{11184810}$
\item $(h,k)=(11,16)$ : $1^{32}$, $5^{30}$, $10^{30}$, $30^{10}$, $75^{10}$, $115^{20}$, $265^{10}$, $285^{10}$, $485^{10}$, $555^{10}$, $1027^{50}$, $1325^{2}$, $2920^{50}$, $2973^{50}$, $3150^{2}$, $5200^{2}$, $7390^{10}$, $7799^{50}$, $10022^{50}$, $15364^{50}$, $20755^{10}$, $32590^{10}$, $34201^{50}$, $40425^{10}$, $45550^{2}$, $60433^{50}$, $198735^{10}$, $235725^{10}$, $253860^{10}$, $260460^{10}$, $265860^{10}$, $286915^{10}$, $339675^{2}$, $456645^{10}$, $541435^{10}$
\item $(h,k)=(11,17)$ : $1^{2}$, $5^{2}$, $10^{2}$, $50^{2}$, $4000^{2}$, $7450^{2}$, $219525^{2}$, $515650^{2}$, $1931150^{2}$, $14099375^{2}$
\item $(h,k)=(11,18)$ : $1^{2}$, $15^{2}$, $25^{4}$, $50^{2}$, $80^{10}$, $300^{2}$, $450^{2}$, $1450^{2}$, $1950^{2}$, $6500^{2}$, $7250^{2}$, $47495^{10}$, $285425^{2}$, $346925^{2}$, $476125^{2}$, $861975^{2}$, $3575225^{2}$, $10975650^{2}$
\item $(h,k)=(11,19)$ : $1^{2}$, $3^{10}$, $20^{10}$, $60^{10}$, $100^{2}$, $200^{2}$, $625^{2}$, $16775875^{2}$
\item $(h,k)=(11,21)$ : $1^{32}$, $3^{50}$, $5^{30}$, $10^{10}$, $15^{60}$, $25^{20}$, $55^{10}$, $200^{50}$, $305^{10}$, $345^{10}$, $435^{10}$, $611^{50}$, $1680^{10}$, $2130^{10}$, $3150^{2}$, $3575^{10}$, $3865^{10}$, $4410^{10}$, $7765^{10}$, $14165^{10}$, $19150^{10}$, $68850^{10}$, $87135^{10}$, $108297^{50}$, $141750^{2}$, $155750^{2}$, $192990^{10}$, $241550^{2}$, $260740^{10}$, $454385^{10}$, $513885^{10}$, $527185^{10}$, $538230^{10}$
\item $(h,k)=(11,22)$ : $1^{2}$, $5^{2}$, $10^{2}$, $150^{2}$, $195^{10}$, $250^{2}$, $1075^{2}$, $1400^{2}$, $1925^{2}$, $5050^{2}$, $5725^{2}$, $8150^{2}$, $16752500^{2}$
\item $(h,k)=(11,23)$ : $1^{2}$, $15^{2}$, $75^{2}$, $300^{2}$, $525^{2}$, $11950^{2}$, $27250^{2}$, $130550^{2}$, $503000^{2}$, $1377300^{2}$, $2945250^{10}$
\item $(h,k)=(11,24)$ : $1^{2}$, $3^{10}$, $32280^{10}$, $659725^{2}$, $5772600^{2}$, $10183475^{2}$
\item $(h,k)=(12,13)$ : $1^{2}$, $3^{11184810}$
\item $(h,k)=(12,14)$ : $1^{2}$, $15^{2}$, $20^{10}$, $75^{2}$, $130^{10}$, $280^{10}$, $675^{2}$, $78470^{10}$, $89375^{2}$, $163600^{2}$, $16128975^{2}$
\item $(h,k)=(12,16)$ : $1^{2}$, $5^{2}$, $10^{2}$, $25^{10}$, $50^{2}$, $2875^{2}$, $7425^{2}$, $16766725^{2}$
\item $(h,k)=(12,17)$ : $1^{32}$, $5^{20}$, $15^{10}$, $20^{10}$, $25^{10}$, $30^{10}$, $31^{50}$, $65^{50}$, $85^{10}$, $200^{2}$, $210^{10}$, $215^{10}$, $239^{50}$, $295^{10}$, $350^{10}$, $580^{10}$, $1080^{10}$, $1735^{50}$, $2070^{10}$, $2325^{10}$, $3585^{10}$, $4535^{10}$, $5780^{10}$, $5985^{10}$, $11421^{50}$, $11430^{10}$, $14130^{10}$, $21022^{50}$, $32945^{10}$, $33225^{2}$, $54178^{50}$, $68360^{10}$, $110625^{2}$, $146515^{10}$, $200315^{10}$, $213290^{10}$, $214950^{10}$, $249100^{10}$, $267260^{10}$, $392775^{2}$, $394045^{10}$, $470660^{10}$, $494425^{10}$
\item $(h,k)=(12,18)$ : $1^{2}$, $3^{10}$, $2175^{2}$, $29650^{2}$, $16745375^{2}$
\item $(h,k)=(12,19)$ : $1^{2}$, $15^{2}$, $25^{2}$, $50^{2}$, $365^{10}$, $2625^{2}$, $441950^{2}$, $5203925^{2}$, $11126800^{2}$
\item $(h,k)=(12,21)$ : $1^{2}$, $5^{2}$, $10^{2}$, $50^{2}$, $500^{2}$, $550^{2}$, $5275^{2}$, $137685^{10}$, $143450^{2}$, $15938950^{2}$
\item $(h,k)=(12,22)$ : $1^{32}$, $4^{50}$, $5^{60}$, $10^{10}$, $16^{50}$, $20^{10}$, $22^{50}$, $65^{10}$, $70^{10}$, $100^{10}$, $130^{10}$, $235^{10}$, $269^{50}$, $305^{10}$, $400^{2}$, $713^{50}$, $840^{10}$, $865^{10}$, $1310^{10}$, $2030^{10}$, $2050^{10}$, $2357^{50}$, $2805^{10}$, $2820^{10}$, $2915^{10}$, $3015^{10}$, $3930^{10}$, $5110^{10}$, $5472^{50}$, $6515^{10}$, $8390^{10}$, $15150^{10}$, $24745^{50}$, $27525^{10}$, $29730^{10}$, $40055^{10}$, $64025^{10}$, $66215^{10}$, $105805^{10}$, $125085^{10}$, $196150^{10}$, $198025^{10}$, $204200^{2}$, $222170^{10}$, $256310^{10}$, $299745^{10}$, $309790^{10}$, $314370^{10}$, $333980^{10}$, $498840^{10}$
\item $(h,k)=(12,23)$ : $1^{2}$, $3^{10}$, $17050^{2}$, $20025^{2}$, $30575^{2}$, $82675^{2}$, $153450^{2}$, $232650^{2}$, $338840^{50}$, $7769775^{2}$
\item $(h,k)=(12,24)$ : $1^{2}$, $15^{2}$, $60^{10}$, $120^{150}$, $900^{1202}$, $1800^{18030}$
\item $(h,k)=(13,14)$ : $1^{2}$, $5^{2}$, $10^{2}$, $25^{2}$, $109^{50}$, $275^{2}$, $300^{2}$, $1175^{2}$, $777080^{10}$, $1014375^{2}$, $11872925^{2}$
\item $(h,k)=(13,16)$ : $1^{2}$, $15^{2}$, $250^{2}$, $615^{10}$, $1075^{2}$, $25045^{10}$, $140675^{2}$, $145150^{2}$, $245765^{10}$, $325275^{2}$, $14807650^{2}$
\item $(h,k)=(13,17)$ : $1^{2}$, $3^{10}$, $5^{10}$, $25^{4}$, $8275^{2}$, $12025^{2}$, $49375^{2}$, $169600^{2}$, $647275^{2}$, $3178115^{10}$
\item $(h,k)=(13,18)$ : $1^{32}$, $5^{20}$, $10^{20}$, $15^{10}$, $20^{10}$, $35^{10}$, $40^{10}$, $45^{10}$, $50^{20}$, $55^{10}$, $140^{10}$, $168^{50}$, $190^{10}$, $355^{10}$, $375^{10}$, $420^{10}$, $610^{10}$, $655^{10}$, $830^{10}$, $1025^{2}$, $1210^{10}$, $1265^{10}$, $1678^{50}$, $2552^{50}$, $2884^{50}$, $3335^{10}$, $4780^{10}$, $5050^{2}$, $6160^{10}$, $9810^{10}$, $13945^{10}$, $15135^{10}$, $16975^{10}$, $26500^{10}$, $45400^{10}$, $46405^{10}$, $46750^{10}$, $67665^{10}$, $73950^{10}$, $95570^{10}$, $100265^{10}$, $140210^{10}$, $203040^{10}$, $228150^{2}$, $307900^{2}$, $315150^{10}$, $347355^{10}$, $348965^{10}$, $378415^{10}$, $444735^{10}$, $453700^{10}$
\item $(h,k)=(13,19)$ : $1^{2}$, $5^{2}$, $10^{2}$, $25^{2}$, $45^{10}$, $75^{2}$, $2525^{2}$, $4225^{2}$, $212710^{10}$, $361375^{2}$, $428350^{2}$, $2457200^{2}$, $4399925^{2}$, $8059725^{2}$
\item $(h,k)=(13,21)$ : $1^{2}$, $15^{2}$, $60^{10}$, $6675^{2}$, $51900^{2}$, $118800^{2}$, $16599525^{2}$
\item $(h,k)=(13,22)$ : $1^{2}$, $3^{10}$, $5^{10}$, $25^{2}$, $50^{2}$, $150^{2}$, $3165^{10}$, $162850^{2}$, $16598275^{2}$
\item $(h,k)=(13,23)$ : $1^{32}$, $5^{10}$, $15^{10}$, $40^{10}$, $60^{10}$, $100^{2}$, $125^{10}$, $210^{10}$, $225^{10}$, $245^{20}$, $295^{10}$, $325^{2}$, $752^{50}$, $1615^{10}$, $2035^{10}$, $2070^{10}$, $2490^{10}$, $2515^{10}$, $2670^{10}$, $2825^{2}$, $4102^{50}$, $5640^{10}$, $6072^{50}$, $8484^{50}$, $8515^{10}$, $10335^{10}$, $10670^{10}$, $10900^{2}$, $12240^{10}$, $16580^{10}$, $21840^{10}$, $33028^{50}$, $46000^{2}$, $51040^{50}$, $51540^{10}$, $53740^{10}$, $56056^{50}$, $73480^{10}$, $87965^{10}$, $118630^{10}$, $165345^{10}$, $189585^{10}$, $191615^{10}$, $265470^{10}$, $307420^{10}$, $404705^{10}$, $439150^{10}$, $482075^{2}$
\item $(h,k)=(13,24)$ : $1^{2}$, $5^{2}$, $10^{2}$, $75^{2}$, $200^{2}$, $3425^{2}$, $13435^{50}$, $28780^{10}$, $324425^{2}$, $338400^{2}$, $364850^{2}$, $2206205^{10}$, $4235025^{2}$
\item $(h,k)=(14,16)$ : $1^{2}$, $3^{10}$, $25^{2}$, $125^{2}$, $300^{2}$, $15600^{2}$, $377950^{2}$, $1205755^{10}$, $10354425^{2}$
\item $(h,k)=(14,17)$ : $1^{2}$, $15^{2}$, $150^{2}$, $2575^{2}$, $36775^{2}$, $75290^{10}$, $337875^{2}$, $719975^{2}$, $15303400^{2}$
\item $(h,k)=(14,18)$ : $1^{2}$, $5^{2}$, $10^{2}$, $25^{2}$, $240^{10}$, $450^{2}$, $650^{2}$, $750^{2}$, $3075^{2}$, $253500^{2}$, $721200^{10}$, $12911550^{2}$
\item $(h,k)=(14,19)$ : $1^{32}$, $5^{10}$, $7^{50}$, $10^{20}$, $15^{80}$, $25^{10}$, $45^{10}$, $50^{10}$, $85^{10}$, $106^{50}$, $130^{10}$, $140^{10}$, $144^{50}$, $205^{10}$, $217^{50}$, $235^{10}$, $290^{20}$, $320^{20}$, $390^{50}$, $480^{10}$, $960^{10}$, $1405^{10}$, $1470^{10}$, $1575^{2}$, $1970^{10}$, $2460^{10}$, $3129^{50}$, $3950^{2}$, $4320^{10}$, $7150^{2}$, $7455^{10}$, $13425^{2}$, $23360^{50}$, $24364^{50}$, $31850^{10}$, $36025^{10}$, $42950^{10}$, $45630^{10}$, $53190^{10}$, $55745^{10}$, $60940^{20}$, $66435^{10}$, $67852^{50}$, $68000^{2}$, $73688^{50}$, $104265^{10}$, $149990^{10}$, $218110^{10}$, $315005^{10}$, $431325^{2}$, $480925^{10}$, $539270^{10}$
\item $(h,k)=(14,21)$ : $1^{2}$, $3^{10}$, $25^{12}$, $125^{2}$, $1000^{2}$, $3355185^{10}$
\item $(h,k)=(14,22)$ : $1^{2}$, $15^{2}$, $200^{10}$, $850^{2}$, $2720^{10}$, $12175^{2}$, $16749575^{2}$
\item $(h,k)=(14,23)$ : $1^{2}$, $5^{2}$, $10^{2}$, $350^{2}$, $1150^{2}$, $1825^{2}$, $2315^{10}$, $61325^{2}$, $135725^{2}$, $16565250^{2}$
\item $(h,k)=(14,24)$ : $1^{32}$, $5^{10}$, $10^{50}$, $15^{20}$, $20^{10}$, $45^{50}$, $60^{10}$, $61^{50}$, $70^{10}$, $85^{10}$, $135^{10}$, $150^{2}$, $175^{12}$, $200^{2}$, $320^{10}$, $335^{20}$, $340^{10}$, $372^{50}$, $385^{10}$, $550^{10}$, $895^{10}$, $1530^{10}$, $1750^{2}$, $2390^{10}$, $3945^{10}$, $4060^{50}$, $4795^{10}$, $5295^{10}$, $5320^{10}$, $7720^{10}$, $15270^{10}$, $18050^{10}$, $22895^{10}$, $28211^{50}$, $35355^{10}$, $37135^{10}$, $44706^{50}$, $46315^{10}$, $62720^{10}$, $69070^{10}$, $71617^{50}$, $87339^{50}$, $117665^{10}$, $144945^{10}$, $146530^{10}$, $378370^{10}$, $464895^{10}$, $470940^{10}$, $539950^{2}$
\item $(h,k)=(16,17)$ : $1^{2}$, $5^{2}$, $10^{2}$, $55^{10}$, $625^{2}$, $2975^{2}$, $5042^{50}$, $6475^{2}$, $8875^{2}$, $12500^{10}$, $36325^{2}$, $807700^{2}$, $2180850^{2}$, $3036750^{2}$, $10507800^{2}$
\item $(h,k)=(16,18)$ : $1^{2}$, $15^{2}$, $25^{2}$, $28600^{2}$, $34790^{10}$, $40405^{10}$, $83575^{2}$, $945050^{2}$, $5051750^{2}$, $10292225^{2}$
\item $(h,k)=(16,19)$ : $1^{2}$, $3^{10}$, $316^{50}$, $12525^{2}$, $19475^{2}$, $1077000^{2}$, $7377250^{2}$, $8283050^{2}$
\item $(h,k)=(16,21)$ : $1^{32}$, $5^{90}$, $15^{10}$, $30^{10}$, $48^{50}$, $50^{10}$, $70^{10}$, $100^{10}$, $270^{10}$, $305^{20}$, $390^{10}$, $930^{10}$, $995^{10}$, $2190^{10}$, $2195^{10}$, $3125^{2}$, $3179^{50}$, $7105^{10}$, $8760^{10}$, $8930^{10}$, $9695^{10}$, $12500^{2}$, $25220^{10}$, $39870^{10}$, $45521^{50}$, $62840^{10}$, $102890^{10}$, $115685^{10}$, $154225^{2}$, $166310^{10}$, $212520^{10}$, $237180^{10}$, $296270^{10}$, $334280^{10}$, $372250^{2}$, $410560^{10}$, $425780^{10}$, $531495^{10}$
\item $(h,k)=(16,22)$ : $1^{2}$, $5^{2}$, $10^{2}$, $38^{50}$, $75^{10}$, $29975^{2}$, $86330^{10}$, $236975^{2}$, $453773^{50}$, $1366800^{2}$, $3366150^{2}$
\item $(h,k)=(16,23)$ : $1^{2}$, $15^{2}$, $125^{2}$, $275^{2}$, $325^{2}$, $350625^{2}$, $7793525^{2}$, $8632325^{2}$
\item $(h,k)=(16,24)$ : $1^{2}$, $3^{10}$, $164^{50}$, $5050^{2}$, $2161525^{2}$, $14606525^{2}$
\item $(h,k)=(17,18)$ : $1^{2}$, $3^{10}$, $50^{2}$, $27180^{10}$, $47000^{2}$, $109925^{2}$, $770225^{2}$, $15714100^{2}$
\item $(h,k)=(17,19)$ : $1^{2}$, $15^{2}$, $115^{10}$, $225^{2}$, $950^{2}$, $2425^{2}$, $15100^{2}$, $202525^{2}$, $219800^{2}$, $685675^{2}$, $1970550^{2}$, $13679375^{2}$
\item $(h,k)=(17,21)$ : $1^{2}$, $5^{2}$, $10^{2}$, $25^{2}$, $100^{4}$, $125^{2}$, $275^{2}$, $425^{2}$, $428^{50}$, $600^{2}$, $2050^{2}$, $2865^{10}$, $34702^{50}$, $68925^{2}$, $72350^{2}$, $120325^{10}$, $2522425^{2}$, $2866750^{2}$, $9748850^{2}$
\item $(h,k)=(17,22)$ : $1^{32}$, $2^{300}$, $5^{30}$, $8^{50}$, $9^{50}$, $10^{20}$, $20^{10}$, $22^{50}$, $45^{10}$, $50^{12}$, $65^{10}$, $70^{10}$, $96^{50}$, $175^{2}$, $305^{10}$, $320^{10}$, $475^{10}$, $570^{10}$, $605^{10}$, $1000^{10}$, $3015^{10}$, $6565^{10}$, $7160^{10}$, $17918^{50}$, $32475^{2}$, $38335^{10}$, $100540^{50}$, $106668^{50}$, $448305^{10}$, $509425^{2}$, $533240^{10}$, $539735^{10}$, $540735^{10}$
\item $(h,k)=(17,23)$ : $1^{2}$, $3^{10}$, $750^{2}$, $7625^{10}$, $27650^{2}$, $1867875^{2}$, $14842800^{2}$
\item $(h,k)=(17,24)$ : $1^{2}$, $5^{10}$, $15^{2}$, $225^{2}$, $275^{2}$, $475^{2}$, $7160^{10}$, $53225^{2}$, $16687175^{2}$
\item $(h,k)=(18,19)$ : $1^{2}$, $5^{2}$, $10^{2}$, $14650^{2}$, $14750^{2}$, $39250^{2}$, $117530^{10}$, $169100^{2}$, $249875^{2}$, $697825^{2}$, $1113600^{2}$, $2778100^{10}$
\item $(h,k)=(18,21)$ : $1^{2}$, $5^{10}$, $15^{2}$, $50^{2}$, $100^{2}$, $7350^{2}$, $9650^{2}$, $25675^{2}$, $29775^{2}$, $362100^{2}$, $501715^{10}$, $2587275^{2}$, $3798325^{2}$, $7448300^{2}$
\item $(h,k)=(18,22)$ : $1^{2}$, $3^{10}$, $10^{10}$, $4350^{2}$, $48800^{2}$, $63935^{10}$, $261725^{2}$, $1982200^{2}$, $2829925^{2}$, $11330475^{2}$
\item $(h,k)=(18,23)$ : $1^{32}$, $5^{20}$, $8^{50}$, $10^{10}$, $15^{10}$, $30^{10}$, $65^{10}$, $160^{10}$, $234^{50}$, $235^{10}$, $285^{10}$, $455^{10}$, $860^{10}$, $955^{10}$, $1107^{50}$, $1155^{10}$, $1455^{10}$, $1475^{2}$, $5390^{10}$, $6375^{2}$, $8770^{10}$, $12690^{50}$, $18555^{10}$, $19800^{10}$, $22070^{10}$, $23015^{10}$, $25675^{2}$, $32977^{50}$, $59185^{10}$, $67190^{10}$, $85650^{2}$, $89325^{10}$, $90025^{10}$, $170725^{10}$, $262405^{10}$, $341180^{10}$, $364390^{10}$, $423025^{2}$, $462200^{10}$, $468080^{10}$, $533925^{10}$
\item $(h,k)=(18,24)$ : $1^{2}$, $5^{2}$, $10^{2}$, $475^{2}$, $900^{4}$, $1250^{2}$, $10675^{2}$, $121025^{2}$, $592350^{2}$, $2183650^{2}$, $13865975^{2}$
\item $(h,k)=(19,21)$ : $1^{2}$, $3^{10}$, $75^{2}$, $250^{10}$, $1750^{2}$, $45700^{2}$, $2243375^{10}$, $5511550^{2}$
\item $(h,k)=(19,22)$ : $1^{2}$, $15^{2}$, $50^{2}$, $125^{2}$, $850^{2}$, $1150^{2}$, $1475^{2}$, $2400^{2}$, $4450^{2}$, $6050^{2}$, $8175^{2}$, $22670^{50}$, $137225^{2}$, $319425^{2}$, $375375^{2}$, $464075^{2}$, $832250^{2}$, $1263650^{2}$, $3262500^{2}$, $9531225^{2}$
\item $(h,k)=(19,23)$ : $1^{2}$, $5^{2}$, $10^{2}$, $25^{10}$, $35^{10}$, $100^{2}$, $3000^{2}$, $17133^{50}$, $111925^{2}$, $173250^{2}$, $16060300^{2}$
\item $(h,k)=(19,24)$ : $1^{32}$, $10^{10}$, $15^{20}$, $20^{30}$, $25^{10}$, $35^{60}$, $40^{10}$, $50^{2}$, $80^{50}$, $90^{10}$, $105^{10}$, $160^{10}$, $425^{2}$, $565^{10}$, $655^{10}$, $670^{10}$, $900^{2}$, $2155^{10}$, $2198^{50}$, $2615^{10}$, $4950^{10}$, $6267^{50}$, $6816^{50}$, $14100^{10}$, $15835^{10}$, $25305^{10}$, $28709^{50}$, $64405^{10}$, $78320^{10}$, $81700^{10}$, $86005^{10}$, $110825^{2}$, $158760^{10}$, $160755^{10}$, $165780^{10}$, $201585^{10}$, $213535^{10}$, $249870^{10}$, $295630^{10}$, $376710^{10}$, $427820^{10}$, $484195^{10}$
\item $(h,k)=(21,22)$ : $1^{2}$, $5^{2}$, $10^{2}$, $100^{2}$, $175^{2}$, $650^{2}$, $1250^{2}$, $1355^{10}$, $1444^{50}$, $8010^{10}$, $311375^{2}$, $1085650^{2}$, $3059015^{10}$
\item $(h,k)=(21,23)$ : $1^{2}$, $15^{2}$, $50^{2}$, $135^{10}$, $300^{2}$, $525^{2}$, $1190^{10}$, $1325^{2}$, $2725^{2}$, $2840^{10}$, $26350^{2}$, $84625^{2}$, $191475^{2}$, $216575^{2}$, $326475^{2}$, $662000^{2}$, $675975^{2}$, $14567975^{2}$
\item $(h,k)=(21,24)$ : $1^{2}$, $3^{10}$, $8915^{10}$, $9550^{2}$, $15675^{2}$, $262925^{2}$, $784845^{10}$, $12520250^{2}$
\item $(h,k)=(22,23)$ : $1^{2}$, $3^{10}$, $5^{50}$, $50^{2}$, $1450^{2}$, $4025^{2}$, $9700^{2}$, $37275^{2}$, $79375^{2}$, $1489850^{2}$, $2798975^{2}$, $12356375^{2}$
\item $(h,k)=(22,24)$ : $1^{2}$, $15^{12}$, $50^{2}$, $5900^{2}$, $8625^{2}$, $26975^{2}$, $36850^{2}$, $2051535^{10}$, $6441050^{2}$
\item $(h,k)=(23,24)$ : $1^{2}$, $5^{2}$, $10^{2}$, $15^{560}$, $17^{950}$, $19^{350}$, $20^{130}$, $25^{6}$, $26^{100}$, $30^{400}$, $35^{30}$, $37^{50}$, $39^{50}$, $41^{50}$, $44^{800}$, $45^{590}$, $46^{650}$, $50^{866}$, $52^{400}$, $54^{100}$, $57^{300}$, $59^{200}$, $60^{590}$, $63^{200}$, $65^{130}$, $66^{350}$, $70^{320}$, $71^{150}$, $72^{50}$, $74^{500}$, $75^{180}$, $76^{200}$, $78^{100}$, $80^{380}$, $81^{200}$, $82^{200}$, $84^{200}$, $90^{700}$, $91^{200}$, $93^{50}$, $97^{300}$, $98^{200}$, $100^{118}$, $105^{40}$, $106^{400}$, $108^{250}$, $110^{290}$, $112^{200}$, $115^{10}$, $117^{500}$, $118^{50}$, $120^{480}$, $124^{150}$, $125^{224}$, $127^{100}$, $130^{510}$, $140^{440}$, $145^{160}$, $149^{150}$, $150^{196}$, $155^{40}$, $160^{100}$, $164^{50}$, $166^{100}$, $170^{20}$, $173^{100}$, $175^{86}$, $176^{100}$, $179^{100}$, $180^{310}$, $182^{100}$, $184^{150}$, $185^{10}$, $188^{150}$, $190^{120}$, $195^{50}$, $200^{340}$, $205^{50}$, $210^{170}$, $215^{100}$, $220^{320}$, $225^{380}$, $228^{50}$, $230^{80}$, $236^{150}$, $238^{50}$, $240^{130}$, $245^{290}$, $246^{50}$, $250^{270}$, $255^{160}$, $256^{50}$, $260^{40}$, $265^{50}$, $270^{150}$, $273^{50}$, $274^{50}$, $275^{308}$, $280^{30}$, $288^{100}$, $290^{60}$, $295^{10}$, $296^{50}$, $299^{100}$, $300^{360}$, $315^{80}$, $319^{50}$, $320^{150}$, $325^{40}$, $330^{160}$, $335^{50}$, $340^{20}$, $345^{110}$, $350^{808}$, $353^{50}$, $354^{50}$, $358^{50}$, $360^{100}$, $362^{50}$, $365^{20}$, $369^{50}$, $370^{170}$, $375^{58}$, $380^{60}$, $385^{60}$, $387^{50}$, $390^{90}$, $399^{50}$, $400^{260}$, $405^{40}$, $410^{20}$, $415^{80}$, $420^{10}$, $440^{20}$, $445^{40}$, $450^{398}$, $460^{20}$, $465^{40}$, $470^{90}$, $475^{100}$, $480^{20}$, $482^{50}$, $490^{70}$, $500^{326}$, $510^{60}$, $515^{10}$, $520^{30}$, $525^{66}$, $535^{40}$, $540^{20}$, $544^{50}$, $545^{70}$, $550^{220}$, $555^{40}$, $559^{50}$, $560^{110}$, $565^{60}$, $570^{80}$, $575^{214}$, $580^{80}$, $598^{50}$, $600^{264}$, $610^{40}$, $615^{10}$, $625^{32}$, $630^{40}$, $645^{20}$, $650^{182}$, $655^{70}$, $670^{30}$, $675^{132}$, $680^{20}$, $690^{40}$, $700^{100}$, $710^{20}$, $720^{20}$, $725^{150}$, $730^{20}$, $750^{228}$, $760^{20}$, $770^{30}$, $775^{112}$, $790^{40}$, $795^{20}$, $800^{140}$, $805^{10}$, $820^{30}$, $825^{50}$, $830^{60}$, $840^{20}$, $850^{106}$, $860^{20}$, $865^{10}$, $870^{50}$, $875^{66}$, $880^{20}$, $885^{40}$, $890^{50}$, $900^{146}$, $920^{50}$, $925^{42}$, $935^{10}$, $945^{30}$, $950^{206}$, $955^{30}$, $960^{40}$, $965^{20}$, $975^{30}$, $980^{10}$, $985^{20}$, $1000^{136}$, $1010^{20}$, $1020^{50}$, $1025^{62}$, $1035^{20}$, $1045^{20}$, $1050^{166}$, $1075^{4}$, $1080^{10}$, $1085^{10}$, $1100^{110}$, $1125^{110}$, $1140^{40}$, $1150^{168}$, $1165^{10}$, $1170^{10}$, $1175^{40}$, $1180^{20}$, $1185^{20}$, $1200^{50}$, $1210^{10}$, $1215^{10}$, $1220^{20}$, $1225^{64}$, $1250^{116}$, $1260^{10}$, $1270^{20}$, $1275^{46}$, $1290^{10}$, $1295^{30}$, $1300^{120}$, $1305^{30}$, $1310^{20}$, $1325^{86}$, $1340^{30}$, $1350^{128}$, $1370^{10}$, $1375^{42}$, $1380^{10}$, $1400^{152}$, $1425^{16}$, $1430^{10}$, $1445^{10}$, $1450^{108}$, $1460^{20}$, $1465^{10}$, $1470^{20}$, $1475^{18}$, $1480^{10}$, $1500^{58}$, $1525^{10}$, $1545^{10}$, $1550^{116}$, $1555^{10}$, $1560^{10}$, $1570^{10}$, $1575^{36}$, $1580^{30}$, $1585^{10}$, $1600^{60}$, $1605^{10}$, $1615^{20}$, $1625^{68}$, $1650^{44}$, $1660^{10}$, $1670^{30}$, $1675^{12}$, $1700^{98}$, $1725^{32}$, $1750^{52}$, $1760^{10}$, $1765^{10}$, $1775^{26}$, $1790^{20}$, $1800^{60}$, $1810^{10}$, $1815^{10}$, $1825^{32}$, $1830^{10}$, $1835^{10}$, $1850^{94}$, $1875^{24}$, $1880^{20}$, $1895^{10}$, $1900^{64}$, $1905^{10}$, $1910^{10}$, $1915^{10}$, $1925^{6}$, $1945^{10}$, $1950^{90}$, $1955^{10}$, $1975^{12}$, $2000^{46}$, $2025^{26}$, $2030^{10}$, $2040^{10}$, $2050^{130}$, $2075^{20}$, $2100^{94}$, $2125^{14}$, $2140^{10}$, $2150^{52}$, $2175^{36}$, $2200^{80}$, $2225^{18}$, $2230^{10}$, $2250^{26}$, $2275^{18}$, $2300^{32}$, $2305^{10}$, $2310^{10}$, $2325^{24}$, $2340^{10}$, $2350^{56}$, $2375^{20}$, $2395^{10}$, $2400^{62}$, $2410^{10}$, $2420^{10}$, $2450^{58}$, $2475^{38}$, $2500^{26}$, $2525^{26}$, $2550^{20}$, $2575^{32}$, $2585^{10}$, $2600^{54}$, $2625^{10}$, $2650^{30}$, $2675^{10}$, $2680^{10}$, $2690^{10}$, $2700^{30}$, $2750^{22}$, $2775^{8}$, $2800^{48}$, $2825^{16}$, $2850^{56}$, $2870^{10}$, $2875^{6}$, $2900^{20}$, $2925^{28}$, $2950^{56}$, $2965^{10}$, $2975^{12}$, $3000^{60}$, $3025^{14}$, $3050^{14}$, $3075^{26}$, $3100^{46}$, $3125^{8}$, $3150^{46}$, $3175^{10}$, $3200^{20}$, $3225^{14}$, $3230^{10}$, $3250^{32}$, $3275^{6}$, $3300^{20}$, $3325^{10}$, $3345^{10}$, $3350^{38}$, $3375^{18}$, $3400^{22}$, $3425^{16}$, $3450^{30}$, $3475^{4}$, $3500^{24}$, $3525^{10}$, $3550^{28}$, $3575^{14}$, $3600^{16}$, $3625^{6}$, $3650^{30}$, $3675^{6}$, $3700^{32}$, $3725^{28}$, $3750^{22}$, $3775^{4}$, $3800^{18}$, $3825^{6}$, $3850^{30}$, $3875^{4}$, $3900^{30}$, $3950^{22}$, $3975^{10}$, $4000^{20}$, $4025^{12}$, $4050^{44}$, $4075^{6}$, $4100^{18}$, $4125^{16}$, $4150^{30}$, $4175^{8}$, $4200^{28}$, $4250^{20}$, $4275^{8}$, $4300^{16}$, $4325^{6}$, $4350^{16}$, $4375^{4}$, $4400^{16}$, $4425^{4}$, $4450^{22}$, $4475^{6}$, $4500^{22}$, $4525^{6}$, $4550^{20}$, $4575^{4}$, $4600^{12}$, $4625^{4}$, $4650^{22}$, $4675^{2}$, $4700^{4}$, $4750^{22}$, $4800^{22}$, $4825^{4}$, $4850^{22}$, $4875^{8}$, $4900^{12}$, $4925^{8}$, $4950^{16}$, $4975^{10}$, $5025^{8}$, $5050^{14}$, $5075^{4}$, $5100^{12}$, $5125^{8}$, $5150^{12}$, $5175^{8}$, $5200^{10}$, $5225^{10}$, $5250^{26}$, $5275^{2}$, $5300^{22}$, $5325^{18}$, $5350^{12}$, $5375^{6}$, $5400^{8}$, $5450^{8}$, $5475^{8}$, $5500^{14}$, $5550^{20}$, $5575^{4}$, $5600^{10}$, $5625^{12}$, $5650^{16}$, $5675^{12}$, $5700^{32}$, $5725^{10}$, $5750^{16}$, $5775^{14}$, $5800^{8}$, $5850^{12}$, $5875^{4}$, $5900^{12}$, $5925^{6}$, $5950^{24}$, $5975^{6}$, $6000^{14}$, $6025^{6}$, $6050^{4}$, $6075^{6}$, $6100^{10}$, $6125^{10}$, $6150^{10}$, $6175^{4}$, $6200^{14}$, $6225^{8}$, $6250^{4}$, $6275^{6}$, $6300^{8}$, $6325^{8}$, $6350^{6}$, $6375^{8}$, $6400^{18}$, $6450^{8}$, $6500^{8}$, $6525^{4}$, $6550^{20}$, $6625^{6}$, $6650^{6}$, $6700^{12}$, $6725^{8}$, $6750^{4}$, $6800^{4}$, $6825^{10}$, $6850^{14}$, $6875^{4}$, $6900^{2}$, $6925^{4}$, $6950^{18}$, $6975^{8}$, $7000^{22}$, $7025^{6}$, $7050^{2}$, $7075^{4}$, $7100^{6}$, $7125^{2}$, $7150^{2}$, $7200^{8}$, $7225^{4}$, $7250^{6}$, $7275^{4}$, $7300^{6}$, $7325^{4}$, $7350^{6}$, $7375^{6}$, $7400^{4}$, $7425^{4}$, $7450^{2}$, $7475^{2}$, $7500^{4}$, $7525^{2}$, $7550^{10}$, $7575^{6}$, $7600^{8}$, $7625^{4}$, $7650^{4}$, $7675^{6}$, $7700^{8}$, $7725^{4}$, $7750^{8}$, $7775^{8}$, $7800^{2}$, $7825^{4}$, $7850^{2}$, $7875^{4}$, $7900^{2}$, $7925^{2}$, $7950^{6}$, $8000^{4}$, $8025^{2}$, $8050^{12}$, $8075^{2}$, $8100^{2}$, $8125^{4}$, $8150^{4}$, $8175^{10}$, $8200^{10}$, $8225^{4}$, $8250^{4}$, $8275^{4}$, $8300^{4}$, $8350^{6}$, $8375^{4}$, $8425^{6}$, $8450^{6}$, $8475^{4}$, $8500^{8}$, $8525^{8}$, $8550^{4}$, $8575^{4}$, $8600^{6}$, $8650^{4}$, $8675^{8}$, $8700^{6}$, $8725^{4}$, $8750^{10}$, $8775^{2}$, $8800^{2}$, $8875^{4}$, $8900^{2}$, $8925^{4}$, $8950^{2}$, $8975^{6}$, $9000^{2}$, $9025^{4}$, $9050^{4}$, $9075^{6}$, $9100^{18}$, $9150^{4}$, $9200^{6}$, $9225^{4}$, $9275^{2}$, $9300^{12}$, $9325^{4}$, $9350^{2}$, $9375^{6}$, $9400^{2}$, $9425^{2}$, $9450^{2}$, $9475^{4}$, $9525^{2}$, $9550^{2}$, $9650^{2}$, $9675^{2}$, $9725^{4}$, $9750^{2}$, $9775^{4}$, $9825^{2}$, $9850^{8}$, $9875^{8}$, $9950^{6}$, $9975^{2}$, $10000^{6}$, $10100^{2}$, $10125^{4}$, $10150^{2}$, $10175^{2}$, $10200^{4}$, $10225^{2}$, $10250^{2}$, $10275^{10}$, $10300^{2}$, $10350^{4}$, $10375^{2}$, $10400^{4}$, $10425^{2}$, $10450^{2}$, $10500^{2}$, $10525^{2}$, $10550^{2}$, $10575^{2}$, $10650^{2}$, $10675^{2}$, $10700^{6}$, $10725^{4}$, $10800^{2}$, $10825^{2}$, $10850^{4}$, $10900^{4}$, $10925^{2}$, $11000^{4}$, $11025^{2}$, $11050^{2}$, $11075^{2}$, $11125^{2}$, $11200^{4}$, $11225^{4}$, $11250^{2}$, $11325^{2}$, $11400^{4}$, $11500^{2}$, $11525^{2}$, $11550^{6}$, $11625^{2}$, $11650^{6}$, $11675^{2}$, $11700^{2}$, $11750^{4}$, $11800^{2}$, $11825^{2}$, $11925^{2}$, $11950^{2}$, $12000^{2}$, $12025^{4}$, $12050^{6}$, $12100^{4}$, $12150^{2}$, $12175^{2}$, $12200^{2}$, $12225^{4}$, $12250^{2}$, $12400^{4}$, $12450^{6}$, $12525^{2}$, $12575^{4}$, $12700^{6}$, $12750^{2}$, $12800^{4}$, $12850^{2}$, $12900^{2}$, $12950^{2}$, $13000^{4}$, $13050^{2}$, $13075^{2}$, $13150^{6}$, $13175^{2}$, $13200^{6}$, $13225^{2}$, $13275^{2}$, $13325^{2}$, $13350^{2}$, $13375^{2}$, $13450^{2}$, $13525^{2}$, $13625^{2}$, $13650^{2}$, $13900^{2}$, $13925^{2}$, $14025^{2}$, $14050^{2}$, $14175^{2}$, $14225^{2}$, $14250^{2}$, $14450^{2}$, $14650^{2}$, $14700^{2}$, $14725^{4}$, $14750^{2}$, $14800^{2}$, $14925^{2}$, $14950^{2}$, $15050^{2}$, $15125^{2}$, $15250^{4}$, $15400^{2}$, $15600^{2}$, $15650^{2}$, $15825^{2}$, $15875^{2}$, $16225^{2}$, $16325^{2}$, $16350^{2}$, $16375^{2}$, $16500^{2}$, $16550^{2}$, $16575^{2}$, $16650^{2}$, $16900^{2}$, $17075^{2}$, $17600^{2}$, $17650^{2}$, $17700^{2}$, $18400^{2}$, $18725^{2}$, $19150^{2}$, $19275^{2}$, $19600^{2}$, $20300^{2}$, $20625^{2}$, $20875^{2}$, $22000^{2}$, $22675^{2}$
\end{itemize}



\end{document}

